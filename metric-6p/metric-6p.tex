\documentclass[landscape, 8pt]{extarticle}
\usepackage{geometry}

% \usepackage{showframe}
% \setlength{\fboxsep}{-\fboxrule}

\usepackage[dvipsnames]{xcolor}

\usepackage{adjustbox}

\colorlet{colour1}{Red}
\colorlet{colour2}{Green}
\colorlet{colour3}{Cerulean}

\geometry{
    a4paper, 
    margin=0.17in
}

\pretolerance=0
\hyphenpenalty=0

\usepackage{lmodern}

\usepackage[fontsize=7pt]{scrextend}

\usepackage{multirow}
\usepackage{graphicx} % Required for inserting images
\usepackage{amsmath}
\usepackage{amsfonts}
\usepackage{amsthm}
\usepackage{amssymb}
% \usepackage{preamble}
\usepackage{enumitem}
\setlist{itemsep=0pt}


\usepackage{multicol}
\usepackage{lipsum}
\usepackage[framemethod=TikZ]{mdframed}
% \usepackage{../thmboxes_white}
\usepackage{../thmboxes_v2}
\usepackage{../customs}
\usepackage{float}
% \usepackage{setspace}
\usepackage[nodisplayskipstretch]{setspace}





\setlength{\parskip}{0.2em}

% Custom Definitions of operators
% \DeclareMathOperator{\im}{im}
% \DeclareMathOperator{\Fix}{Fix}
% \DeclareMathOperator{\Orb}{Orb}
% \DeclareMathOperator{\Stab}{Stab}
% \DeclareMathOperator{\send}{send}
% \DeclareMathOperator{\dom}{dom}
% \DeclareMathOperator{\Maps}{Maps}
% \DeclareMathOperator{\sgn}{sgn}
% \DeclareMathOperator{\Mat}{Mat}
% \DeclareMathOperator{\scale}{sc}
% \DeclareMathOperator{\Hom}{Hom}
% \DeclareMathOperator{\id}{id}
% \DeclareMathOperator{\rk}{rk}
% \DeclareMathOperator{\Tr}{tr}
% \DeclareMathOperator{\diag}{diag}
% \DeclareMathOperator{\can}{can}

\usepackage{hyperref} % note: this is the final package

\parindent = 0pt

\renewcommand\labelitemi{\tiny$\bullet$}

\begin{document}

\setlength{\abovedisplayskip}{3.5pt}
\setlength{\belowdisplayskip}{3.5pt}
\setlength{\abovedisplayshortskip}{3.5pt}
\setlength{\belowdisplayshortskip}{3.5pt}

\begin{multicols}{3}
\raggedcolumns


\section*{\huge Metric Spaces Exam Notes}
Made by Leon :) \textit{Note: Any reference numbers are to the lecture notes}

\vspace{-5pt}
\section{Introduction to Metric Spaces}

\begin{thm}[Definition of a Metric]{def:metric}{}
    \vspace{-5pt}
    Let $X$ be a non-empty set. A function $d: X \times X \to \mathbb{R} $ is called a \textbf{metric} iff for all $x,\,y,\,z\in X$,
    \vspace{-3pt}
    \begin{itemize}
        \item $d(x,y)\ge 0$ and $d(x,y)=0 \iff x = y$
        \item $d(x,y)=d(y,x)$
        \item $d(x,y)\le d(x,z)+d(z,y)$ (Triangle Inequality)
    \end{itemize}
    \vspace{-5pt}
    A non-empty set $X$ equipped with a metric $d$ is a \textbf{metric space}
\end{thm}

\vspace{-5pt}
\begin{dfn}[Real Vector Spaces]{dfn:modules-and-vector-spaces}{}
    \vspace{-5pt}
    A \textbf{real vector space $V$} is a set with two operations $(X, +, \cdot)$, where:
    \vspace{-15pt}
    \begin{itemize}[leftmargin=*]
        \item $+$ is addition, and $\cdot$ is scalar multiplication
        \item $(X, +)$ is an abelian group - i.e. for all (vectors) $x,y,z\in X$:
            \vspace{-5pt}
            \begin{itemize}
                \item \textbf{Closure}: $x + y\in X$
                \item \textbf{Commutativity}: $x + y = y + x$
                \item \textbf{Associativity}: $x + (y + z) = (x + y) + z$
                \item \textbf{Identity}: $\exists 0\in X$ s.t. for all $x\in X$ we have $0 + x = x + 0 = x$
                \item \textbf{Inverse}: $\forall x\in X$ we have $-x$ s.t. $x + (-x) = (-x) + x = 0$
            \end{itemize}
            
            \vspace{-5pt}
            \item Vector space axioms: for all $x,y,z\in X$ and $\mu, \lambda \in \mathbb{R}$ we have:
                \vspace{-5pt}
                \begin{itemize}
                    \item \textbf{Closure-ish thing}: $\lambda x\in X$
                    \item \textbf{Distributivity 1}: $\lambda(x + y) = \lambda x + \lambda y$
                    \item \textbf{Distributivity 2}:$(\lambda + \mu)x = \lambda y + \mu x$
                    \item \textbf{Associativity}: $\lambda (\mu x) = (\lambda \mu) x$
                    \item \textbf{Identity}: $1x = x$
                \end{itemize}
    \end{itemize}
\end{dfn}

\vspace{-5pt}
\begin{dfn}[Normed and Inner Product Spaces]{def:normed-space-inner-space}{}
    \vspace{-5pt}
    \textrule{\textbf{Normed Vector Spaces}}

    \vspace{-3pt}
    A \textbf{normed vector space} is a real vector space $X$ equipped with a \textbf{norm}, i.e. a function that assigns to every vector $x\in X$ a real number $\lVert x \rVert$ so that, for all vectors $x$ and $y$ in $X$ and all real scalars $a$:

    \vspace{-5pt}
    \begin{itemize}
        \item $\lVert x\rVert\ge 0 $ and $\lVert x \rVert = 0 \iff x = 0$
        \item $\lVert ax \rVert = \lvert a \rvert\lVert x \rVert$
        \item $\lVert x+y \rVert\le \lVert x \rVert + \lVert y \rVert$
    \end{itemize}

    \vspace{-7pt}
    \longrule{0.08ex}
    \textbf{Remark}: If $(X, \lVert \cdot \rVert)$ is a normed vector space then
    \[d(x,y) = \lVert x - y \rVert\]
    defines a metric in $X$

    \textbf{Remark}: This is a generalisation of the "length of a vector"


    \textrule{\textbf{Inner Product Spaces}}
    \vspace{-3pt}
    Let $X$ be a real vector space. An \textbf{inner product} on $X$ is a function that assigns to every pair $(x,y)\in X \times X $ a real number denoted by $\langle x,y \rangle$ and has the following properties:
    \vspace{-3pt}
    \begin{itemize}
        \item $\langle x,x \rangle\ge 0$ and $\langle x,x \rangle = 0 \iff x = 0$
        \item $\langle x,y \rangle = \langle y,x \rangle$
        \item $ax + by, z = a\langle x,z \rangle + b\langle y,z \rangle$
    \end{itemize}

    \vspace{-5pt}
    \longrule{0.08ex}
    A \textbf{real inner product space} is a real vector space equipped with an inner product.
    If $\langle \cdot, \cdot \rangle$ is an inner product on $X$, then
    \vspace{-5pt}
    \begin{itemize}
        \item $\lVert x \rVert = \sqrt{\langle x,x \rangle}$ defines a norm in $X$
        \item $d(x,y) = \lVert x - y \rVert$ defines a metric in $X$
    \end{itemize}
    \vspace{-5pt}
    \textbf{Remark}: This is a generalisation of the dot product
\end{dfn}


\vspace{-5pt}

\begin{dfn}[$n$-dimensional Euclidean space]{def:ndim-euclidean-space}{}
    \vspace{-5pt}
    Let $X = \mathbb{R}^{n} = \{(x_{1},x_{2},\dots,x_{n}) : x_{1},x_{2},\dots,x_{n}\in\mathbb{R}\}$
    \newline
    For $x = (x_{1},x_{2},\dots,x_{n})$, $y=(y_{1},y_{2},\dots,y_{n})$ in $\mathbb{R}^{n}$, define
    \[\langle x,y \rangle = x_{1}y_{1} + x_{2}y_{2} + \cdots + x_{n}y_{n} \text{ (inner product)}\]
    \[\lVert x \rVert_{2} = \langle x,x \rangle^{1/2} = \sqrt{x^{2}_{1} + x^{2}_{2} + \cdots + c^{2}_{n}}\text{(norm)}\]
\end{dfn}
\vspace{-5pt}

\begin{xmp}[Examples of Metric Spaces]{xmp:metric-spaces}{}
    \vspace{-5pt}
    Unless stated otherwise let $X = \mathbb{R}^{n}$. The case $X = \mathbb{R}^{2}$ is listed in \textcolor{red}{red}

    \def\arraystretch{1.5}
    \begin{center}
        \begin{tabular}{|c|l|}
        \hline
        Name & Norm and Metric \\
        \hline
        \multirow{2}{*}{Standard} & $\lvert x \rvert = \sqrt{\textcolor{red}{x^{2}_{1} + x^{2}_{2}} + \cdots + x^{2}_{n}}$ \\
        & $d(x, y) = \sqrt{\textcolor{red}{(x_{1} \! - \! y_{1})^{2} \! + \! (x_{2} \! - \! y_{2})^{2}} \! + \cdots + \! (x_{n} \! - \! y_{n})^{2}}$ \\

        \hline
        \multirow{2}{*}{Taxicab} & $\lVert x \rVert_{1} = \textcolor{red}{\lvert x_{1} \rvert + \lvert x_{2} \rvert} + \cdots \lvert x_{n} \rvert$ \\
                                 & $d_{1}(x, y) = \textcolor{red}{\lvert x_{1} - y_{1} \rvert + \lvert x_{2} - y_{2} \rvert} + \cdots + \lvert x_{n} - y_{n} \rvert$ \\
        \hline
        \multirow{2}{*}{Euclidean} & $\lVert x \rVert_{2} = \sqrt{\textcolor{red}{\lvert x_{1} \rvert^{2} + \lvert x_{2} \rvert^{2}} + \cdots \lvert x_{n} \rvert^{2}}$ \\
                                   & $d_{2}(x, y) = \sqrt{\textcolor{red}{\lvert x_{1} \! - \! y_{1} \rvert^{2} \! + \! \lvert x_{2} \! - \! y_{2} \rvert^{2}} \! + \cdots + \! \lvert x_{n} \! - \! y_{n} \rvert^{2}}$ \\
        \hline
        \multirow{2}{*}{$p$-metric} & $\lVert x \rVert_{p} = \displaystyle \left( \sum_{k = 1}^{n} \lvert x_{k} \rvert^{p}\right)^{1 /p}$ \\
                          & $d_{p}(x, y) = \displaystyle \left( \sum_{k = 1}^{n} \lvert x_{k} - y_{k} \rvert^{p}\right)^{1 /p}$ \\
        \hline
        \multirow{2}{*}{Chebyshev} & $\lVert x \rVert_{\infty} = \max \{\textcolor{red}{\lvert x_{1} \rvert, \lvert x_{2} \rvert},\dots, \lvert x_{n} \rvert\}$ \\
                                   & $d(x, y) = \max \{\textcolor{red}{\lvert x_{1} - y_{1} \rvert, \lvert x_{2} - y_{2} \rvert},\dots, \lvert x_{n} - y_{n} \rvert\}$ \\

        \hline
        \multirow{2}{*}{Discrete} & Not induced by a metric \\
                          & $d(x, y) = \begin{cases}
                              0 & x = y \\
                              1 & x \ne y
                          \end{cases}$ \\
        \hline
        \multirow{2}{*}{Post Office} & Not induced by a metric \\
                          & $d(x, y) = \begin{cases}
                              \lVert x \rVert_{2} + \lVert y \rVert_{2} & x = y \\
                              1 & x \ne y
                          \end{cases}$ \\
        \hline
    \end{tabular}
    \end{center}

    % TODO: cauchy shwarz?

    % This is the \textbf{Cauchy-Schwarz Inequality}. Various ways to prove this (watch lecture 1)

    \textrule{\textbf{The complex plane}}

    Let $X = \mathbb{C},\, d: \mathbb{C} \times \mathbb{C} \to \mathbb{R} $
    \[d(z,w) = \lvert z - w \rvert\]
    If $z = a+ib, w = c+id,\, a,b,c,d\in\mathbb{R}$, then
    \[z - w = (a - c) + i(b - d)\]
    therefore,
    \[d(z,w) = \sqrt{(a-c)^{2} + (b-d)^{2}}\]

\end{xmp}



\begin{xmp}[Sequence Spaces]{xmp:sequence-spaces}{}
    \vspace{-5pt}
    \textrule{\textbf{The space $\ell^{1}$}}

    $\ell^{1}$ is the set of real sequences $(x_{n})_{n\in\mathbb{N}}$ where $\sum_{n = 1}^{\infty}\lvert x_{n} \rvert$ converges.

    For $x = (x_{1},\dots,x_{n},\dots)\in \ell^{1}$, $y = (y_{1},\dots,y_{n},\dots)\in \ell^{1}$ we define
    \vspace{-2pt}
    \begin{itemize}[leftmargin=*]
        \item \textbf{Norm}: $\lVert x \rVert_{1} = \sum_{n = 1}^{\infty} \lvert x_{n} \rvert$
        \item \textbf{Metric}: $d_{1}(x, y) = \lVert x - y \rVert_{1} = \sum_{n = 1}^{\infty} \lvert x_{n} - y_{n} \rvert$
    \end{itemize}
        
    \vspace{-2pt}
    \textrule{\textbf{The space $\ell^{2}$}}

    $\ell^{2}$ is the set of real seqs $(x_{n})_{n\in N}$ where $\sum_{n = 1}^{\infty} \lvert x_{n} \rvert^{2}$ converges

    For $x = (x_{1},\dots,x_{n},\dots)\in \ell^{2}$, $y = (y_{1},\dots,y_{n},\dots)\in \ell^{2}$ we define
    \vspace{-6pt}

    \begin{itemize}[leftmargin=*]
        \item \textbf{Inner product}: $\langle x, y \rangle = \sum_{n = 1}^{\infty} x_{n}y_{n}$
        \item \textbf{Norm}: $\lVert x \rVert_{2} = \left(\sum_{n = 1}^{\infty}\lvert x_{n} \rvert^{2}\right)^{1 /2}$
        \item \textbf{Metric}: $d_{2}(x, y) = \lVert x - y \rVert_{2} = \left(\sum_{n = 1}^{\infty}\lvert x_{n} - y_{n} \rvert^{2}\right)^{1 /2}$
    \end{itemize}

    \vspace{-3pt}
    \textbf{Thm}: $\ell^{2}$ is a real vector space

    \textrule{\textbf{The space $\ell^{\infty}$}}

    $\ell^{\infty}$ is the set of all bounded sequences of real numbers
    For $x = (x_{1},\dots,x_{n},\dots),\,y = (y_{1},\dots,y_{n},\dots)\in \ell^{\infty}$
    \vspace{-3pt}
    \begin{itemize}[leftmargin=*]
        \item \textbf{Norm}: $\lVert x \rVert_{\infty} = \sup \{\lvert x_{1} \rvert,\dots,\lvert x_{n} \rvert,\dots\}$
        \item \textbf{Metric}: $\lVert x - y \rVert_{\infty} = \sup \{\lvert x_{1} - y_{1} \rvert,\dots,\lvert x_{n} - y_{n} \rvert,\dots\}$
    \end{itemize}

    \vspace{-3pt}
    \textrule{\textbf{The space $C([a, b])$}}

    $X = C([a, b])$ is the set of all continuous functions $f : [a, b]\to \mathbb{R}$
    \vspace{-3pt}
    \begin{itemize}[leftmargin=*]
        \item \textbf{Norm}: $\lVert f \rVert_{\infty} = \max \{ \lvert f(x) \rvert : a \le x \le b\}$
        \item \textbf{Metric}: $d_{\infty}(f, g) = \lVert f - g \rVert = \max \{\lvert f(x) - g(x) \rvert : a \le x \le b\}$
    \end{itemize}

    \vspace{-3pt}
    \textrule{\textbf{The $L^{1}$ metric}}

    $X$ is the set of all continuous functions $f : [a, b] \to \mathbb{R}$
    \vspace{-3pt}
    \begin{itemize}
        \item \textbf{Norm}: $\lVert f \rVert_{1} = \int_{a}^{b} \lvert f(x) \rvert dx$
        \item \textbf{Metric}: $d_{2}(f, g) = \lVert f - g \rVert_{1} = \int_{a}^{b} \lvert f(x) - g(x) \rvert dx$
    \end{itemize}
    \vspace{-3pt}

    \textrule{\textbf{The $L^{2}$ metric}}

    $X$ is the set of all continuous functions $f : [a, b] \to \mathbb{R}$
    \vspace{-3pt}
    \begin{itemize}
        \item \textbf{Inner Product}: $\langle f, g \rangle = \int_{a}^{b} f(x)g(x) dx$
        \item \textbf{Norm}: $\lVert f \rVert_{2} = \langle f, f \rangle^{1 /2} = \left(\int_{a}^{b} \lvert f(x) \rvert^{2} dx\right)^{1 /2}$
        \item \textbf{Metric}: $d_{1}(f, g) = \left(\int_{a}^{b} \lvert f(x) - g(x) \rvert^{2} dx\right)^{1 /2}$
    \end{itemize}
\end{xmp}

\begin{dfn}[Metric Subspaces]{dfn:metric-subspace}{}
    Let $(X, d)$ be a metric space and $Y$ a non-empty subset of $X$. Define
    \begin{itemize}
        \item $d_{Y} : Y \times Y \to \mathbb{R}$
        \item $d_{y}(y, y') = d(y, y')$
    \end{itemize}

    Then $d_{Y}$ is a metric on $Y$. $d_{Y}$ is called the \textbf{induced} or \textbf{inherited} metric, and $(Y, d_{Y})$ is said to be a metric subspace of the metric space $(X, d)$
\end{dfn}


\begin{dfn}[Open Ball]{def:open-ball}{}
    Let $(X, d)$ be a metric space, $c$ be a point in $X$, and $r > 0$. The \textbf{open ball} with center $c$ and radius $r$ is defined by
    \[B(c,r) = \{x\in X: d(c,x) < r\}\]
\end{dfn}

\newpage




% Note: there are lots of different notations for this, e.g. calling it a sphere
%
% \textbf{Example:} on the real line with the standard metric
% \[b(c,r) = \{x\in\mathbb{R} : \lvert x - c \rvert < r\} = (c - r, c + r)\]
%
% \textbf{Example:} on the real plane with the Euclidean metric, $X = \mathbb{R}^{2}$m
% \[d_{2}(x,y) = \sqrt{\lvert x_{1} - y_{1} \rvert^{2} + \lvert x_{2} - y_{2} \rvert}^{2}\]
% $B(c,r)$ is the open disc with center $c$ and radius $r$
%
% % \longrule{0.08ex}
%
% Watch lecture recording for examples of open balls on:
% \begin{itemize}
%     \item Discrete metric
%     \item $\mathbb{R}^{2}$ with the $d_{1}$ metric
%     \item $\mathbb{R}^{2}$ with the $d_{\infty}$ metric
% \end{itemize}
%

\section{Convergence}

\subsection{Convergent Sequences in Metric Spaces}
On the real line, $x_{n}\to x$ iff for every positive $\epsilon$, there exists an index $N$ such that for all indices $n$ where $n\ge N$, we have $\lvert x_{n} - x \rvert < \epsilon$.

\begin{dfn}[Convergent Sequence]{metric-convergence}{}
    Let $(X,d)$ be a metric space, $(x_{n})^{\infty}_{n=1}$ be a sequence in $X$, and $x\in X$. We say that $(x_{n})^{\infty}_{n=1}$ converges to $x$ iff for every positive $\epsilon$, there exists an index $N$ s.t. for all indices $n$ with $n\ge N$a we have $d(x_{n}, x) < \epsilon$.

    Observe that:
    \begin{itemize}
        \item $d(x_{n}, x)< \epsilon$ is equivalent to $x_{n}\in B(x,\epsilon)$.
        \item $x_{n}\to x$ in $(X,d)$ iff $d(x_{n}, x)\to 0$ on the real line
    \end{itemize}
\end{dfn}

\begin{thm}[Uniqueness of metric limit]{thm:uniqueness-metric-limit}{}
    \begin{itemize}
        \item Let $(X,d)$ be a metric space, and $x,x'\in X,\,x\ne x'$. Then there exists a positive radius $r$ s.t. $B(x,r)\cap B(x',r) = \emptyset$
        \item A sequence in a metric space can have at most one limit
    \end{itemize}
\end{thm}

%
% \textbf{Proof of first:} $d(x,x') > 0$ because $x \ne x'$. Choose any $r$ with $0  < r \le \frac{d(x,x')}{2}$. If $y\in B(x,r)$, then $d(y,x)<r$, therefore
% \[d(y,x' \ge d(x,x') - d(y,x) > d(x,x') - r)\]
% and $d(x,x') - r \ge r$, therefore
% \[d(y,x') > r\]
% Therefore, $y\not\in B(x', r)$
%
% \textbf{Proof of second: } Let $x_{n} \to x$ and $x_{n}\to x'$ in a metric space $(X,d)$. We claim that $x = x'$.
% Assume $x\ne x'$. Let $r > 0$ be s.t.
% \[B(x,r) \cap B(x',r) = \emptyset\]
% Since $x_{n}\to x$, there exists $N$ s.t. for all $n$ with $n \ge N$ we have
% \[x_{n} \in B(x,r)\]
% Since $x_{n}\to x$, there exists $N'$ s.t. for all $n$ with $n \ge N'$ we have
% \[x_{n} \in B(x',r)\]
% For any $n$ with $n \ge \max \{N, N'\}$, the term $x_{n}$ belongs to both balls - contradiction
%

\begin{xmp}[convergence in ($\mathbb{R}^N, d_2$)]{xmp:convergence}{}
    A sequence
    \begin{align*}
        x_{1} &= (x_{11}, \dots,x_{1j}, \dots x_{1N}) \\
        x_{2} &= (x_{21}, \dots,x_{2j}, \dots x_{2N}) \\
        \vdots & \\
        x_{n} &= (x_{n1}, \dots,x_{nj}, \dots x_{nN}) \\
        \vdots \\
        \downarrow \\
        x &= (x_{1}, \dots, x_{j}, \dots, x_{N})
    \end{align*}
    in $\mathbb{R}^{N}, d_{2}$ converges to $x = (x_{1}, \dots, x_{j}, \dots, x_{N})$ iff for each $j$,
    \[x_{nj}\xrightarrow[j\to+\infty]{} x_{j}\]
\end{xmp}

% \longrule{0.08ex}

% Watch lecture recording 23/01 for examples of:
% \begin{itemize}
%     \item Convergence in $\ell^{2}$
%     \item Convergecnce in $C([a,b])$
% \end{itemize}


\begin{dfn}[Bounded Sequence]{def:bounded-sequence}{}
    A sequence in a metric space is said to be \textbf{bounded} iff there exists an open ball that contains all of its terms

    \longrule{0.08ex}
    \textbf{Note}: this is the same definition as "sequence is bounded if there is upper and lower bound", as open ball implies the same thing
\end{dfn}


\begin{thm}[]{thm:convergent-seq-bound}{}
    Every convergence is bounded
    % TODO: lol
\end{thm}

% \textbf{Proof:} Let $x_{n}\to x$ in a metric space $(X,d)$. There exists an index $N$ s.t. for all $n$ with $n\ge N$,
% \[x_{n}\in B(x,1)\]
% Let $r$ be any positive number such that
% \[r>1,\,r>d(x,x_{1}),\dots,r>d(x,x_{N-1})\]
% Then, for all $n$,
% \[d(x_{n}, x) < r\]
% therefore
% \[x_{n}\in B(x,r)\]
%
\subsection{Cauchy Sequences}
Convergence: For every $\epsilon$, there is an $N$ such that for $n\ge N$, $d(x_{n}, x) < \epsilon$
\[x_{1}\quad x_{2} \quad \cdots \quad x_{N} \quad \cdots x_{n} \quad \cdots \quad \to x\]
Replace $x$ by any $x_{m}$ with $m \ge N$
\[x_{1}\quad x_{2} \quad \cdots \quad x_{N} \quad \cdots x_{n} \quad \cdots \quad x_{m} \quad \cdots\]
'$d(x_{n}, x) < \epsilon$' becomes '$\forall m \ge N,\,d(x_{n}, x_{m}) < \epsilon$'

\begin{dfn}[Cauchy Sequence]{def:cauchy-seq}{}
    A sequence $(x_{n})^{\infty}_{n=1}$ in a metric space $(X,d)$ is said to be a \textbf{Cauchy sequence} iff for every positive $\epsilon$, there exists an index $N$, s.t. for all indices $n,m$ with $n,m\ge N$,
    \[d(x_{n}, x_{m}) < \epsilon\]
\end{dfn}

\begin{thm}[]{thm:convergent-to-cauchy}{}
    If a sequence in a metric space converges, then it is a Cauchy sequence
\end{thm}

% TODO: The converse is not true.

\begin{dfn}[Complete Metric Spaces]{def:complete-ms}{}
    A metric space is said to be \textbf{complete} if and only if every Cauchy Sequence is convergent
\end{dfn}

\textbf{Examples:}
\begin{itemize}
    \item $\mathbb{R}$ with the standard metric is complete
    \item $\mathbb{Q}$ with the standard metric is not complete
    \item $(0,1)$ with the standard metric is not complete
    \item $[0,1]$ with the standard metric is complete
    \item $\mathbb{R}^{n},\,\ell^{p},\,C([a,b])$ is complete (proof later)
\end{itemize} 

\subsection{Open sets and closed sets}

\begin{dfn}[Open Sets and Closed Sets]{def:open-closed-set}{}
    Let $(X,d)$ be a metric space.
    \begin{itemize}
        \item A subset $G$ of $X$ is said to be \textbf{open} iff for every point $x$ in $G$ there exists a positive radius $r$ such that $B(x,r)\subseteq G$.
        \item A subset $F$ of $X$ is said to be \textbf{closed} iff $F^{c}$ is open
    \end{itemize}
\end{dfn}

\begin{dfn}[Discrete Metric Space]{def:discrete-ms}{}
    A metric space is called \textbf{discrete} iff all its subsets are open (equiv. all subsets are closed)

    \textbf{Example:} $[0,1]\cap (2,3)$
\end{dfn}


\begin{thm}[Properties of open sets]{thm:open-set-props}{}
    Let $(X,d)$ be a metric space
    \begin{enumerate}
        \item The union of any family of open sets is an open set
        \item The intersection of finitely many open sets is an open set
    \end{enumerate}
\end{thm}

\begin{thm}[Infinite open sets]{thm:infinite-open-sets}{}
    The intersection of infinitely many open sets is not always an open set
    
    For example, let $G_{n} = (- \frac{1}{n}, \frac{1}{n}), n = 1,2,\dots$ on the real line with the standard metric.

    Each $G_{n}$ is open but
    \[\bigcap\limits_{n = 1}^{\infty} G_{n} = \{0\}\]
\end{thm}

\begin{thm}[Relatively open sets]{thm:relatively-open}{}
    Let $(X,d)$ be a metric space and $A$ be a non-empty subset of $X$ equipped with the induced metric $d_{A}$. Let $G\subseteq A$. $G$ is open in $(A, d_{A})$ iff there exists a subset $O$ of $X$, open in $(X,d)$, such that $G = A \cap O$

    The open sets of $(A, d_{A})$ are sometimes referred to as \textbf{relatively open}
\end{thm}

\begin{thm}[]{thm:open-set-converge}{}
    Let $(X, d)$ be a metric space, $(x_{n})^{\infty}_{n=1}$ be a sequence in $X$ and $x$ be a point in $X$.

    $x_{n}\to x$ iff every open set that contains $x$ contains eventually all terms of the sequence
\end{thm}

\begin{dfn}[Neighbourhoods of points]{def:neighbourhood}{}
    An \textbf{open neighbourhood} of a point $x$ is any open set that contains $x$. $x_{n}\to x$ iff every open neighbourhood of $x$ contains eventually all terms of the sequence.

    \longrule{0.08ex}

    A \textbf{neighbourhood} of a point $x$ is a set that contains an open neighbourhood of $x$.  $x_{n}\to x$ iff every neighbourhood of $x$ contains eventually all terms of the sequence.
\end{dfn}

\begin{thm}[Properties of Closed sets]{thm:closed-set-props}{}
    Let $(X, d)$ be a metric space.
    \begin{enumerate}
        \item The intersection of any family of closed sets is a closed set
        \item The union of finitely many closed sets is a closed set.
    \end{enumerate}
\end{thm}

\begin{thm}[Infinite closed sets]{thm:infinite-closed-sets}{}
    The union of infinitely many closed sets is not always a closed set.

    For example, let $F_{n} = [\frac{1}{n}, 1], n=1,2,\dots,$ on the real line with the standard metric.
    Each $F_{n}$ is closed but
    \[\bigcup\limits_{n = 1}^{\infty}F_{n} = (0,1]\]
    is not closed.
\end{thm}

\begin{thm}[]{thm:closed-ms-convergence}{}
    A subset $F$ of a metric space is closed iff the limit of every convergent sequence of elements of $F$ belongs to $F$
\end{thm}

\newpage

\begin{itemize}
    \item In any metric space $(X,d)$, singletons $F = \{x\}$ are closed.
    \item In any metric space, any finite set is closed because
        \[\{x_{1},\dots,x_{n}\} = \{x_{1}\}\cup \cdots \cup \{x_{n}\}\]
\end{itemize}


\subsection{Closure}

\begin{dfn}[Closure]{def:closure}{}
    Let $(X, d)$ be a metric space and $A \subseteq X$. The \textbf{closure} of $A$, deonted by $\overline{A}$, is the smallest closed subset of $X$ that contains $A$

    There exists at least one closed subset of $X$ that contains $A$, namely $X$ itself. The smallest closed subset of $X$ that contains $A$ is
    \[\bigcap\limits_{A \subseteq F \subseteq X,\,F \text{closed}} F\]
\end{dfn}

\begin{thm}[Properties of Closure]{thm:closure-props}{}
    Let $(X, d)$ be a metric space and $A,\,B \subseteq X$.
    \begin{enumerate}
        \item $\overline{\emptyset} = \emptyset$ and $\overline{X} = X$
        \item $A \subseteq \overline{A}$ and $\overline{A}$ is closed
        \item $A$ is closed iff $A = \overline{A}$
        \item $\overline{\overline{A}} = \overline{A}$
        \item If $A \subseteq B$, then $\overline{A} \subseteq \overline{B}$
        \item $\overline{A \cup B} = \overline{A} \cup \overline{B}$
    \end{enumerate}
\end{thm}

\begin{itemize}
    \item On the real line with the standard metric, $\overline{(a,b)} = [a,b]$
    \item In $\mathbb{R}^{n}$ with the Euclidean metric $d_{2}$, the closure of the open ball $B(c,r)$ is the closed ball $\{x\in\mathbb{R}^{n}: d_{2}(x,c)\le r\}$
    \item On the complex plane with its standard metric, the closure of an open disc is the corresponding closed disc
    \item Let $X$ be a non-empty set with the discrete metric, $c\in X$ and $r =1$. Then $B(c,1) = \{c\}$, therefore $\overline{B(c,1)} - \overline{\{c\}} = \{c\}$, while
        \[\{x\in X : d(x,c) \le 1\} = X\]
        The closure of an open ball is not always equal to the corresponding closed ball
    \item $X = \mathbb{R},\,d(x,y) = \lvert x - y \rvert$. $\overline{\mathbb{Q}} = \mathbb{R}$
\end{itemize}

\begin{dfn}[Dense Subset of a Metric Space]{def:dense-subset}{}
    Let $(X, d)$ be a metric space. A subset $D$ of $X$ is said to be \textbf{dense} iff $\overline{D} = X$
\end{dfn}

Random fact: In $\mathbb{R}^{n}$ with the Euclidean metric $d_{2}$, $\mathbb{Q}^{n}$ is dense.

\begin{thm}[Closure Equivalence]{thm:closure-equivalence}{}
    Let $(X, d)$ be a metric space, $A \subseteq X, x\in X$. The following are equivalent
    \begin{enumerate}
        \item $x\in\overline{A}$
        \item For every positive $r$, $B(x,r) \cap A \ne \emptyset$
        \item There exists a sequence $(a_{n})_{n\in \mathbb{N}}$ with $a_{n}\in A$ for all $n$, such that $a_{n}\to x$
    \end{enumerate}
    A point $x$ with any of these properties is called an \textbf{adherent point} of $A$. So, $\overline{A}$ is the set of all adherent points of $A$.
\end{thm}

\begin{dfn}[Limit points of sets]{def:limit-point}{}
    Let $(X, d)$ be a metric space, $A \subseteq X$ and $x\in X$. We say that $x$ is a \textbf{limit point} or an \textbf{accumulation point} of $A$ iff every open ball centered at $x$ contains an element of $A$ distinct from $x$, i.e.
    \[\forall r > 0 \quad (B(x,r) \backslash \{x\}) \cap A \ne \emptyset\]
    The set of all limit points of $A$ is called the \textbf{derived set} of $A$ and is denoted by $A'$ or $\tilde{A}$.
\end{dfn}

\subsection{Continuous functions between metric spaces}

\begin{dfn}[Continuity at a point]{def:metric-point-continuity}{}
    Let $(X, d_{X}),\, (Y,d_{Y})$ be metric spaces and $f: X \to Y $ be a function. We say that $f$ is \textbf{continuous at a point} $x_{0}$ in $X$ iff for for every positive $\epsilon$, there exists a positive $\delta$, s.t., for all $x\in X$ with $d_{X}(x,x_{0}) < \delta$ we have $d_{Y}(f(x), f(x_{0})) < \epsilon$
    
    \longrule{0.08ex}
    Alternatively, $f$ is \textbf{continuous at a point} $x_{0}\in X$ iff, for every positive $\epsilon$, there exists a positive $\delta$, such that, for all $x\in B_{X}(x_{0}, \delta)$ we have $f(x)\in B_{Y}(f(x_{0}), \epsilon)$
\end{dfn}

\begin{dfn}[Continuity of a function]{def:metric-continuity}{}
    Let $(X, d_{X}), (Y, d_{Y})$ be metric spaces. A function $f: X \to Y $ is said to be \textbf{continuous} iff it is continuous at every point in $X$
\end{dfn}

\begin{thm}[]{}{}
    Let $(X,d_{X}), (Y, d_{Y})$ be metric spaces, $f : X \to Y$ be a function and $x_{0}$ be a point in $X$. Then $f$ is continuous at $x_{0}$ iff for every open neighbourhood $G$ of $f(x_{0})$ there exists an open neighbourhood $O$ of $x_{0}$ such that, for all $x\in O$, we have $f(x) \in G$
\end{thm}

\begin{thm}[Continuity and Convergence]{thm:continuity-conv-equivalence}{}
    Let $(X, d_{X}),\, (Y, d_{Y})$ be metric spaces, $x_{0}$ be a point in $X$, and $f : X \to Y$ be a function. The following are equivalent:
    \begin{enumerate}
        \item $f$ is continuous at $x_{0}$
        \item For every sequence $(x_{n})^{\infty}_{n = 1}$ in $X$, if $x_{n}\xrightarrow[n\to +\infty]{}$ in $(X, d_{X})$, then $f(x_{n}) \xrightarrow[n\to +\infty]{} f(x_{0})$ in $(Y, d_{Y})$
    \end{enumerate}
\end{thm}

\begin{thm}[Continuity and Open Sets]{def:continuity-open-sets}{}
    Let $(X, d_{X}),\,(Y, d_{Y})$ be metric spaces. A function $f: X \to Y $ is continuous iff the inverse image $f^{-1}(G)$ of any open subset $G$ of $Y$ is an open subset of $X$
\end{thm}

\section{Topology!!!}
\subsection{Homeomorphisms and Topological Properties}
\begin{dfn}[Topological Space]{def:topological-space}{}
    A \textbf{topological space} is a set $X$ together with a family $\mathcal{T}$ of subsets of $X$ that has the following properties:
    \begin{itemize}
        \item $\emptyset,X\in \mathcal{T}$
        \item Any union of elements of $\mathcal{T}$ is an element of $\mathcal{T}$
        \item Any finite intersection of elements of $\mathcal{T}$ is an element of $\mathcal{T}$
    \end{itemize}
    $\mathcal{T}$ is called a \textbf{topology} and the elements of $\mathcal{T}$ are called \textbf{open sets}
\end{dfn}

\begin{dfn}[Continuity of Topological Spaces]{def:topological-continuity}{}
    Let $(X, \mathcal{T}_{X})$ and $(Y, \mathcal{T}_{Y})$ be two topological spaces. A function $f : X \to Y$ is said to be \textbf{continuous} iff for every $G$ in $\mathcal{T}_{Y}$ the pre-image $f^{-1}(G)$ is an element of $\mathcal{T}_{X}$.

    $f$ is said to be a \textbf{homeomorphism} iff it is a continuous bijection and its inverse is continuous.

    If such a homeomorphism exists then $(X, \mathcal{T}_{X})$ and $(Y, \mathcal{T}_{Y})$ are said to be \textbf{homeomorphic}
\end{dfn}

If $(X, \mathcal{T}_{X})$ and $(Y, T_{Y})$ are homeomorphic, and one of them is compact or connected or separable etc etc, then so is the other

Properties that are preserved by homeomorphisms are called topological properties


\begin{thm}[$d : X \times X \to \mathbb{R}$ is continuous]{thm:metric-space-continuity}{}
    Let $(X, d)$ be a metric space. The function $f: X \times X \to \mathbb{R} $ is continuous.

    $\mathbb{R}$ is equipped with the standard metric. $X \times X$ is equipped with the product metric
\end{thm}

\subsubsection{Continuity of linear operators between normed vector spaces}
Let $(X, \lVert \cdot \rVert_{X}),\,(Y, \lVert \cdot \rVert_{Y})$ be normed vector spaces. Recall that $d_{X} : X \times X \to \mathbb{R},\,d(x, x') = \lVert x - x' \rVert_{X}$, and $d_{Y} : Y \times Y \to \mathbb{R},\, d_{Y}(y, y') = \lVert  y - y' \rVert_{Y}$ are metrics

\begin{dfn}[Bounded Linear Operators]{def:bounded-linear-operators}{}
    A linear operator $T : X \to Y$ is said to be \textbf{bounded} iff there exists a positive constant $C$ such that, for all $x\in X$,
    \[\lVert T(x) \rVert_{Y} \le C \lVert x \rVert_{X}\]
\end{dfn}

\begin{thm}[Linear Operator Equivalence]{def:linear-operator-equiv}{}
    Let $T: X \to Y $ be a linear operator. The following are equivalent:
    \begin{enumerate}
        \item $T$ is continuous
        \item $T$ is continuous at $0$
        \item $T$ is bounded
    \end{enumerate}
\end{thm}

\begin{itemize}
    \item Let $(X, \lVert  \cdot \rVert)$ be a normed vector space and define $f: \mathbb{R} \times X \to X $ by $f(\lambda, x) = \lambda x$. Define $g: X \times X \to X $ by $g(x,y) = x + y$. $f$ and $g$ are continuous
\end{itemize}


\newpage

\subsection{Fixed Points and Lipschitz}

\begin{dfn}[Lipschitz Functions]{def:lipschitz-functions}{}
    Let $(X, d_{X}),\,(Y, d_{Y})$ be metric spaces. A function $f : X \to Y$ is said to be a \textbf{Lipschitz} function iff there exists a constant $L$ such that for all $x,x'\in X$,
    \[d_{Y}(f(x), f(x')) \le L d_{X}(x,x')\]
    If $L < 1$, $f$ is said to be a \textbf{contraction}
\end{dfn}

\textbf{Note}: Magnus uses non-standard terminology here:
\begin{itemize}
    \item When the equation is satisifed and $L < 1$, Magnus calls $f$ a \textbf{strict contraction}
    \item He uses \textbf{contraction} for a functino $f$ that satisfies the weaker condition: for all $x, x' \in X$ with $x \ne x'$
        \[d_{Y}(f(x), f(x')) < d_{X}(x,x')\]
\end{itemize}

\begin{thm}[Lipschitz Continuity]{thm:lipschitz-continuity}{}
    Every Lipschitz function is continuous
\end{thm}

\begin{dfn}[Fixed Points]{def:fixed-points}{}
    A \textbf{fixed point} of a function $f: S \to S $ where $S$ is a non-empty set, is any element $x$ of $S$ such that $f(x) = x$

    Solving equations can sometimes be reduced to finding fixed points
\end{dfn}

Watch lecture recording 06/03 for more in-depth examples
\begin{itemize}
    \item Newton's Method for solving $f(x) = 0$
    \item Picard's Method for solving the Initial Value Problem
\end{itemize}

\begin{thm}[Banach's Fixed Point Theorem]{thm:complete-ms-fixed-point}{}
    Let $(X, d)$ be a complete metric space and let $f : X \to X$ be a contraction. Then $f$ has a unique fixed point
\end{thm}

\subsection{Equivalent Metrics}
\begin{dfn}[Equivalent Metrics]{def:equivalent-metrics}{}
    Two metrics on the same non-empty set $X$ are said to be \textbf{equivalent} iff they have the same open sets
\end{dfn}

\begin{thm}[Equivalent Metrics Theorem]{thm:equivalent-metrics}{}
    Let $d_{1}$ and $d_{2}$ be metrics on the same non-empty set $X$. If there exist positive constants $C$ and $C'$ such that for all $x, y$ in $X$,
    \[Cd_{1}(x, y) \le d_{2}(x, y) \le C'd_{1}(x, y)\]
    then $d_{1}$ and $d_{2}$ are equivalent
\end{thm}

\begin{dfn}[Limits of functions between metric spaces]{dfn:functions-metric-spaces}{}
    Let $(X, d_{X})$ and $(Y, d_{Y})$ be metric spaces, $x_{0}$ be a limit point of $X$, $y_{0}\in Y$ and $f : X \to Y$ be a function. We say that $\lim_{x\to x_{0}} f(x) = y_{0}$ iff
    \[\forall \epsilon > 0 \quad \exists \delta > 0 \quad \forall x\in B_{X}(x_{0}, \delta) \backslash \{x_{0}\} \quad f(x)\in B_{Y}(y_{0}, \epsilon)\]
\end{dfn}

\begin{thm}[Completeness of the Classical Spaces]{thm:completeness-of-classical-spaces}{}
    \vspace{-5pt}
    Some examples of complete metric spaces:
    \vspace{-12pt}
    \begin{multicols}{4}
    \begin{itemize}
        \item $(\mathbb{R}^{n}, d_{2})$
        \item $\ell^{2}$
        \item $C([a, b])$
        \item $\ell^{\infty}$
    \end{itemize}
    \end{multicols}
    \vspace{-5pt}
    \longrule{0.08ex}
    \vspace{-5pt}

    Let $(X, d_{X})$ and $(Y, d_{Y})$ be two metric spaces and assume that $(Y, d_{Y})$ is complete.

    Let $C(X, Y)$ be the set of all continuous and bounded functions from $X$ to $Y$. For $f, g\in C(X, Y)$ define
    \[D(f, g) = \sup \{d_{Y}(f(x), g(x)) : x\in X\}\]
    $D$ is a metric and the metric space $(C(X, Y), D)$ is complete
\end{thm}

\begin{dfn}[The product space \texorpdfstring{$X^{n}$}{Xn}]{dfn:x-n-product-space}{}
    Let $(X, d)$ be a metric space and $n\in \mathbb{N}$. Define $D : X^{n} \to \mathbb{R}$ by
    \[D(x_{1}, x_{2}) = d(x_{11}, x_{21}) + d(x_{12}, x_{22}) + \cdots + d(x_{1n}, x_{2n})\]
    \textbf{Lemma}: $D$ is a metric and a sequence converges in $(X^{n}, D)$ iff it converges componentwise

    \textbf{Lemma}: If $(X, d)$ is complete then $(X^{n}, D)$ is complete
\end{dfn}

\begin{dfn}[The product space \texorpdfstring{$X^{\mathbb{N}}$}{X^N}]{dfn:x-natural-product-space}{}
    Let $B^{A}$, where $A, B$ are sets, be the set of all functions from $A$ to $B$

    \longrule{0.08ex}
    \textbf{Def:} Let $(X, d)$ be a metric space. Define a metric $D : X^{\mathbb{N}} \times X^{\mathbb{N}} \to \mathbb{R}$ by
    \[D(x_{1}, x_{2}) = \sum_{n = 1}^{\infty} \frac{1}{2^{n}} \frac{d(x_{1n}, x_{2n})}{1 + d(x_{1n}, x_{2n})}\]
    where $x_{1} = (x_{11},\dots, x_{1n},\dots)$, $x_{2} = (x_{21},\dots,x_{2n},\dots)$

    $(X^{\mathbb{N}}, D)$ is called a \textbf{product space}

\end{dfn}

\begin{thm}[Convergence of Product spaces]{thm:product-space-convergence}{}
    Let $(X, d)$ be a metric space, let $(x_{k})_{k=1}^{\infty}$ be a sequence in $X^{\mathbb{N}}$ and let $x\in X^{\mathbb{N}}$. Write $x_{k} = (x_{k 1},\dots, x_{kn},\dots)$ and $x = (l_{1},\dots,l_{n},\dots)$. Then, $x_{k} \xrightarrow[k \to +\infty]{(X^{\mathbb{N}}, D)} x$ if and only if, for all $n$, $x_{kn}\xrightarrow[k \to +\infty]{(X^{\mathbb{N}} l_{n}}$
\end{thm}

\begin{thm}[Completeness of product spaces]{thm:product-space-completeness}{}
    Let $(X, d)$ be a complete metric space. Then the product space $(X^{\mathbb{N}}, D)$ is complete.
\end{thm}

\begin{thm}[Completeness of \texorpdfstring{$\mathbb{R}$}{R}]{thm:completeness-of-R}{}
    \vspace{-5pt}
    \begin{itemize}[leftmargin=*]
        \item \textbf{Thm (Least Upper Bound Principle)}: Every non-empty bounded above subset of $\mathbb{R}$ has a least upper bound
        \item \textbf{Thm (Monotone Convergence)}: Every bounded monotone sequence of real numbers has a limit
        \item \textbf{Thm ($\epsilon$-convergence)}: Let $A$ be a non-empty boudned subset of $\mathbb{R}$ and let $\epsilon$ be positive. If the distance between any two elements of $A$ is $< \epsilon$, then
        \[\sup(A) - \inf(A) \le \epsilon\]
        \item \textbf{Thm}: Every Cauchy sequence of real numbers is convergent
    \end{itemize}
\end{thm}

\begin{dfn}[Limit Superior and Inferior]{dfn:limsup-liminf}{}
    Let $(x_{n})^{\infty}_{n=1}$ be a Cauchy sequence in $\mathbb{R}$. Then $(x_{n})^{\infty}_{n=1}$ is bounded.

    Define:
    \[I_{n} = \inf \{x_{n}, x_{n+1},\dots\} \quad S_{n} = \sup \{x_{n}, x_{n+1},\dots\}\]

    \longrule{0.08ex}
    \textbf{Thm}: $(S_{n})^{\infty}_{n=1}$ and $(I_{n})^{\infty}_{n=1}$ are monotone and bounded
    \[I_{1} \le I_{n} \le S_{n} \le S_{1}, \quad n = 1,2,\dots\]
    Therefore $I_{n} \to I$ and $S_{n} \to S$ for some reals $I$ and $S$. Since $S_{n} - I_{n} \to 0$ we have $S = I$. We also have $x_{n}\to S = I$

    \textrule{\textbf{Limsup and Liminf}}
    \vspace{-8pt}

    \begin{itemize}[leftmargin=*]
        \item The limit of the sequence $(I_{n})^{\infty}_{n=1}$ is called the \textbf{limit inferior} of $(x_{n})^{\infty}_{n=1}$ and is denoted by $\liminf x_{n}$
            \[\liminf x_{n} = \lim_{n\to +\infty} I_{n} = \lim_{n\to +\infty} \inf \{x_{n}, x_{n+1},\dots\}\]
        \item The limit of the sequence $(S_{n})^{\infty}_{n=1}$ is called the \textbf{limit superior} of $(x_{n})^{\infty}_{n=1}$ and is denoted by $\limsup x_{n}$
            \[\limsup x_{n} = \lim_{n\to +\infty} S_{n} = \lim_{n\to +\infty} \sup \{x_{n}, x_{n+1},\dots\}\]
    \end{itemize}

    \longrule{0.08ex}
    \begin{itemize}[leftmargin=*]
        \item $\liminf x_{n}$ is the smallest subsequential limit of $(x_{n})^{\infty}_{n=1}$
        \item $\limsup x_{n}$ is the largest subsequential limit of $(x_{n})^{\infty}_{n=1}$
        \item $(x_{n})^{\infty}_{n=1}$ converges iff $\liminf x_{n} = \limsup x_{n}$
    \end{itemize}
\end{dfn}

\newpage
\section{Compactness}

\begin{dfn}[Compactness]{dfn:compactness}{}
    Let $X = \mathbb{R}$ and $d$ be the standard metric. A subset $K$ of $\mathbb{R}$ is said to be \textbf{compact} iff every sequence of elements of $K$ has a subsequence that converges to an element of $K$
\end{dfn}

\begin{xmp}[Examples of compactness]{xmp:compact-sets}{}
    \vspace{-13pt}
    \setlength{\columnseprule}{0.5pt}
    \begin{multicols}{2}
        \textbf{Compact sets}
        \begin{itemize}[leftmargin=*]
            \item $[a, b]$ is compact
            \item $\emptyset$ is compact
            \item $\mathbb{R} \cup \{-\infty, +\infty\}$ is compact!
        \end{itemize}

        \textbf{Not Compact sets}
        \begin{itemize}[leftmargin=*]
            \item $(0, 1)$ is not compact
            \item $\mathbb{R}$ is not compact
        \end{itemize}
    \end{multicols}
    \vspace{-3pt}
\end{xmp}

\vspace{-5pt}
\begin{thm}[Heine-Borel Theorem]{thm:heine-borel}{}
    On the real line with the standard metric, a set is compact if and only if it is closed and bounded

    \longrule{0.08ex}
    In $\mathbb{R}^{n}$ with the Euclidean metric, a set is compact if and only if it is closed and bounded
\end{thm}

\vspace{-5pt}
\begin{thm}[Continuous Functions on Compact Sets]{thm:cont-functions-on-compact-sets}{}
    Let $K \subseteq \mathbb{R}$ be compact, and $f : K \to \mathbb{R}$ be a continuous function. Then $f$ is bounded
\end{thm}

\begin{thm}[Extreme Value Theorem]{thm:evt}{}
    Let $K \subseteq \mathbb{R}$ be compact and $f : K \to \mathbb{R}$ be a continuous function. Then $f$ has a maximum and a minimum
\end{thm}

\begin{thm}[Open Covers]{thm:open-covers}{}
    An \textbf{open cover} of a set $S$ in a metric space is a family $(G_{i})_{i\in I}$ of open sets such that $S \subset \bigcup\limits_{i \in I} G_{i}$. A \textbf{subcover} of an open cover $(G_{i})_{i\in I}$ is a sub-family $(G_{i})_{i\in I'}$ where $I' \subset I$, such that $S \subseteq \bigcup_{i\in I'} G_{i}$

    \longrule{0.08ex}
    \textbf{Thm}: On the real line with the standard metric, a set $K$ is compact iff every open cover of $K$ has a finite subcover

    \longrule{0.08ex}
    \textbf{Lemma}: Every open cover of the interval $[a, b]$, where $a, b\in \mathbb{R}$, $a \le b$ has a finite cover

    \longrule{0.08ex}
    \textbf{Thm}: Let $K \subseteq \mathbb{R}$ and assume that every open cover of $K$ has a finite subcover. Then $K$ is closed and bounded, hence compact
\end{thm}

\begin{dfn}[Sequentially compact sets]{dfn:sequentially-compact}{}
    Let $(X, d)$ be a metric space and $K \subseteq X$
    \begin{itemize}[leftmargin=*]
        \item We say that $K$ is \textbf{sequentially compact} iff every sequence in $K$ has a subsequence that converges to an element of $K$
        
            For $K = X$ this becomes: $X$ is compact iff every sequence in $X$ has a convergent subsequence

        \item We say that $K$ is \textbf{compact} iff every open cover of $K$ has a finite subcover
    \end{itemize}

    These two notions of compactness are equivalent
\end{dfn}

\begin{thm}[idk more thms]{thm:compact-thms}{}
    \textbf{Thm}: Let $(X, d)$ be a metric space and $K \subseteq X, K\ne\emptyset$, $d_{K}$ be the induced metric on $K$. $K$ is a (sequentially) compact subset of $X$ iff the metric space $(K, d_{K})$ is (sequentially compact)

    \longrule{0.08ex}
    \textbf{Thm}: Let $(X, d)$ be a metric space and $K \subseteq X$ be sequentially compact. Then $K$ is closed and bounded

    \longrule{0.08ex}
    \textbf{Thm}: Let $(X, d)$ be a sequentially compact metric space and $K \subseteq X$. The set $K$ is sequentially compact iff it is closed

    \longrule{0.08ex}
    \textbf{Thm}: If a metric space $(X, d)$ is sequentially compact, then it is complete
\end{thm}

\begin{thm}[Extreme Value Theorem again]{thm:evt-again}{}
    Let $(X, d)$ be a metric space, $K$ be a sequentially compact subset of $X$ and $f : K \to \mathbb{R}$ be a continous function. Then $f$ has a maximum and a minimum. In particular, $f$ is bounded.
\end{thm}

\begin{dfn}[Uniform Continuity]{dfn:uniform-continuity}{}
    Let $(X, d_{X}), (Y, d_{Y})$ be metric spaces. A function $f : X \to Y$ is said to be \textbf{uniformly continuous} iff for every positive $\epsilon$ there exists a positive $\delta$ such that, for all $x, x'$ in $X$ with $d_{X}(x, x') < \delta$ we have $d_{Y}(f(x), f(x')) < \epsilon$

    \longrule{0.08ex}
    \textbf{Thm}: Let $(X, d_{X})$ be a sequentially compact metric space, $(Y, d_{Y})$ be a metric space and $f : X\to Y$ be a continuous function. Then $f$ is uniformly continuous
\end{dfn}

\begin{thm}[yet another compactness thm]{thm:compactness-thm-againnnnn}{}
    If a metric spasce $(X, d)$ is compact, then it is sequentially compact

    If a subset $K$ of $X$ is compact, prove that $K$ is sequentially compact

    Let $(X, d)$ be a compact metric space and $A$ be an infinite subset of $X$. Then $A$ has at least one limit point
\end{thm}


\begin{thm}[Lebesgue's Lemma]{thm:lebesgue-lemma}{}
    Let $(X, d)$ be a sequentially compact metric space and $X = \bigcup_{i\in I} G_{i}$ be an open cover of $X$. There exists a positive $\delta$ such that for any two points $x, y\in X$ with $d(x, y) < \delta$ there exists an $i$ such that $x, y\in G_{i}$. Any such $\delta$ is called a \textbf{Lebesgue number} of the open cover

    \textbf{Lemma}: Let $(D, d)$ be a sequentially compact metric space and $X = \bigcup_{i\in I} G_{i}$ be an open cover of $X$. Then there exists a $\delta > 0$ s.t. any nonempty subset of $X$ of diameter $< \delta$ can be covered by a single $G_{i}$
\end{thm}

\begin{dfn}[Totally bounded spaces]{thm:totally-bounded}{}
    A metric space $(X, d)$ is said to be \textbf{totally bounded} iff for every positive $\delta$, $X$ can be covered by a finite number of open balls of radius $\delta$.

    \textbf{Note}: If $(X, d)$ is totally bounded then it is bounded, but the converse is not necessarily true
\end{dfn}

\begin{thm}[Sequentially compactness boundedness]{thm:seq-compact-bound}{}
    If a metric space is sequentially compact, then it is totally bounded

    \longrule{0.08ex}
    \textbf{Thm}: Every sequentially compact metric space is compact. (From now on, refer to sequentially compact spaces as compact)

    \longrule{0.08ex}
    \textbf{Thm}: A metric space is compact iff it is complete and totally bounded
\end{thm}

\begin{dfn}[Countable and Uncountable Sets]{dfn:countable-sets}{}
    A set $S$ is said to be:
    \begin{itemize}
        \item \textbf{Infinitely coutnable} iff there is a bijection $f : \mathbb{N} \to S$
        \item \textbf{Countable} if it is finite or infinitely countable
        \item \textbf{Uncountable} iff it isn't countable
    \end{itemize}

    \textrule{\textbf{Examples}}
    \begin{itemize}
        \item $\{1, 2, 3\}$ and $\mathbb{R}$ are countable sets
        \item $\mathbb{Q}$ is infinitely countable
        \item $\mathbb{R}$ is uncountable
    \end{itemize}
\end{dfn}

\begin{thm}[Dense Subset equivalence]{thm:dense-equivalence}{}
    Let $(X, d)$ be a metric space and $D \subseteq X$. The following are equivalent:
    \begin{enumerate}
        \item $D$ is dense
        \item For every $x\in X$ and $\epsilon > 0$ there exists $y\in D$ s.t. $d(x, y) < \epsilon$
        \item For every $x\in X$ there is a sequence $(y_{n})^{\infty}_{n = 1}$ of elements of $D$ s.t. $y_{n}\to x$
        \item For every element $x\in X$ and every open nbhd $G$ of $x$, $G \cap D \ne \emptyset$
        \item $D$ intersects every non-empty open set
    \end{enumerate}
\end{thm}

\newpage
\begin{dfn}[Separable spaces]{dfn:separable-spaces}{}
    A metric space is said to be \textbf{separable} iff it has a countable dense subset

    \textrule{\textbf{Examples}}
    \begin{itemize}
        \item $\mathbb{R}$ with the standard metric is a separable metric because $\mathbb{Q}$ is dense and countable
        \item $\mathbb{R}^{n}$ with the Euclidean metric is a separable metric space because $\mathbb{Q}^{n}$ is dense and countable
        \item $\mathbb{C}$ with its standard metric is a separable metric space because $\{ z\in\mathbb{C} : \text{Re}(z), \text{Im}(z)\in \mathbb{Q}\}$
        \item $\ell^{2}$ is separable, and $\ell^{p}$ is separable for $1 \le p < \infty$
    \end{itemize}
\end{dfn}

\begin{thm}[Weierstrass Approximation Theorem]{thm:weierstrass-approximation}{}
    Let $f : [a, b]\to \mathbb{R}$ be a continuous function and $\epsilon > 0$. There exists a polynomial $p$ with \textit{real} coefficients s.t. for all $x\in [a,b]$
    \[\lvert f(x) - p(x) \rvert < \epsilon\]

    \longrule{0.08ex}
    Let $f : [a,b]\to \mathbb{R}$ be a continuous function and $\epsilon > 0$. There exists a polynomial $p$ with \textit{rational} coefficients s.t. for all $x\in [a,b]$
    \[\lvert f(x) - p(x) \rvert < \epsilon\]
\end{thm}

\begin{thm}[more theorems]{thm:more-thms}{}
    The set of all polynomials with rational coefficients is countable
    
    \textbf{Thm}: $C([a,b])$ is separable
\end{thm}

\begin{thm}[Separability of subspaces]{thm:separability-of-subspaces}{}
    Let $(X, d)$ be a separable metric space, $A \subseteq X$, $A \ne \emptyset$, and $d_{A}$ be the induced metric on $A$. Then the metric space $(A, d_{A})$ is separable

    \longrule{0.08ex}
    \textbf{Thm}: Every compact metric space is separable (compact $\implies$ separable)
\end{thm}

\begin{thm}[Open Ball countability]{thm:open-ball-countable}{}
    Let $(X, d)$ be a separable metric space and let $D$ be a countable dense subset of $X$. Let
    \[\mathcal{B} = \{B(c, r): c\in D, r\in \mathbb{Q}^{+}\}\]
    be the set of all open balls with centers in $D$ and rational radii. Then $\mathcal{B}$ is countable and every open set in $X$ can be written as a union of elements of $\mathcal{B}$
\end{thm}

\begin{dfn}[Open Bases and Second Countability]{dfn:open-bases-second-countable}{}
    \textrule{\textbf{Open Bases}}

    Let $(X, \mathcal{T})$ be a topological space. An \textbf{open base} (or \textbf{base}) for the topology $\mathcal{T}$ is a family $\mathcal{B}$ of open sets such that every open set in $\mathcal{T}$ can be written as a union of elements of $\mathcal{B}$

    \textrule{\textbf{Second Countability}}

    A topological space $(X, \mathcal{T})$ is said to satisfy the \textbf{second Axiom of Countability}, or is \textbf{second countable} iff it has a countable open base

    \longrule{0.08ex}
    \textbf{Thm}: In a separable metric space, every family of pairwise disjoint non-empty open sets is countable

    \longrule{0.08ex}
    \textbf{Thm}: On the real line with the standard metric, every open set can be written as a countable union of disjoint open intervals
\end{dfn}

\begin{dfn}[Continuous Extensions]{dfn:continuous-extensions}{}
    Let $(X, d_{X}), (Y, d_{Y})$ be metric spaces, $D$ be a dense subset of $X$, $f, g : X \to Y$ continuous functions s.t. $f(x) = g(x)$ for all $x\in D$. Then $f = g$

    \longrule{0.08ex}
    Let $(X, d_{X}), (Y, d_{Y})$ be metric spaces, $D \subseteq X$ be dense, $f : D \to Y$ be uniformly continuous, and assume that $(Y, d_{Y})$ is complete. Then $f$ has a unique continuous extension $F : X \to Y$
\end{dfn}

\begin{thm}[complete ms props]{thm:complete-ms-props}{}
    Let $(X, d)$ be a metric space, $F$ be a nonempty subset of $X$ and $d_{F}$ be the induced metric on $F$. If the metric space $(F, d_{F})$ is complete then $F$ is a closed subset of $X$

    \longrule{0.08ex}
    \textbf{Thm}: Let $(X, d)$ be a complete metric space, $F$ be a nonempty subset of $X$, and $d_{F}$ be the induced metric on $F$. If $F$ is a closed subset of $X$, then the metric space $(F, d_{F})$ is complete

    \longrule{0.08ex}
    \textbf{Thm}: Let $(X, d)$ be a complete metric space, $A \subseteq X$, $A \ne \emptyset$. Then
    \begin{enumerate}
        \item The metric space $(\overline{A}, d_{\overline{A}})$ is complete
        \item If $A \subseteq B \subseteq X$ and $(B, d_{B})$ is complete, then $\overline{A} \subseteq B$
    \end{enumerate}
\end{thm}

\begin{dfn}[Isometries]{dfn:isometries}{}
    Let $(X, d_{X})$ and $(Y, d_{Y})$ be metric spaces. A function $f : X \to Y$ is called a \textbf{isometry} iff for all $x_{1}, x_{2}\in X$,
    \[d_{Y}(f(x_{1}), f(x_{2})) = d_{X}(x_{1}, x_{2})\]

    \longrule{0.08ex}
    \textbf{Thm}: Let $(X, d_{X})$ and $(Y, d_{Y})$ be metric spaces and $f : X \to Y$ be an isometry. Then $f$ is an injection. If, moreover, $f$ is a surjection (hence $f$ bij.) then $f^{-1} : Y \to X$ is also an isometry

    \longrule{0.08ex}
    \textbf{Thm}: The metric spaces $(X, d_{X})$ and $(Y, d_{Y})$ are said to be \textbf{isometric} iff there exists an isometry $f$ from $X$ onto $Y$

    \longrule{0.08ex}
    \textbf{Thm}: if two metric spaces are isometric and one of them is complete/compact/connected/... then so is the other
\end{dfn}

\begin{thm}[Isometry completion]{thm:isometry-completion}{}
    Let $(X, d)$ be a bounded metric space and let $C(X, \mathbb{R})$ be the set of all bounded continuous functions $f : X \to \mathbb{R}$ equipped with the metric
    \[d_{\infty}(f_{1}, f_{2}) = \sup \{\lvert f_{1}(t) - f_{2}(t) \rvert : t\in X\}\]
    For each $x\in X$, define $F_{X} : X \to \mathbb{R}$ be $F_{X}(x') = d(x, x')$. Then
    \begin{enumerate}
        \item $F_{X} \in C(X, \mathbb{R})$
        \item The map $X \to C(X, \mathbb{R}), x \mapsto F_{X}$ is an isometry
        \item $X^{*} = \{F_{X} : x\in X\}$, equipped with the induced metric, is a subspae of $C(X \mathbb{R})$ isometric to $X$
        \item The closure $\overline{X^{*}}$ of $X^{*}$ in $C(X, \mathbb{R})$, equipped with the induced metric, is a complete metric space
        \item $X^{*}$ is dense in $\overline{X^{*}}$
    \end{enumerate}
\end{thm}

\begin{dfn}[Completion of a Metric Space]{dfn:completion-ms}{}
    Let $(X, d)$ be a metric space. A \textbf{completion} of $(X, d_{X})$ is any metric space $(Y, d_{Y})$ with the following properties
    \begin{enumerate}
        \item $(Y, d_{Y})$ is complete
        \item $(Y, d_{Y})$ has a subspace $X^{*}$ isometric to $(X, d_{X})$
        \item $X^{*}$ is dense in $Y$
    \end{enumerate}

    It can be shown that any two completions of $X$ are isometric to each other, i.e. a completion is unique up to isometries
\end{dfn}


\begin{dfn}[Construction of Completion via Cauchy]{dfn:completion-cauchy}{}
    Let $(X, d)$ be a metric space and let $\mathcal{C}$ be the set of all Cauchy sequences of elements of $X$

    We define an equivalence relation $\sim$ in $\mathcal{C}$ as follows: Let $x = (x_{n})_{n\in \mathbb{N}}$, $y=(y_{n})_{n\in \mathbb{N}}\in \mathcal{C}$. We say that $x \sim y$ iff $d(x_{n}, y_{n})\to 0$

    Distinct equivalence classes are disjoint and partition $\mathcal{C}$

    The set of all equivalence classes is called the \textbf{quotient space}, denoted $\mathcal{C} / \sim$
    
    \longrule{0.08ex}
    Define a metric $D$ on $\mathcal{C} / \sim$ as follows:

    Let $\alpha, \beta\in \mathcal{C} /\sim$. Then 
    \[\alpha = [(x_{1},\dots,x_{n},\dots)] \text{ and } \beta = [(y_{1},\dots,y_{n},\dots)]\]
    for some $(x_{1},\dots,x_{n},\dots), (y_{1},\dots,y_{n},\dots)\in \mathcal{C}$. Define
    \[D(\alpha, \beta) = \lim_{n\to +\infty} d(x_{n}, y_{n})\]
    $(\mathcal{C} / \sim, D)$ is complete. Additionally, the following is an isometry:
    \[X \to \mathcal{C} / \sim \qquad x \mapsto ([x,x,\dots,x,\dots])\]
    Let $X^{*}$ be its range. The metric space $(X^{*}, D_{X^{*}})$ is isometric to $(X, d), (\overline{X^{*}}, D_{\overline{X^{*}}})$ is a complete metric space, and $X^{*}$ is dense in $\overline{X^{*}}$
\end{dfn}

\newpage
\begin{dfn}[Connected and Disconnected Spaces]{dfn:connected-spaces}{}
    A metric space $(X, d)$ is said to be \textbf{disconnected} iff there exists non-empty disjoint open sets $G_{1}$ and $G_{2}$ such that
    \[X = G_{1} \cup G_{2}\]
    Otherwise the metric space is called \textbf{connected}

    \longrule{0.08ex}
    A non-empty subset $A$ of a metric space $(X, d)$ is said to be disconnected iff the metric space $(A, d_{A})$, where $d_{A}$ is the induced metric, is disconnected
\end{dfn}

\begin{thm}[Connected Theorems]{thm:connection}{}
    A subset $O$ of $A$ is open in $(A, d_{A})$ iff $O = A \cup G$ for some $G$ that is open in $X$ % TODO: exercise 18

    Therefore, $A$ is disconnected iff there exist open subsets $G_{1}, G_{2}$ of $X$ s.t.
    \begin{itemize}[leftmargin=*]
        \item $A = (A \cap G_{1}) \cup (A \cap G_{2})$, which is equivalent to $A \subseteq G_{1} \cup G_{2}$
        \item $A \cap G_{1} \ne \emptyset,\, A \cap G_{2}\ne \emptyset$
        \item $(A \cap G_{1}) \cap (A \cap G_{2}) = \emptyset$, which is equivalent to $A \cap G_{1} \cap G_{2} = \emptyset$
    \end{itemize}

    \longrule{0.08ex}
    \textbf{Thm}: $\mathbb{R}$ with the standard metric is connected

    \textbf{Thm}: On the real line with the standard metric, all intervals are connected sets

    \textbf{Thm}: A non-empty subset of the real line is connected iff it is an interval

    \textbf{Thm}: A metric space $(X, d)$ is connected iff the only subsets of $X$ with empty boundary are $\emptyset$ and $X$

    \textbf{Thm}: Let $(X, d_{X})$ be a connected metric space, $(Y, d_{Y})$ be a metric space and $f : X \to Y$ be a continuous surjection. Then $(Y, d_{Y})$ is connected as well
\end{thm}

\begin{thm}[Intermediate Value Theorem]{thm:ivt}{}
    Let $(X, d)$ be a connected metric space and $f : X \to \mathbb{R}$ be a continuous function. If $x_{1}, x_{2}\in X$ with $f(x_{1} \ne f(x_{2}))$ and $y$ is a real number between $f(x_{1})$ and $f(x_{2})$, then there exists an $x\in X$ such that $f(x) = y$
\end{thm}

\begin{thm}[Clopen]{thm:clopen}{}
    A metric space $(X, d)$ is connected iff the only clopen subsets are $\emptyset, X$
\end{thm}

\begin{dfn}[Connected Components]{dfn:connected-components}{}
    Let $(X, d)$ be a metric space. We define an equivalence relation $\sim$ in $X$ as follows: $x \sim x'$ iff there exists a connected subset $C$ of $X$ that contains both $x$ and $x'$

    \textbf{Thm}: If $(C_{i})_{i\in I}$ is a family of connected subsets of $X$ with nonempty intersection, then $\bigcup_{i\in I} C_{i}$ is connected
\end{dfn}

\begin{thm}[Big equivalence classes]{thm:big-equivalence-classes}{}
    The equivalence class of any point in $X$ is the largest connected subset of $X$ that contains that point (what point?)
\end{thm}

\begin{dfn}[Path Connected Metric Spaces]{dfn:path-connected}{}
    Let $(X, d)$ be a metric space and $x_{0}, x_{1}\in X$. A \textbf{path} in $X$ from $x_{0}$ to $x_{1}$ is a continuous function $\gamma : [0, 1] \to X$ s.t. $\gamma(0) = x_{0}$, $\gamma(1) = x_{1}$

    $(X, d)$ is said to be \textbf{path-connected} iff for any two points $x_{0}, x_{1}$ in $X$ there is a path in $X$ from $x_{0}$ to $x_{1}$

    A non-empty subset $A$ of $X$ is said to be \textbf{path-connected} iff the metric space $(A, d_{A})$, where $d_{A}$ is the induced metric, is path connected
\end{dfn}

\begin{thm}[Path connected theorem]{thm:path-connected-thm}{}
    Every path-connected metric space is connected

    Not every connected metric space is necessarily path-connected
\end{thm}


\section{Applications}

% TODO: newton and heron's method

\begin{dfn}[Equivalent Norms]{dfn:equivalent-norms}{}
    Two norms on the same real vector space are said to be equivalent iff their corresponding metrics are equivalent

    \longrule{0.08ex}
    If $\lVert \cdot \rVert_{1}$ and $\lVert \cdot \rVert_{2}$ are norms on the same real vector space $X$ and there exist positive constants $C$ and $C'$ such that, for all $x\in X$,
    \[D \lVert x \rVert_{1} \le \lVert x \rVert_{2} \le C' \lVert x \rVert_{1}\]
    then they are equivalent
\end{dfn}

\begin{thm}[p metric again?]{dfn:p-metric}{}
    For any $p$ with $1 \le p < \infty$ and any $x\in \mathbb{R}^{n}$ we define
    \[\lVert x \rVert_{p} = (\lvert x_{1} \rvert^{p} + \cdots + \lvert x_{n} \rvert^{p})^{1 /p}\]

    \textrule{Young's Inequality}
    
    Let $1 \le p, q \le \infty$ s.t. $\frac{1}{p} + \frac{1}{q} = 1$, and $a, b\le 0$. Then
    \[ab \le \frac{a^{p}}{p} + \frac{b^{q}}{q}\]

    \textrule{Holder Inequality}

    Let $1 \le p, q \le \infty$ s.t. $\frac{1}{p} + \frac{1}{q} = 1$ and $x, y\in \mathbb{R}^{n}$. Then
    \[\sum_{i = 1}^{n} \lvert x_{i}y_{i} \rvert \le \lVert x \rVert_{p} \lVert y \rVert_{q}\]

    \textrule{Equivalence of $p$-metrics}
    
    \textbf{Thm}: Any of the following norms are equivalent:
    \begin{align*}
        \lVert x \rVert_{p} &= (\lvert x_{1} \rvert^{p} + \cdots + \lvert x_{n} \rvert^{p})^{1 /p},\,x\in \mathbb{R}^{n},\, 1\le p < \infty\\
        \lVert x \rVert_{\infty} &= \max \{\lvert x_{1} \rvert,\dots,\lvert x_{n} \rvert\},\,x\in\mathbb{R}^{n}
    \end{align*}

    \textbf{Thm}: Let $1 \le p \le q < \infty$. For all $x\in\mathbb{R}^{n}$:
    \[\lVert x \rVert_{q} \le \lVert x \rVert_{p}\]
    As a consequence,
    \[\lVert x \rVert_{\infty} \le \lVert x \rVert_{q} \le \lVert x \rVert_{p} \le \lVert x \rVert_{1}\]

    \textbf{Thm}: All norms in $\mathbb{R}^{n}$ are equivalent
\end{thm}

\begin{thm}[Picard's Theorem]{thm:picard}{}
    Let $f : \mathbb{R} \times \mathbb{R} \to \mathbb{R}$ be a continuous and boudned function, and $t_{0}, x_{0}$ be real numbers. Assume that there exists a positive constant $L$ s.t. for all real $t, x_{1}, x_{2}$ we have:
    \[\lvert f(t, x_{1}) - f(t, x_{2}) \rvert \le L\lvert x_{1} - x_{2} \rvert\]
    Then, there exists a positive $\delta$ and a unique differentiable function $x : [t_{0} - \delta, t_{0} + \delta]\to \mathbb{R}$ s.t. for all $t\in [t_{0}-\delta, t_{0} + \delta]$,
    \[x'(t) = f(t, x(t)) \quad \text{and} \quad x(t_{0}) = x_{0}\]
\end{thm}

% TODO: solving linear systems
% TODO: Fredholm Intergral Equation

\begin{dfn}[Lipschitz Functions again]{dfn:lipshitz-function}{}
    Let $(X, d_{X}),\,(Y, d_{Y})$ be metric spaces. A function $f : X \to Y$ is said to be a \textbf{Lipschitz} function iff there exists a constant $L$ such that for all $x,x'\in X$,
    \[d_{Y}(f(x), f(x')) \le L d_{X}(x,x')\]
    If $L < 1$, $f$ is said to be a \textbf{contraction}
    
    \longrule{0.08ex}
    If $f : \mathbb{R} \to \mathbb{R}$ is a Lipschitz function and $x_{0}$ is any point in $\mathbb{R}$, then for any $x\in \mathbb{R}$ we have
    \[\lvert f(x) - f(x_{0}) \rvert \le L\lvert x - x_{0} \rvert\]
    For $x \ge x_{0}$ this can be expanded to
    \[f(x_{0}) - L(x - x_{0}) \le f(x) \le f(x_{0}) + L(x - x_{0})\]

    \longrule{0.08ex}
    Let $(X, d_{X})$ and $(Y, d_{Y})$ be two metric spaces, and $f : X \to Y$ be a Lipschitz function. Then there exists a smallest Lipschitz constant of $f$

    \longrule{0.08ex}
    \textbf{Thm}: Let $I$ be a non-degenerate open interval on the real line and let $f : I \to \mathbb{R}$ be a differentiable function. Then $f$ is Lipschitz iff $f'$ is bounded. When that is the case,
    \[\lvert f \rvert_{\text{Lip}} = \sup \{\lvert f'(x) \rvert : x\in I\}\]
\end{dfn}

\lipsum[1-12]
\end{multicols}
\end{document}
