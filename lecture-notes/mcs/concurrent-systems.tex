\documentclass{article}
% \usepackage{showframe}

% \usepackage[dvipsnames]{xcolor}
% custom colour definitions
% \colorlet{colour1}{Red}
% \colorlet{colour2}{Green}
% \colorlet{colour3}{Cerulean}

\usepackage{geometry}
% margins
\geometry{
    a4paper,
    bottom=70pt,
    % margin=70pt
}

\usepackage{import}
\usepackage{pdfpages}
\usepackage{transparent}
\usepackage{xcolor}
\usepackage{quiver}

\newcommand{\incfig}[2][1]{%
    \def\svgwidth{#1\columnwidth}
    \import{./figures/}{#2.pdf_tex}
}

\usepackage{graphicx} % Required for inserting images
\usepackage{amsmath}
\usepackage{amsfonts}
\usepackage{amssymb}
\usepackage{mathtools}
% \usepackage{preamble}
\usepackage{multicol}
\usepackage{lipsum}
\usepackage{bbding}
\usepackage{prftree}
\usepackage{float}
\usepackage[nodisplayskipstretch]{setspace}

% tikz and theorem boxes
\usepackage[framemethod=TikZ]{mdframed}
\usepackage{../../thmboxes_v2}
\usepackage{../../customs}


\usepackage{hyperref} % note: this is the final package
\parindent = 0pt
\linespread{1.1}

% Custom Definitions of operators
\DeclareMathOperator{\Ima}{im}
\DeclareMathOperator{\Fix}{Fix}
\DeclareMathOperator{\Orb}{Orb}
\DeclareMathOperator{\Stab}{Stab}
\DeclareMathOperator{\send}{send}
\DeclareMathOperator{\dom}{dom}

\title{Modelling Concurrent Systems Notes}
\author{Leon Lee}
\renewcommand\labelitemi{\tiny$\bullet$}

\begin{document}

\maketitle
\newpage
\tableofcontents
\newpage

\section{Introduction to Concurrent Systems}
\subsection{Lecture 1 - Specification and Implementation}
The main topics that are covered in this course:
\begin{itemize}
    \item Formalising specifications as well as implementations of concurrent systems
    \item Studying the criteria for deciding whether an implementation meets a specification
    \item Techniques for proving whether an implementation meets a specification
\end{itemize}

Both specifications and implementations can be represented by means of \textbf{models of concurrency} such as \textbf{Labelled Transition Systems (LTSs)} or \textbf{Process Graphs}.

\begin{dfn}[Process Graphs and LTSs]{dfn:lts}{}
    A \textbf{process graph} is a triple $(S, I, \to)$, defined by the following:

    \begin{itemize}
        \item $S$ is a set of \textbf{states}
        \item $I\in S$ is an \textbf{initial state}
        \item $\to$ is a set of triples $(s, a, t)$ with $s, t\in S$, and $a$ an \textbf{action} drawn from a set $\mathtt{Act}$
    \end{itemize}

    \longrule{0.08ex}

    A \textbf{Labelled Transition System(LTS)} is a \textit{process graph} without the initial state (but sometimes LTS is used as a synonym for process graph i.e. with the initial state)

    \longrule{0.08ex}

    Sometimes we will use process graphs with a fourth component $\checkmark \subseteq S$ indicating the \textbf{final} states of the process: those in the system can terminate successfully
\end{dfn}

Specifications and implementations can not only be represented by LTSs or other models of concurrency, but also by \textbf{process algebraic expressions}, where complex processes are built up from constants for atomic actions using operators for \textit{sequential}, \textit{alternative}, and \textit{parallel composition}.

The most popular process algebraic languages from literature are:
\begin{itemize}
    \item \textbf{CCS}: the Calculus of Communicating Systems
    \item \textbf{CSP}: Communicating Sequential Processes
    \item \textbf{ACP}: the Algebra of Communicating Processing
\end{itemize}

We will be using ACP, focusing on the \textit{partially synchronous parallel composition} operator

\newpage
\begin{dfn}[ACP]{dfn:acp}{}
    The syntax of \textbf{ACP}, the \textbf{Algebra of Communicating Processes} features the operations
    \begin{enumerate}
        \item $\epsilon$: \textbf{Successful termination} (only in the optional extension $ACP_{\epsilon}$)
        \item $\delta$: \textbf{Deadlock}
        \item $a$: \textbf{Action constant} for each action $a$
        \item $P\cdot Q$: \textbf{Sequential Composition}
        \item $P + Q$: \textbf{Summation}, \textbf{Choice}, or \textbf{Alternative Composition}
        \item $P | | Q$: \textbf{Parallel Composition}
        \item $\partial_H(P)$: \textbf{Restriction}, or \textbf{Encapsulation} for each set of visible actions $H$
        \item $\tau_{I}(P)$: \textbf{Abstraction} for each set of visible actions $I$ (only in the optional extension $ACT_{\tau}$)
    \end{enumerate}
\end{dfn}

\textbf{Note}: There is also left and right parallel operators, $P $

The atomic actions of ACP consist of all $a,b,c$ etc from a given set $A$ of visible actions, and one special action $\tau$, that is meant to be internal and invisible to the outside world.

For each application, a partial \textbf{communication function} $\gamma : A \times A \to A$ is chosen that tells for each two visible actions $a$ and $b$ whether they synchronise (namely if $\gamma$ is defined), and if so, what is result of their synchronisation: $\gamma(a,b)$. The communication function is required to be commutative and associative. The invisible action cannot take part in synchronisations.


\newpage
\begin{dfn}[ACP in terms of Process Graphs]{dfn:acp-process}{}
    Below is the \textbf{ACP} operations in terms of process graphs extended with a predicate $\checkmark$ that signals successful termination

    \begin{itemize}
        \item $\epsilon$ is the graph with one state and no transition. This one state is the initial state, and is marked with $\checkmark$
        \item $\delta$ is the graph with one state and no transitions. This one state is the initial state. It is not makred as terminating.
        \item $a$ is a graph with two states (and initial and a final one) and one transition between them, labelled $a$. The final state is marked with $\checkmark$
        \item $G \cdot H$ is the process that first performs $G$, and upon successful termination of $G$ proceed with $H$. 
        \item $G + H$ is obtained by taking the union of copies of $G$ and $H$ with disjoint sets of states, and adding a fresh state \textbf{root} which will be the initial state of $G + H$. For each transition $I_{G} \xrightarrow{a} s$ in $G$, where $I_{G}$ denotes the initial state of $G$, there will be an extra transition \textbf{root} $\xrightarrow{a} s$, and likewise, for each transition $I_{H} \xrightarrow{a} s$ in $H$, where $I_{H}$ denotes the initial state of $h$, there will be an extra transition \textbf{root} $\xrightarrow{a} s$

            \textbf{root} is labelled with $\checkmark$ if either $I_{G}$ or $I_{H}$ is.
        \item $G | | H$ is obtained by taking the Cartesian product of the states of $G$ and $H$; that is, the states of $G | H$ are pairs $(s, t)$ with $s$ a state from $G$ and $t$ a state from $H$. The initial state of $G | | H$ is the pair of initial states of $G$ and $H$. A state $(s, t)$ is labelled $\checkmark$ iff both $s$ and $t$ are labelled $\checkmark$. The transitions are
            \begin{itemize}
                \item $(s, t) \xrightarrow{a} (s', t)$ whenever $s \xrightarrow{a} s'$ is a transition in $G$
                \item $(s, t) \xrightarrow{a} (s, t')$ whenever $t \xrightarrow{a} t'$ is a transition in $H$
                \item $(s, t) \xrightarrow{c} (s', t')$ whenever $s \xrightarrow{a} s$ is a transition in $G$, and $t \xrightarrow{b} t'$ is a transition in $H$, and $\gamma(a,b) = c$
            \end{itemize}
        Intuitively, $G | | H$ allows all possible interleavings of actions from $G$ and actions from $H$. In addition, it enables actions to synchronise with their communication partners.
        \item $\partial_{H}(G)$ is just $G$, but with all actions in $G$ omitted. It is used to remove the remnants of unsuccessful commmunication, so that the synchronisation that is enabled by parallel composition, is enforced.
        \item $\tau_{I}(G)$ is just $G$, but with all actions in $I$ renamed into $\tau$.
    \end{itemize}
\end{dfn}

These semantics are of the \textbf{denotional} kind. Here "denotional" entails that each constant denotes a process graph (up to \textbf{isomorphism}) and each ACP operator denotes an operation on process graphs (creating a new graph out of one or two argument graphs)

\subsection{Deadlocks}
Deadlock means that a process isn't ``terminated'' but it cannot go further

ACP-$\epsilon$ is an alternative modele where non-finishing states can also terminate

\newpage

\subsection{Relations}

\begin{dfn}[Equivalence Relation]{dfn:equivalence-relation}{}
    A binary relation $\sim$ on a set $D$ is an \textbf{equivalence relation} if it is
    \begin{itemize}
        \item \textbf{Reflexive}: $p \sim p$ for all processes $p\in D$
        \item \textbf{Symmetric}: If $p \sim q$ then $q \sim p$
        \item \textbf{Transitive}: If $p \sim q$ and $q \sim r$ then $p \sim r$
    \end{itemize}
\end{dfn}

\begin{dfn}[Equivalence Class]{dfn:equiv-class}{}
    If $\sim$ is an equivalence relation on $D$, and $p$ is in $D$, then $[p]_{\sim}$ denotes the set of all processes in $D$ that are equivalent to $p$. Such a set is called an \textbf{equivalent class}

    \longrule{0.08ex}

    Equivalence relations that split a set into more sets are called \textbf{finer} or \textbf{more discriminating}, and equivalence relations that split a set into less sets are called \textbf{coarser} or \textbf{more identifying}. The finest relation is $e$ - every element has its own class, and the coarsest relation is the universal equivalence - everything is equivalent to itself
\end{dfn}

\begin{dfn}[Equivalence Relation]{dfn:preorder}{}
    A binary relation on a set $D$ is a \textbf{preorder} ($\sqsubseteq$) if it is
    \begin{itemize}
        \item \textbf{Reflexive}: $p \sqsubseteq p$ for all processes $p\in D$
        \item \textbf{Transitive}: If $p \sqsubseteq q$ and $q \sqsubseteq r$ then $p \sqsubseteq r$
    \end{itemize}
\end{dfn}

\subsection{Reactive Systems}

\newpage
\section{CCS}

\newpage
\section{Congruence Closure}
\subsection{}

\begin{dfn}[Congruence Closure]{dfn:congruence-closure}{}
    For an equivalence $\sim$, we have $\sim^{c}$, its congruence closure where
    \[P \sim^{c} Q \implies \forall C [],\, C[P] \sim C[Q]\]

\end{dfn}

% https://q.uiver.app/#q=WzAsOCxbMCwwLCJCIl0sWzMsMiwiQkIiXSxbMywzLCJXQiJdLFswLDQsIkNUXntDfSJdLFswLDUsIkNUXntcXGluZnR5fSJdLFszLDYsIkNUXnsqfSJdLFsyLDEsIkJCXntDfSJdLFsyLDIsIldCXntDfSJdLFs0LDNdLFsyLDFdLFszLDBdLFs1LDJdLFs1LDRdLFs2LDBdLFsxLDZdLFsyLDddLFs3LDZdXQ==
\[\begin{tikzcd}
	B \\
	&& {BB^{C}} \\
	&& {WB^{C}} & BB \\
	&&& WB \\
	{CT^{C}} \\
	{CT^{\infty}} \\
	&&& {CT^{*}}
	\arrow[from=2-3, to=1-1]
	\arrow[from=3-3, to=2-3]
	\arrow[from=3-4, to=2-3]
	\arrow[from=4-4, to=3-3]
	\arrow[from=4-4, to=3-4]
	\arrow[from=5-1, to=1-1]
	\arrow[from=6-1, to=5-1]
	\arrow[from=7-4, to=4-4]
	\arrow[from=7-4, to=6-1]
\end{tikzcd}\]

\subsubsection{Showing Congruences}

\textbf{Example}: Show that $=_{PT}$ is a congruence for $\partial_{H}$
\[P =_{PT} Q \implies \partial_{H}(P) =_{PT} \partial_{H}(Q)\]
WTS
\[PT(P) = PT(Q) \implies PT(\partial_{H}(P)) = PT(\partial_{H}(Q))\]
\[PT(\partial_{H}(P)) = \{\sigma \in PT(P) \mid \text{$\sigma$ does not contain any action from $H$}\}\]
We can just replace $PT(P)$ with $PT(Q)$ and the implication follows automatically

\textbf{Example}: Show that $=_{CT}$ is \textbf{not} a congruence for $\partial_{H}$

For $P = ab + ac$, we have $CT(\partial_{b}(P)) = a + ac$

For $Q = a(b + c)$, we have $CT(\partial_{b}(Q)) = ac$

\subsection{Equational Logic}

Equations - of the form $P = Q$

Equational logic - comprised of axioms, and rules

Rules use the rules listed below
\begin{itemize}
    \item Reflexive: $P = P$
    \item Symmetric: $P = Q$ and $Q = P$
    \item Transitive: $P = Q$ and $Q = R$ implies $P = R$
    \item Substitution: $P = P + P$ implies $a.b = a.b + a.b$
    \item Congruence: $P = Q \implies C[P] = C[Q]$
\end{itemize}

\[Ax \vdash P = Q \text{ if $P = Q$ follows from the ...} \]

\begin{thm}[]{thm:}{}
    \[P =_{B} Q \iff Ax \vdash P = Q\]
\end{thm}

You want proofs to be sound and complete

Operations of the $+$ in CCS
\begin{itemize}
    \item $P + Q = Q + P$
    \item $(P + Q) + R = R + (Q + R)$
    \item $P + P = P$
    \item $P + 0 = P$
\end{itemize}

Operations of $P | Q$:
\[P = \sum_{i = 1}^{k} a_{i} P_{i} = a_{1}P_{1} + a_{2}P_{2}+\cdots+a_{k}P_{k}\]
\[Q = \sum_{j = 1}^{l} b_{j} Q_{j}\]

\begin{thm}[Expansion Law]{thm:expansion-law}{}
    \[
        P | Q = \sum_{i = 1}^{k} a_{i} . (P_{i} | Q) + \sum_{j = 1}^{l} b_{j} . (P | Q_{j}) + \underbrace{\sum_{i = 1}^{k}\sum_{j = 1}^{l}}_{(a_{i} = \overline{b}_{j})} \tau (P_{i} | Q_{j})
    \]

        \[\gamma : A \times A \to (A \uplus \{\delta\})\]
        $\gamma(a_{i},b_{j}) = \delta$ if they do not communicate
\end{thm}

\begin{dfn}[Head normal form]{dfn:head-norml}{}
    Something is in head normal form if it can be written in the form
    \[P = \sum a_{i} . P_{i}\]
\end{dfn}

\textbf{Example}: We can write
\begin{align*}
    (a.b | c) &= a(b | c) + c(a .b) \\
              &= a(bc + c b) + c(a .b)
\end{align*}

\textbf{Example}: $\partial_{H}(P + Q) = \partial_{H}(P) + \partial_{H}(Q)$

\newpage
\subsection{General Structural Operational Semantics}

\[\prftree{x \to w}{ f(x,y,z) \to g(h(v,w))}\]

\subsection{Hennessy-Milner Logic}
HML is a set of formulas in the following syntax:
\[\phi ::= \top \mid \bot \mid \phi_{1} \wedge \phi_{2} \mid \phi_{1} \vee \phi_{2} \mid \neg \phi \mid \langle a \rangle\phi \mid [a]\phi\]
\begin{itemize}
    \item $P \models  \phi$: ``Process $P$ satisfies formula $\phi$ if $\phi$ holds for $P$''
    \item $P \models \langle a \rangle\phi$: iff there is a $P'$ such that $P \prightarrow{a} P'$ and $P \models \phi$
    \item $P \models [a]\phi$ iff for all $P \prightarrow{a} P'$ we have $P' \models \phi$
\end{itemize}

\newpage
\subsection{Safety}

\begin{dfn}[Canonical Safety Property]{dfn:can-safety}{}
    A process $p$ satisfies the \textbf{canonical safety property}, notation $P \models \mathrm{safety}(b)$ if no trace of $P$ contains the action $b$
\end{dfn}

\[P \models \mathrm{safety}(B) \iff PT(P) \cup B = \emptyset\]

\begin{dfn}[Safety Equivalence]{dfn:safety-equiv}{}
    $P$ is a safety equivalence to $Q$ if for all safety properties, if $P \models \mathrm{safety}(B)$ then so does $Q$
\end{dfn}

\begin{thm}[]{thm:safety-pt}{}
    \[P \equiv_{\mathrm{safety}} Q \iff P =_{PT} Q\]

    \longrule{0.08ex}

    \[P \sqsubseteq Q \iff PT(Q) \subseteq PT(P)\]
\end{thm}

% https://q.uiver.app/#q=WzAsMTksWzAsMCwiXFxidWxsZXQiXSxbMCwxLCJcXGJ1bGxldCJdLFswLDIsIlxcYnVsbGV0Il0sWzAsMywibm9ybWFsIFxcXFxwcm9jZXNzIl0sWzIsMCwiXFxidWxsZXQiXSxbMiwxLCJcXGJ1bGxldCJdLFsyLDIsIlxcYnVsbGV0Il0sWzIsMywiRGVhZGxvY2siXSxbMywxLCJcXGJ1bGxldCJdLFs1LDAsIlxcYnVsbGV0Il0sWzUsMSwiXFxidWxsZXQiXSxbNSwyLCJcXGJ1bGxldCJdLFs2LDEsIlxcYnVsbGV0Il0sWzgsMCwiXFxidWxsZXQiXSxbNSwzLCJMaXZlbG9jayBcXFxcIERpdmVyZ2VudCJdLFs4LDEsIlxcYnVsbGV0Il0sWzgsMiwiXFxidWxsZXQiXSxbOCwzLCJFc2NhcGFibGUgXFxcXCBEaXZlcmdlbmNlIl0sWzYsNCwiTm90IEJCRSJdLFswLDEsImEiXSxbMSwyLCJnIl0sWzQsNSwiYSJdLFs1LDYsImciXSxbNSw4LCJcXHRhdSJdLFs5LDEwLCJhIl0sWzEwLDExLCJnIl0sWzEwLDEyLCJcXHRhdSJdLFsxMiwxMiwiXFx0YXUiXSxbMTMsMTUsImEiXSxbMTUsMTYsImciXSxbMTUsMTUsIlxcdGF1IiwwLHsiYW5nbGUiOjkwfV1d
\[\begin{tikzcd}[cramped]
	\bullet && \bullet &&& \bullet &&& \bullet \\
	\bullet && \bullet & \bullet && \bullet & \bullet && \bullet \\
	\bullet && \bullet &&& \bullet &&& \bullet \\
	\begin{array}{c} normal \\process \end{array} && Deadlock &&& \begin{array}{c} Livelock \\ Divergent \end{array} &&& \begin{array}{c} Escapable \\ Divergence \end{array} \\
	&&&&&& {Not BBE}
	\arrow["a", from=1-1, to=2-1]
	\arrow["a", from=1-3, to=2-3]
	\arrow["a", from=1-6, to=2-6]
	\arrow["a", from=1-9, to=2-9]
	\arrow["g", from=2-1, to=3-1]
	\arrow["\tau", from=2-3, to=2-4]
	\arrow["g", from=2-3, to=3-3]
	\arrow["\tau", from=2-6, to=2-7]
	\arrow["g", from=2-6, to=3-6]
	\arrow["\tau", from=2-7, to=2-7, loop, in=55, out=125, distance=10mm]
	\arrow["\tau", from=2-9, to=2-9, loop, in=325, out=35, distance=10mm]
	\arrow["g", from=2-9, to=3-9]
\end{tikzcd}\]


\end{document}
