\documentclass{article}
% \usepackage{showframe}

% \usepackage[dvipsnames]{xcolor}
% custom colour definitions
% \colorlet{colour1}{Red}
% \colorlet{colour2}{Green}
% \colorlet{colour3}{Cerulean}

\usepackage{geometry}
% margins
\geometry{
    a4paper,
    bottom=70pt,
    % margin=70pt
}

\usepackage{graphicx} % Required for inserting images
\usepackage{amsmath}
\usepackage{amsfonts}
\usepackage{amssymb}
% \usepackage{preamble}
\usepackage{multicol}
\usepackage{lipsum}
\usepackage{bbding}
\usepackage{float}
\usepackage[nodisplayskipstretch]{setspace}

% tikz and theorem boxes
\usepackage[framemethod=TikZ]{mdframed}
\usepackage{../../thmboxes_v2}
\usepackage{../../customs}


\usepackage{hyperref} % note: this is the final package
\parindent = 0pt
\linespread{1.1}

% Custom Definitions of operators
\DeclareMathOperator{\Ima}{im}
\DeclareMathOperator{\Fix}{Fix}
\DeclareMathOperator{\Orb}{Orb}
\DeclareMathOperator{\Stab}{Stab}
\DeclareMathOperator{\send}{send}
\DeclareMathOperator{\dom}{dom}

\title{Modelling Concurrent Systems Notes}
\author{Leon Lee}
\renewcommand\labelitemi{\tiny$\bullet$}

\begin{document}

\maketitle
\newpage
\tableofcontents
\newpage

\section{Introduction to Concurrent Systems}
\subsection{Lecture 1 - Specification and Implementation}
The main topics that are covered in this course:
\begin{itemize}
    \item Formalising specifications as well as implementations of concurrent systems
    \item Studying the criteria for deciding whether an implementation meets a specification
    \item Techniques for proving whether an implementation meets a specification
\end{itemize}

Both specifications and implementations can be represented by means of \textbf{models of concurrency} such as \textbf{Labelled Transition Systems (LTSs)} or \textbf{Process Graphs}.

\begin{dfn}[Process Graphs and LTSs]{dfn:lts}{}
    A \textbf{process graph} is a triple $(S, I, \to)$, defined by the following:

    \begin{itemize}
        \item $S$ is a set of \textbf{states}
        \item $I\in S$ is an \textbf{initial state}
        \item $\to$ is a set of triples $(s, a, t)$ with $s, t\in S$, and $a$ an \textbf{action} drawn from a set $\mathtt{Act}$
    \end{itemize}

    \longrule{0.08ex}

    A \textbf{Labelled Transition System(LTS)} is a \textit{process graph} without the initial state (but sometimes LTS is used as a synonym for process graph i.e. with the initial state)

    \longrule{0.08ex}

    Sometimes we will use process graphs with a fourth component $\checkmark \subseteq S$ indicating the \textbf{final} states of the process: those in the system can terminate successfully
\end{dfn}

Specifications and implementations can not only be represented by LTSs or other models of concurrency, but also by \textbf{process algebraic expressions}, where complex processes are built up from constants for atomic actions using operators for \textit{sequential}, \textit{alternative}, and \textit{parallel composition}.

The most popular process algebraic languages from literature are:
\begin{itemize}
    \item \textbf{CCS}: the Calculus of Communicating Systems
    \item \textbf{CSP}: Communicating Sequential Processes
    \item \textbf{ACP}: the Algebra of Communicating Processing
\end{itemize}

We will be using ACP, focusing on the \textit{partially synchronous parallel composition} operator

\newpage
\begin{dfn}[ACP]{dfn:acp}{}
    The syntax of \textbf{ACP}, the \textbf{Algebra of Communicating Processes} features the operations
    \begin{enumerate}
        \item $\epsilon$: \textbf{Successful termination} (only in the optional extension $ACP_{\epsilon}$)
        \item $\delta$: \textbf{Deadlock}
        \item $a$: \textbf{Action constant} for each action $a$
        \item $P\cdot Q$: \textbf{Sequential Composition}
        \item $P + Q$: \textbf{Summation}, \textbf{Choice}, or \textbf{Alternative Composition}
        \item $P | | Q$: \textbf{Parallel Composition}
        \item $\partial_H(P)$: \textbf{Restriction}, or \textbf{Encapsulation} for each set of visible actions $H$
        \item $\tau_{I}(P)$: \textbf{Abstraction} for each set of visible actions $I$ (only in the optional extension $ACT_{\tau}$)
    \end{enumerate}
\end{dfn}

\textbf{Note}: There is also left and right parallel operators, $P $

The atomic actions of ACP consist of all $a,b,c$ etc from a given set $A$ of visible actions, and one special action $\tau$, that is meant to be internal and invisible to the outside world.

For each application, a partial \textbf{communication function} $\gamma : A \times A \to A$ is chosen that tells for each two visible actions $a$ and $b$ whether they synchronise (namely if $\gamma$ is defined), and if so, what is result of their synchronisation: $\gamma(a,b)$. The communication function is required to be commutative and associative. The invisible action cannot take part in synchronisations.


\newpage
\begin{dfn}[ACP in terms of Process Graphs]{dfn:acp-process}{}
    Below is the \textbf{ACP} operations in terms of process graphs extended with a predicate $\checkmark$ that signals successful termination

    \begin{itemize}
        \item $\epsilon$ is the graph with one state and no transition. This one state is the initial state, and is marked with $\checkmark$
        \item $\delta$ is the graph with one state and no transitions. This one state is the initial state. It is not makred as terminating.
        \item $a$ is a graph with two states (and initial and a final one) and one transition between them, labelled $a$. The final state is marked with $\checkmark$
        \item $G \cdot H$ is the process that first performs $G$, and upon successful termination of $G$ proceed with $H$. 
        \item $G + H$ is obtained by taking the union of copies of $G$ and $H$ with disjoint sets of states, and adding a fresh state \textbf{root} which will be the initial state of $G + H$. For each transition $I_{G} \xrightarrow{a} s$ in $G$, where $I_{G}$ denotes the initial state of $G$, there will be an extra transition \textbf{root} $\xrightarrow{a} s$, and likewise, for each transition $I_{H} \xrightarrow{a} s$ in $H$, where $I_{H}$ denotes the initial state of $h$, there will be an extra transition \textbf{root} $\xrightarrow{a} s$

            \textbf{root} is labelled with $\checkmark$ if either $I_{G}$ or $I_{H}$ is.
        \item $G | | H$ is obtained by taking the Cartesian product of the states of $G$ and $H$; that is, the states of $G | H$ are pairs $(s, t)$ with $s$ a state from $G$ and $t$ a state from $H$. The initial state of $G | | H$ is the pair of initial states of $G$ and $H$. A state $(s, t)$ is labelled $\checkmark$ iff both $s$ and $t$ are labelled $\checkmark$. The transitions are
            \begin{itemize}
                \item $(s, t) \xrightarrow{a} (s', t)$ whenever $s \xrightarrow{a} s'$ is a transition in $G$
                \item $(s, t) \xrightarrow{a} (s, t')$ whenever $t \xrightarrow{a} t'$ is a transition in $H$
                \item $(s, t) \xrightarrow{c} (s', t')$ whenever $s \xrightarrow{a} s$ is a transition in $G$, and $t \xrightarrow{b} t'$ is a transition in $H$, and $\gamma(a,b) = c$
            \end{itemize}
        Intuitively, $G | | H$ allows all possible interleavings of actions from $G$ and actions from $H$. In addition, it enables actions to synchronise with their communication partners.
        \item $\partial_{H}(G)$ is just $G$, but with all actions in $G$ omitted. It is used to remove the remnants of unsuccessful commmunication, so that the synchronisation that is enabled by parallel composition, is enforced.
        \item $\tau_{I}(G)$ is just $G$, but with all actions in $I$ renamed into $\tau$.
    \end{itemize}
\end{dfn}

These semantics are of the \textbf{denotional} kind. Here "denotional" entails that each constant denotes a process graph (up to \textbf{isomorphism}) and each ACP operator denotes an operation on process graphs (creating a new graph out of one or two argument graphs)

\subsection{Deadlocks}
Deadlock means that a process isn't ``terminated'' but it cannot go further

ACP-$\epsilon$ is an alternative modele where non-finishing states can also terminate


\newpage




\end{document}
