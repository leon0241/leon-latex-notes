\documentclass{article}
% \usepackage{showframe}

% \usepackage[dvipsnames]{xcolor}
% custom colour definitions
% \colorlet{colour1}{Red}
% \colorlet{colour2}{Green}
% \colorlet{colour3}{Cerulean}

\usepackage{geometry}
% margins
\geometry{
    a4paper,
    bottom=70pt,
    % margin=70pt
}

\usepackage{graphicx} % Required for inserting images
\usepackage{amsmath}
\usepackage{amsfonts}
\usepackage{amssymb}
% \usepackage{preamble}
\usepackage{multicol}
\usepackage{lipsum}
\usepackage{bbding}
\usepackage{float}
\usepackage[nodisplayskipstretch]{setspace}

% tikz and theorem boxes
\usepackage[framemethod=TikZ]{mdframed}
\usepackage{../../thmboxes_v2}
\usepackage{../../customs}


\usepackage{hyperref} % note: this is the final package
\parindent = 0pt
\linespread{1.1}

% Custom Definitions of operators
\DeclareMathOperator{\Ima}{im}
\DeclareMathOperator{\Fix}{Fix}
\DeclareMathOperator{\Orb}{Orb}
\DeclareMathOperator{\Stab}{Stab}
\DeclareMathOperator{\send}{send}
\DeclareMathOperator{\dom}{dom}

\title{Modelling Concurrent Systems Notes}
\author{Leon Lee}
\renewcommand\labelitemi{\tiny$\bullet$}

\begin{document}

\maketitle
\newpage
\tableofcontents
\newpage

\section{Introduction to Concurrent Systems}
\subsection{Lecture 1 - Specification and Implementation}
The main topics that are covered in this course:
\begin{itemize}
    \item Formalising specifications as well as implementations of concurrent systems
    \item Studying the criteria for deciding whether an implementation meets a specification
    \item Techniques for proving whether an implementation meets a specification
\end{itemize}

Both specifications and implementations can be represented by means of \textbf{models of concurrency} such as \textbf{Labelled Transition Systems (LTSs)} or \textbf{Process Graphs}.

\begin{dfn}[Process Graphs and LTSs]{dfn:lts}{}
    A \textbf{process graph} is a triple $(S, I, \to)$, defined by the following:

    \begin{itemize}
        \item $S$ is a set of \textbf{states}
        \item $I\in S$ is an \textbf{initial state}
        \item $\to$ is a set of triples $(s, a, t)$ with $s, t\in S$, and $a$ an \textbf{action} drawn from a set $\mathtt{Act}$
    \end{itemize}

    \longrule{0.08ex}

    A \textbf{Labelled Transition System(LTS)} is a \textit{process graph} without the initial state (but sometimes LTS is used as a synonym for process graph i.e. with the initial state)

    \longrule{0.08ex}

    Sometimes we will use process graphs with a fourth component $\checkmark \subseteq S$ indicating the \textbf{final} states of the process: those in the system can terminate successfully
\end{dfn}

\end{document}
