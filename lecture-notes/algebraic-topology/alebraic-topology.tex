\documentclass{article}
% \usepackage{showframe}

% \usepackage[dvipsnames]{xcolor}
% custom colour definitions
% \colorlet{colour1}{Red}
% \colorlet{colour2}{Green}
% \colorlet{colour3}{Cerulean}

\usepackage{geometry}
% margins
\geometry{
    a4paper,
    bottom=70pt,
    % margin=70pt
}

\usepackage{graphicx} % Required for inserting images
\usepackage{amsmath}
\usepackage{amsfonts}
\usepackage{amsthm}
\usepackage{amssymb}
\usepackage{mathtools}
% \usepackage{preamble}
\usepackage{multicol}
\usepackage{lipsum}
\usepackage{float}
\usepackage[nodisplayskipstretch]{setspace}

% tikz and theorem boxes
\usepackage[framemethod=TikZ]{mdframed}
\usepackage{../../thmboxes_v3}
\usepackage{../../customs}


\usepackage{hyperref} % note: this is the final package
\parindent = 0pt
\linespread{1.1}

% Custom Definitions of operators
\DeclareMathOperator{\Ima}{im}
\DeclareMathOperator{\Fix}{Fix}
\DeclareMathOperator{\Orb}{Orb}
\DeclareMathOperator{\Stab}{Stab}
\DeclareMathOperator{\send}{send}
\DeclareMathOperator{\dom}{dom}

\title{Algebraic Topology Notes}
\author{Leon Lee}
\renewcommand\labelitemi{\tiny$\bullet$}

\begin{document}

\maketitle
\newpage
\tableofcontents
\newpage

\section{Introduction to Algebraic Topology}
\subsection{Topologies to Algebra}

We want to turn topological spaces into algebraic objects through operations called Invariants. An example is that if two topological spaces $X$ and $Y$ are isomorphic, the translated algebraic object should also be isomorphic
\begin{align*}
    \mathrm{TOP} &\rightsquigarrow \mathrm{ALG}\\
    X &\mapsto A(X) \quad\text{``algebraic objects''}\\
    X \cong Y &\mapsto A(X) \cong A(Y)
\end{align*}

\begin{xmp}[Examples of Algebraic Objects]{xmp:algebraic-objects}{}
    Some examples of algebraic objects:
    \begin{itemize}
        \item The set of Connected Components $\pi_{0}(X)$
        \item The Fundamental Group $\pi_{1}(X)$
        \item Higher homotopy groups $\pi_{n}(X)$
    \end{itemize}
\end{xmp}

Note: the more involved the algebraic invariant is, the more topology it sees.

Computability problem leads to Homology Theory (this is non-examinable)

\subsection{Connected Spaces}

\begin{rcl}[Topologies]{rcl:topologies}{}
    A topology on $X$, $\mathcal{T}$, is a family of subsets s.t.
    \begin{itemize}
        \item $\emptyset, X \in \mathcal{T}$
        \item Closed under finite intersection, $U_{1}, U_{2}\in \mathcal{T} \implies U_{1} \cap U_{2} \in \mathcal{T}$
        \item Closed under arbitrary unions
    \end{itemize}
\end{rcl}

Examples of topological spaces:
\begin{itemize}
    \item Trivial topology $\mathcal{T} = \{\emptyset, X\}$
    \item Discrete Topology $\mathcal{T} = \mathcal{P}(X)$
    \item $\mathbb{R}$ or anything made from a metric space
\end{itemize}

\begin{dfn}[Connected Spaces]{dfn:connected-space}{}
    A topological space $X$ is \textbf{connected} if $X = A \uplus B$ ($A$ and $B$ are open) means that $A = \emptyset$ or $A = X$
\end{dfn}

\begin{ppn}[Connected Spaces and Clopens]{ppn:connected-clopen}{}
    $X$ is connected iff the only clopens are $\emptyset, X$
\end{ppn}

\begin{proof} $ $\newline
$(\implies)$: A clopen then $X = A \uplus A^{C} \implies A = \emptyset, X$ (both $A$ and $A^{C}$ open)

$(\impliedby)$: $A \uplus B \implies A = B^{C} \implies$ $A$ is clopen
\end{proof}

\textbf{Examples}:
\begin{itemize}
    \item $\mathbb{R}$ is connected. Opens are generated by intervals like $(-\infty, a)$, $(a, b)$, $(a, \infty)$.
    \item The trivial topology is connected. (by definition since there are only two sets).
    \item The discrete topology is \textit{not} connected, unless $X = \emptyset$ or $X = \{*\}$ in which case it coincides with the trivial topology.
\end{itemize}

\begin{ppn}[Connectedness of Maps]{ppn:maps-connected}{}
    For a continuous map $f : X \to Y$, and $X$ connected, we have that $f(X)$ is connected.
\end{ppn}

\begin{proof}
    $f(X) = U \uplus V \implies f^{-1}(U) \uplus f^{-1}(V) = X \implies f^{-1}(U) = \emptyset, X$
\end{proof}

\begin{crl}[]{crl:homeomorphisms-connected}{}
    If $X\cong Y$ are homeomorphic, then $X$ is connected iff $Y$ is connected
\end{crl}

\begin{ppn}[]{ppn:}{}
    The relation ($x \sim y$ if $\exists$ connected subset $A \subseteq X$ s.t. $x,y\in A$) is an equivalence relation.
\end{ppn}

\begin{proof} We show the relation fulfils all requirements for an equivalence relation:
    \begin{itemize}
        \item \textbf{Reflexivity}: $x \sim x$: $x \in \{x\} \subseteq X$
        \item \textbf{Symmetry}: $x \sim y \iff y \sim x$ tautological (we don't specify between $x$ and $y$ so just take $y = x$ and $x = y$)
        \item \textbf{Transitivity}: $x \sim y \wedge y\sim z \implies x \sim z$, $x,y\in A$, $y,z\in B$. Claim: $A \cup B$ is connected. Proof in workshop
    \end{itemize}
\end{proof}

\begin{dfn}[Components]{dfn:component}{}
    The equivalence classes of the above proposition are called \textbf{components}
\end{dfn}

\subsection{Path-Connectedness}

\begin{dfn}[Path]{dfn:path}{}
    A \textbf{path} in $X$ is a continuous map $\alpha : I \to X$ for $I = \mathcal{T}(0, 1)$.

    $x \sim y \iff \exists \alpha : I \prightarrow{\text{path}} X \text{ s.t. } \alpha(0) = x, \alpha(1) = y$
\end{dfn}

$x \sim y$ is an equivalence relation due to the following operations on paths:
\begin{enumerate}
    \item Constant path. If $x \in X$, $c_{X} : I \to X$, $c_{x}(t) := X$
    \item Path reversal. Let $\alpha : I \to X$ be a path. Then $\overline{\alpha} : I \to X, t \mapsto \alpha(1 - t)$
    \item Path concatenation: $\alpha : I \to X$, $\beta : I \to X$ s.t. $\alpha(1) = \beta(0)$. Then
        \[(a * b)(t) = \begin{cases}
            \alpha(2t), \quad 0 \le t \le \frac{1}{2} \\
            \beta(2t - 1), \frac{1}{2} \le t \le 1
        \end{cases}\]
\end{enumerate}

\begin{dfn}[Connected Components]{dfn:connected-component}{}
    The set of path-connected components (equivalence classes) is denoted by $\pi_{0}(X)$
\end{dfn}

\textbf{Remarks}:
\begin{itemize}
    \item We have that $X \cong Y \implies \pi_{0}(X) \cong \pi_{0}(Y)$
    \item Path-connected $\implies$ Connected (but not vice-versa). Counterexample: Pick 
        \[X = \{(x, \sin(\frac{1}{x})) \mid 0 < x < 1\}\]
        is connected but not path connected
\end{itemize}

\end{document}
