\documentclass{article}
% \usepackage{showframe}

% \usepackage[dvipsnames]{xcolor}
% custom colour definitions
% \colorlet{colour1}{Red}
% \colorlet{colour2}{Green}
% \colorlet{colour3}{Cerulean}

\usepackage{geometry}
% margins
\geometry{
    a4paper,
    bottom=70pt,
    % margin=70pt
}

\usepackage{graphicx} % Required for inserting images
\usepackage{amsmath}
\usepackage{amsfonts}
\usepackage{amsthm}
\usepackage{amssymb}
\usepackage{mathtools}
\usepackage{../../preamble-notes}
\usepackage{multicol}
\usepackage{lipsum}
\usepackage{float}
\usepackage[nodisplayskipstretch]{setspace}

% tikz and theorem boxes
\usepackage[framemethod=TikZ]{mdframed}
\usepackage{../../thmboxes_v3}
\usepackage{../../customs}


\usepackage{hyperref} % note: this is the final package
\parindent = 0pt
\linespread{1.1}

% Custom Definitions of operators
\DeclareMathOperator{\Ima}{im}
\DeclareMathOperator{\Fix}{Fix}
\DeclareMathOperator{\Orb}{Orb}
\DeclareMathOperator{\Stab}{Stab}
\DeclareMathOperator{\send}{send}
\DeclareMathOperator{\dom}{dom}

\title{Algebraic Topology Notes}
\author{Leon Lee}
\renewcommand\labelitemi{\tiny$\bullet$}

\begin{document}

\maketitle
\newpage
\tableofcontents
\newpage

\section{Introduction to Algebraic Topology}
\subsection{Topologies to Algebra}

We want to turn topological spaces into algebraic objects through operations called Invariants. An example is that if two topological spaces $X$ and $Y$ are isomorphic, the translated algebraic object should also be isomorphic
\begin{align*}
    \mathrm{TOP} &\rightsquigarrow \mathrm{ALG}\\
    X &\mapsto A(X) \quad\text{``algebraic objects''}\\
    X \cong Y &\mapsto A(X) \cong A(Y)
\end{align*}

\begin{xmp}[Examples of Algebraic Objects]{xmp:algebraic-objects}{}
    Some examples of algebraic objects:
    \begin{itemize}
        \item The set of Connected Components $\pi_{0}(X)$
        \item The Fundamental Group $\pi_{1}(X)$
        \item Higher homotopy groups $\pi_{n}(X)$
    \end{itemize}
\end{xmp}

Note: the more involved the algebraic invariant is, the more topology it sees.

Computability problem leads to Homology Theory (this is non-examinable)

\subsection{Connected Spaces}

\begin{rcl}[Topologies]{rcl:topologies}{}
    A topology on $X$, $\mathcal{T}$, is a family of subsets s.t.
    \begin{itemize}
        \item $\emptyset, X \in \mathcal{T}$
        \item Closed under finite intersection, $U_{1}, U_{2}\in \mathcal{T} \implies U_{1} \cap U_{2} \in \mathcal{T}$
        \item Closed under arbitrary unions
    \end{itemize}
\end{rcl}

Examples of topological spaces:
\begin{itemize}
    \item Trivial topology $\mathcal{T} = \{\emptyset, X\}$
    \item Discrete Topology $\mathcal{T} = \mathcal{P}(X)$
    \item $\mathbb{R}$ or anything made from a metric space
\end{itemize}

\begin{dfn}[Connected Spaces]{dfn:connected-space}{}
    A topological space $X$ is \textbf{connected} if $X = A \uplus B$ ($A$ and $B$ are open) means that $A = \emptyset$ or $A = X$
\end{dfn}

\begin{ppn}[Connected Spaces and Clopens]{ppn:connected-clopen}{}
    $X$ is connected iff the only clopens are $\emptyset, X$
\end{ppn}

\begin{proof} $ $\newline
$(\implies)$: A clopen then $X = A \uplus A^{C} \implies A = \emptyset, X$ (both $A$ and $A^{C}$ open)

$(\impliedby)$: $A \uplus B \implies A = B^{C} \implies$ $A$ is clopen
\end{proof}

\textbf{Examples}:
\begin{itemize}
    \item $\mathbb{R}$ is connected. Opens are generated by intervals like $(-\infty, a)$, $(a, b)$, $(a, \infty)$.
    \item The trivial topology is connected. (by definition since there are only two sets).
    \item The discrete topology is \textit{not} connected, unless $X = \emptyset$ or $X = \{*\}$ in which case it coincides with the trivial topology.
\end{itemize}

\begin{ppn}[Connectedness of Maps]{ppn:maps-connected}{}
    For a continuous map $f : X \to Y$, and $X$ connected, we have that $f(X)$ is connected.
\end{ppn}

\begin{proof}
    $f(X) = U \uplus V \implies f^{-1}(U) \uplus f^{-1}(V) = X \implies f^{-1}(U) = \emptyset, X$
\end{proof}

\begin{crl}[]{crl:homeomorphisms-connected}{}
    If $X\cong Y$ are homeomorphic, then $X$ is connected iff $Y$ is connected
\end{crl}

\begin{ppn}[]{ppn:}{}
    The relation ($x \sim y$ if $\exists$ connected subset $A \subseteq X$ s.t. $x,y\in A$) is an equivalence relation.
\end{ppn}

\begin{proof} We show the relation fulfils all requirements for an equivalence relation:
    \begin{itemize}
        \item \textbf{Reflexivity}: $x \sim x$: $x \in \{x\} \subseteq X$
        \item \textbf{Symmetry}: $x \sim y \iff y \sim x$ tautological (we don't specify between $x$ and $y$ so just take $y = x$ and $x = y$)
        \item \textbf{Transitivity}: $x \sim y \wedge y\sim z \implies x \sim z$, $x,y\in A$, $y,z\in B$. Claim: $A \cup B$ is connected. Proof in workshop
    \end{itemize}
\end{proof}

\begin{dfn}[Components]{dfn:component}{}
    The equivalence classes of the above proposition are called \textbf{components}
\end{dfn}

\subsection{Path-Connectedness}

\begin{dfn}[Path]{dfn:path}{}
    A \textbf{path} in $X$ is a continuous map $\alpha : I \to X$ for $I = \mathcal{T}(0, 1)$.

    $x \sim y \iff \exists \alpha : I \prightarrow{\text{path}} X \text{ s.t. } \alpha(0) = x, \alpha(1) = y$
\end{dfn}

$x \sim y$ is an equivalence relation due to the following operations on paths:
\begin{enumerate}
    \item Constant path. If $x \in X$, $c_{X} : I \to X$, $c_{x}(t) := X$
    \item Path reversal. Let $\alpha : I \to X$ be a path. Then $\overline{\alpha} : I \to X, t \mapsto \alpha(1 - t)$
    \item Path concatenation: $\alpha : I \to X$, $\beta : I \to X$ s.t. $\alpha(1) = \beta(0)$. Then
        \[(a * b)(t) = \begin{cases}
            \alpha(2t), \quad 0 \le t \le \frac{1}{2} \\
            \beta(2t - 1), \frac{1}{2} \le t \le 1
        \end{cases}\]
\end{enumerate}

\begin{dfn}[Connected Components]{dfn:connected-component}{}
    The set of path-connected components (equivalence classes) is denoted by $\pi_{0}(X)$
\end{dfn}

\textbf{Remarks}:
\begin{itemize}
    \item We have that $X \cong Y \implies \pi_{0}(X) \cong \pi_{0}(Y)$
    \item Path-connected $\implies$ Connected (but not vice-versa). Counterexample: Pick 
        \[X = \{(x, \sin(\frac{1}{x})) \mid 0 < x < 1\}\]
        is connected but not path connected
\end{itemize}

\begin{dfn}[Homotopy]{dfn:homotopy}{}
    Let $f, g : X \to Y$ continuous maps. A \textbf{homotopy} from $f$ to $g$ is a continuous map $h : X \times I \to Y$ s.t.
    \begin{align*}
        h(-, 0) &= f \iff h(x,0) = f(x),\, \forall x \\
        h(-, 1) &= g
    \end{align*}
\end{dfn}

\textbf{Terminology}: $f$ is homotopy equivalent to $g$ if there exists a homotopy $h$

homotopies on homotopies - horizontal composition

% https://q.uiver.app/#q=WzAsMyxbMCwwLCJYIl0sWzIsMCwiWSJdLFs0LDAsIloiXSxbMCwxLCJmIiwyXSxbMSwyLCJnIiwyXSxbMCwxLCJmJyIsMCx7Im9mZnNldCI6LTMsImN1cnZlIjotM31dLFsxLDIsImcnIiwwLHsib2Zmc2V0IjotMywiY3VydmUiOi0zfV0sWzMsNSwiaF8xIiwyLHsic2hvcnRlbiI6eyJzb3VyY2UiOjIwLCJ0YXJnZXQiOjIwfX1dLFs0LDYsImhfMiIsMix7InNob3J0ZW4iOnsic291cmNlIjoyMCwidGFyZ2V0IjoyMH19XV0=
\[\begin{tikzcd}[cramped]
	X && Y && Z
	\arrow[""{name=0, anchor=center, inner sep=0}, "f"', from=1-1, to=1-3]
	\arrow[""{name=1, anchor=center, inner sep=0}, "{f'}", shift left=3, curve={height=-18pt}, from=1-1, to=1-3]
	\arrow[""{name=2, anchor=center, inner sep=0}, "g"', from=1-3, to=1-5]
	\arrow[""{name=3, anchor=center, inner sep=0}, "{g'}", shift left=3, curve={height=-18pt}, from=1-3, to=1-5]
	\arrow["{h_1}"', shorten <=3pt, shorten >=3pt, Rightarrow, from=0, to=1]
	\arrow["{h_2}"', shorten <=3pt, shorten >=3pt, Rightarrow, from=2, to=3]
\end{tikzcd}\]

Vertical composition
% https://q.uiver.app/#q=WzAsMixbMCwwLCJYIl0sWzIsMCwiWSJdLFswLDEsImciLDAseyJsYWJlbF9wb3NpdGlvbiI6NzB9XSxbMCwxLCJmIiwwLHsib2Zmc2V0IjotMywiY3VydmUiOi0zfV0sWzAsMSwiayIsMix7Im9mZnNldCI6MywiY3VydmUiOjN9XSxbMywyLCJoXzEiLDIseyJzaG9ydGVuIjp7InNvdXJjZSI6MjAsInRhcmdldCI6MjB9fV0sWzIsNCwiaF8yIiwyLHsic2hvcnRlbiI6eyJzb3VyY2UiOjIwLCJ0YXJnZXQiOjIwfX1dXQ==

\[\begin{tikzcd}[cramped]
	X && Y
	\arrow[""{name=0, anchor=center, inner sep=0}, "g"{pos=0.7}, from=1-1, to=1-3]
	\arrow[""{name=1, anchor=center, inner sep=0}, "f", shift left=3, curve={height=-18pt}, from=1-1, to=1-3]
	\arrow[""{name=2, anchor=center, inner sep=0}, "k"', shift right=3, curve={height=18pt}, from=1-1, to=1-3]
	\arrow["{h_1}"', shorten <=3pt, shorten >=3pt, Rightarrow, from=1, to=0]
	\arrow["{h_2}"', shorten <=3pt, shorten >=3pt, Rightarrow, from=0, to=2]
\end{tikzcd}\]

\newpage
\subsection{Homotopy Equivalence}

\begin{dfn}[Homotopy Equivalence]{dfn:homotopy-equivalence}{}
    Two spaces $X$, $Y$ are called \textbf{homotopy equivalent} or \textbf{of the same homotopy type}, and denoted by $X \simeq Y$, if there exists a homotopy equivalence $f : X \to Y$
\end{dfn}

\textbf{Note}: We use $\cong$ for homeomorphisms and $\simeq$ for homotopy equivalences.

\begin{lma}[Homotopy inverses]{lma:homotopy-inverses}{}
    Let $f : X \to Y$ and $g : Y \to Z$ with homotopy inverses $\tilde{f} : Y \to X$ and $\tilde{g} : Z \to Y$ respectively. Then, $\tilde{f} \circ \tilde{g} : Z \to X$ is a homotopy inverse of $g \circ f : X \to Z$.

    In particular, $X \simeq Y$ and $Y \simeq Z$ implies $X \simeq Z$.
\end{lma}

\begin{dfn}[Contractible Spaces]{dfn:contractible}{}
    A space $X$ is called \textbf{contractible} if it is homotopy equivalent to a point, i.e. $X\simeq *$
\end{dfn}

\textbf{Example}: $\mathbb{R}^{n}$ is contractible. Let $x_{0}$ be a fixed point in $\mathbb{R}^{n}$ and define the (straight line) homotopy $h : c_{x_{0}}\simeq \mathrm{id}_{\mathbb{R}^{n}}$ by
\[h(x, t) = (1 - t)x_{0} + tx\]

\begin{rem}[]{rem:contractible-pc}{}
    \begin{enumerate}
        \item Contractible spaces are path-connected
        \item The converse does not hold. For example $X = \mathbb{S}^{1}$ will lead to a counterexample.
        \item A contractible space $X$ is contractible at any point $x_{0}$. Since $X$ is path-connected a path from $x$ to $x'$ defines a homotopy $c_{x} \simeq c_{x'}$
        \item Any two maps $f, g : X \to Y$ are homotopic if $Y$ is contractible.
    \end{enumerate}
\end{rem}

\newpage

\section{Retractions and Deformations}

\begin{dfn}[Retractions and Detractions]{dfn:retract-detract}{}
    \begin{itemize}
        \item A \textbf{retract} of $X$ onto a subspace $A \subset X$ is a map $r : X \to A$ such that $r |_{A} = \mathrm{id}_{A}$. Equivalently, this is a map $r : X \to X$ such that $r^{2} = r$ and $r(X) = A$
        \item A \textbf{deformation retract} of $X$ onto $A$ is the additional datum of a homotopy $h : \mathrm{id}_{X} \simeq i \circ r$, where $i : A \hookrightarrow X$ denotes the inclusion
    \end{itemize}
\end{dfn}

In other words, a deformation retract is a homotopy $h : X \times I \to X$ such that $h(x, 0) = x$ and $h(x, 1)\in A$ for all $x\in X$ and $h(a,1) = a$ for all $a\in A$

Not all retracts can form deformation retracts. For instance, notice that the retract $X$ onto a point $\{x_{0}\}$ can be a deformation retract if and only if $X$ is contractible.

\begin{ppn}[Deformation Retracts cause Homotopy Equivalence]{ppn:retracts-to-homotopy-equivalence}{}
    A deformation retract of $X$ onto $A$ induces a homotopy equivalence $X \simeq A$.
\end{ppn}

\subsection{Quotient spaces}
\begin{dfn}[Quotient Space]{dfn:quotient-space}{}
    Let $X$ be a topological space and let $\sim$ be an equivalence relation on $X$. Then $X / \sim$ is equipped with the quotient topology and called a \textbf{quotient space}. If $Z$ is a closed subset in $X$, then we can also define the quotient space $X / Z$.
\end{dfn}

\textbf{Examples of Quotient Spaces}
\begin{itemize}
    \item The quotient of the $n$-dimensional closed disk by its boundary is the $n$-sphere, i.e. 
        \[ \mathbb{D}^{n} / \partial \mathbb{D}^{n} \cong \mathbb{S}^{n} \]
    \item The $2$-torus: $\mathbb{R}^{2} / \mathbb{Z}^{2}$
    \item The projective space $\mathbb{R}P^{n} = \mathbb{R}^{n+1} - \{0\} / \sim$ by the relation $x \sim y$ iff there exists some $\lambda\in \mathbb{R}^{\times}$ such that $x = \lambda y$. This corresponds to the space of lines through the origin in $\mathbb{R}^{n+1}$.
\end{itemize}

\begin{dfn}[Alternate Quotient Space]{dfn:alternate-quotient-space}{}
    Let $f : Z \to Y$ be a continuous map between a closed subset $Z \subset X$ and $Y$. Then
    \[X \amalg_{f} Y = X \amalg Y / f(z) \sim y\]

    Additionally,
    \begin{itemize}
        \item Its \textbf{mapping cylinder} is defined as the topological space
            \[M_{f} := (X \times I) \amalg Y / \sim\]
            where the quotient identifies $(x, 0) \sim f(x)$ for any $x\in X$
        \item Its \textbf{cone} is the further quotient
            \[C_{f} := M_{f} / X \times \{1\}\]
        \item The \textbf{cone} of a topological space $X$ is:
            \[C_{X} := C_{\mathrm{id}_{X}} = X \times I / X \times \{1\}\]
    \end{itemize}
\end{dfn}

\begin{rem}[Commutative Diagram of the Mapping Cylinder]{rem:cylinder-cd}{}
    In other words, the mapping cylinder of $f : X \times Y$ is the pushout of the diagram
% https://q.uiver.app/#q=WzAsNCxbMCwwLCJYIFxcdGltZXMgXFx7MFxcfSJdLFswLDIsIlggXFx0aW1lcyBJIl0sWzIsMiwiTV9mIl0sWzIsMCwiWSJdLFswLDMsImYiXSxbMCwxXSxbMywyXSxbMSwyXV0=
\[\begin{tikzcd}[cramped]
	{X \times \{0\}} && Y \\
	\\
	{X \times I} && {M_f}
	\arrow["f", from=1-1, to=1-3]
	\arrow[from=1-1, to=3-1]
	\arrow[from=1-3, to=3-3]
	\arrow[from=3-1, to=3-3]
\end{tikzcd}\]
\end{rem}

\begin{lma}[Iclusion Map of the Mapping Cylinder]{lma:cylinder-iclusion}{}
    Let $f : X \to Y$ and $M_{f}$ its mapping cylinder. The iclusion map $i : Y \hookrightarrow M_{f}$ is a strong deformation retract.
\end{lma}

\subsection{Examples of Deformation Retracts}
\begin{xmp}[Shhere]{xmp:sphere-deformation}{}
    Consider the $n$-sphere $\mathbb{S}^{n}$ with the standard embedding $\mathbb{R}^{n+1} \backslash \{ 0\}$. Then the map
    \[r : \mathbb{R}^{n+1} \backslash_{0} \{0\} \to \mathbb{S}^{n},\, x \mapsto \frac{x}{\lvert x \rvert}\]
    is a retract. Indeed, if $x$ has norm $\lvert x \rvert = 1$, then $r(x) = x$. For a deformation retract one needs to find a homotopy $h : i \circ r \simeq \mathrm{id}_{X}$. this can easily be realized by the following straight line homotopy:
    \[h : \mathbb{R}^{n+1} \backslash \{0\} \times I \to \mathbb{R}^{n+1} \backslash \{0\}, \quad (x, t) \mapsto (1-t) \frac{x}{\lvert x \rvert} + tx\]
    Indeed $h(x,0) = r(x)$ and $h(x, 1) = x$ for all $x$

    In fact, one can easily check that the above forms a strong deformation retract as $h(x, t) = x$ for all $x\in \mathbb{S}^{n}$ and $t\in I$. Note that one could have also constructed a deformation retract that is not strong, for example by rotating in time.
\end{xmp}

\newpage


\begin{rem}[]{rem:path-homotopy-properties}{}
    Ordinary tomotopy are not interesting for paths, e.g. $\alpha : I \to X$ is homotopic to a constant path
\end{rem}

\begin{ppn}[]{ppn:path-concat-unital}{}
    Path concatenation is unital and asssociative up to relative union
\end{ppn}

\begin{lma}[]{lma:}{}
    Let $\alpha : I \to X$ be a path, and $\lambda : I \to I$ continuous s.t. $\lambda(0) = 0$ and $\lambda(1) = 1$.

    Then, $\alpha \circ \lambda \cong \alpha$ (relative to $\{0,1\}$)
\end{lma}

\begin{dfn}[Fundamental Group]{dfn:fundamental-group}{}
    The fundamental group of $X$ at $x_{0}\in X$ is the homotophy equivalence class of ``loops'' at $x_{0}$. i.e. paths in $X$ s.t. $\alpha(0) = \alpha(0) = x$
\end{dfn}

\end{document}
