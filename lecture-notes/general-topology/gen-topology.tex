\documentclass{article}
% \usepackage{showframe}

% \usepackage[dvipsnames]{xcolor}
% custom colour definitions
% \colorlet{colour1}{Red}
% \colorlet{colour2}{Green}
% \colorlet{colour3}{Cerulean}

\usepackage{geometry}
% margins
\geometry{
    a4paper,
    bottom=70pt,
    % margin=70pt
}

\usepackage{graphicx} % Required for inserting images
\usepackage{amsmath}
\usepackage{amsfonts}
\usepackage{amssymb}
\usepackage{amsthm}
% \usepackage{preamble}
\usepackage{multicol}
\usepackage{lipsum}
\usepackage{float}
\usepackage[nodisplayskipstretch]{setspace}

% tikz and theorem boxes
\usepackage[framemethod=TikZ]{mdframed}
\usepackage{../../thmboxes_v2}
\usepackage{../../customs}


\usepackage{hyperref} % note: this is the final package
\parindent = 0pt
\linespread{1.1}

% Custom Definitions of operators
\DeclareMathOperator{\Ima}{im}
\DeclareMathOperator{\Fix}{Fix}
\DeclareMathOperator{\Orb}{Orb}
\DeclareMathOperator{\Stab}{Stab}
\DeclareMathOperator{\send}{send}
\DeclareMathOperator{\dom}{dom}
\DeclareMathOperator{\Int}{int}

\title{General Topology Math Notes}
\author{Leon Lee}
\renewcommand\labelitemi{\tiny$\bullet$}

\begin{document}

\maketitle
\newpage
\tableofcontents
\newpage

\section{Intro to Topology}
\subsection{Why Topology?}

Topology can appear where we least expect it...
\begin{itemize}
    \item Algebraic Number Theory - Next to Euclidean topology, can define other topologies on $\mathbb{Q}$ (related to how often primes divide a number). Extends to Adeles, Langlands programme, etc
    \item Arithmetic Progressions in the Integers - An arithmetic progression of length $k$ is a set $\{a, a+d,\dots,a+(k-1)d\}$
        Finding subsets of $\mathbb{N}$ that contain arbitrarily long APs:
        \begin{itemize}
            \item $2\mathbb{N}$ or $\mathbb{N}$
            \item Primes (Green-Tao Theorem, 2007). Green-Tao theorem relies on \textbf{Szemeredi's Theorem}: Any dense enough subset of $\mathbb{N}$ contains arbitrarily long APs

                Furstenburg's idea: Get from $A \subseteq\mathbb{N}$ to $(a_{i}\in \{0, 1\}^{\mathbb{N}})$ with $a_{i} \begin{cases}
                    1 & i\in A \\
                    0 & \text{else}
                \end{cases}$

                Use topological dynamics to study this: A topological dynamical system is a triple of $X$ cpt, $T : X \to X$ continuous, and a probability measure $\mu$ preserved by $T$ (what)
        \end{itemize}
\end{itemize}

\subsection{Topological Spaces and Examples}
\begin{dfn}[Topological Space]{dfn:top-space}{}
    A \textbf{topological space} is a pair $(X, \mathcal{T})$, where $X$ is a nonempty set, and $\mathcal{T}$ is a collection of subsets of $X$ which satisfies:
    \begin{enumerate}
        \item $\emptyset\in \mathcal{T}$ and $X\in \mathcal{T}$
        \item if $U_{\lambda}\in \mathcal{T}$ for each $\lambda\in A$ (where $A$ is some indexing set), then $\bigcup\limits_{\lambda \in A} U_{\lambda}\in \mathcal{T}$
        \item If $U_{1}, U_{2}\in \mathcal{T}$, then $U_{1} \cap U_{2}\in \mathcal{T}$
    \end{enumerate}
\end{dfn}

\subsubsection{Examples of Topological Spaces}
\begin{enumerate}
    \item $\mathbb{R}^{n}$ with the Euclidean Topology - induced by the Euclidean Metric
    \item For any set $X$, $\mathcal{T} = \mathcal{P}(X)$ (discrete topology)
    \item For any set $X$, $\mathcal{T} = \{\emptyset, X\}$ (indiscrete topology)
    \item $X = \{0,1,2\}$ with $\mathcal{T} = \{\emptyset, X, \{0\}, \{0,1\}, \{0,2\}\}$
    \item $X = \mathbb{R}$ and $U$ open (aka, in $\mathcal{T}$) if $R \backslash U$ is finite or $U = \emptyset$
\end{enumerate}
Proof for $5$:
\begin{enumerate}
    \item $\emptyset\in \mathcal{T}$, $\emptyset$ is finite $\implies X \in \mathcal{T}$
    \item Intersections of finite sets are finite
    \item Unions of finite sets are finite
\end{enumerate}

\begin{dfn}[Neighbourhood of a point]{dfn:neighbourhood}{}
    A \textbf{neighbourhood} of a point $x\in X$ is a subset $N \subseteq X$ s.t. $x\in U \subseteq N$ for some open subset $U \subseteq X$
\end{dfn}

\begin{dfn}[Metric Space]{dfn:metric-space}{}
    A \textbf{metric space} $(X, d)$ is a nonempty set $X$ together with a function
    \[d : X \times X \to \mathbb{R}\]
    with the following properties:
    \begin{enumerate}
        \item $d(x, y) \ge 0$ for all $x, y\in X$ and $d(x, y) = 0$ iff $x = y$
        \item $d(x, y) = d(y, x)$ for all $x, y \in X$
        \item $d(x, z) \le d(x, y) + d(y, z)$ for all $x,y,z\in X$
    \end{enumerate}
    The function $d$ is called the metric. Point $3$ is called the \textit{triangle inequality}

    \longrule{0.08ex}
    
    For any $x\in X$ and any positive real number $r$ the \textbf{open ball} in $X$ with centre $x$ and radius $r$ is defined by
    \[B(x, r) = \{y\in X | d(x, y) < r\}\]
    We declare a subset $U$ of $X$ to be \textit{open in the metric topology given by d} iff for each $a\in U$ there is an $r > 0$ such that $B(a, r) \subseteq U$

    \longrule{0.08ex}

    If $(X, \mathcal{T})$ is a topological space, and if $X$ admits a metric whose metric topology is precisely $\mathcal{T}$ we say that $(X, \mathcal{T})$ is \textbf{metrisable}. Thus the euclidean spaces with their usual topologies are metrisable
\end{dfn}

\subsubsection{Examples of Metric Spaces}
\begin{enumerate}
    \item Any set $X$ with $d(x, y) = \begin{cases}
            0, & x = y \\
            1, & \text{else}
    \end{cases}$
    \item $\mathbb{R}^{n}$ with $d(x, y) = \lvert x - y \rvert = \sqrt{\sum_{i = 1}^{n} (x_{i} - y_{i})^{2}}$
    \item $C([0,1])$ with $d(f, g) = \max_{t\in[0,1]} \lvert f(t) - g(t) \rvert$
    \item $C([0,1])$ with $d(f, g) = \sqrt{\int_{0}^{n}\lvert f(t) - g(t) \rvert^{2} dt}$
\end{enumerate}

\subsubsection{Topologies on Metric spaces}
We want to define a topology on $(X, d)$. For this, we want open balls to be open in the topology

\begin{dfn}[Base]{dfn:base}{}
    For a set $X$, a basis $\mathcal{B}$ is a collection of subsets such that
    \begin{enumerate}
        \item $\bigcup_{B\in\mathcal{B}} B = X$
        \item $B_{1}\cap B_{2}\in \mathcal{B}$ for all $B_{1},B_{2}\in \mathcal{B}$
    \end{enumerate}
    The \textbf{topology generated by $\mathcal{B}$} is
    \[\mathcal{T} := \{\bigcup_{i\in I} B_{i}, \text{$I$ index set},\, B_{i}\in \mathcal{B}\}\]
\end{dfn}

\textbf{Note}: This is a topology because
\[(\cup_{i\in I} B_{i}) \cap (\cup_{j\in J} B_{j}) = \bigcup_{i\in I,j\in J} \underbrace{B_{i} \cap B_{j}}_{\in \mathcal{B}}\in \mathcal{t}\]

\begin{dfn}[Metric Topology]{dfn:metric-topology}{}
    \[\text{Let } \mathcal{B} = \{\bigcap_{i = 1}^{n} Br_{1}(x_{1}), n\in \mathbb{N},\, r_{i} > 0,\, x_{i}\in X,\,\forall i\}\]
    The \textbf{metric topology} is the topology generated by this basis
\end{dfn}

\textbf{Observation} A set $U$ is open in the metric topology $\iff$ $\forall x\in U,\, \exists r > 0$ s.t. $Br(x) \subseteq U$

\begin{itemize}
    \item $\impliedby$: For each $x\in U$, let $r_{x}$ s.t. $B_{r_{x}}(x) \subseteq U$. Then $U = \bigcup\limits_{x \subseteq U} B_{r_{x}}(x)$ is open
    \item $\implies$: Let $x\in U$ be given. Knwo that $x\in B_{r_{1}}(x_{1}) \cup \dots \cup B_{r_{n}}(x_{n})$ for some $n,r_{1},x_{1}$. For each $i$, there is $\delta_{i} >0$ s.t. $B_{\delta_{i}}(x) \le B_{r_{1}}(x_{1})$.
\end{itemize}

huh?

\begin{thm}[random ms prop]{thm:ms-prop-idk}{}
    If $X$ carries metrics $d, \tilde{d}$ such that $ad(x, y) \le \tilde{d}(x,y) \le Ad(x, y)$ for some $a, A > 0$, then the induced topologies agree
\end{thm}


\begin{dfn}[Subspace Topology]{dfn:subspace-topology}{}
    Let $(X, \mathcal{T})$ be a topological space, and let $A \subseteq X$ be any subset. Then the \textbf{subspace topology} on $A$ consists of all sets of the form $U \cap A$ where $U \in \mathcal{T}$
\end{dfn}

\textbf{Example}: $(-1, 1) \subseteq \mathbb{R}$ with euclidean topology. The subspace topology is
\[\{(-1, 1) \cap U,\, U \subseteq \mathbb{R} \text{ open}\}\]
$(-1, 1)$ is closed in the subspace topology

\begin{thm}[Topology Lemmas]{thm:topology-lemmas}{}
    \begin{itemize}
        \item[\textbf{1.3}] If $(X, \mathcal{T})$ is a topological space and $U_{1},\dots,U_{n}$ are open sets, then the intersection $\bigcap_{i=1}^{n}u_{i}$ is also open
        \item[1.6] In order to show that a set $U \subseteq X$ is open, it is enough to show that for every $x\in U$ there is an open set $V$ with $x\in V \subseteq U$
        \item[1.6] A subset $U$ of $\mathbb{R}^{n}$ is \textit{open for the usual topology} iff for each $a\in U$ there exists an $r > 0$ s.t.
            \[\lvert x -a \rvert < r \implies x \in U\]
            The collection of open sets thus defined is called the \textbf{usual topology on $\mathbb{R}^{n}$}. Note that open balls are open sets under this definition
    \end{itemize}
\end{thm}

\begin{dfn}[Topology Small Definitions]{dfn:topology-small-defs}{}
    \begin{itemize}
        \item
    \end{itemize}
\end{dfn}


\newpage

\subsection{Closed sets, Closure, Interior, and Boundary}

\begin{dfn}[Closed Subsets]{dfn:closed-subset}{}
    Let $(X, \mathcal{T})$ be a topological space. A subset $A \subseteq X$ is \textbf{closed} iff its complement $X \backslash A = A^{C} := \{x \in X \ x\not\in A\}$ is open in $X$
\end{dfn}

\textbf{Note}: A set being ``closed'' has no connection with ``not being open''

\subsubsection{Examples of open and closed sets}
\begin{itemize}
    \item A set that is neither open nor closed: $[0, 1) \subseteq \mathbb{R}$ under Euclidean topology
    \item A set that is both closed and open: $\emptyset$ or $X$
\end{itemize}

\begin{thm}[]{thm:topology-union-lemmas}{}
    Let $(X, \mathcal{T})$ be a topological space. Then
    \begin{enumerate}
        \item $\emptyset$ and $X$ are closed.
        \item The union of finitely many closed sets is a closed set
        \item The intersection of any collection of closed sets is a closed set
    \end{enumerate}
\end{thm}

$\bigcup_{i\in I} A_{i}$ is not necessarily closed.
\[A_{n} = \left[-2 + \frac{1}{n}, 2 - \frac{1}{n}\right] \text{ in } \mathbb{R}, \bigcup_{i = 1}^{n} A_{i} = (-2, 2)\]

\subsubsection{Topological proof that there are infinitely many primes (Furstenberg)}
\begin{proof}
    Look at $\mathbb{Z}$ with

    \[\mathcal{B} := \{S(a, b),\, a\ne 0,\,b\in \mathbb{Z}\} \quad\text{ and }\quad S(a,b) = \{an + b,\,n\in \mathbb{Z}\}\]

    Let the open sets be the one generated by this basis. We can show
    \begin{enumerate}
        \item $S(a,b)$ is both open and closed.
        \item All open sets are infinite.
    \end{enumerate}

    \vspace{-3pt}
    \noindent\rule{\textwidth}{0.08ex}\vspace{-8pt}

    \begin{enumerate}
        \item $S(a,b) = \mathbb{Z} \backslash \bigcup_{i=1}^{a - 1} S(a, b-1)$
        \item Clear
    \end{enumerate}

    Thus:
    \[\underbrace{\mathbb{Z} \backslash \{-1, 1\}}_{\text{not closed}} = \bigcup_{\text{$p$ primes}} \overbrace{S(p, 0)}^{\text{closed}}\]
    If there were only finitely many primes, right hand side would be closed
\end{proof}

\newpage
\subsection{Closure and stuff}

\begin{dfn}[Closure, Interior, Boundary]{dfn:closure-interior-boundary}{}
    Let $(X, \mathcal{T})$ be a topological space.
    \begin{enumerate}
        \item The \textbf{closure} of a subset $A \subseteq X$ is the smallest closed set such that $A \subseteq \overline{A}$.
            \[\overline{A} := \bigcap\limits_{\substack{C \subseteq X \text{closed} \\ A \subseteq C}} C\]
        \item The (open) \textbf{interior} of a subset $A \subseteq X$ is the biggest open set $U$ contained in $A$
            \[\Int{A} = A^{\circ} := \bigcap\limits_{\substack{U \subseteq X \text{open} \\ U \subseteq A}} C\]
        \item The \textbf{boundary} or \textbf{frontier} of a subset $A \subseteq X$ is
            \[\partial A := \overline{A} \backslash A^{\circ}\]
        \item A subset $A$ of $X$ is \textbf{dense} in $X$ iff $\overline{A} = X$

            E.g.: $\mathbb{Q} \subseteq \mathbb{R}$ with the Euclidean topology
    \end{enumerate}
\end{dfn}

\begin{thm}[Closure and Interior of Complement]{thm:complement-properties}{}
    Let $(X, \mathcal{T})$ be a topological space and $A \subseteq X$. Then
    \begin{enumerate}
        \item The closure of the complement is the complement of the interior:
            \[\overline{X \backslash A} = X \backslash (A^{\circ})\]
        \item the interior of the complement is the complement of the closure:
            \[(X \backslash A)^{\circ} = X \backslash \overline{A}\]
    \end{enumerate}
\end{thm}

\begin{dfn}[Limits in Topological spaces]{dfn:topological-limits}{}
    A sequence $(x_{n})$ converges to $x\in X$ if $\forall U$ open with $x\in U$, $\exists N$ s.t. $x_{n}\in U$ for all $n \ge N$
\end{dfn}

\begin{dfn}[Limit Set]{dfn:limitset}{}
    $\overline{A}$ can be thought as the set of limits. Formally define the \textbf{Limit Set} as
    \[A' = \{x\in X : \forall U \subseteq X, x\in U, \text{ open } \exists a\in A,\,\text{s.t. } a\in U\}\]
\end{dfn}

But, limits in general are much worse behaved in topological spaces, e.g. no unique limit point

\textbf{Example}: a topological space $X$ and a sequence $(x_{n})$ which does not have a unique limit (i.e. $\exists x\ne y$ s.t. $x_{n} \to x$ and $x_{n}\to y$ in the sense defined): Nontrivial $X$ with the indiscrete topology $\{\emptyset, X\}$


\newpage
\subsection{Hausdorff Spaces}

\textbf{Problem}: Non-unique limits are nasty :(

\begin{dfn}[Hausdorff Space]{dfn:hausdorff}{}
    A topological space $(X, \mathcal{T})$ is \textbf{Hausdorff} if for each $x, y\in X$ with $x\ne y$ there exist \textit{disjoint} open sets $U$ and $V$ s.t. $x\in U$ and $y\in V$

    This space has \textit{unique limits}!
\end{dfn}

If $(X, d)$ is a metric space then it is automatically Hausdorff, so any metrisable space is Hausdorff. The trivial topology on a set with more than one element is not Hausdorff. Not every Hausdorff space is metrisable

Non-Hausdorff spaces are a lot more annoying to work with - for example you can have multiple limits in non-Hausdorff spaces

\begin{thm}[Open sets on \texorpdfstring{$\mathbb{R}$}{R} with Euclidean Topology]{thm:r-open-sets}{}
    \begin{itemize}
        \item A set $U$ is open iff there are open intervals $I_{j}$ s.t.
    \[U = \bigcup_{j=1}^{\infty} I_{j}\]
    \item A set $A$ is closed iff there are $F_{j}$ (union of two closed intervals) s.t.
        \[A = \bigcap_{j = 1}^{\infty} F_{j}\]
    \end{itemize}
\end{thm}


\begin{dfn}[Convergence of Hausdorff Spaces]{dfn:haursdoff-convergence}{}
    A sequence $(x_{n})$ of members of a topological space $X$ converges to $x\in X$ if for every open set $U$ containing $x$, there exists an $N$ such that $n\ge N \implies x_{n}\in U$
\end{dfn}

\begin{thm}[Haussdorf Convergence Uniqueness]{thm:hausdorff-conv-uniqueness}{}
    Suppose $(X, \mathcal{T})$ is Hausdorff. Then a sequence $(x_{n})$ can converge to at most one limit.
\end{thm}

Being Hausdorff is what's called a \textit{topological property}, which means whether or not it is true in a particular case depends only on the open sets of the space in question.

In contrast, the property of \textit{completeness} of a metric space is not a topological property as there exist sets upon which one can put two distinct metrics, one complete and one not, yet for which the metric topologies coincide

\newpage
\begin{dfn}[Cauchy Sequences]{dfn:cauchy-seq}{}
    Let $(X, d)$ be a metric space
    \begin{enumerate}
        \item A \textbf{Cauchy Sequence} is a sequence $(x_{n})$ with each $x_{n}\in X$ with the property that for each $\epsilon > 0$, there exists an $N$ s.t. $m,n\ge N \implies d(x_{m}, x_{n}) < \epsilon$
        \item $(X, d)$ is \textbf{complete} if every Cauchy Sequence converges
    \end{enumerate}
\end{dfn}

\textbf{Caveat}: In general, this does not have to converge to an $x\in X$

\textbf{Example}: $\mathbb{Q}$ with the Euclidean metric. 

\begin{dfn}[Complete Space]{dfn:complete-space}{}
    A metric space is called \textbf{complete} if all Cauchy sequences converge to a point in the space
\end{dfn}

\begin{dfn}[Closure in Metric Spaces]{dfn:metric-closure}{}
    Let $(X, d)$ be a complete metric space and $A \subseteq X$. A point $x$ is in the \textbf{closure} of $A \iff \exists x_{i} \to x$ with $x\in A$
\end{dfn}

\end{document}
