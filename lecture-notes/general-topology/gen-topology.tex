\documentclass{article}
% \usepackage{showframe}

% \usepackage[dvipsnames]{xcolor}
% custom colour definitions
% \colorlet{colour1}{Red}
% \colorlet{colour2}{Green}
% \colorlet{colour3}{Cerulean}

\usepackage{geometry}
% margins
\geometry{
    a4paper,
    bottom=70pt,
    % margin=70pt
}

\usepackage{graphicx} % Required for inserting images
\usepackage{amsmath}
\usepackage{amsfonts}
\usepackage{amssymb}
% \usepackage{preamble}
\usepackage{multicol}
\usepackage{lipsum}
\usepackage{float}
\usepackage[nodisplayskipstretch]{setspace}

% tikz and theorem boxes
\usepackage[framemethod=TikZ]{mdframed}
\usepackage{../../thmboxes_v2}
\usepackage{../../customs}


\usepackage{hyperref} % note: this is the final package
\parindent = 0pt
\linespread{1.1}

% Custom Definitions of operators
\DeclareMathOperator{\Ima}{im}
\DeclareMathOperator{\Fix}{Fix}
\DeclareMathOperator{\Orb}{Orb}
\DeclareMathOperator{\Stab}{Stab}
\DeclareMathOperator{\send}{send}
\DeclareMathOperator{\dom}{dom}

\title{General Topology Math Notes}
\author{Leon Lee}
\renewcommand\labelitemi{\tiny$\bullet$}

\begin{document}

\maketitle
\newpage
\tableofcontents
\newpage

\section{Intro to Topology}
\subsection{Why Topology?}

Topology can appear where we least expect it...
\begin{itemize}
    \item Algebraic Number Theory - Next to Euclidean topology, can define other topologies on $\mathbb{Q}$ (related to how often primes divide a number). Extends to Adeles, Langlands programme, etc
    \item Arithmetic Progressions in the Integers - An arithmetic progression of length $k$ is a set $\{a, a+d,\dots,a+(k-1)d\}$
        Finding subsets of $\mathbb{N}$ that contain arbitrarily long APs:
        \begin{itemize}
            \item $2\mathbb{N}$ or $\mathbb{N}$
            \item Primes (Green-Tao Theorem, 2007). Green-Tao theorem relies on \textbf{Szemeredi's Theorem}: Any dense enough subset of $\mathbb{N}$ contains arbitrarily long APs

                Furstenburg's idea: Get from $A \subseteq\mathbb{N}$ to $(a_{i}\in \{0, 1\}^{\mathbb{N}})$ with $a_{i} \begin{cases}
                    1 & i\in A \\
                    0 & \text{else}
                \end{cases}$

                Use topological dynamics to study this: A topological dynamical system is a triple of $X$ cpt, $T : X \to X$ continuous, and a probability measure $\mu$ preserved by $T$ (what)
        \end{itemize}
\end{itemize}

\subsection{Topological Spaces and Examples}
\begin{dfn}[Topological Space]{dfn:top-space}{1.1}
    A \textbf{topological space} is a pair $(X, \mathcal{T})$, where $X$ is a nonempty set, and $\mathcal{T}$ is a collection of subsets of $X$ which satisfies:
    \begin{enumerate}
        \item $\emptyset\in \mathcal{T}$ and $X\in \mathcal{T}$
        \item if $U_{\lambda}\in \mathcal{T}$ for each $\lambda\in A$ (where $A$ is some indexing set), then $\bigcup\limits_{\lambda \in A} U_{\lambda}\in \mathcal{T}$
        \item If $U_{1}, U_{2}\in \mathcal{T}$, then $U_{1} \cap U_{2}\in \mathcal{T}$
    \end{enumerate}
\end{dfn}

\subsubsection{Examples of Topological Spaces}
\begin{enumerate}
    \item $\mathbb{R}^{n}$ with the Euclidean Topology - induced by the Euclidean Metric
    \item For any set $X$, $\mathcal{T} = \mathcal{P}(X)$ (discrete topology)
    \item For any set $X$, $\mathcal{T} = \{\emptyset, X\}$ (indiscrete topology)
    \item $X = \{0,1,2\}$ with $\mathcal{T} = \{\emptyset, X, \{0\}, \{0,1\}, \{0,2\}\}$
    \item $X = \mathbb{R}$ and $U$ open (aka, in $\mathcal{T}$) if $R \backslash U$ is finite or $U = \emptyset$
\end{enumerate}
Proof for $5$:
\begin{enumerate}
    \item $\emptyset\in \mathcal{T}$, $\emptyset$ is finite $\implies X \in \mathcal{T}$
    \item Intersections of finite sets are finite
    \item Unions of finite sets are finite
\end{enumerate}

reading for next time: 
section 1.2 def 1.17 - 1.19
section 1.4
section 1.1

\end{document}
