\documentclass{article}
% \usepackage{showframe}

% \usepackage[dvipsnames]{xcolor}
% custom colour definitions
% \colorlet{colour1}{Red}
% \colorlet{colour2}{Green}
% \colorlet{colour3}{Cerulean}

\usepackage{geometry}
% margins
\geometry{
    a4paper,
    bottom=70pt,
    % margin=70pt
}

\usepackage{graphicx} % Required for inserting images
\usepackage{amsmath}
\usepackage{amsthm}
\usepackage{amsfonts}
\usepackage{amssymb}
% \usepackage{preamble}
\usepackage{multicol}
\usepackage{lipsum}
\usepackage{float}
\usepackage[nodisplayskipstretch]{setspace}

% tikz and theorem boxes
\usepackage[framemethod=TikZ]{mdframed}
\usepackage{../../thmboxes_v3}
\usepackage{../../customs}


\usepackage{hyperref} % note: this is the final package
\parindent = 0pt
\linespread{1.1}

% Custom Definitions of operators
\DeclareMathOperator{\Ima}{im}
\DeclareMathOperator{\Fix}{Fix}
\DeclareMathOperator{\Orb}{Orb}
\DeclareMathOperator{\Stab}{Stab}
\DeclareMathOperator{\send}{send}
\DeclareMathOperator{\dom}{dom}

\title{Galois Theory Notes}
\author{Leon Lee}
\renewcommand\labelitemi{\tiny$\bullet$}

\begin{document}

\maketitle
\newpage
\tableofcontents
\newpage

\section{Introduction to Galois Theory}
\subsection{Complex Numbers over \texorpdfstring{$\mathbb{R}$}{R}}

Think of $\mathbb{C}$ as $\mathbb{C} = \{a + bi \mid a,b\in \mathbb{R}\}$ where the only thing you know about $i$ is that
\[i^{2} + 1 = 0\]
However, this equation has two roots (no way to distinguish $i$ and $-i$). ezample $z = i$

\begin{dfn}[Conjugate]{dfn:complex-conj}{}
    $z,z'\in \mathbb{Z}$ are \textbf{conjugate} over $\mathbb{R}$ if $\forall p(t) \in \mathbb{R}[t]$ (polynomials with coefficients in $\mathbb{R}$)
    \[p(z) = 0 \iff p(z') = 0\]
\end{dfn}

\begin{lma}[Characterising Conjugates]{lma:charaterising-conjugates}{}
    $z,z'\in \mathbb{C}$ are conjugate over $\mathbb{R}$ iff either $z = z'$ or $\overline{z} = z'$
\end{lma}

\begin{proof}
    Assume $z,z'$ are conjugate.
    \[z = x + iy \quad x,y\in \mathbb{R}\]
    Take the equation
    \begin{align*}(z - x) ^{2} + y^{2} &= 0 \\ p(t) = (t - x) ^{2} + y^{2} &\in \mathbb{R}[t]\end{align*}
\end{proof}

\subsection{Complex Numbers over \texorpdfstring{$\mathbb{Q}$}{Q}}
\begin{dfn}[Conjugacy in \texorpdfstring{$\mathbb{Q}$}{Q}]{dfn:conj-q}{}
    $z,z'\in \mathbb{C}$ are \textbf{conjugate over $\mathbb{Q}$} if $\forall p(t) \in \mathbb{Q}[t]$
    \[p(z) = 0 \iff p(z') = 0\]
\end{dfn}
\textbf{Example}: $\sqrt{2}$ is not conjugate to $-\sqrt{2}$ over $\mathbb{R}$. e.g.
\[p(t) = t - \sqrt{2}\in \mathbb{R}[t]\]

\textbf{Example}: $\sqrt{2}$ is conjugate to $-\sqrt{2}$ over $\mathbb{Q}$

Treat $\sqrt{2}$ like i
\[\mathbb{Q}(\sqrt{2}) = \{a + b\sqrt{2} \mid a,b\in \mathbb{Q}\}\]
This is also a field.

We want the analogue of complex conjugates. Show it is well-defined:
\[a + b \sqrt{2} = a' + b' \sqrt{2}, \quad a,b,a',b'\in \mathbb{Q}, \quad a - a' = (b ' - b) \sqrt{2}\]
..some more stuff

\textbf{Example}: $\sqrt{2}$ is not conjugate to $\sqrt{3}$ over $\mathbb{R}$
\[p(t) = t^{2} - 2\]
\textbf{Example}: $p$ prime, $\omega = \exp\left( \frac{z\pi i}{p} \right)$ (primitive $p$-th root of unity)
\[\omega^{p} = 1\]

$\omega^{k}$ is conjugate over $\mathbb{Q}$ to $\omega$, $\forall k = 1,\dots,p-1$

$\omega^{k}$ is conjugate over $\mathbb{R}$ to $\omega$ iff $k \equiv 1, p - 1 \mod p$

\begin{dfn}[Conjugacy for sets]{dfn:set-conjugacy}{}
    $(z_{1},\dots,z_{n}), z_{i}, z_{i}' \in \mathbb{C}$
    is conjugate over $\mathbb{Q}$ to $(z_{1}',\dots,z_{n}')$
    if $\forall p(t_{1},\dots,t_{n})\in \mathbb{Q}[t_{1},\dots,t_{n}]$

    \longrule{0.08ex}

    Additionally, if $(z_{1},\dots,z_{n})$ conjugate to $(z_{1}',\dots,z_{n}')$, then $z_{i}$ is conjugate to $z_{i}'$ for all $i$
\end{dfn}

\begin{dfn}[Galois Group]{dfn:galois-group}{}
    If we have a $f\in \mathbb{Q}[t]$ where $\alpha_{1},\dots,\alpha_{n}$ are roots, then the Galois group of $f$, $\mathrm{Gal}(f)$ is
    \[\mathrm{Gal}(g) = \{\sigma \in S_{n} \mid (\alpha_{1},\dots,\alpha_{n}) \text{ conjugate to } \{\alpha_{S(1)},\dots,\alpha_{\sigma(n)}\}\}\]
\end{dfn}


\end{document}
