\documentclass{article}
% \usepackage{showframe}

% \usepackage[dvipsnames]{xcolor}
% custom colour definitions
% \colorlet{colour1}{Red}
% \colorlet{colour2}{Green}
% \colorlet{colour3}{Cerulean}

\usepackage{geometry}
% margins
\geometry{
    a4paper,
    bottom=70pt,
    % margin=70pt
}

\usepackage{graphicx} % Required for inserting images
\usepackage{amsmath}
\usepackage{amsfonts}
\usepackage{amssymb}
% \usepackage{preamble}
\usepackage{multicol}
\usepackage{lipsum}
\usepackage{float}
\usepackage[nodisplayskipstretch]{setspace}

% tikz and theorem boxes
\usepackage[framemethod=TikZ]{mdframed}
\usepackage{../../thmboxes_v2}
\usepackage{../../customs}


\usepackage{hyperref} % note: this is the final package
\parindent = 0pt
\linespread{1.1}

% Custom Definitions of operators
\DeclareMathOperator{\Ima}{im}
\DeclareMathOperator{\Fix}{Fix}
\DeclareMathOperator{\Orb}{Orb}
\DeclareMathOperator{\Stab}{Stab}
\DeclareMathOperator{\send}{send}
\DeclareMathOperator{\dom}{dom}

\title{Software Testing Notes}
\author{Leon Lee}
\renewcommand\labelitemi{\tiny$\bullet$}

\begin{document}

\maketitle
\newpage
\tableofcontents
\newpage

\section{Intro}
\subsection{Structure}
\begin{itemize}
    \item Lecture material - 2 lectures per week
    \item Two tutorial sessions per two-week chunk
        \begin{itemize}
            \item First week the group will meet on its own without a tutor to discuss and prepare for the tutorial in the following week that weill be done without a tutor
            \item The focus will be on what needs to be done to build the part of your coursework portfolio that demonstrates you have achieved the learning outcome
        \end{itemize}
    \item Reading: the relevant chapters of the course book will be specified and selected parts of the standard ISO/IEC/IEEE 29119 parts 1-4, 6, and 11
    \item Maybe some guest lectures on relevant topics
\end{itemize}

Office hours 0900-1000 on Tuesdays at 3.46 in Informatics Forum

\subsection{Coursework}
100\% Coursework!

Pick a project you want to work with in testing and development - either your own project or the provided software project

You will Analyse and Test this software to demonstrate you have achieved the learning outcome of the course

\begin{itemize}
    \item Write short justifications that the work you have done demonstrates you have achieved each learning outcome
    \item The justifications make up the 'portfolio' you submit. Has a short introduction where you outline the software and one section for each learning outcome - at most 3p long
    \item In addition, you submit your self assessment of your work following a detailed grading scheme
    \item Your grading will be audited to check it is accurate. Inaccuracies will be corrected by the auditor. These will be agreed in 1-1 summative feedback sessions in the two weeks after submissions
\end{itemize}

\textbf{Deadline 23/01/25 at 12:00}

\subsubsection{Support}

For each two week chunk of the course you will have one unsupervised tutorial group meeting to work on preparing for the following week's tutored meeting

The goal the two sessions will be the development of the section justifying the attainment of the relevant learning outcome.

The grading scheme does allow you to change the balance of the assessment across sections.

The sections in the portfolio will refer to work you have done but the work you have done need not be included in the portfolio

\newpage
\section{Stuff}
\subsection{Challenges of Software}
For many systems there are also many stakeholders often with competing and conflicting requirements

Systems might depend on unreliable infrastructure (e.g. libraries, operating systems, comms)

Software often has complex, exploitable, errors e.g. buffer overruns, ``one off'' issues in the use of arrays (e.g. UoE ISG mail problem)

Software is often ``open'' to a wide range of potential users and abusers

\subsection{New technologies}
\begin{itemize}
    \item Can reduce error e.g. Natural Language AI models can do very accurate autocomplete so avoiding a range of issues
    \item But can lead to errors of over reliance or introduce new issues e.g. some of the Natural Language models can be quite racist depending on their training
\end{itemize}
New development approaches can introduce new kinds of faults
\begin{itemize}
    \item Machine Learning introduces a wide range of ethical issues
    \item Poorly designed web services can be subject to a wide range of resource issues
\end{itemize}

\newpage
\subsection{W3 Lecture 2}

\end{document}
