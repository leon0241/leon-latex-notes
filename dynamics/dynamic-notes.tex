\documentclass{article}
% \usepackage{showframe}

% \usepackage[dvipsnames]{xcolor}
% custom colour definitions
% \colorlet{colour1}{Red}
% \colorlet{colour2}{Green}
% \colorlet{colour3}{Cerulean}

\usepackage{geometry}
% margins
\geometry{
    a4paper,
    bottom=70pt,
    % margin=70pt
}

\usepackage{graphicx} % Required for inserting images
\usepackage{amsmath}
\usepackage{amsfonts}
\usepackage{amssymb}
\usepackage{preamble}
\usepackage{multicol}
\usepackage{lipsum}
\usepackage{float}
\usepackage[nodisplayskipstretch]{setspace}

% tikz and theorem boxes
\usepackage[framemethod=TikZ]{mdframed}
\usepackage{../thmboxes_v2}
% \usepackage{thmboxes_col}


\usepackage{hyperref} % note: this is the final package
\parindent = 0pt

% Custom Definitions of operators
\DeclareMathOperator{\Ima}{im}
\DeclareMathOperator{\Fix}{Fix}
\DeclareMathOperator{\Orb}{Orb}
\DeclareMathOperator{\Stab}{Stab}
\DeclareMathOperator{\send}{send}
\DeclareMathOperator{\dom}{dom}

\title{Dynamics and Vector Calculus Notes}
\author{Leon Lee}
\renewcommand\labelitemi{\tiny$\bullet$}


\begin{document}

\maketitle
\newpage
\tableofcontents
\newpage

\section{Couple Oscillations and normal modus}

% \begin{figure}[h!]
%     \centering
%     \includegraphics[width=\linewidth]{}
%     \caption{idk how to do diagrams in LaTeX}
%     \label{spring}
% \end{figure}
some diagram idk

\noindent\rule{\textwidth}{0.2pt}

where $x_{1}$ and $x_{2}$ are displacements from equilibrium

\textbf{For mass 1}
\begin{itemize}
    \item Force to the left: $-k_{1}x_{1}$
    \item Force to the right: $-k_{2}(x_{2}-x_{1})$
\end{itemize}

\[m \frac{d^{2}x_{1}}{dt^{2}} = -k_{1}x_{1} + k_{2}(x_{2} - x_{1}) - k_{3}x_{2}\]

Write this in matrix form
\[m \frac{d^{2}}{dt^{2}}\begin{pmatrix}
    x_{1} \\
    x_{2}
\end{pmatrix} = \begin{pmatrix}
k_{1} + k_{2} & -k_{2} \\
-k_{2} & k_{2} + k_{3}
\end{pmatrix} \begin{pmatrix}
    x_{1} \\
    x_{2}
\end{pmatrix} \implies m \frac{d^{2}x}{dt^{2}} = -K x\]

\begin{dfn}[Normal Mode Solution]{normal-mode}{}
    \textbf{Normal Mode Solution}: All co-ordinates (here $x_{1},\,x_{2}$) oscillate with the same frequency
\end{dfn}

$x(t) = \cos{(\omega t - \phi)} \underline{b}$

$\underline{b}$ is constant vector, $\omega$ to be determined

sub in matrixeq??
\begin{align*}
    &-m \omega^{2} \cos(\omega t - \phi) \underline{b} + K \cos(\omega t - \varnothing) \underline{b} = 0\\
    & -m \omega^{2} \underline{b} + K \underline{b} = 0 \to K \underline{b} = \lambda \underline{b} \quad \lambda = m \omega^{2}
\end{align*}
where $\lambda$ is eigenvalue, and $b$ is eigenvector

For simplicity, take $k_{1} = k_{2} = k_{3} = k$

Then,
\[K = \begin{pmatrix}
    2k & -k \\
    -k & 2k
\end{pmatrix} \qquad (K - \lambda \mathbb{1}) \underline{b} = 0 \implies \lvert k - \lambda \mathbb{1} \rvert = 0\]

\[\begin{vmatrix}
    2k - \lambda & -k \\
    -k & 2k - x
\end{vmatrix} = 0 \implies (2k - \lambda)^{2} - k^{2} = 0\]

This is called the "Characteristic Equation"

$(2k - \lambda) = \pm k \quad \lambda = 2k \mp k$

Therefore, $\lambda = k, 3k$

Mode A: $\lambda_{A} = k \quad (K - k \mathbb{1}) \underline{b} = 0$
\[\begin{pmatrix}
    k & -k \\
    -k & k
\end{pmatrix} \begin{pmatrix}
    b_{1} \\
    b_{2}
\end{pmatrix} = 0 \quad \underline{b}_{A} = Ct \begin{pmatrix}
    1 \\
    1
\end{pmatrix}\]

Usually, choose a constant s.t. $\underline{b} \cdot \underline{b} = 1$

Mode B: $\lambda_{A} = 3k \quad (K - 3k \mathbb{1}) \underline{b} = \begin{pmatrix}
    -k & -k \\
    -k & -k
\end{pmatrix}\begin{pmatrix}
    b_{1} \\
    b_{2}
\end{pmatrix} = 0$

and some stuff more i forgor to write

\noindent\rule{\textwidth}{0.2pt}

[diagram thing]

Normal mode $\underline{x}(t) = \underline{b}\cos(\omega t - \phi)\to (K - \kappa \mathbb{1})\underline{b} = 0 \quad \lambda = m \omega^{2}$
\[\lambda_{A} = k,\, \underline{b}_{A} = \frac{1}{\sqrt{2}}\begin{pmatrix}
    1\\
    1
\end{pmatrix} \quad \lambda_{B} = 3k,\, \underline{b}_{B} = \frac{1}{\sqrt{2}}\begin{pmatrix}
    1\\
    -1
\end{pmatrix}\]

So we have $2$ independent solutions 

General solution: $\underline{x}(t) = A \underline{b}_{A} \cos(\omega_{A} t - \phi_{A}) + B \underline{b}_{B}\cos(\omega_{B}t - \varnothing_{B})$

So there are $4$ constants $A,\,B,\,\phi_{A},\,\varnothing_{B}$ to be fixed


\subsection{Motion in Normal modes}

\begin{align*}
    \text{Mode A} & x_{1} = x_{2} \quad \text{"unphase"} \quad \omega_{A} = \left(\frac{k}{m}\right)^{2} \\
    \text{Mode B} & x_{1} = -x_{2} \quad \text{"antiphase"} \quad \omega_{A} > \omega{B}
\end{align*}

\textbf{Normal Co-ordinates}
Take scalar product

\begin{align*}
    (1, 1) \begin{pmatrix}
    x_{1} \\
    x_{2}
\end{pmatrix} &= x_{1} + x_{2} = 2A\cos(\omega_{A} - \phi_{A}) \\
    (1, -1) \begin{pmatrix}
    x_{1} \\
    -x_{2}
\end{pmatrix} &= x_{1} + x_{2} = 2B\cos(\omega_{B}t - \phi_{B}) \\
\end{align*}

\noindent\rule{\textwidth}{0.2pt}

\textbf{Define}

\begin{align*}
    z_{1} = \frac{1}{\sqrt{2}}(x_{1} + x_{2}) = \alpha^{1}\cos(w_{A}t - \phi) \quad z_{1}+\omega^{2}_{A} z_{1} = 0 \quad\text{(SHO)} \\
    z_{2} = \frac{1}{\sqrt{2}}(x_{1} - x_{2}) = \beta^{1}\cos(w_{B}t - \phi) \quad z_{2}+\omega^{2}_{B} z_{2} = 0\quad\text{(SHO)}
\end{align*}

$z_{1}$ and $z_{2}$ are each independent simple harmonic motions, and energy is preserved in each one

\[E_{A} = \frac{1}{2} m (z_{1})^{2} + \frac{1}{2} k z^{2}_{1} = \text{constant in time}\]

\subsection{Summmary: properties of Normal Modes}
\[\]
\begin{itemize}
    \item $\underline{x}_{\alpha} = A_{\alpha} \underline{b}_{A} \cos(\omega_{\alpha} t - \phi_{\alpha}))$
    \item All coordinates oscillate at the same frequency
    \item constants $A_{\alpha}, \phi_{\alpha}$ are fixed by  ic (???)
    \item General motion is superposition of normal modes
    \item Normal coordinates $z_{\alpha} = \underline{b}_{\alpha} \cdot \underline{x}$
    \item Transforming to the noraml coordinates $\to$ diagonalise $k$ (see notes i.e. ask alice or fiona for them)
    \item Energy in each normal mode conserved, mode with lowest $\omega$ is the most symmetric
\end{itemize}




\end{document}
