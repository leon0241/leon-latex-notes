\documentclass[landscape, 8pt]{extarticle}

\usepackage{../../preamble}
\usepackage[separate]{../../rss/thmboxes_v4}
\usepackage{symbols}


\begin{document}
\setlength{\abovedisplayskip}{3.5pt}
\setlength{\belowdisplayskip}{3.5pt}
\setlength{\abovedisplayshortskip}{3.5pt}
\setlength{\belowdisplayshortskip}{3.5pt}
\setlength{\multicolsep}{0pt}% 50% of original values
\setlist{topsep=3pt, itemsep=0pt}

\begin{multicols*}{3}
\raggedcolumns
\section*{\huge Group Theory Notes}

\section{Revision of Groups}
\begin{dfn}[Definition of a Group]{dfn:group-definition}{1.1.1}
	A \textbf{group} consists of a set $G$ together with a function $G \times G zto G$ which maps an ordered pair $(g, h) \in G \times G$ to an element $g \star h \in G$. The following axioms must be satisfied:
	\begin{enumerate}
	    \item \textbf{Associativity}: $(g \star h) \star k = g \star (h \star k)$ for each triple $(g,h,k)\in G \times G \times G$
	    \item \textbf{Identity}: $\exists e\in G$ such that $e\star g = g = g \star e$ for each $g\in G$
	    \item \textbf{Inverse}: To each element $g\in G$, there is an element $g^{-1} \in G$ such that $g\star h = e = h \star g$
	\end{enumerate}
	\textbf{Note}: The closure axiom follows from the definition of a function.
\end{dfn}

\begin{xmp}[Examples of Groups]{xmp:groups}{1.2.A}
	\begin{itemize}
	    \item[\textbf{1.2.1})] $S_{n}$, the \textbf{$n$-th symmetric group}, is the group of permutations of $\{1,2,\dots,n\}$, with composition of functions.
	    \item[\textbf{1.2.2})] $D_{n}$, the \textbf{$n$-th dihedral group}, is the group of symmetries of the $n$-gon. It has $2n$ elements: $n$ rotations, and $n$ reflections.
	    \item[\textbf{1.2.3})] The \textbf{free group} on the letters $x, y$ is written as $G = \langle x, y \rangle$. The elements of $G$ are \textbf{words} in the symbols $x, y, x^{-1}, y^{-1}$. The group operation $\star$ is \textbf{concatenation}: so $x x x^{-1}y \star y^{-1} x = x x x^{-1} y y^{-1} x$. $e$ is the \textbf{empty word} with $0$ letters, and $x^{-1}$ and $y^{-1}$ is the inverse of $x$ and $y$ respectively. Thus $x x x^{-1} y = xy$.
	    \item[\textbf{1.2.4})] $(\mathbb{Z}, +)$ is a group, with $e = 0$. It is a \textbf{cyclic} group, and is \textbf{generated} by $1$.
	    \item[\textbf{1.2.5})] $\mathbb{Z} /n$, the set of integers modulo $n$, is a group under $+$.
	\end{itemize}
\end{xmp}

\begin{dfn}[Abelian Group]{dfn:abelian}{1.2.6}
	A group $(G, \star)$ is \textbf{abelian} if $g \star h = h \star g$ for all $g, h\in G$.

	\textbf{Note}: Often, when $(G, \star)$ is ablian, we write $g+h$ as the group operation.
\end{dfn}

\begin{dfn}[Subgroup]{dfn:subgroup}{1.3.1}
	If $H$ is a nonempty subset of $G$ then $H$ is a \textbf{subgroup} provided that:
	\begin{itemize-zero}
	    \item $hk \in H$ for all $h, k\in H$.
	    \item $h^{-1} \in H$ for all $h\in H$.
	\end{itemize-zero}

	We usually just say ``$H$ is closed under the group operations''. Note that $e\in H$ follows from the definition, and associativity follows from the fact that $G$ is a group. Any subgroup $H$ of $G$ is a group using the same product as that of $G$.

	We write $H \le G$ when $H$ is a subgroup of $G$ (as opposed to $H \subseteq G$ which just means $H$ is a subset of $G$). The notation $H < G$ means that $H$ is a subgroup of $G$ and $H \ne G$. A subgroup $H$ is \textbf{proper} if $H \ne G$ and is \textbf{non-trivial} if $H \ne \{e\}$.
\end{dfn}

\begin{xmp}[Examples of Subgroups]{xmp:subgroups}{1.3.A}
	\begin{itemize}
	    \item[\textbf{1.3.2})] The subsets $\{e\}$ and $G$ are always subgroups of $G$
		\item[\textbf{1.3.3})] The group of rotations of an $n$-gon is a subgroup of $D_{n}$
		\item[\textbf{1.3.4})] The \textbf{$n$-th alternating group}, $A_{n}$ is the subgroup of $S_{n}$ consisting of all permutations that can be written as the product of an even number of $2$-cycles.
		\item[\textbf{1.3.5})] Let $G$ be a group and $g\in G$. Then $\langle g \rangle := \{g^{n} : n \in \mathbb{Z}\}$ is a subgroup of $G$, called the \textbf{subgroup generated by $g$}. If $G = \langle g \rangle$ for some $g\in G$, then $G$ is \textbf{cyclic}.
	\end{itemize}
\end{xmp}

\begin{dfn}[Coset]{dfn:coset}{1.3.6}
	Let $H \le G$ and $g\in G$. Then the \textbf{left coset} of $H$ determined by $g$ is the set 
	\[gH := \{gh \mid h\in H\}\]
	Similarly, the \textbf{right coset} of $H$ determind by $g$ is the set
	\[Hg := \{hg \mid h\in H\}\]
\end{dfn}


\end{multicols*}
\end{document}
