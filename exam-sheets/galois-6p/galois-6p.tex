\documentclass[landscape, 8pt]{extarticle}

\usepackage{../../preamble}
\usepackage{symbols}

\begin{document}

\setlength{\abovedisplayskip}{3.5pt}
\setlength{\belowdisplayskip}{3.5pt}
\setlength{\abovedisplayshortskip}{3.5pt}
\setlength{\belowdisplayshortskip}{3.5pt}

\begin{multicols}{3}
\raggedcolumns

\section*{\huge Galois Theory Notes}
Made by Leon :) \textit{Note: Any reference numbers are to the lecture notes}

\section{Galois Groups}

\begin{dfn}[Conjugate Numbers]{dfn:conjugate-numbers}{1.1.1}
    Two complex numbers $z$ and $z'$ are \textbf{conjugate over $\mathbb{R}$} if for all polynomials $p$ with coefficients in $\mathbb{R}$,
    \[p(z) = 0 \iff p(z') = 0\]
\end{dfn}

\begin{lma}[Characterising Conjugates]{lma:charaterising-conjugates}{1.1.2}
    $z,z'\in \mathbb{C}$ are conjugate over $\mathbb{R}$ iff either $z = z'$ or $\overline{z} = z'$
\end{lma}

\begin{dfn}[Conjugacy in \texorpdfstring{$\mathbb{Q}$}{Q}]{dfn:conj-q}{1.1.9}
    $z,z'\in \mathbb{C}$ are \textbf{conjugate over $\mathbb{Q}$} if $\forall p(t) \in \mathbb{Q}[t]$
    \[p(z) = 0 \iff p(z') = 0\]
\end{dfn}

\begin{dfn}[Conjugacy for sets]{dfn:set-conjugacy}{1.1.9}
    $(z_{1},\dots,z_{n}), z_{i}, z_{i}' \in \mathbb{C}$
    is conjugate over $\mathbb{Q}$ to $(z_{1}',\dots,z_{n}')$
    if $\forall p(t_{1},\dots,t_{n})\in \mathbb{Q}[t_{1},\dots,t_{n}]$

    \longrule{0.08ex}

    Additionally, if $(z_{1},\dots,z_{n})$ conjugate to $(z_{1}',\dots,z_{n}')$, then $z_{i}$ is conjugate to $z_{i}'$ for all $i$
\end{dfn}

% TODO: idk what's happening here

\begin{dfn}[Galois Group]{dfn:galois-group}{1.2.1}
    Let $f$ be a polynomia lwith coefficients in $\mathbb{Q}$. Write $\alpha_{1},\dots,\alpha_{k}$ for its distinct roots in $\mathbb{C}$. The \textbf{Galois group} of $f$ is
    \[\mathrm{Gal}(g) = \{\sigma \in S_{n} \mid (\alpha_{1},\dots,\alpha_{n}) \text{ conjugate to } (\alpha_{S(1)},\dots,\alpha_{\sigma(n)})\}\]
    \textbf{Note}: distinct roots mean that we ignore any repetition of roots.
\end{dfn}

\begin{dfn}[Solvability (Simple Definition)]{dfn:solvability}{1.3.0}
    A complex number is \textbf{radical} if it can be obtained from the rationals using only the usual arithmetic operations and $k$th roots. A polynomial over $\mathbb{Q}$ is \textbf{solvable (or soluble) by radicals} if all of its complex roots are radical.
\end{dfn}

\begin{thm}[Galois]{thm:galois}{1.3.5}
    Let $f$ be a polynomial over $\mathbb{Q}$. Then
    \[f \text{ is solvable by radicals} \iff \mathrm{Gal}(f) \text{ is a solvable group.}\]
\end{thm}

\section{Groups, Rings, and Fields}

\begin{dfn}[Group Action]{dfn:group-action}{2.1.1}
    Let $G$ be a group and $X$ a set. An \textbf{action} of $G$ on $X$ is a function $G \times X \to X$, written as $(g,x)\mapsto gx$ such that
    \[(gh)x = g(hx)\]
    for all $g, h\in G$ and $x\in X$ and
    \[1x = x\]
    for all $x\in X$, where $1$ is the identity of $G$
\end{dfn}

\begin{dfn}[Faithful Actions]{dfn:faithful-actions}{2.1.7}
    An action of a group $G$ on a set $X$ is \textbf{faithful} if for $g,\,h\in G$,
    \[gx = hx \text{ for all } x\in X \implies g = h\]
    Faithfulness means that if two elements of the group \textit{do} the same, they \textit{are} the same.
\end{dfn}

\begin{lma}[Faithful Properties]{lma:faithful-props}{2.1.8}
    For an action of a group $G$ on a set $X$, the following are equivalent:
    \begin{enumerate-tight}
        \item The action is faithful
        \item For $g\in G$, if $gx = x$ for all $x\in X$ then $g = 1$
        \item The homomorphism $\Sigma : G \to \sym(X)$ is injective
        \item $\ker \Sigma$ is trivial.
    \end{enumerate-tight}
\end{lma}

% Actions examples

\begin{lma}[Isomorphisms of Faithful Groups]{lma:faith-isos}{2.1.11}
    Let $G$ be a group acting faithfully on a set $X$. then $G$ is isomorphic to the subgroup
    \[\im \Sigma = \{\overline{g} \mid g\in G\}\]
    of $\sym(X)$, where $\Sigma : G \to \sym(X)$ and $\overline{g}$ are defined as above.
\end{lma}

\begin{dfn}[Fixed Set]{dfn:fixed-set}{2.1.1}
    Let $G$ be a group acting on a set $X$. Let $S \subseteq G$. The \textbf{fixed set} of $S$ is
    \[\fix(S) = \{ x\in X \mid sx = x \text{ for all } s\in S\}\]
\end{dfn}

\begin{lma}[Normal Fixed Sets]{lma:normal-fixed}{2.1.15}
    Let $G$ be a group acting on a set $X$, let $S \subseteq G$, and let $g\in G$. Then $\fix(gSg^{-1}) = g\fix(S)$.

    Here, $gSg^{-1} = \{gs g^{-1} \mid s\in S\}$ and $g\fix(S) = \{gx \mid x \in \fix(S)\}$
\end{lma}

\begin{dfn}[Ring Homomorphism]{dfn:ring-homomorphism}{2.2.1}
    Given rings $R$ and $S$, a \textbf{homomorphism} from $R$ to $S$ is a function $\phi : R \to S$ satisfying the following equations for all $r,\,r'\in R$:
    \begin{multicols}{2}
    \begin{itemize-tight}
        \item $\phi(r+r') = \phi(r) + \phi(r')$
        \item $\phi(rr') = \phi(r)\phi(r')$
        \item $\phi(0) = 0$, $\phi(1) = 1$
        \item $\phi(-r) = -\phi(r)$
    \end{itemize-tight}
    \end{multicols}

    \longrule{0.08ex}
    A \textbf{subring} of a ring $R$ is a subset $S \subseteq R$ that contains $0$ and $1$ and is closed under addition, multiplication, and negatives. Whenever $S$ is a subring of $R$, the inclusion $\iota : S \to R$ (defined by $\iota(s) = s$) is a homomorphism.
\end{dfn}

\begin{lma}[Intersection of Subrings]{lma:subring-intersections}{2.2.3}
    Let $R$ be a ring and let $\mathcal{S}$ be any set (perhaps infinite) of subrings of $R$. Then their intersection $\bigcap_{S\in \mathcal{S}} S$ is also a subring of $R$.
\end{lma}

% TODO: ExaPLE 2.2

\begin{rcl}[Ideals and Quotient Rings]{rcl:ideal}{}
    Let $R$ be a ring. $I \subseteq R$ is an \textbf{ideal}, $I \unlhd R$, if the following hold:
    \begin{enumerate}
        \setlength\itemsep{0em}
        \item $I \ne \emptyset$
        \item $I$ is closed under subtraction
        \item for all $i\in I$ and $r\in R$ we have $ri, ir\in I$
    \end{enumerate}
    Every ring homomorphism $\phi : R \to S$ has an image $\im \phi$, which is a subring of $S$, and a kernel $\ker \phi$, which is an ideal of $R$.

    \longrule{0.08ex}

    Given an ideal $I \unlhd R$, we obtain the quotient ring $R / I$ and the canonical homomorphism $\pi_{I} : R \to R /I$ which is surjective and hs kernel $I$. 

    \textbf{Universal Prop}: Given any ring $S$ and any homomorphism $\phi : R \to S$ satisfying $\ker \phi \supseteq I$, there is exactly one homomorphism $\overline{\phi} : R /I \to S$ such that this diagram communutes.
% https://q.uiver.app/#q=WzAsMyxbMCwwLCJSIl0sWzAsMiwiUiAvIEkiXSxbMiwyLCJTIl0sWzAsMSwiXFxwaV9JIiwyXSxbMSwyLCJcXG92ZXJsaW5le1xccGhpfSIsMl0sWzAsMiwiXFxwaGkiXV0=
\[\begin{tikzcd}[cramped,column sep=scriptsize]
	R \\
	\\
	{R / I} && S
	\arrow["{\pi_I}"', from=1-1, to=3-1]
	\arrow["\phi", from=1-1, to=3-3]
	\arrow["{\overline{\phi}}"', from=3-1, to=3-3]
\end{tikzcd}\]
\end{rcl}

\begin{rcl}[Integral Domain]{rcl:integral-domain}{}
    An \textbf{integral domain} is a ring $R$ such that $0_{R} \ne 1_{R}$ and for $r,\,r'\in R$,
    \[rr' = 0 \implies r = 0 \text{ or } r' = 0\]
\end{rcl}

\newpage
\begin{rcl}[Generated Ideal]{rcl:generated-ideal}{}
    Let $Y$ be a subset of a ring $R$. The \textbf{ideal $\langle Y \rangle$ generated by $Y$} is defined as the intersection of all the ideals of $R$ containing $Y$.

    \longrule{0.08ex}
    \begin{itemize-zero}
        \item Ideals of the form $\langle r \rangle$ are called \textbf{principal ideals}. A \textbf{principle ideal domain} is an integral domain where every ideal is principal.
        \item Let $r$ and $s$ be elements of a ring $R$. We say that $r$ \textbf{divides} $s$, and write $r \mid s$ if there exists $a\in R$ such that $s = ar$. This condition is equivalent to $s\in \langle r \rangle$, and to $\langle s \rangle \supseteq \langle r \rangle$.
        \item An element $u\in R$ is a \textbf{unit} if it has a multiplicative inverse, or equivalently, if $\langle u \rangle = R$. The units form a group $R^{\times}$ under multiplication.
        \item Elements $r$ and $s$ of a ring are \textbf{coprime} if for $a\in R$,
            \[a \mid r \text{ and } a \mid s \implies a \text{ is a unit}\]
    \end{itemize-zero}
\end{rcl}

\begin{lma}[Characterisation of Generated Ideals]{lma:generated-ideal}{2.2.11}
    Let $R$ be a ring and let $Y = \{r_{1},\dots,r_{n}\}$ be a finite subset. Then
    \[\langle Y \rangle = \{a_{1}r_{1} + \cdots + a_{n}r_{n} : a_{1},\dots, a_{n} \in R\}\]
\end{lma}

\begin{ppn}[Coprime and PIDs]{ppn:coprime-pid}{2.2.16}
    Let $R$ be a principal ideal domain and $r,\,s\in R$. Then
    \[r \text{ and } s \text{ are coprime} \iff ar + bs = 1 \text{ for some } a,\,b\in R\]
\end{ppn}

\lipsum[1-12]
\end{multicols}
\end{document}
