\documentclass[landscape, 8pt]{extarticle}

\usepackage{../../preamble}
\usepackage{symbols}

\begin{document}

\setlength{\abovedisplayskip}{3.5pt}
\setlength{\belowdisplayskip}{3.5pt}
\setlength{\abovedisplayshortskip}{3.5pt}
\setlength{\belowdisplayshortskip}{3.5pt}

\begin{multicols}{3}
\raggedcolumns

\section*{\huge Galois Theory Notes}
Made by Leon :) \textit{Note: Any reference numbers are to the lecture notes}

\section{Galois Groups}

\begin{dfn}[Conjugate Numbers]{dfn:conjugate-numbers}{1.1.1}
    Two complex numbers $z$ and $z'$ are \textbf{conjugate over $\mathbb{R}$} if for all polynomials $p$ with coefficients in $\mathbb{R}$,
    \[p(z) = 0 \iff p(z') = 0\]
\end{dfn}

\begin{lma}[Characterising Conjugates]{lma:charaterising-conjugates}{1.1.2}
    $z,z'\in \mathbb{C}$ are conjugate over $\mathbb{R}$ iff either $z = z'$ or $\overline{z} = z'$
\end{lma}

\begin{dfn}[Conjugacy in \texorpdfstring{$\mathbb{Q}$}{Q}]{dfn:conj-q}{1.1.9}
    $z,z'\in \mathbb{C}$ are \textbf{conjugate over $\mathbb{Q}$} if $\forall p(t) \in \mathbb{Q}[t]$
    \[p(z) = 0 \iff p(z') = 0\]
\end{dfn}

\begin{dfn}[Conjugacy for sets]{dfn:set-conjugacy}{1.1.9}
    $(z_{1},\dots,z_{n}), z_{i}, z_{i}' \in \mathbb{C}$
    is conjugate over $\mathbb{Q}$ to $(z_{1}',\dots,z_{n}')$
    if $\forall p(t_{1},\dots,t_{n})\in \mathbb{Q}[t_{1},\dots,t_{n}]$

    \longrule{0.08ex}

    Additionally, if $(z_{1},\dots,z_{n})$ conjugate to $(z_{1}',\dots,z_{n}')$, then $z_{i}$ is conjugate to $z_{i}'$ for all $i$
\end{dfn}

% TODO: idk what's happening here

\begin{dfn}[Galois Group]{dfn:galois-group}{1.2.1}
    Let $f$ be a polynomia lwith coefficients in $\mathbb{Q}$. Write $\alpha_{1},\dots,\alpha_{k}$ for its distinct roots in $\mathbb{C}$. The \textbf{Galois group} of $f$ is
    \[\mathrm{Gal}(g) = \{\sigma \in S_{n} \mid (\alpha_{1},\dots,\alpha_{n}) \text{ conjugate to } (\alpha_{S(1)},\dots,\alpha_{\sigma(n)})\}\]
    \textbf{Note}: distinct roots mean that we ignore any repetition of roots.
\end{dfn}

\begin{dfn}[Solvability (Simple Definition)]{dfn:solvability}{1.3.0}
    A complex number is \textbf{radical} if it can be obtained from the rationals using only the usual arithmetic operations and $k$th roots. A polynomial over $\mathbb{Q}$ is \textbf{solvable (or soluble) by radicals} if all of its complex roots are radical.
\end{dfn}

\begin{thm}[Galois]{thm:galois}{1.3.5}
    Let $f$ be a polynomial over $\mathbb{Q}$. Then
    \[f \text{ is solvable by radicals} \iff \mathrm{Gal}(f) \text{ is a solvable group.}\]
\end{thm}

\section{Groups, Rings, and Fields}

\begin{dfn}[Group Action]{dfn:group-action}{2.1.1}
    Let $G$ be a group and $X$ a set. An \textbf{action} of $G$ on $X$ is a function $G \times X \to X$, written as $(g,x)\mapsto gx$ such that
    \[(gh)x = g(hx)\]
    for all $g, h\in G$ and $x\in X$ and
    \[1x = x\]
    for all $x\in X$, where $1$ is the identity of $G$
\end{dfn}

\begin{dfn}[Faithful Actions]{dfn:faithful-actions}{2.1.7}
    An action of a group $G$ on a set $X$ is \textbf{faithful} if for $g,\,h\in G$,
    \[gx = hx \text{ for all } x\in X \implies g = h\]
    Faithfulness means that if two elements of the group \textit{do} the same, they \textit{are} the same.
\end{dfn}

\begin{lma}[Faithful Properties]{lma:faithful-props}{2.1.8}
    For an action of a group $G$ on a set $X$, the following are equivalent:
    \begin{enumerate-tight}
        \item The action is faithful
        \item For $g\in G$, if $gx = x$ for all $x\in X$ then $g = 1$
        \item The homomorphism $\Sigma : G \to \sym(X)$ is injective
        \item $\ker \Sigma$ is trivial.
    \end{enumerate-tight}
\end{lma}

% Actions examples

\begin{lma}[Isomorphisms of Faithful Groups]{lma:faith-isos}{2.1.11}
    Let $G$ be a group acting faithfully on a set $X$. then $G$ is isomorphic to the subgroup
    \[\im \Sigma = \{\overline{g} \mid g\in G\}\]
    of $\sym(X)$, where $\Sigma : G \to \sym(X)$ and $\overline{g}$ are defined as above.
\end{lma}

\begin{dfn}[Fixed Set]{dfn:fixed-set}{2.1.1}
    Let $G$ be a group acting on a set $X$. Let $S \subseteq G$. The \textbf{fixed set} of $S$ is
    \[\fix(S) = \{ x\in X \mid sx = x \text{ for all } s\in S\}\]
\end{dfn}

\begin{lma}[Normal Fixed Sets]{lma:normal-fixed}{2.1.15}
    Let $G$ be a group acting on a set $X$, let $S \subseteq G$, and let $g\in G$. Then $\fix(gSg^{-1}) = g\fix(S)$.

    Here, $gSg^{-1} = \{gs g^{-1} \mid s\in S\}$ and $g\fix(S) = \{gx \mid x \in \fix(S)\}$
\end{lma}

\begin{dfn}[Ring Homomorphism]{dfn:ring-homomorphism}{2.2.1}
    Given rings $R$ and $S$, a \textbf{homomorphism} from $R$ to $S$ is a function $\phi : R \to S$ satisfying the following equations for all $r,\,r'\in R$:
    \begin{multicols}{2}
    \begin{itemize-tight}
        \item $\phi(r+r') = \phi(r) + \phi(r')$
        \item $\phi(rr') = \phi(r)\phi(r')$
        \item $\phi(0) = 0$, $\phi(1) = 1$
        \item $\phi(-r) = -\phi(r)$
    \end{itemize-tight}
    \end{multicols}

    \longrule{0.08ex}
    A \textbf{subring} of a ring $R$ is a subset $S \subseteq R$ that contains $0$ and $1$ and is closed under addition, multiplication, and negatives. Whenever $S$ is a subring of $R$, the inclusion $\iota : S \to R$ (defined by $\iota(s) = s$) is a homomorphism.
\end{dfn}

\begin{lma}[Intersection of Subrings]{lma:subring-intersections}{2.2.3}
    Let $R$ be a ring and let $\mathcal{S}$ be any set (perhaps infinite) of subrings of $R$. Then their intersection $\bigcap_{S\in \mathcal{S}} S$ is also a subring of $R$.
\end{lma}

% TODO: ExaPLE 2.2

\begin{rcl}[Ideals and Quotient Rings]{rcl:ideal}{}
    Let $R$ be a ring. $I \subseteq R$ is an \textbf{ideal}, $I \unlhd R$, if the following hold:
    \begin{enumerate}
        \setlength\itemsep{0em}
        \item $I \ne \emptyset$
        \item $I$ is closed under subtraction
        \item for all $i\in I$ and $r\in R$ we have $ri, ir\in I$
    \end{enumerate}
    Every ring homomorphism $\phi : R \to S$ has an image $\im \phi$, which is a subring of $S$, and a kernel $\ker \phi$, which is an ideal of $R$.

    \longrule{0.08ex}

    Given an ideal $I \unlhd R$, we obtain the quotient ring $R / I$ and the canonical homomorphism $\pi_{I} : R \to R /I$ which is surjective and hs kernel $I$. 

    \textbf{Universal Prop}: Given any ring $S$ and any homomorphism $\phi : R \to S$ satisfying $\ker \phi \supseteq I$, there is exactly one homomorphism $\overline{\phi} : R /I \to S$ such that this diagram communutes.
% https://q.uiver.app/#q=WzAsMyxbMCwwLCJSIl0sWzAsMiwiUiAvIEkiXSxbMiwyLCJTIl0sWzAsMSwiXFxwaV9JIiwyXSxbMSwyLCJcXG92ZXJsaW5le1xccGhpfSIsMl0sWzAsMiwiXFxwaGkiXV0=
\[\begin{tikzcd}[cramped,column sep=scriptsize]
	R \\
	\\
	{R / I} && S
	\arrow["{\pi_I}"', from=1-1, to=3-1]
	\arrow["\phi", from=1-1, to=3-3]
	\arrow["{\overline{\phi}}"', from=3-1, to=3-3]
\end{tikzcd}\]
\end{rcl}

\begin{rcl}[Integral Domain]{rcl:integral-domain}{}
    An \textbf{integral domain} is a ring $R$ such that $0_{R} \ne 1_{R}$ and for $r,\,r'\in R$,
    \[rr' = 0 \implies r = 0 \text{ or } r' = 0\]
\end{rcl}

\newpage
\begin{rcl}[Generated Ideal]{rcl:generated-ideal}{}
    Let $Y$ be a subset of a ring $R$. The \textbf{ideal $\langle Y \rangle$ generated by $Y$} is defined as the intersection of all the ideals of $R$ containing $Y$.

    \longrule{0.08ex}
    \begin{itemize-zero}
        \item Ideals of the form $\langle r \rangle$ are called \textbf{principal ideals}. A \textbf{principle ideal domain} is an integral domain where every ideal is principal.
        \item Let $r$ and $s$ be elements of a ring $R$. We say that $r$ \textbf{divides} $s$, and write $r \mid s$ if there exists $a\in R$ such that $s = ar$. This condition is equivalent to $s\in \langle r \rangle$, and to $\langle s \rangle \supseteq \langle r \rangle$.
        \item An element $u\in R$ is a \textbf{unit} if it has a multiplicative inverse, or equivalently, if $\langle u \rangle = R$. The units form a group $R^{\times}$ under multiplication.
        \item Elements $r$ and $s$ of a ring are \textbf{coprime} if for $a\in R$,
            \[a \mid r \text{ and } a \mid s \implies a \text{ is a unit}\]
    \end{itemize-zero}
\end{rcl}

\begin{lma}[Characterisation of Generated Ideals]{lma:generated-ideal}{2.2.11}
    Let $R$ be a ring and let $Y = \{r_{1},\dots,r_{n}\}$ be a finite subset. Then
    \[\langle Y \rangle = \{a_{1}r_{1} + \cdots + a_{n}r_{n} : a_{1},\dots, a_{n} \in R\}\]
\end{lma}

\begin{ppn}[Coprime and PIDs]{ppn:coprime-pid}{2.2.16}
    Let $R$ be a principal ideal domain and $r,\,s\in R$. Then
    \[r \text{ and } s \text{ are coprime} \iff ar + bs = 1 \text{ for some } a,\,b\in R\]
\end{ppn}

\begin{rcl}[Field]{rcl:field}{2.3.0}
    A \textbf{field} is a ring $K$ in which $0 \ne 1$ and every nonzero element is a unit. Equivalently, it is a ring such that $K^{\times} = K \backslash \{0\}$. Every field is an integral domain.

    A field $K$ has exactly two ideals: $\{0\}$ and $K$.

    A \textbf{subfield} of a field $K$ is a subring that is a field
\end{rcl}

\begin{xmp}[Rational Expressions]{xmp:rational-expressions}{2.3.2}
    Let $K$ be a field. A \textbf{rational expression} over $K$ is a ratio of two polynomials
    \[\frac{f(t)}{g(t)}\]
    where $f(t),\,g(t)\in K[t]$ with $g\ne 0$. Two such expressions, $f_{1} /g_{1}$ and $f_{2} /g_{2}$ are regarded as equal if $f_{1}g_{2} = f_{2}g_{1}$ in $K[t]$. i.e. equivalence class. The set of rational expressions over $K$ is denoted by $K(t)$
\end{xmp}

\begin{lma}[Homomorphisms between fields]{lma:homomorphisms-fields}{2.3.3}
    Every (ring) homomorphism between fields is injective.
\end{lma}

\begin{lma}[Images of Subfields]{lma:subfield-images}{2.3.6}
    Let $\phi : K \to L$ be a homomorphism between fields.
    \begin{enumerate-tight}
        \item For any subfield $K'$ of $K$, the image $\phi K'$ is a subfield of $L$
        \item For any subfield $L'$ of $L$, the preimage $\phi^{-1}L'$ is a subfield of $K$
    \end{enumerate-tight}
\end{lma}

\begin{dfn}[Equaliser]{dfn:equaliser}{2.3.7}
    Let $X$ and $Y$ be sets, and let $S \subseteq \{ \text{ functions } X \to Y\}$. The \textbf{equalizer} of $S$ is
    \[\eq(S) = \{x\in X \mid f(x) = g(x) \text{ for all } f,\,g\in S\}\]
    i.e., it is the part of $X$ where all the functions in $S$ are equal.
\end{dfn}

\begin{lma}[Equalisers are Subfields]{lma:equaliser-subfield}{2.3.8}
    Let $K$ and $L$ be fields, and let $S \subseteq \{\text{homomorphisms} K \to L\}$. Then $\eq(S)$ is a subfield of $K$.
\end{lma}

\begin{rcl}[Characteristic]{rcl:characteristic}{2.3.9}
    Let $R$ be any ring. There is a unique homomorphism $\chi : \mathbb{Z} \to R$. Its kernel is an ideal of the principal ideal domain $\mathbb{Z}$. Hence $\ker \chi = \langle n \rangle$ for a unique integer $n \ge 0$. This $n$ is called the \textbf{characteristic} of $R$, and written as $\Char R$. So for $m\in \mathbb{Z}$, we have that $m \cdot 1_{R} = 0$ iff $m$ is a multiple of $\Char R$. Or equivalently,
    \[\Char R = \begin{cases}
        \text{the least $n > 0$ s.t. $n \cdot 1_{R} = 0_{R}$}, & \text{if such an $n$ exists}\\
        0 & \text{otherwise}
    \end{cases}\]
\end{rcl}

\begin{lma}[Characteristic of Integral Domains]{lma:integral-domain-char}{2.3.11}
    The characteristic of an integral domain is $0$ or a prime number.
\end{lma}

\begin{lma}[Characteristics of Homomorphisms]{lma:homomorphism-characteristic}{2.3.12}
    Let $\phi : K \to L$ be a homomorphism of fields. Then $\Char K = \Char L$.
\end{lma}

\begin{rcl}[Prime Subfield]{rcl:prime-subfield}{2.3.C}
    The \textbf{prime subfield} of $K$ is the inersection of all the subfields of $K$. Any intersection of subfields is a subfield, and is the smallest subfield of $K$, in teh sense that any other subfield of $K$ contains it. Concretely, the prime subfield of $K$ is
    \[\left\{ \frac{m \cdot 1_{K}}{n \cdot 1_{K}} \mid m,\,n\in \mathbb{Z} \text{ with } n \cdot 1_{K} \ne 0\right\}\]
\end{rcl}

\begin{lma}[Prime Subfields]{lma:only-prime-subfields}{2.3.16}
    Let $K$ be a field.
    \begin{itemize-tight}
        \item If $\Char K = 0$ then the prime subfield of $K$ is (iso to) $\mathbb{Q}$.
        \item If $\Char K = p > 0$ then the prime subfield of $K$ is (iso to) $\mathbb{F}_{p}$
    \end{itemize-tight}
\end{lma}

\begin{lma}[Characteristic of Finite Fields]{lma:finite-field-characteristic}{2.3.17}
    Every finite field has positive characteristic.
\end{lma}

\begin{lma}[Prime Division]{lma:prime-division}{2.3.19}
    Let $p$ be a prime and $0 < i < p$. Then $p \mid \binom{p}{i}$
\end{lma}

\begin{ppn}[Characteristics and Primes]{ppn:characteristics-prime}{2.3.20}
    Let $p$ be a prime number and $R$ a ring of characteristic $p$.
    \begin{enumerate-tight}
        \item The function
            \[\theta : R \to R \quad r \mapsto r^{p}\]
            is a homomorphism.
        \item If $R$ is a field then $\theta$ is injective.
        \item If $R$ is a finite field then $\theta$ is an automorphism of $R$
    \end{enumerate-tight}

    \longrule{0.08ex}
    The homomorphism $\theta : r \mapsto r^{p}$ is called the \textbf{Frobenius map}, or, in the case of finite fields, the \textbf{Frobenius Automorphism}.
\end{ppn}

\begin{crl}[Roots by Characteristic]{crl:root-characteristic}{2.3.22}
    Let $p$ be a prime number.
    \begin{enumerate-tight}
        \item In a field of characteristic $p$, every element has \textit{at most} one $p$th root.
        \item In a finite field of characteristic $p$, every element has \textit{exactly} one $p$th root.
    \end{enumerate-tight}
\end{crl}

% TODO: examples

\begin{rcl}[Reducible Elements]{rcl:reducible-element}{2.3.D}
    An element $r$ of a ring $R$ is \textbf{irreducible} if $r$ is not $0$ or a unit, and if for $a,\,b\in R$.
    \[r = ab \implies a \text{ or } b \text{ is a tu}\]
    For example, the irreducibles in $\mathbb{Z}$ are $\pm 2,\,\pm 3,\,\pm 5,\dots$. An element of a ring is \textbf{reducible} if it is not $0$, a unit, or irreducible.

    \longrule{0.08ex}
    \textbf{Warning}: The $0$ and units of a ring are neither reducible nor irreducible, in much the same way that the integers $0$ and $1$ are neither prime nor composite.
\end{rcl}

\newpage

\begin{ppn}[]{ppn:irreducible-generated-field}{2.3.26}
    Let $R$ be a principal ideal domain and $0 \ne r \in R$. Then
    \[r \text{ is irreducible } \iff R / \langle r \rangle \text{ is a field}\]
    This lets us construct fields from irreducible elements of a PID.
\end{ppn}

\section{Polynomials}

\begin{dfn}[Polynomial Ring]{dfn:polynomial-ring}{3.1.1}
    Let $R$ be a ring. A \textbf{polynomial over $R$} is an infinite sequence $(a_{0},a_{1},a_{2},\dots)$ of elements of $R$ s.t. $\{i \mid a_{i} \ne 0\}$ is finite.

    The set of polynomials over $R$ forms a ring as follows:
    \begin{align*}
        (a_{0},\, a_{1} ,\dots) + (b_{0},b_{1},\dots) &= (a_{0}+ b_{0},a_{1}+b_{1},\dots), \\
        (a_{0},\, a_{1} ,\dots) \cdot (b_{0},b_{1},\dots) &= (c_{0},c_{1},\dots), \\
        \text{where } c_{k} &= \sum_{i,j : i + j = k} a_{i}b_{j}
    \end{align*}
    The zero is $(0,0,\dots)$ and the mult. identity is $(1,0,0,\dots)$.

    The set of polynomials over $R$ is written as $R[t]$. Since $R[t]$ is itself a ring $S$, we can consider the ring $S[u] = (R[t,u])[v]$, etc.

    Polynomials are typically written as $f$ or $f(t)$, interchangeable. A polynomial $f = (a_{0},a_{1},\dots)$ over $R$ gives rise to a function
    \begin{align*}
        R &\to R \\
        r &\mapsto a_{0}+a_{1}r+a_{2}r^{2}+\cdots.
    \end{align*}
\end{dfn}

% warnings

\begin{rem}[Rational Functions vs Expressions]{rem:rational-function-expression}{3.1.5}
    $K(t)$ is the field of \textit{rational expressions} over a field $K$. These are \textbf{not} functions, e.g. $1 /(t - 1)$ is a totally respectable element of $K(t)$, and you don't need to worry about $t=1$.
\end{rem}

\begin{ppn}[Universal Property of the Polyring]{ppn:uniprop-polyring}{3.1.6}
    Let $R$ and $B$ be rings. For every homomorphism $\phi : R \to B$ and every $b\in B$, there is exactly one homomorphism $\theta : R[t] \to B$ such that
    \begin{align*}
        \theta(a) &= \phi(a) \text{ for all } a\in R\\
        \theta(t) &= b
    \end{align*}
\end{ppn}

\begin{dfn}[Induced Homomorphism]{dfn:induced-homomorphism}{3.1.7}
    Let $\phi : R \to S$ be a ring homomorphism. The \textbf{induced homomorphism}
    \[\phi_{\ast} : R[t] \to S[t]\]
    is the unique homomorphism $R[t] \to S[t]$ s.t. $\phi_{\ast} = \phi(a)$ for all $a\in R$ and $\phi_{\ast}(t) = t$
\end{dfn}

% TODO: Explanationfor this, and evaluation.

\begin{dfn}[Degree]{dfn:degree}{3.1.9}
    The \textbf{degree}, $\deg(f)$, of a nonzero polynomial $f(t) = \sum a_{i}t^{i}$ is the largest $n \ge 0$ s.t. $a_{n}\ne 0$. By convention, $\deg(0) = -\infty$, where $-\infty$ is a formal symbol which we give the properties
    \[-\infty < n, \quad (-\infty) + n = -\infty, \quad (-\infty) + (-\infty) = -\infty\]
    for all integers $n$
\end{dfn}

\begin{lma}[Degree and Integral Domains]{lma:degree-intdoms}{3.1.11}
    Let $R$ be an integral domain. Then:
    \begin{enumerate-tight}
        \item $\deg(fg) = \deg(f) + \deg(g)$ for all $f,\,g\in R[t]$
        \item $R[t]$ is an integral domain.
    \end{enumerate-tight}

    \longrule{0.08ex}
    The one and only polynomial of degree $-\infty$ is the zero polynomial. The polynomials of degree $0$ are the nonzero constants. The polynomials of degree $>0$ are therefore the nonconstant polynomials.
\end{lma}

\begin{lma}[]{lma:nonzero-constant-stuff}{3.1.14}
    Let $K$ be a field. Then
    \begin{enumerate-tight}
        \item The units in $K[t]$ are the nonzero constants
        \item $f\in K[t]$ is irreducible iff $f$ is nonconstant and cannot be expressed as a product of two nonconstant polynomials.
    \end{enumerate-tight}
\end{lma}

\begin{ppn}[Uniqueness of Poly Division]{ppn:poly-division-uniqueness}{3.2.1}
    Let $K$ be a field and $f,\,g\in K[t]$ with $g\ne 0$. Then there is exactly one pair of polynomials $q,\,r\in K[t]$ such that $f = qg + r$ and $\deg(r) < \deg(g)$
\end{ppn}

\begin{ppn}[Polynomial PIDs]{ppn:polynomial-pid}{3.2.2}
    Let $K$ be a field. Then $K[t]$ is a principal ideal domain.
\end{ppn}

\begin{crl}[Irreducibility and Fields]{crl:irreducible-fields}{3.2.5}
    Let $K$ be a field and let $0 \ne f \in K[t]$. Then
    \[f \text{ is irreducible} \iff K[t] / \langle f \rangle \text{ is a field.}\]
\end{crl}

\begin{lma}[Divisibility by Irreducibles]{lma:divisibility-by-irreducibles}{3.2.6}
    Let $K$ be a field and let $f(t) \in K[t]$ be a nonconstant polynomial. Then $f(t)$ is divisible by some irreducible in $K[t]$
\end{lma}

\begin{lma}[Divisibility of Products]{lma:divisibility-by-products}{3.2.7}
    Let $K$ be a field and $f,\,g,\,h\in K[t]$. Suppose that $f$ is irreducible and $f \mid gh$. Then $f \mid g$ or $f \mid h$
\end{lma}

\begin{thm}[Unique Determination of Polys]{thm:poly-unique-determination}{3.2.8}
    Let $K$ be a field and $0 \ne f \in K[t]$. Then
    \[f = af_{1}f_{2}\cdots f_{n}\]
    for some $n \ge 0$, $a\in K$, and monic irreducibles $f_{1},\dots,f_{n}\in K[t]$. Moreover, $n$ and $a$ are uniquely determined by $f$, and $f_{1},\dots,f_{n}$ are uniquely determind up to reordering.

    \longrule{0.08ex}
    \textbf{Monic} means that the leading coefficient is $1$
\end{thm}

\begin{lma}[Root Finding]{lma:root-finding}{3.2.9}
    One way to find an irreducible factor of a polynomial $f(t)\in K[t]$ is to find a \textbf{root}. Let $K$ be a field, $f(t) \in K[t]$, and $a\in K$. Then
    \[f(a) = 0 \iff (t - a) \mid f(t).\]

    \longrule{0.08ex}
    A field is \textbf{algebraically closed} if every nonconstant polynomial has at least one root.
\end{lma}

\begin{lma}[Algebraically Closed Field]{lma:algebraically-closed-field}{3.2.10}
    Let $K$ be an algebraically closed field and $0 \ne f \in K[t]$. then
    \[f(t) = c(t - a_{1})^{m_{1}} \cdots (t - a_{k})^{m_{k}},\]
    where $c$ is the leading coefficient of $f$, and $a_{1},\dots,a_{k}$ are the distinct roots of $f$ in $K$, and $m_{1},\dots,m_{k}\ge 1$
\end{lma}

\begin{lma}[Degrees and Irreducibility]{lma:degrees-irreducibility}{3.3.1}
    Let $K$ be a field and $f \in K[t]$.
    \begin{enumerate-tight}
        \item If $f$ is constant then $f$ is not irreducible.
        \item If $\deg(f) = 1$ then $f$ is irreducible.
        \item If $\deg(f) \ge 2$ and $f$ has a root then $f$ is reducible.
        \item If $\deg(f)\in \{2,3\}$ and $f$ has no root then $f$ is irreducible.
    \end{enumerate-tight}

    \textbf{Warning}: To show a polynomial is irreducible, it's generally \textit{not} enough to show it has no root. The converse of $3$ is false!
\end{lma}

\begin{dfn}[Primitive Polynomial]{dfn:primitive-polynomial}{3.3.6}
    A polynomial over $\mathbb{Z}$ is \textbf{primitive} if its coefficients have no common divisor except for $\pm 1$.
\end{dfn}

\newpage

\begin{lma}[Existence of Primitive Polynomials]{lma:primitive-existence}{3.3.7}
    Let $f(t)\in \mathbb{Q}[t]$. Then there exists a primitive polynomial $F(t)\in \mathbb{Z}[t]$ and $\alpha\in \mathbb{Q}$ such that $f = \alpha F$.
\end{lma}


\begin{rem}[Irreducibility over]{rem:irreducibility-over}{3.3.7A}
    If the coefficients of a polynomial $f(t)\in \mathbb{Q}[t]$ happen to all be integers, the word ``irreducible'' could mean two things: irreducibility in the ring $\mathbb{Q}[t]$ or in the ring $\mathbb{Z}[t]$. We say that $f$ is irreducible \textbf{over} $\mathbb{Q}$ or $\mathbb{Z}$ to distinguish between the two.
\end{rem}

\begin{lma}[Gauss' Lemma]{lma:gauss-lemma}{3.3.8}
    \begin{enumerate-tight}
        \item The product of two primitive polynomials over $\mathbb{Z}$ is primitive.
        \item If a nonconstant polynomial over $\mathbb{Z}$ is irreducible over $\mathbb{Z}$, it is irreducible over $\mathbb{Q}$
    \end{enumerate-tight}
\end{lma}

\begin{ppn}[Mod \texorpdfstring{$p$}{p} method]{ppn:mod-p-method}{3.3.9}
    Let $f(t) = a_{0} + a_{1}t + \cdots + a_{n}t^{n}\in \mathbb{Z}[t]$. If there is some prime $p$ such that $p \nmid a_{n}$ and $\overline{f} \in \mathbb{F}_{p}[t]$ is irreducible, then $f$ is irreducible over $\mathbb{Q}$.

    \textbf{Warning}: This only tells you that a polynomial is \textit{irreducible} over $\mathbb{Q}$ and says nothing about whether it is \textit{reducible}.
\end{ppn}

\begin{ppn}[Eisenstein's Criterion]{ppn:eisenstein}{3.3.12}
    Let $f(t) = a_{0}+\cdots + a_{n}t^{n}\in \mathbb{Z}[t]$, with $n \ge 1$. Suppose there exists a prime $p$ such that
    \begin{itemize-tight}
        \item $p\nmid a_{n}$
        \item $p \mid a_{i}$ for all $i\in \{0,\dots, n - 1\}$
        \item $p^{2} \nmid a_{0}$
    \end{itemize-tight}
    Then $f$ is irreducible over $\mathbb{Q}$.
\end{ppn}

% TODO: codegree

\begin{xmp}[Cyclotomic Polynomial]{xmp:cyclotoic}{3.3.16}
    Let $p$ be a prime. The \textbf{$p$th cyclotomic polynomial} is
    \[\Phi_{p}(t) = 1 + t + \cdots + t^{p-1} = \frac{t^{p} - 1}{t - 1}\]
    $\Phi_{p}$ is irreducible.
\end{xmp}

\section{Field Extensions}

\begin{rem}[Inclusion Funtion]{rem:inclusion-function}{4.1.A}
    Given a set $A$ and a subset $B \subseteq A$, there is an \textbf{inclusion} function $\iota : B \to A$ defined by $\iota(b) = b$ for all $b\in B$.

    On the other hand, given any injective funtion between sets, say $\phi : X \to A$, the image $\im A$ is a subset of $A$, and there is a bijection $\phi' : X \to \im \phi$ given by $\phi'(x) = \phi(x)$ $(x\in X)$. Hence the set $X$ is isomorphic to (in bijection with) the subset $\im \phi$ of $A$.

    So given any subset of $A$, we get an injection into $A$, and vice versa. These two back-and-forth processes are mutually inverse (up to iso), so subsets and injections are more or less the same thing. (wtf?)
\end{rem}

% TODO: what does this mean

\begin{dfn}[Field Extension]{dfn:field-extension}{4.1.1}
    Let $K$ be a field. An \textbf{extension} of $K$ is a field $M$ together with a homomorphism $\iota : K \to M$.

    We can write $M : K$ to mean that $M$ is an extension of $K$, not bothering to mention $\iota$.
\end{dfn}

% TODO: examples

\begin{dfn}[Generated Subfield]{dfn:generated-subfield}{4.1.4}
    Let $K$ be a field and $X$ a subset of $K$. The subfield of $K$ \textbf{generated by} $X$ is the intersection of all the subfields of $K$ containing $X$.

    Let $F$ be the subfield of $K$ generated by $X$. $F$ contains $X$, and $F$ is also the \textit{smallest} subfield of $K$ containing $X$ (in the sense that any subfield of $K$ containing $X$ contains $F$)
\end{dfn}

% TODO: examples

\begin{dfn}[Adjoined Subfields]{dfn:adjointed-subfield}{4.1.8}
    Let $M : K$ be a field extension and $Y \subseteq M$. We write $K(Y)$ for the subfield of $M$ generated by $K \cup Y$. We call it $K$ with $Y$ \textbf{adjoined}, or the subfield of $M$ \textbf{generated by $Y$ over $K$}

    \longrule{0.08ex}
    So, $K(Y)$ is the smallest subfield of $M$ containing both $K$ and $Y$. When $Y$ is a finite set $\{\alpha_{1},\dots,\alpha_{n}\}$, we write $K(\{\alpha_{1},\dots,\alpha_{n}\})$ as $K(\alpha_{1},\dots,\alpha_{n})$
\end{dfn}

% TODO: examples

% TODO: warning 4.1.10

\begin{rem}[Algebraic Number]{rem:algebraic-number}{4.2.A}
    A complex number $\alpha$ is said to be ``algebraic'' if
    \[a_{0} + a_{1}\alpha + \cdots + a_{n}a^{n} = 0\]
    for some rational numbers $a_{i}$, not all zero. This concept generalises to arbitrary field extensions:
\end{rem}

\begin{dfn}[Algebraic Numbers for Extensions]{dfn:algebraic-extension}{4.2.1}
    Let $M : K$ be a field extension and $\alpha\in M$. Then $\alpha$ is \textbf{algebraic} over $K$ if there exists $f \in K[t]$ s.t. $f(\alpha) = 0$ but $f \ne 0$, and \textbf{transcendental} otherwise.

    \longrule{0.08ex}
    Let $M : K$ be a field extension and $\alpha\in M$. An \textbf{annihilating polynomial} of $\alpha$ is a polynomial $f \in K[t]$ such that $f(\alpha)=0$. So, $\alpha$ is algebraic iff it has some nonzero annihilating polynomial.
\end{dfn}

% TODO: transcendental numbers examples

\begin{lma}[Annihilaters]{lma:annihilator}{4.2.6}
    Let $M : K$ be a field extension and $\alpha\in M$. Then there is a polynomial $m(t) \in K[t]$ such that
    \begin{equation}\label{eq:annihilator}\langle m \rangle = \{\text{annihilating polynomials of $\alpha$ over $K$}\}.\end{equation}
    If $\alpha$ is transcendental over $K$ then $m = 0$. If $\alpha$ is algebraic over $K$ then there is a unique monic polynomial $m$ satisfying \eqref{eq:annihilator}.
\end{lma}

\begin{dfn}[Minimal Polynomial]{dfn:minimal-polynomial}{4.2.7}
    Let $M : K$ be a field extension and let $\alpha\in M$ be algebraic over $K$. The \textbf{minimal polynomial} of $\alpha$ is the unique monic polynomial satisfying \eqref{eq:annihilator}.

    \textbf{Warning}: We do not define the minimal polynomial for a transcendental element. Therefore, some elements of $M$ may have no minimal polynomial
\end{dfn}

\begin{lma}[Minimal Polynomial Conditions]{lma:minimal-polynomial-conds}{4.2.10}
    Let $M : K$ be a field extension, let $\alpha \in M$ be algebraic over $K$ and let $m\in K[t]$ be a monic polynomial. The following are equivalent:
    \begin{enumerate-tight}
        \item $m$ is the minimal polynomial of $\alpha$ over $K$
        \item $m(\alpha) = 0$ and $m \mid f$ for all annihilating polynomials $f$ of $\alpha$ over $K$
        \item $m(\alpha) = 0$ and $\deg(m) \le \deg (f)$ for all nonzero annihilating polynomials.
        \item $m(\alpha) = 0$ and $m$ is irreducible over $K$.
    \end{enumerate-tight}
    Part $3$ says the minimal polynomial is a monic annihilating polynomial of least degree.
\end{lma}

%TODO: examples, these are probably important

\begin{dfn}[]{dfn:minimal-polynomial-element}{4.3.1}
    Let $K$ be a field.
    \begin{enumerate-tight}
        \item Let $m\in K[t]$ be monic and irreducible. Write $\alpha\in K[t] /\langle m \rangle$ for the imge of $t$ under the canonical homomorphism $K[t] \to K[t] / \langle m \rangle$. Then $\alpha$ has minimal polynomial $m$ over $K$, and $K[t] /\langle m \rangle$ is generated by $\alpha$ over $K$.
        \item The element $t$ of the field $K(t)$ of rational expressions over $K$ is transcendental over $K$, and $K(t)$ is generated by $t$ over $K$
    \end{enumerate-tight}
    In part $1$, we are viewing $K[t] / \langle m \rangle$ as an extension of $K$.
\end{dfn}

\newpage
\begin{dfn}[Homomorphism over Fields]{dfn:homomorphism-over-field}{4.3.3}
    Let $K$ be a field, and let $\iota : K \to M$, and $\iota' : K \to M'$ be extensions of $K$. A homomorphism $\phi : M \to M'$ is said to be a \textbf{homomorphism over $K$} if
% https://q.uiver.app/#q=WzAsMyxbMCwwLCJNIl0sWzIsMCwiTSwiXSxbMSwxLCJLIl0sWzAsMSwiXFxwaGkiXSxbMiwwLCJcXGlvdGEiXSxbMiwxLCJcXGlvdGEnIiwyXV0=
\[\begin{tikzcd}[cramped]
	M && {M,} \\
	& K
	\arrow["\phi", from=1-1, to=1-3]
	\arrow["\iota", from=2-2, to=1-1]
	\arrow["{\iota'}"', from=2-2, to=1-3]
\end{tikzcd}\]
commutes.
\end{dfn}

\begin{lma}[Uniqueness of Field Homomorphisms]{lma:homomorphisms-over-unique}{4.3.6}
    Let $M$ and $M'$ be extensions of a field $K$, and let $\phi, \psi : M \to M'$ be homomorphisms over $K$. Let $Y$ be a subset of $M$ such that $M = K(Y)$. If $\phi(\alpha) = \psi(\alpha)$ for all $\alpha\in Y$ then $\phi = \psi$.
\end{lma}

\begin{ppn}[Universal Props of \texorpdfstring{$K[t] /\langle m \rangle$}{K[t]/<m>}, \texorpdfstring{$K(t)$}{K(t)}]{ppn:universal-prop-simples}{4.3.7}
    Let $K$ be a field
    \begin{enumerate-tight}
        \item Let $m\in K[t]$ be monic and irreducible, let $L : K$ be an extension of $K$, and let $\beta\in L$ with minimal polynomial $m$. Write $\alpha$ for the image of $t$ under the canonical homomorphism $K[t] \to K[t] / \langle m \rangle$. Then there is exactly one homomorphism $\phi : K[t] / \langle m \rangle \to L$ over $K$ such that $\phi(a) = \beta$
        \item Let $L : K$ be an extension of $K$, and let $\beta\in L$ be transcendental. Then there is exactly one homomorphism $\phi : K(t) \to L$ over $K$ such that $\phi(t) = \beta$.
    \end{enumerate-tight}
    \begin{figure}[H]
        \centering
% https://q.uiver.app/#q=WzAsNSxbMCwyLCJLW3RdL1xcbGFuZ2xlIG0gXFxyYW5nbGUiXSxbMywxLCJMIl0sWzIsMywiSyJdLFswLDEsIlxcYWxwaGEiXSxbMywwLCJcXGJldGEiXSxbMCwxLCJcXHBoaSIsMCx7InN0eWxlIjp7ImJvZHkiOnsibmFtZSI6ImRvdHRlZCJ9fX1dLFsyLDBdLFsyLDFdLFszLDQsIiIsMix7InN0eWxlIjp7InRhaWwiOnsibmFtZSI6Im1hcHMgdG8ifSwiYm9keSI6eyJuYW1lIjoiZG90dGVkIn19fV1d
\[\begin{tikzcd}[cramped]
	&&& \beta \\
	\alpha &&& L \\
	{K[t]/\langle m \rangle} \\
	&& K
	\arrow[dotted, maps to, from=2-1, to=1-4]
	\arrow["\phi", dotted, from=3-1, to=2-4]
	\arrow[from=4-3, to=2-4]
	\arrow[from=4-3, to=3-1]
\end{tikzcd}\]
        \caption{Diagram for 1}
        \label{fig:simple-extension-uniprop}
    \end{figure}
\end{ppn}

% TODO: example - LOOK AT THIS!!!

\begin{rem}[Isomorphism Over a Field]{rem:isomorphism-over-field}{4.3.A}
    Let $M$ and $M'$ be extensions of a field $K$. A homomorphism $\phi : M \to M'$ is an \textbf{isomorphism over $K$} if it is a homomorphism over $K$ and an isomorphism of fields. If such a $\phi$ exists, we say that $M$ and $M'$ are \textbf{isomorphic over $K$}.
\end{rem}

\begin{crl}[Uniqueness of Isomorphisms]{crl:isomorphism-uniqueness}{4.3.11}
    Let $K$ be a field.
    \begin{enumerate-tight}
        \item Let $m\in K[t]$ be monic and irreducible, let $L : K$ be an extension of $K$, and let $\beta\in L$ with minimal polynomical $m$ and with $L = K(\beta)$. Write $\alpha$ for the image of $t$ under the canonical homomorphism $K[t] \to K[t] / \langle m \rangle$. then there is exactly one isomorphism $\phi : K[t] / \langle m \rangle \to L$ over $K $ such that $\phi(\alpha) = \beta$.
        \item Let $L : K$ be an extension of $K$, and let $\beta \in L$ be transcendental with $L = K(\beta)$. Then there is exactly one isomorphism $\phi : K(t) \to L$ over $K$ such that $\phi(t) = \beta$.
    \end{enumerate-tight}
\end{crl}

\begin{dfn}[Simple Extension]{dfn:simple-extension}{4.3.13}
    A field extension $M : K$ is \textbf{simple} if there exists $\alpha \in M$ such that $M = K(\alpha)$.
\end{dfn}

\begin{thm}[Classification of Simple Extensions]{thm:simple-extension-classification}{4.3.16}
    Let $K$ be a field
    \begin{enumerate-tight}
        \item Let $m\in K[t]$ be a monic irreducible polynomial. Then there exists an extension $M : K$ and an algebraic element $\alpha\in M$ such that $M = K(\alpha)$ and $\alpha$ has minimal polynomial $m$ over $K$. 

            Moreover, if $(M, \alpha)$ and $(M',\alpha')$ are two such pairs, there is exactly one isomorphism $\phi : M \to M'$ over $K$ such that $\phi(\alpha) = \alpha'$
        \item There exists an extension $M : K$ and a transcendental element $\alpha \in M$ such that $M = K(\alpha)$.

            Moreover, if $(M, \alpha)$ and $(M',\alpha')$ are two such pairs, there is exactly one isomorphism $\phi : M \to M'$ over $K$ such that $\phi(\alpha) = \alpha'$.
    \end{enumerate-tight}
\end{thm}

\begin{rem}[Field Extension Explanation]{rem:field-extension-explanation}{4.3.C}
    Given any field $K$ and any monic irreducible $m(t) \in K[t]$, we can say the words ``adjoin to $K$ a root $\alpha$ of $m$'', and this unambiguously defines an extension $K(\alpha) : K$. Similarly, we can unambiguously adjoin to $K$ a transcendental element.
\end{rem}

\begin{rem}[Field Extensions as Vector Spaces]{rem:field-extensions-as-vector-spaces}{5.1.A}
    Let $M : K$ be a field extension. Then $M$ is a vector space over $K$ in a natural way. Addition and subtraction in the vector space $M$ are the same as in the field $M$. Scalar multiplication in the vector space is just multiplication of elements of $M$ by elements of $K$, which makes sen because $K$ is embedded as a subfield of $M$.

    \longrule{0.08ex}
    When we view $M$ as a vector space over $K$ rather than an extension, we forget how to multiply together elements of $M$ that aren't in $K$.
\end{rem}

\begin{dfn}[Degree of a Field Extension]{dfn:field-extension-degree}{5.1.1}
    The \textbf{degree $[M : K]$} of a field extension $M : K$ is the dimension of $M$ as a vector space over $K$.

    If $M$ is an \textit{infinite-dimensional} vector space over $K$, we write $[M : K] = \infty$, where $\infty$ is a formal symbol which we give the properties
    \[n < \infty, \quad n \cdot \infty = \infty\:(n \ge 1), \quad \infty \cdot \infty = \infty\]
    for integers $n$. An extension $M : K$ is \textbf{finite} if $[M : K] < \infty$.
\end{dfn}

\begin{rem}[Degree over itself]{rem:degree-over-itself}{5.1.4}
    The degree $[K : K]$ of $K$ over itself is $1$, not $0$. Degrees of extensions are never $0$.
\end{rem}

\begin{thm}[Basis of Field Extensions]{thm:field-extension-basis}{5.1.5}
    Let $K(\alpha) : K$ be a simple extension.
    \begin{enumerate-zero}
        \item Suppose that $\alpha$ is algebraic over $K$. Write $m\in K[t]$ for the minimal polynomial of $\alpha$ and $n = \deg(m)$. Then
            \[1,\alpha,\dots,\alpha^{n-1}\]
            is a basis of $K(\alpha)$ over $K$. In particular, $[K(\alpha) : K] = \deg(m)$
        \item Suppose that $\alpha$ is transcendental over $K$. Then $1,\alpha,\alpha^{2},\dots$ are linearly independent over $K$. In particular, $[K(\alpha) : K] = \infty$
    \end{enumerate-zero}
\end{thm}

%TODO: Example

\begin{crl}[Degree and Alegebraicness]{crl:degree-algebraic}{5.1.10}
    Let $M : K$ be a field extension and $\alpha\in M$, the \textbf{degree} of $\alpha$ over $K$ is $[K(\alpha) : K]$. We write it as $\deg_{K}(\alpha)$. Then
    \[\deg_{K}(\alpha) < \infty \iff \alpha \text{ is algebraic over $K$.}\]
    If $\alpha$ is algebraic over $K$ then the degree of $\alpha$ over $K$ is the degree of the minimal polynomial of $\alpha$ over $K$.
\end{crl}

% TODO: visualization of 5.1.12

\begin{crl}[Size of Nested Extensions]{crl:nested-extension-size}{5.1.12}
    Let $M : L : K$ be a field extension and $\beta \in M$. Then
    \[
        [L(\beta) : L] \le [K(\beta) : K]
    \]
\end{crl}

\begin{crl}[Polynomial Form for Extensions]{crl:extension-poly-form}{5.1.14}
    Let $M : K$ be a field extension. Let $\alpha_{1},\dots, \alpha_{n}\in M$, when $\alpha_{i}$ algebraic over $K$ of degree $d_{i}$. Then every element $\alpha\in K(\alpha_{1},\dots,\alpha_{n})$ can be expressed as a polynomial in $\alpha_{1},\dots,\alpha_{n}$ over $K$. More exactly,
    \[\alpha = \sum_{r_{1},\dots,r_{n}} c_{r_{1},\dots,r_{n}}a^{r_{1}}_{1} \cdots a^{r_{n}}_{n}\]
    for some $c_{r_{1},\dots,r_{n}}\in K$, where $r_{i}$ ranges over $0,\dots,d_{i}-1$.
\end{crl}

\newpage
\begin{thm}[Tower Law]{thm:tower-law}{5.1.17}
    Let $M : L : K$ be field extensions.
    \begin{enumerate-tight}
        \item If $(\alpha_{i})_{i\in I}$ is a basis of $L$ over $K$ and $(\beta_{j})_{j\in J}$ is a basis of $M$ over $L$, then $(\alpha_{i}\beta_{j})_{(i,j)\in I \times J}$ is a basis of $M$ over $K$.
        \item $M : K$ is finite $\iff$ $M : L$ and $L : K$ are finite.
        \item $[M : K] = [M : L][L : K]$
    \end{enumerate-tight}
    The sets $I$ and $J$ here could be infinite. A family $(\alpha_{i})_{i\in I}$ of elements of a field is \textbf{finitely supported} if the set $\{i\in I \mid \alpha_{i} \ne 0\}$ is finite.
\end{thm}

% examples

\begin{crl}[Dividing Extensions]{crl:dividing-extensions}{5.1.19}
    Let $M : L' : L : K$ be field extensions. If $M : K$ is finite, then $[L' : L]$ divides $[M : K]$
\end{crl}

\begin{crl}[Triangle Tower Inequality]{crl:triangle-tower}{5.1.21}
    Let $M : K$ be a field extension and $\alpha_{1},\dots,\alpha_{n}\in M$. Then
    \[[K(\alpha_{1},\dots,\alpha_{n}) : K] \le [K(\alpha_{1}) : K] \cdots [K(\alpha_{n}) : K].\]
\end{crl}

% TODO: the diagram thing

\begin{dfn}[Finitely Generated Extensions]{dfn:finitely-generated}{5.2.1}
    A field extension $M : K$ is \textbf{finitely generated} if $M = K(Y)$ for some finite subset $Y \subseteq M$.
\end{dfn}

\begin{dfn}[Algebraic Extensions]{dfn:algebraic-extensions}{5.2.2}
    A field extension $M : K$ is \textbf{algebraic} if every element of $M$ is algebraic over $K$.
\end{dfn}

\begin{ppn}[Algebraic and Finiteness]{ppn:algebraic-finite-extension}{5.2.4}
    The following conditions on a field extension $M : K$ are equivalent:
    \begin{enumerate-tight}
        \item $M : K$ is finite
        \item $M : K$ is finitely generated and algebraic
        \item $M = K(\alpha_{1},\dots,\alpha_{n})$ for some finite set $\{\alpha_{1},\dots,\alpha_{n}\}$ of elements of $M$ algebraic over $K$.
    \end{enumerate-tight}
\end{ppn}

\begin{crl}[Algebraic and Finiteness (SEs)]{crl:algebraic-finite-simple}{5.2.6}
    Let $K(\alpha) : K$ be a simple extension. The following are equivalent:
    \begin{enumerate-tight}
        \item $K(\alpha) : K$ is finite
        \item $K(\alpha) : K$ is algebraic
        \item $\alpha$ is algebraic over $K$.
    \end{enumerate-tight}

\end{crl}

\begin{ppn}[]{ppn:qbar-subfield-c}{5.2.7}
    $\overline{\mathbb{Q}}$ is a subfield of $\mathbb{C}$.
\end{ppn}

\begin{rem}[Iterated Quadratic]{rem:iterated-quadratic}{5.3.A}
    For a subfield $K \subseteq \mathbb{R}$, an extension $K : \mathbb{Q}$ is \textbf{iterated quadratic} if there is some finite sequence of subfields
    \[\mathbb{Q} = K_{0} \subseteq K_{1} \subseteq \cdots \subseteq K_{n} = K\]
    such that $[K_{i} : K_{i-1}] = 2$ for all $i\in \{1,\dots,n\}$
\end{rem}

\begin{dfn}[Compositum]{dfn:compositum}{5.3.3}
    Let $L$ and $L'$ be subfields of a field $M$. The \textbf{compositum $LL'$} of $L$ and $L'$ is the subfield of $M$ generated by $L \cup L'$

    That is, $LL'$ is the smallest subfield of $M$ containing both $L$ and $L'$.
\end{dfn}

\begin{lma}[]{lma:compositum-two}{5.3.6}
    Let $M : K$ be a field extension and let $L,\, L'$ be subfields of $M$ containing $K$. If $[L : K] = 2$ then $[LL':L'] \in \{1,2\}$.
    %TODO: generality of this
\end{lma}

\begin{lma}[]{lma:encompassing-iterated-quad}{5.3.8}
    Let $K$ and $L$ be subfields of $\mathbb{R}$ such that the extensions $K : \mathbb{Q}$ and $L : \mathbb{Q}$ are iterated quadratic. Then there is some subfield $M$ of $\mathbb{R}$ such that the extension $M : \mathbb{Q}$ is iterated quadratic and $K,\, L \subseteq M$.
\end{lma}

\begin{ppn}[Iteratic Quadratics from Points]{ppn:iterated-quadratic-point}{5.3.9}
    Let $(x,y)\in \mathbb{R}^{2}$. If $(x,y)$ is constructable from $\{(0,0),\, (1,0)\}$ then there is an iterated quadratic extension of $\mathbb{Q}$ containing $x$ and $y$.
\end{ppn}

\begin{thm}[Quadratics and Constructability]{thm:}{5.3.10}
    Let $(x,y)\in \mathbb{R}^{2}$. If $(x,y)$ is constructible from $\{(0,0),\, (1,0)\}$ then $x$ and $y$ are algebraic over $\mathbb{Q}$, and their degrees over $\mathbb{Q}$ are powers of $2$.
\end{thm}

\begin{dfn}[Extending Homomorphism]{dfn:extending-homomorphism}{6.1.1}
    Let $\iota : K \to M$ and $\iota : K' \to M'$ be field extensions. Let $\psi : K \to K'$ be a homomorphism of fields. A homomorphism $\phi : M \to M'$ \textbf{extends} $\psi$ if the square
% https://q.uiver.app/#q=WzAsNCxbMCwwLCJNIl0sWzIsMCwiTSciXSxbMCwyLCJLJyJdLFsyLDIsIksiXSxbMiwzLCJcXHBzaSIsMl0sWzIsMCwiXFxpb3RhIl0sWzAsMSwiXFxwaGkiXSxbMywxLCJcXGlvdGEnIiwyXV0=
\[\begin{tikzcd}[cramped, column sep=scriptsize]
	M && {M'} \\
	\\
	{K'} && K
	\arrow["\phi", from=1-1, to=1-3]
	\arrow["\iota", from=3-1, to=1-1]
	\arrow["\psi"', from=3-1, to=3-3]
	\arrow["{\iota'}"', from=3-3, to=1-3]
\end{tikzcd}\]
commutes $(\phi \circ \iota = \iota' \circ \psi)$. Most of the time we view $K$ as a subset of $M$, and $K'$ as a subset of $M'$, with $\iota$ and $\iota'$ be the inclusions. In this case, for $\phi$ to extend $\psi$ just means that
\[\pi(a) = \psi(a) \text{ for all } a\in K\]
\end{dfn}

\begin{lma}[Induced Homomorphism as sum]{lma:induced-homomorphism-sum}{6.1.3}
    Let $M : K$ and $M' : K'$ be field extensions, let $\phi : K \to K'$ be a homomorphism, and let $\phi : M \to M'$ be a homomorphism extending $\psi$. Let $\alpha\in M$ and $f(t) \in K[t]$. Then
    \[f(\alpha) = 0 \iff (\psi_{\ast} f)(\phi(\alpha)) = 0.\]
\end{lma}

\begin{ppn}[Unique Extending Isomorphisms]{ppn:extending-isomorphisms}{6.1.6}
    Let $\psi : K \to K'$ be an isomorphism of fields. Let $K(\alpha) : K$ be a simple extension where $\alpha$ has minimal polynomial $m$ over $K$, and let $K'(\alpha') : K'$ be a simple extension where $\alpha'$ has minimal polynomial $\psi_{\ast}m$ over $K'$. Then there is exactly one isomorphism $\phi : K(\alpha) \to K'(\alpha')$ that extends $\psi$ and satisfies $\phi(\alpha) = \alpha'$.
% https://q.uiver.app/#q=WzAsNCxbMCwwLCJLKFxcYWxwaGEpIl0sWzIsMCwiSycoXFxhbHBoYScpIl0sWzAsMSwiSyciXSxbMiwxLCJLJyJdLFsyLDMsIlxccHNpLCBcXGNvbmcnIiwyXSxbMiwwXSxbMCwxLCJcXHBoaSwgXFxjb25nJyIsMCx7InN0eWxlIjp7ImJvZHkiOnsibmFtZSI6ImRvdHRlZCJ9fX1dLFszLDFdXQ==
\[\begin{tikzcd}[cramped]
	{K(\alpha)} && {K'(\alpha')} \\
	{K'} && {K'}
    \arrow["{\phi}", "\cong"', dotted, from=1-1, to=1-3]
	\arrow[from=2-1, to=1-1]
	\arrow["\psi"', "\cong", from=2-1, to=2-3]
	\arrow[from=2-3, to=1-3]
\end{tikzcd}\]
A dotted arrow is used to denote a map whose existence is part of the conclusion of a theorem.
\end{ppn}

\begin{dfn}[Splitting Polynomial]{dfn:splitting-polynomial}{6.2.2}
    Let $f$ be a polynomial over a field $M$. Then $f$ \textbf{splits} in $M$ if
    \[f(t) = \beta(t - \alpha_{1}) \cdots (t - a_{n})\]
    for some $n \ne 0$ and $\beta, \alpha_{1},\dots,\alpha_{n}\in M$.

    Equivalently, $f$ splits in $M$ if all its irreducible factors in $M[t]$ are linear.
\end{dfn}

% Example 6.2.3


\newpage
\begin{dfn}[Splitting Field]{dfn:splitting-field}{6.2.6}
    Let $f$ be a nonzero polynomial over a field $K$. A \textbf{splitting field} of $f$ over $K$ is an extension $M$ of $K$ such that:
    \begin{enumerate-tight}
        \item $f$ splits in $M$
        \item $M = K(\alpha_{1},\dots,\alpha_{n})$, where $\alpha_{1},\dots,\alpha_{n}$ are the roots of $f$ in $M$.
    \end{enumerate-tight}

    $2$ can be replaced by ``If $L$ is a subfield of $M$ containing $K$, and $f$ splits in $L$, then $L = M$''
\end{dfn}

\begin{lma}[Size Limits of Splitting Fields]{lma:splitting-field-deg}{6.2.10}
    Let $f\ne 0$ be a polynomial over a field $K$. Then there exists a splitting field $M$ of $f$ over $K$ such that $[M : K] \le \deg(f)!$
\end{lma}

\begin{ppn}[Splitting Fields and Isomorphisms]{ppn:splitting-field-iso}{6.2.11}
    Let $\psi : K \to K'$ be an isomorphism of fields, let $0 \ne f \in K[t]$, let $M$ be a splitting field of $f$ over $K$, and let $M'$ be a splitting field of $\psi_{\ast}f$ over $K'$. Then
    \begin{enumerate-tight}
        \item There exists an isomorphism $\phi : M \to M'$ extending $\psi$.
        \item There are at most $[M : K]$ such extensions $\phi$.
    \end{enumerate-tight}
    We often use this result when $K' = K$ and $\psi = \id_{K}$.
\end{ppn}

\begin{thm}[Isos and Autos of a Splitting Field]{thm:isos-autos-splitting-field}{6.2.13}
    Let $f$ be a nonzero polynomial over a field $K$. Then
    \begin{enumerate-zero}
        \item There exists a splitting field of $f$ over $K$
        \item Any two splitting fields of $f$ are isomorphic over $K$
        \item When $M$ is a splitting field of $f$ over $K$,
            \[\text{number of automorphisms of $M$ over $K$}\le [M : K] \le \deg(f)\]
    \end{enumerate-zero}
\end{thm}

\begin{lma}[]{lma:something-splitting}{6.2.14}
    \begin{enumerate-tight}
        \item Let $M : S : K$ be field extensions, $0 \ne f \in K[t]$, and $Y \subseteq M$. Suppose that $S$ is the splitting field of $f$ over $K$. Then $S(Y)$ is the splitting field of $f$ over $K(Y)$
        \item Let $f\ne 0$ be a polynomial over a field $K$, and let $L$ be a subfield of $\mathrm{SF}_{K}(f)$ containing $K$ (so that $\mathrm{SF}_{K}(f) : L : K$). Then $\mathrm{SF}_{K}(f)$ is the splitting field of $f$ over $L$.
    \end{enumerate-tight}
\end{lma}

\begin{dfn}[Galois Group of an Extension]{dfn:galois-group-extension}{6.3.1}
    The \textbf{Galois Group} $\Gal(M : K)$ of a field extension $M : K$ is the group of automorphisms of $M$ over $K$, with composition as the group operation.

    In other words, an element of $\Gal(M : K)$ is an isomorphism $\theta : M \to M$ such that $\theta(a) = a$ for all $a\in K$.
\end{dfn}

\begin{dfn}[Galois Group of a Polynomial]{dfn:galois-group-polynomial}{6.3.5}
    Let $f$ be a nonzero polynomial over a field $K$. The \textbf{Galois Group} $\Gal_{K}(f)$ of $f$ over $K$ is $\Gal(\SF_{K}(f) : K)$

    So the definitions fit together like this:
    \[\text{polynomial} \quad \longmapsto\quad \text{field extension}\quad\longmapsto\quad\text{group}\]
\end{dfn}

\begin{rem}[Degree Size of Galois Group]{rem:degree-size-galois}{6.3.A}
    Via Theoerem $6.2.13$,
    \[\lvert \Gal_{K}(f) \rvert \le [\SF_{K}(f) :0K] \le \deg(f)!\]
    In particular, $\Gal_{K}(f)$ is always a finite group.
\end{rem}

\begin{lma}[Restriction of Actions on GGs]{lma:action-restriction}{6.3.7}
    Let $f$ be a nonzero polynomial over a field $K$. Then the action of $\Gal_{K}(f)$ on $\SF_{K}(f)$ restricts to an action on the st of roots of $f$ in $\SF_{K}(f)$.

    \longrule{0.08ex}
    \textbf{Terminology}: Given a group $G$ acting on a set $X$ and a subset $A \subseteq X$, the action \textbf{restricts} to $A$ if $ga \in A$ for all $g\in G$ and $a\in A$.
\end{lma}

\begin{lma}[Faithful Action of Galois Groups]{lma:faithful-actions}{6.3.8}
    Let $f$ be a nonzero polynomial over a field $K$. Then the action of $\Gal_{K}(f)$ on the roots of $f$ is \textbf{faithful}.
\end{lma}

\begin{rem}[What Galois Group Means]{rem:what-galois-group-means}{6.3.B}
    An element of the Galois group of $f$ is completely determined by how it permutes the roots of $f$. So you can view elements of the Galois group as \textit{being} permutations of the roots.

    However, not every permutation of the roots belongs to the Galois group. Suppose $f\in K[t]$ has distinct roots $\alpha_{1},\dots,\alpha_{k}$ in its splitting field. For each $\Theta\in \Gal_{K}(f)$ there is a permutation $\sigma_{\theta}\in S_{k}$ defined by
    \[\theta(\alpha_{i}) = \alpha_{\sigma_{\theta}(i)} \quad\text{ for } i\in \{1,\dots,k\}\]
    Then $\Gal_{K}(f)$ is isomorphic to the subgroup $\{\sigma_{\theta} \mid \theta \in \Gal_{K}(f)\}$ of $S_{K}$. The isomorphism is given by $\theta \mapsto \sigma_{\theta}$.
\end{rem}

\begin{dfn}[Conjugacy]{dfn:conjugacy}{6.3.9}
    Let $M : K$ be a field extension, let $k \ge 0$, and let $(\alpha_{1},\dots,\alpha_{k})$ and $(\alpha_{1}',\dots,\alpha_{k}')$ be $k$-tuples of elements of $M$. Then $(\alpha_{1},\dots,\alpha_{k})$ and $(\alpha_{1}',\dots,\alpha_{k}')$ are \textbf{conjugate} over $K$ if for all $p\in K[t_{1},\dots,t_{k}]$,
    \[p(\alpha_{1},\dots,\alpha_{k})= 0 \iff p(\alpha_{1}',\dots,\alpha_{k}') = 0\]
    If $k=1$ we omit the brackets and say $\alpha$ and $\alpha'$ are conjugate.
\end{dfn}

\begin{ppn}[Permutation Definition of GG]{ppn:permutation-galois-group}{6.3.10}
    Let $f$ be a nonzero polynomial over a field $K$ with distinct roots $\alpha_{1},\dots,\alpha_{k}$ in $\SF_{k}(f)$. Then
    \[\{\sigma\in S_{k} \mid (\alpha_{1},\dots,\alpha_{k})\text{ and }(\alpha_{\sigma(1)},\dots,\alpha_{\sigma(k)}) \text{ are conj. over $K$}\}\]
    is a subgroup of $S_{k}$ isomorphic to $\Gal_{K}(f)$
\end{ppn}

\begin{crl}[Galois Groups and Extensions]{crl:gal-group-extension}{6.3.12}
    Let $L : K$ be a field extension and $0 \ne f \in K[t]$. Then $\Gal_{L}(f)$ is isomorphic to a subgroup of $\Gal_{K}(f)$.
\end{crl}

\begin{crl}[Division of Roots and GGs]{crl:gal-group-division-roots}{6.3.14}
    Let $f$ be a nonzero polynomial over a field $K$, with $k$ distinct roots in $\SF_{K}(f)$. Then $\lvert \Gal_{K}(f) \rvert$ divides $k!$.
\end{crl}

% TODO: actually read page 97 lol



\lipsum[1-6]
\end{multicols}
\end{document}
