\documentclass[landscape, 8pt]{extarticle}

\usepackage{../../preamble}
\usepackage{symbols}

\setlength{\parskip}{3pt}
\usepackage[fontsize=6.5pt]{scrextend}

\begin{document}

% No idea what this thing actually does lmao but it compacts stuff slightly
\setlength{\abovedisplayskip}{3.5pt}
\setlength{\belowdisplayskip}{3.5pt}
\setlength{\abovedisplayshortskip}{3.5pt}
\setlength{\belowdisplayshortskip}{3.5pt}

\begin{multicols}{3}
\raggedcolumns

\section*{\huge Galois Theory Notes}
Made by Leon :) \textit{Note: Any reference numbers are to the lecture notes}

\vspace{-7pt}
\section{Galois Groups}

\begin{dfn}[Conjugate Numbers]{dfn:conjugate-numbers}{1.1.1}
    \vspace{-5pt}
    Two complex numbers $z$ and $z'$ are \textbf{conjugate over $\mathbb{Q}$} \textit{(exact same def. for $\mathbb{R}$ but we usually use $\mathbb{Q}$)} iff either $z = z'$ or $\overline{z} = z'$. Alternatively, if for all polynomials $p$ with coefficients in $\mathbb{Q}$,
    \[p(z) = 0 \iff p(z') = 0\]
    $(z_{1},\dots,z_{k})$, and $(z_{1}',\dots,z_{k}')$ $k$-tuples in $\mathbb{C}$ are \textbf{conjugate over $\mathbb{Q}$} if for all polynomials $p(t_{1},\dots,z_{k})$ over $\mathbb{Q}$ in $k$ variables,
    \[p(z_{1},\dots,z_{k}) = 0 \iff p(z_{1}',\dots,z_{k}') = 0\]
    \vspace{-15pt}
    \longrule{0.08ex}
    Additionally, if $(z_{1},\dots,z_{n})$ conjugate to $(z_{1}',\dots,z_{n}')$, then $z_{i}$ is conjugate to $z_{i}'$ for all $i$
\end{dfn}

% TODO: idk what's happening here

% \begin{dfn}[Galois (Overview)]{dfn:galois-group}{1.2.1}
%     Let $f$ be a polynomial with coefficients in $\mathbb{Q}$. Write $\alpha_{1},\dots,\alpha_{k}$ for its distinct roots in $\mathbb{C}$. The \textbf{Galois group} of $f$ is
%     \[\mathrm{Gal}(g) = \{\sigma \in S_{n} \mid (\alpha_{1},\dots,\alpha_{n}) \text{ conjugate to } (\alpha_{S(1)},\dots,\alpha_{\sigma(n)})\}\]
%     \textbf{Note}: distinct roots mean that we ignore any repetition of roots.
%
%     \textrule{Solvability}
%     A complex number is \textbf{radical} if it can be obtained from the rationals using only the usual arithmetic operations and $k$th roots. A polynomial over $\mathbb{Q}$ is \textbf{solvable (or soluble) by radicals} if all of its complex roots are radical.
%
%     \textrule{The Fundamental Theorem of Galois Theory}
%     Let $f$ be a polynomial over $\mathbb{Q}$. Then
%     \[f \text{ is solvable by radicals} \iff \mathrm{Gal}(f) \text{ is a solvable group.}\]
% \end{dfn}

\vspace{-15pt}
\section{Groups, Rings, and Fields}
\vspace{-3pt}
\begin{dfn}[Group Action]{dfn:group-action}{2.1.1}
    \vspace{-5pt}
    Let $G$ be a group and $X$ a set. An \textbf{action} of $G$ on $X$ is a function $G \times X \to X$, written as $(g,x)\mapsto gx$ such that
    \[(gh)x = g(hx) \text{ and } 1x = x\]
    for all $g, h\in G$ and $x\in X$, where $1$ is the identity of $G$
\end{dfn}

\vspace{-6pt}
\begin{dfn}[Faithful Actions]{dfn:faithful-actions}{2.1.7}
    \vspace{-5pt}
    An action of a group $G$ on a set $X$ is \textbf{faithful} if for $g,\,h\in G$,
    \[gx = hx \text{ for all } x\in X \implies g = h\]
    \par\vspace{-7pt}
    \textit{``If two elements of the group \textit{do} the same, they \textit{are} the same.''}
    \textrule{Lemma 2.1.8: Properties of Faithful Actions}
    \par\vspace{-12pt}
    For an action of a group $G$ on a set $X$, the following are equal:
    \vspace{-7pt}
    \begin{enumerate-tight}
        \item The action is faithful\vspace{-2pt}
        \item For $g\in G$, if $gx = x$ for all $x\in X$ then $g = 1$\vspace{-2pt}
        \item The homomorphism $\Sigma : G \to \sym(X)$ is injective\vspace{-2pt}
        \item $\ker \Sigma$ is trivial.
    \end{enumerate-tight}
    \vspace{-5pt}
    \textrule{Lemma 2.1.11: Isomorphisms of Faithful Groups}
    \par\vspace{-12pt}
    Let $G$ be a group acting faithfully on a set $X$. Then $G$ is isomorphic to the subgroup of $\sym(X)$, where $\Sigma : G \to \sym(X)$
    \[\im \Sigma = \{\overline{g} \mid g\in G\}, \text{ where } \overline{g} : X \to X \text{ and } \overline{g}(x) = gx\]
\end{dfn}

\vspace{-6pt}
\begin{dfn}[Fixed Set]{dfn:fixed-set}{2.1.1}
    \vspace{-5pt}
    For a group $G$ acting on a set $X$, let $S \subseteq G$. The \textbf{fixed set} of $S$ is
    \vspace{-2pt}
    \[\Fix(S) = \{ x\in X \mid sx = x \text{ for all } s\in S\}\]
    \vspace{-10pt}
    \textrule{Lemma 2.1.15: Normal Fixed Sets}
    \par\vspace{-12pt}
    Let $G$ be a group acting on a set $X$, let $S \subseteq G$, and let $g\in G$. Then $\Fix(gSg^{-1}) = g\Fix(S)$.

    Here, $gSg^{-1} = \{gs g^{-1} \mid s\in S\}$ and $g\Fix(S) = \{gx \mid x \in \Fix(S)\}$
\end{dfn}

\vspace{-6pt}
\begin{dfn}[Ring Homomorphism]{dfn:ring-homomorphism}{2.2.1}
    \vspace{-5pt}
    Given rings $R$ and $S$, a \textbf{homomorphism} from $R$ to $S$ is a function $\phi : R \to S$ satisfying the following equations for all $r,\,r'\in R$:
    \vspace{-15pt}
    \begin{multicols}{2}
    \begin{itemize-tight}
        \item $\phi(r+r') = \phi(r) + \phi(r')$
        \item $\phi(rr') = \phi(r)\phi(r')$
        \item $\phi(0) = 0$, $\phi(1) = 1$
        \item $\phi(-r) = -\phi(r)$
    \end{itemize-tight}
    \end{multicols}
    \par\vspace{-18pt}
    \longrule{0.08ex}
    A \textbf{subring} of a ring $R$ is a subset $S \subseteq R$ that contains $0$ and $1$ and is closed under addition, multiplication, and negatives. When $S$ is a subring of $R$, the inclusion $\iota : S \to R$ is a homomorphism.
    \vspace{-4pt}
    \textrule{Lemma 2.2.3: Intersection of Subrings}
    Let $R$ be a ring and let $\mathcal{S}$ be any set (perhaps infinite) of subrings of $R$. Then their intersection $\bigcap_{S\in \mathcal{S}} S$ is also a subring of $R$.
\end{dfn}

% TODO: Example 2.2

\vspace{-6pt}
\begin{rcl}[Ideals and Quotient Rings]{rcl:ideal}{}
    \vspace{-5pt}
    Let $R$ be a ring. $I \subseteq R$ is an \textbf{ideal}, $I \unlhd R$, if the following hold:
    \vspace{-16pt}
    \begin{multicols}{2}
    \begin{enumerate-zero}
        \setlength\itemsep{0em}
        \item $I \ne \emptyset$
        \item $I$ is closed under subtraction
    \end{enumerate-zero}
    \end{multicols}
    \vspace{-21pt}
    \begin{enumerate-zero}
        \setlength\itemsep{0em}
        \setcounter{enumi}{2}
        \item for all $i\in I$ and $r\in R$ we have $ri, ir\in I$
    \end{enumerate-zero}
    \vspace{-7pt}
    Every ring homomorphism $\phi : R \to S$ has an image $\im \phi$, which is a subring of $S$, and a kernel $\ker \phi$, which is an ideal of $R$.
    \par\vspace{-7pt}
    \longrule{0.08ex}
    Given an ideal $I \unlhd R$, define the quotient ring $R / I$ and canonical homomorphism $\pi_{I} : R \to R /I$ which is surjective and has kernel $I$. 


    \begin{vwcol}[widths={0.7,0.3}, rule=0pt]
    \textbf{Universal Property of Factor Rings}: 
        Given a ring $S$ and any homomorphism $\phi : R \to S$ satisfying $\ker \phi \supseteq I$, there is exactly one homomorphism $\overline{\phi} : R /I \to S$ s.t. this diagram commutes.
        \columnbreak
% https://q.uiver.app/#q=WzAsMyxbMCwwLCJSIl0sWzAsMiwiUiAvIEkiXSxbMiwyLCJTIl0sWzAsMSwiXFxwaV9JIiwyXSxbMSwyLCJcXG92ZXJsaW5le1xccGhpfSIsMl0sWzAsMiwiXFxwaGkiXV0=
\[\begin{tikzcd}[ampersand replacement=\&,cramped, column sep=small, row sep=scriptsize]
	R \\
	\\
	{R / I} \&\& S
	\arrow["{\pi_I}"', from=1-1, to=3-1]
	\arrow["\phi", from=1-1, to=3-3]
	\arrow["{\overline{\phi}}"', from=3-1, to=3-3]
    \end{tikzcd}\]
    \end{vwcol}
    \vspace{-28pt}
\end{rcl}

\vspace{-6pt}
\begin{rcl}[Integral Domains and Generators]{rcl:generated-ideal}{}
    \vspace{-7pt}
    An \textbf{integral domain} is a ring $R$ s.t. $0_{R} \ne 1_{R}$, and for $r,\,r'\in R$,
    \vspace{-3pt}
    \[rr' = 0 \implies r = 0 \text{ or } r' = 0.\]
    \vspace{-13pt}
    \textrule{Generated Ideals}
    Let $Y$ be a subset of a ring $R$. The \textbf{ideal $\langle Y \rangle$ generated by $Y$} is defined as the intersection of all the ideals of $R$ containing $Y$.
    \par\vspace{-8pt}
    \longrule{0.08ex}
    \vspace{-18pt}
    \begin{itemize-zero}
        \item \textbf{Principal ideals} are ideals of the form $\langle r \rangle$. A \textbf{principle ideal domain} is an integral domain where every ideal is principal.  \vspace{-2pt}
        \item Let $r$ and $s$ be elements of a ring $R$. $r$ \textbf{divides} $s$, or $r \mid s$, if $\exists a\in R$ s.t. $s = ar$. This is equivalent to $s\in \langle r \rangle$, and $\langle s \rangle \supseteq \langle r \rangle$. \vspace{-2pt}
        \item An element $u\in R$ is a \textbf{unit} if it has a multiplicative inverse, i.e. if $\langle u \rangle = R$. The units form a group $R^{\times}$ under multiplication. \vspace{-2pt}
        \item Elements $r$ and $s$ of a ring are \textbf{coprime} if for $a\in R$,
            \vspace{-2pt}
            \[a \mid r \text{ and } a \mid s \implies a \text{ is a unit}\]
    \end{itemize-zero}
    \vspace{-14pt}
    \begin{enumerate}[leftmargin=3.5em]
        \setlength\itemsep{0em}
        \item[\textbf{2.2.11})] For a ring $R$ and a finite subset $Y = \{r_{1},\dots,r_{n}\}$. Then
            \[\langle Y \rangle = \{a_{1}r_{1} + \cdots + a_{n}r_{n} : a_{1},\dots, a_{n} \in R\}\]
        \vspace{-14pt}
        \item[\textbf{2.2.16})] Let $R$ be a principal ideal domain and $r,\,s\in R$. Then
            \[r \text{ and } s \text{ are coprime} \iff ar + bs = 1 \text{ for some } a,\,b\in R\]
    \end{enumerate}
\end{rcl}

\vspace{-7pt}
\begin{rcl}[Fields, Fieldeals, and Subfields]{rcl:field}{2.3.A}
    \vspace{-5pt}
    A \textbf{field} is a ring $K$ in which $0 \ne 1$ and every nonzero element is a unit. Equivalently, it is a ring such that $K^{\times} = K \backslash \{0\}$. Every field is an integral domain. A field $K$ has exactly two ideals: $\{0\}$ and $K$. A \textbf{subfield} of a field $K$ is a subring that is a field
\end{rcl}

\vspace{-7pt}
\begin{xmp}[Rational Expressions]{xmp:rational-expressions}{2.3.2}
    \vspace{-5pt}
    Let $K$ be a field. A \textbf{rational expression} over $K$ is a ratio of two polynomials
    \vspace{-5pt}
    \[f(t) /g(t)\]
    where $f(t),\,g(t)\in K[t]$ with $g\ne 0$\footnote{Note that these are \textbf{not} functions, e.g. $1 /(t - 1)$ is a valid element of $K(t)$, and you don't need to worry about $t=1$.}. Two such expressions, $f_{1} /g_{1}$ and $f_{2} /g_{2}$ are regarded as equal if $f_{1}g_{2} = f_{2}g_{1}$ in $K[t]$. i.e. equivalence class. The set of rational expressions over $K$ is called $K(t)$
    \vspace{-5pt}
\end{xmp}

\vspace{-7pt}
\begin{dfn}[Equaliser]{dfn:equaliser}{2.3.7}
    \vspace{-5pt}
    For sets $X$ and $Y$, and $S \subseteq \{ \text{ functions } X \to Y\}$, the \textbf{equalizer} of $S$ is \textit{``the part of $X$ where all the functions in $S$ are equal''}, i.e.
    \[\eq(S) = \{x\in X \mid f(x) = g(x) \text{ for all } f,\,g\in S\}\]
\end{dfn}

\vspace{-7pt}
\begin{lma}[Ring Homomorphism Properties]{lma:homomorphisms-fields}{2.3.B}
    \vspace{-5pt}
    \begin{itemize-tight}
        \item[\textbf{2.3.3})] Every (ring) homomorphism between fields is injective.
        \item[\textbf{2.3.6})] Let $\phi : K \to L$ be a homomorphism between fields.
            \begin{enumerate}[leftmargin=0em]
                \vspace{-6pt}
                \setlength\itemsep{0em}
                \item For a subfield $K'$ of $K$, the image $\phi K'$ is a subfield of $L$\vspace{-1pt}
                \item For a subfield $L'$ of $L$, the preimage $\phi^{-1}L'$ is a subfield of $K$
            \end{enumerate}
            \vspace{-5pt}
        \item[\textbf{2.3.8})] Let $K$ and $L$ be fields, and let
            \vspace{-2pt}
            \[
                S \subseteq \{\text{homomorphisms } K \to L\}
            \]
            \par\vspace{-5pt}
            Then $\eq(S)$ is a subfield of $K$.
    \end{itemize-tight}
\end{lma}

\vspace{-5pt}
\begin{rcl}[Characteristic]{rcl:characteristic}{2.3.9}
    \vspace{-5pt}
    For a ring $R$, there is a unique homomorphism $\chi : \mathbb{Z} \to R$ whose kernel is an ideal of the PID $\mathbb{Z}$. Hence $\ker \chi = \langle n \rangle$ for a unique integer $n \ge 0$. $n$ is the \textbf{characteristic} of $R$ ($\Char R$). So for $m\in \mathbb{Z}$, we have that $m \cdot 1_{R} = 0$ iff $m$ is a multiple of $\Char R$. Or:
    \[\Char R = \begin{cases}
        \text{the least $n > 0$ s.t. $n \cdot 1_{R} = 0_{R}$}, & \text{if such an $n$ exists}\\
        0 & \text{otherwise}
    \end{cases}\]
    \longrule{0.08ex}
    \vspace{-15pt}
    \begin{enumerate}[leftmargin=3.5em]
        \setlength\itemsep{0em}
        \item[\textbf{2.3.11})] The characteristic of an integral domain is $0$ or prime.
        \item[\textbf{2.3.12})] Let $\phi : K \to L$ be a homomorphism of fields. Then $\Char K = \Char L$.
    \end{enumerate}
\end{rcl}

\vspace{-5pt}
\begin{rcl}[Prime Subfield]{rcl:prime-subfield}{2.3.C}
    \vspace{-5pt}
    The \textbf{prime subfield} of $K$ is the intersection of all the subfields of $K$. Concretely, the prime subfield of $K$ is
    \[\left\{ \frac{m \cdot 1_{K}}{n \cdot 1_{K}} \mid m,\,n\in \mathbb{Z} \text{ with } n \cdot 1_{K} \ne 0\right\}\]
    \textrule{Lemma 2.3.16}
    Let $K$ be a field.
    \vspace{-7pt}
    \begin{itemize-tight}
        \item If $\Char K = 0$ then the prime subfield of $K$ is (iso to) $\mathbb{Q}$.
        \item If $\Char K = p > 0$ then the prime subfield of $K$ is (iso to) $\mathbb{F}_{p}$
    \end{itemize-tight}
    \vspace{-10pt}
    \longrule{0.08ex}
    \vspace{-5pt}
    \textbf{Lemma 2.3.17}: Every finite field has positive characteristic.
    \vspace{4pt}
\end{rcl}

\begin{ppn}[The Frobenius Map]{lma:frobenius-map}{2.3.19}
    \vspace{-5pt}
    \textbf{Lemma 2.3.19}: Let $p$ be a prime and $0 < i < p$. Then $p \mid \binom{p}{i}$
    \par\vspace{-5pt}
    \longrule{0.08ex}
    Let $p$ be a prime number and $R$ a ring of characteristic $p$. Let the \textbf{Frobeinus Map} be the homomorphism $\theta : R \to R \quad r \mapsto r^{p}$.
    \vspace{-5pt}
    \begin{enumerate-tight}
        \item The Frobenius map is a homomorphism.
        \item If $R$ is a field then $\theta$ is injective.
        \item If $R$ is a finite field then $\theta$ is an automorphism of $R$. In this case we call $\theta$ the \textbf{Frobenius Automorphism}
    \end{enumerate-tight}
    \vspace{-7pt}
    \textrule{Corollary 2.3.22: Roots by Characteristic}
    \vspace{-2pt}
    Let $p$ be a prime number, and $K$ be a field with characteristic $p$.
    \vspace{-5pt}
    \begin{enumerate-zero}
        \item Every element in $K$ has \textit{at most} one $p$th root.
        \item If $K$ is a finite field, every element has \textit{exactly} one $p$th root.
    \end{enumerate-zero}
\end{ppn}

% TODO: examples

\vspace{-5pt}
\begin{rcl}[Reducible Elements]{rcl:reducible-element}{2.3.D}
    \vspace{-5pt}
    An element $r$ of a ring $R$ is \textbf{irreducible} if $r$ is not $0$ or a unit, and if for $a,\,b\in R$.
    \[r = ab \implies a \text{ or } b \text{ is an unit}\]
    For example, the irreducibles in $\mathbb{Z}$ are $\pm 2,\,\pm 3,\,\pm 5,\dots$. An element of a ring is \textbf{reducible} if it is not $0$, a unit, or irreducible.
    \par\vspace{-7pt}
    \longrule{0.08ex}
    \textbf{Warning}: The $0$ and units of a ring are neither reducible nor irreducible, in much the same way that the integers $0$ and $1$ are neither prime nor composite.
\end{rcl}

\vspace{-5pt}
\begin{ppn}[]{ppn:irreducible-generated-field}{2.3.26}
    \vspace{-5pt}
    Let $R$ be a principal ideal domain and $0 \ne r \in R$. Then
    \[r \text{ is irreducible } \iff R / \langle r \rangle \text{ is a field}\]
    This lets us construct fields from irreducible elements of a PID.
\end{ppn}

\newpage
\section{Polynomials}
\vspace{-2pt}
\begin{dfn}[Polynomial Ring]{dfn:polynomial-ring}{3.1.1}
    \vspace{-5pt}
    Let $R$ be a ring. A \textbf{polynomial over $R$} is an infinite sequence $(a_{0},a_{1},a_{2},\dots)$ of elements of $R$ s.t. $\{i \mid a_{i} \ne 0\}$ is finite.

    The set of polynomials over $R$, written $R[t]$, forms a ring:
    \begin{align*}
        (a_{0},\, a_{1} ,\dots) + (b_{0},b_{1},\dots) &= (a_{0}+ b_{0},a_{1}+b_{1},\dots), \\
        (a_{0},\, a_{1} ,\dots) \cdot (b_{0},b_{1},\dots) &= (c_{0},c_{1},\dots), \\
        \text{where } c_{k} &= \sum_{i,j : i + j = k} a_{i}b_{j}
    \end{align*}
    % The zero is $(0,0,\dots)$ and the mult. identity is $(1,0,0,\dots)$.


    Polynomials are typically written as $f$ or $f(t)$ interchangeably. A polynomial $f = (a_{0},a_{1},\dots)$ over $R$ gives rise to a function
    \begin{equation*}
        R \to R, \quad r \mapsto a_{0}+a_{1}r+a_{2}r^{2}+\cdots.
    \end{equation*}
\end{dfn}

% warnings

\vspace{-5pt}
\begin{ppn}[Universal Property of the Polyring]{ppn:uniprop-polyring}{3.1.6}
    \vspace{-5pt}
    Let $R,\, B$ be rings. For every homomorphism $\phi : R \to B$ and every $b\in B$, there is exactly one homomorphism $\theta : R[t] \to B$ such that
    \begin{align}
        \tag{3.4}\theta(a) &= \phi(a) \text{ for all } a\in R\\
        \tag{3.5}\theta(t) &= b
    \end{align}
\end{ppn}

\begin{dfn}[Induced Homomorphism]{dfn:induced-homomorphism}{3.1.7}
    \vspace{-5pt}
    Let $\phi : R \to S$ be a ring homomorphism. We define
    \[\phi_{\ast} : R[t] \to S[t]\]
    as the \textbf{induced homomorphism}, which is the unique homomorphism $R[t] \to S[t]$ s.t. $\phi_{\ast} = \phi(a)$ for all $a\in R$ and $\phi_{\ast}(t) = t$.
\end{dfn}

% TODO: Explanationfor this, and evaluation.

\vspace{-5pt}
\begin{dfn}[Degree of a Polynomial]{dfn:degree-polynomial}{3.1.9}
    \vspace{-5pt}
    The \textbf{degree}, $\deg(f)$, of a nonzero polynomial $f(t) = \sum a_{i}t^{i}$ is the largest $n \ge 0$ s.t. $a_{n}\ne 0$. By convention, $\deg(0) = -\infty$, where $-\infty$ is a formal symbol which we give the properties for all $n\in \mathbb{Z}$:
    \[-\infty < n, \quad (-\infty) + n = -\infty, \quad (-\infty) + (-\infty) = -\infty\]
    \vspace{-12pt}
    \textrule{Lemma 3.1.11}
    Let $R$ be an integral domain. Then:
    \vspace{-7pt}
    \begin{enumerate-tight}
        \item $\deg(fg) = \deg(f) + \deg(g)$ for all $f,\,g\in R[t]$
        \item $R[t]$ is an integral domain.
    \end{enumerate-tight}
    \vspace{-8pt}
    $\deg(-\infty)$ implies the (unique) zero polynomial, $\deg(0)$ implies the nonzero constants, $\deg(>0)$ implies the nonconstant polynomials.
    \vspace{-2pt}
    \textrule{Lemma 3.1.14}
    Let $K$ be a field. Then
    \vspace{-7pt}
    \begin{enumerate-tight}
        \item The units in $K[t]$ are the nonzero constants
        \item $f\in K[t]$ is irreducible iff $f$ is nonconstant and cannot be expressed as a product of two nonconstant polynomials.
    \end{enumerate-tight}
    \vspace{-7pt}
    \textrule{Lemma 3.2.1 - Uniqueness of Poly Division}
    For a field $K$ and $f,\,g\in K[t]$ with $g\ne 0$, there is exactly one pair of polynomials $q,\,r\in K[t]$ s.t. $f = qg + r$ and $\deg(r) < \deg(g)$
\end{dfn}

\vspace{-5pt}
\begin{lma}[Facts about Fields]{lma:field-facts}{3.2.A}
    \vspace{-5pt}
    \begin{itemize-tight}
        \item[\textbf{3.2.2})] Let $K$ be a field. Then $K[t]$ is a principal ideal domain.
        \item[\textbf{3.2.5})] Let $K$ be a field and let $0 \ne f \in K[t]$. Then
            \[f \text{ is irreducible} \iff K[t] / \langle f \rangle \text{ is a field.}\]
        \item[\textbf{3.2.6})] Let $K$ be a field and let $f(t) \in K[t]$ be a nonconstant polynomial. Then $f(t)$ is divisible by some irreducible in $K[t]$
        \item[\textbf{3.2.7})] Let $K$ be a field and $f,\,g,\,h\in K[t]$. Suppose that $f$ is irreducible and $f \mid gh$. Then $f \mid g$ or $f \mid h$.
    \end{itemize-tight}
    
\end{lma}


\vspace{-5pt}
\begin{thm}[Unique Determination of Polys]{thm:poly-unique-determination}{3.2.8}
    \vspace{-5pt}
    Let $K$ be a field and $0 \ne f \in K[t]$. Then
    \[f = af_{1}f_{2}\cdots f_{n}\]
    for some $n \ge 0$, $a\in K$, and monic\footnote{Monic means that the highest order element has coefficient $1$.} irreducibles $f_{1},\dots,f_{n}\in K[t]$. Moreover, $n$ and $a$ are uniquely determined by $f$, and $f_{1},\dots,f_{n}$ are uniquely determind up to reordering.
    \vspace{-3pt}
\end{thm}

\vspace{-5pt}
\begin{lma}[Root Finding]{lma:root-finding}{3.2.9}
    \vspace{-5pt}
    One way to find an irreducible factor of a polynomial $f(t)\in K[t]$ is to find a \textbf{root}. Let $K$ be a field, $f(t) \in K[t]$, and $a\in K$. Then
    \[f(a) = 0 \iff (t - a) \mid f(t).\]
\end{lma}

\begin{lma}[Algebraically Closed Field]{lma:algebraically-closed-field}{3.2.10}
    \vspace{-5pt}
    A field is \textbf{algebraically closed} if every nonconstant polynomial has at least one root.
    \par\vspace{-7pt}
    \longrule{0.08ex}
    Let $K$ be an algebraically closed field and $0 \ne f \in K[t]$. then
    \[f(t) = c(t - a_{1})^{m_{1}} \cdots (t - a_{k})^{m_{k}},\]
    where $c$ is the leading coefficient of $f$, and $a_{1},\dots,a_{k}$ are the distinct roots of $f$ in $K$, and $m_{1},\dots,m_{k}\ge 1$
\end{lma}

\vspace{-5pt}
\begin{lma}[Degrees and Irreducibility]{lma:degrees-irreducibility}{3.3.1}
    \vspace{-5pt}
    Let $K$ be a field and $f \in K[t]$.
    \vspace{-8pt}
    \begin{enumerate-tight}
        \item If $f$ is constant then $f$ is not irreducible.
        \item If $\deg(f) = 1$ then $f$ is irreducible.
        \item If $\deg(f) \ge 2$ and $f$ has a root then $f$ is reducible.
        \item If $\deg(f)\in \{2,3\}$ and $f$ has no root then $f$ is irreducible.
    \end{enumerate-tight}

    \vspace{-7pt}
    \textbf{Warning}: To show a polynomial is irreducible, it's generally \textit{not} enough to show it has no root. The converse of $3$ is false!
\end{lma}

\vspace{-5pt}
\begin{dfn}[Primitive Polynomial]{dfn:primitive-polynomial}{3.3.6}
    \vspace{-5pt}
    A polynomial over $\mathbb{Z}$ is \textbf{primitive} if its coefficients have no common divisor except for $\pm 1$.
    \textrule{Lemma 3.3.7: Existence of Primitives}
    Let $f(t)\in \mathbb{Q}[t]$. Then there exists a primitive polynomial $F(t)\in \mathbb{Z}[t]$ and $\alpha\in \mathbb{Q}$ such that $f = \alpha F$.
\end{dfn}


\vspace{-5pt}
\begin{rem}[Irreducibility over]{rem:irreducibility-over}{3.3.A}
    \vspace{-5pt}
    If the coefficients of a polynomial $f(t)\in \mathbb{Q}[t]$ happen to all be integers, the word ``irreducible'' could mean two things: irreducibility in the ring $\mathbb{Q}[t]$ or in the ring $\mathbb{Z}[t]$. We say that $f$ is irreducible \textbf{over} $\mathbb{Q}$ or $\mathbb{Z}$ to distinguish between the two.
\end{rem}

\vspace{-5pt}
\begin{lma}[Irreducibility Tests]{lma:irreducibility-tests}{3.3.B}
    \vspace{-5pt}
    \textrule{Lemma 3.3.8: Gauss' Lemma}
    \par\vspace{-20pt}
    \begin{enumerate-zero}
        \item The product of two primitive polynomials over $\mathbb{Z}$ is primitive.\vspace{-3pt}
        \item If a nonconstant polynomial over $\mathbb{Z}$ is irreducible over $\mathbb{Z}$, it is irreducible over $\mathbb{Q}$
    \end{enumerate-zero}
    \vspace{-9pt}
    \textrule{Lemma 3.3.9: Mod-$p$ Method}
    \par\vspace{-12pt}
    Let $f(t) = a_{0} + a_{1}t + \cdots + a_{n}t^{n}\in \mathbb{Z}[t]$. If there is some prime $p$ s.t. $p \nmid a_{n}$ and $\overline{f} \in \mathbb{F}_{p}[t]$ is irreducible, then $f$ is irreducible over $\mathbb{Q}$.

    \textbf{Warning}: This only tells you that a polynomial is \textit{irreducible} over $\mathbb{Q}$ and says nothing about whether it is \textit{reducible}.
    \vspace{-2pt}
    \textrule{Lemma 3.3.12: Eisenstein's Criterion}
    Let $f(t) = a_{0}+\cdots + a_{n}t^{n}\in \mathbb{Z}[t]$, with $n \ge 1$. Suppose there exists a prime $p$ such that
    \[\bullet \: p\nmid a_{n} \qquad\quad \bullet\: p \mid a_{i},\,\forall i\in \{0,\dots, n - 1\} \qquad\quad \bullet\:p^{2} \nmid a_{0} \]
    Then $f$ is irreducible over $\mathbb{Q}$.
\end{lma}

% TODO: co-degree

% TODO: put this into examples
% \begin{xmp}[Cyclotomic Polynomial]{xmp:cyclotoic}{3.3.16}
%     Let $p$ be a prime. The \textbf{$p$th cyclotomic polynomial} is
%     \[\Phi_{p}(t) = 1 + t + \cdots + t^{p-1} = \frac{t^{p} - 1}{t - 1}\]
%     $\Phi_{p}$ is irreducible.
% \end{xmp}

% ╭──────────────────────────────────────────────────────────╮
% │                          PAGE 3                          │
% ╰──────────────────────────────────────────────────────────╯

\section{Field Extensions}
\vspace{-3pt}
\begin{dfn}[Field Extension]{dfn:field-extension}{4.1.1}
    \vspace{-5pt}
    It is sometimes easier to think of a subset as an injection. Given a set $A$ and a subset $B \subseteq A$, define an \textbf{inclusion} function
    \vspace{-2pt}
    \[
        \iota : B \to A \text{ defined by } \iota(b) = b \text{ for all } b\in B.
    \]
    Let $K$ be a field. An \textbf{extension} of $K$ is a field $M$ together with a homomorphism $\iota : K \to M$. We write $M : K$ to mean that $M$ is an extension of $K$, not bothering to mention $\iota$.
\end{dfn}

\vspace{-8pt}
\begin{xmp}[Examples of Field Extensions]{xmp:examples of field extensions}{4.1.2}
    \vspace{-12pt}
    \[\iota_{1}: \mathbb{Q} \to \mathbb{R}, \quad \iota_{2} : \mathbb{R} \to \mathbb{C}, \quad \iota_{3} : \mathbb{Q} \to \mathbb{C}\]
    $\iota_{4} : Q \to K$, where $K = \{a + b\sqrt{2} \mid a,\,b\in \mathbb{Q}\}$ (we call this $\mathbb{Q}(\sqrt{2})$)
\end{xmp}

% TODO: examples

\vspace{-8pt}
\begin{dfn}[Generated Subfields]{dfn:generated-subfield}{4.1.4}
    \vspace{-5pt}
    For a field $K$, and $X$ a subset of $K$, the subfield of $K$ \textbf{generated by} $X$ is the intersection of all subfields of $K$ containing $X$. Let $F$ be the subfield of $K$ generated by $X$. $F$ contains $X$, and $F$ is also the \textit{smallest} subfield of $K$ containing $X$ (i.e. any subfield of $K$ containing $X$ contains $F$)

    \vspace{-3pt}
    \textrule{Definition 4.1.8: Adjoined Subfields}
    \par\vspace{-13pt}
    For a field extension $M : K$, and $Y \subseteq M$, we write $K(Y)$ for the subfield of $M$ generated by $K \cup Y$. We call it the subfield of $M$ \textbf{generated by $Y$ over $K$}, or $K$ with $Y$ \textbf{adjoined}.
    \par\vspace{-7pt}
    \longrule{0.08ex}
    $K(Y)$ is the smallest subfield of $M$ containing both $K,\,Y$. If $Y$ is a finite set $\{\alpha_{1},\dots,\alpha_{n}\}$, write $K(\{\alpha_{1},\dots,\alpha_{n}\})$ as $K(\alpha_{1},\dots,\alpha_{n})$
\end{dfn}

% TODO: examples

% TODO: warning 4.1.10

\vspace{-8pt}
\begin{dfn}[Algebraic Numbers]{dfn:algebraic-extension}{4.2.1}
    \vspace{-5pt}
    A complex number $\alpha\in \mathbb{C}$ is said to be ``algebraic'' if
    \vspace{-3pt}
    \[a_{0} + a_{1}\alpha + \cdots + a_{n}a^{n} = 0\]
    \par\vspace{-5pt}
    for some rational numbers $a_{i}$, not all zero
    \textrule{Algebraic Numbers for Arbitrary Fields}
    \par\vspace{-14pt}
    For a field extension $M : K$, and $\alpha\in M$, $\alpha$ is \textbf{algebraic} over $K$ if $\exists f \in K[t]$ s.t. $f(\alpha) = 0$ but $f \ne 0$, \textbf{transcendental} otherwise.
\end{dfn}

% TODO: transcendental numbers examples

\vspace{-6pt}
\begin{lma}[Annihilators]{lma:annihilator}{4.2.6}
    \vspace{-5pt}
    Let $M : K$ be a field extension and $\alpha\in M$. An \textbf{annihilating polynomial} of $\alpha$ is a polynomial $f \in K[t]$ such that $f(\alpha)=0$. So, $\alpha$ is algebraic iff it has some nonzero annihilating polynomial.
    \par\vspace{-7pt}
    \longrule{0.08ex}
    For a field extension $M : K$ and $\alpha\in M$, there is a polynomial $m(t) \in K[t]$ such that
    \vspace{-5pt}
    \begin{equation}\label{eq:annihilator}\tag{4.2}
        \langle m \rangle = \{\text{annihilating polynomials of $\alpha$ over $K$}\}.
    \end{equation}
    \par\vspace{-5pt}
    If $\alpha$ is transcendental over $K$ then $m = 0$. If $\alpha$ is algebraic over $K$ then there is a unique monic polynomial $m$ satisfying \eqref{eq:annihilator}.
\end{lma}

\vspace{-6pt}
\begin{dfn}[Minimal Polynomial]{dfn:minimal-polynomial}{4.2.7}
    \vspace{-5pt}
    Let $M : K$ be a field extension and let $\alpha\in M$ be \textit{algebraic} over $K$. The \textbf{minimal polynomial} of $\alpha$ is the unique monic polynomial satisfying \eqref{eq:annihilator}.
    \newline\textbf{Warning}: This isn't defined over transcendentals, therefore some elements of $M$ might not have a minimal polynomial.
    \textrule{Lemma 4.2.10: Minimal Polynomial Conditions}
    \par\vspace{-13pt}
    Let $M : K$ be a field extension, let $\alpha \in M$ be algebraic over $K$ and let $m\in K[t]$ be a monic polynomial. The following are equivalent:
    \vspace{-7pt}
    \begin{enumerate-zero}
        \item $m$ is the minimal polynomial of $\alpha$ over $K$\vspace{-2pt}
        \item $m(\alpha) = 0$, $m \mid f$ for all annihilating polynomials $f$ of $\alpha$ over $K$\vspace{-2pt}
        \item $m(\alpha) = 0$ and $\deg(m) \le \deg (f)$ for all nonzero annihilating polynomials. \textit{``monic annihilating polynomial of least degree.''}\vspace{-2pt}\vspace{-1pt}
        \item $m(\alpha) = 0$ and $m$ is irreducible over $K$.
    \end{enumerate-zero}
\end{dfn}

%TODO: examples, these are probably important

\newpage
\vspace{-7pt}
\begin{dfn}[]{dfn:minimal-polynomial-element}{4.3.1}
    \vspace{-5pt}
    Let $K$ be a field.
    \vspace{-7pt}
    \begin{enumerate}[leftmargin=1.5ex]
        \item Let $m\in K[t]$ be monic and irreducible. Write $\alpha\in K[t] /\langle m \rangle$ for the image of $t$ under the canonical homomorphism $K[t] \to K[t] / \langle m \rangle$. Then $\alpha$ has minimal polynomial $m$ over $K$, and $K[t] /\langle m \rangle$ is generated by $\alpha$ over $K$.\vspace{-3pt}
        \item The element $t$ of the field $K(t)$ of rational expressions over $K$ is transcendental over $K$, and $K(t)$ is generated by $t$ over $K$
    \end{enumerate}
\end{dfn}

\vspace{-7pt}
\begin{dfn}[Homomorphism over Fields]{dfn:homomorphism-over-field}{4.3.3}
    \vspace{-5pt}
    \begin{vwcol}[widths={0.67,0.33}, rule=0pt]
    For a field $K$, and let $\iota : K \to M$, $\iota' : K \to M'$ be extensions of $K$. A homomorphism $\phi : M \to M'$ is called a \textbf{homomorphism over $K$} if the following diagram commutes:
        \columnbreak
% https://q.uiver.app/#q=WzAsMyxbMCwwLCJNIl0sWzIsMCwiTSwiXSxbMSwxLCJLIl0sWzAsMSwiXFxwaGkiXSxbMiwwLCJcXGlvdGEiXSxbMiwxLCJcXGlvdGEnIiwyXV0=
\[\begin{tikzcd}[ampersand replacement=\&,cramped,column sep=scriptsize]
	M \&\& {M,} \\
	\& K
	\arrow["\phi", from=1-1, to=1-3]
	\arrow["\iota", from=2-2, to=1-1]
	\arrow["{\iota'}"', from=2-2, to=1-3]
\end{tikzcd}\]
    \end{vwcol}
\end{dfn}

\vspace{-7pt}
\begin{lma}[Uniqueness of Field Homomorphisms]{lma:homomorphisms-over-unique}{4.3.6}
    \vspace{-5pt}
    Let $M$ and $M'$ be extensions of a field $K$, and let $\phi, \psi : M \to M'$ be homomorphisms over $K$. Let $Y$ be a subset of $M$ such that $M = K(Y)$. If $\phi(\alpha) = \psi(\alpha)$ for all $\alpha\in Y$ then $\phi = \psi$.
\end{lma}

\vspace{-7pt}
\begin{ppn}[Universal Props of \texorpdfstring{$K[t] /\langle m \rangle$}{K[t]/<m>}, \texorpdfstring{$K(t)$}{K(t)}]{ppn:universal-prop-simples}{4.3.7}
    \vspace{-5pt}
    \textrule{Universal Property of $K[t] / \langle m \rangle$}
    \vspace{-12pt}
    \begin{vwcol}[widths={0.4,0.6}, rule=0pt]
        Let $K$ be a field, and:
        \vspace{-15pt}
        \begin{itemize}[leftmargin=*, rightmargin=0.43\linewidth]\setlength\itemsep{-1.4em}
            \item $m\in K[t]$ monic and irreducible
            \item $L : K$ an extension of $K$
            \item $\beta\in L$ with minimal polynomial $m$
            \item Write $\alpha$ for the image of $t$ under the canonical homomorphism $K[t] \to K[t] / \langle m \rangle$.
            \item Then there is exactly one homomorphism $\phi : K[t] / \langle m \rangle \to L$ over $K$ such that $\phi(a) = \beta$.
        \end{itemize}
        \columnbreak
% https://q.uiver.app/#q=WzAsNSxbMCwyLCJLW3RdL1xcbGFuZ2xlIG0gXFxyYW5nbGUiXSxbMiwxLCJMIl0sWzEsMywiSyJdLFswLDEsIlxcYWxwaGEiXSxbMiwwLCJcXGJldGEiXSxbMCwxLCJcXHBoaSIsMCx7InN0eWxlIjp7ImJvZHkiOnsibmFtZSI6ImRvdHRlZCJ9fX1dLFsyLDBdLFsyLDFdLFszLDQsIiIsMix7InN0eWxlIjp7InRhaWwiOnsibmFtZSI6Im1hcHMgdG8ifSwiYm9keSI6eyJuYW1lIjoiZG90dGVkIn19fV1d
        \[\begin{tikzcd}[ampersand replacement=\&,cramped, column sep=small, row sep=scriptsize]
            \&\& \beta \\
            \alpha \&\& L \\
            {K[t]/\langle m \rangle} \\
            \& K
            \arrow[dotted, maps to, from=2-1, to=1-3]
            \arrow["\phi", dotted, from=3-1, to=2-3]
            \arrow[from=4-2, to=2-3]
            \arrow[from=4-2, to=3-1]
        \end{tikzcd}\]
    \end{vwcol}
    \vspace{-18pt}
    \textrule{Universal Property of $K(t)$}
    \vspace{-12pt}

        For $L : K$ an extension of $K$, and transcendental $\beta\in L$, there is exactly one homomorphism $\phi : K(t) \to L$ over $K$ s.t. $\phi(t) = \beta$.

    
\end{ppn}

% TODO: example - LOOK AT THIS!!!
\vspace{-5pt}
\begin{crl}[Isomorphisms and Uniqueness]{crl:isomorphism-uniqueness}{4.3.11}
    \vspace{-5pt}
    Let $M$ and $M'$ be extensions of a field $K$. A homomorphism $\phi : M \to M'$ is an \textbf{isomorphism over $K$} if it is a homomorphism over $K$ and an isomorphism of fields. If such a $\phi$ exists, we say that $M$ and $M'$ are \textbf{isomorphic over $K$}.
    \par\vspace{-8pt}
    \longrule{0.08ex}
    Let $K$ be a field.
    \vspace{-7pt}
    \begin{enumerate-zero}
        \item Let the conditions from 4.3.7 apply, alongside the condition that $L = K(\beta)$. Then there is exactly one isomorphism $\phi : K[t] / \langle m \rangle \to L$ over $K $ such that $\phi(\alpha) = \beta$.
        \item Let $L : K$ be an extension of $K$, and let $\beta \in L$ be transcendental with $L = K(\beta)$. Then there is exactly one isomorphism $\phi : K(t) \to L$ over $K$ such that $\phi(t) = \beta$.
    \end{enumerate-zero}
\end{crl}

\vspace{-7pt}
\begin{dfn}[Simple Extension]{dfn:simple-extension}{4.3.13}
    \vspace{-5pt}
    A field extension $M : K$ is \textbf{simple} if $\exists \alpha \in M$ s.t. $M = K(\alpha)$.
\end{dfn}

\vspace{-7pt}
\begin{thm}[Classification of Simple Extensions]{thm:simple-extension-classification}{4.3.16}
    \vspace{-5pt}
    Let $K$ be a field.
    \vspace{-7pt}
    \begin{enumerate-zero}
        \item Let $m\in K[t]$ be a monic irreducible polynomial. Then there exists an extension $M : K$ and an algebraic element $\alpha\in M$ such that $M = K(\alpha)$ and $\alpha$ has minimal polynomial $m$ over $K$. Moreover, if $(M, \alpha)$ and $(M',\alpha')$ are two such pairs, there is exactly one isomorphism $\phi : M \to M'$ over $K$ s.t. $\phi(\alpha) = \alpha'$
        \item There exists an extension $M : K$ and a transcendental element $\alpha \in M$ such that $M = K(\alpha)$. Moreover, if $(M, \alpha)$ and $(M',\alpha')$ are two such pairs, there is exactly one isomorphism $\phi : M \to M'$ over $K$ such that $\phi(\alpha) = \alpha'$.
    \end{enumerate-zero}
\end{thm}

\vspace{-7pt}
\section{Degree}
\begin{dfn}[Degree of a Field Extension]{dfn:field-extension-degree}{5.1.1}
    \vspace{-5pt}
    Let $M : K$ be a field extension. Then $M$ can be seen as a vector space over $K$. When we view $M$ as a vector space over $K$ rather than an extension, we forget how to multiply together elements of $M$ that aren't in $K$.
    \par\vspace{-9pt}
    \longrule{0.08ex}
    The \textbf{degree $[M : K]$} of a field extension $M : K$ is the dimension of $M$ as a vector space over $K$. If $M$ is an \textit{infinite-dimensional} vector space over $K$, we write $[M : K] = \infty$, where $\infty$ is a formal symbol with the properties
    \vspace{-3pt}
    \[n < \infty, \quad n \cdot \infty = \infty\:(n \ge 1), \quad \infty \cdot \infty = \infty\]
    \par\vspace{-5pt}
    for integers $n$. An extension $M : K$ is \textbf{finite} if $[M : K] < \infty$.
    \vspace{-5pt}
    \textrule{Warning 5.1.4}
    The degree $[K : K]$ of $K$ over itself is $1$, not $0$. Degrees of extensions are never $0$.
\end{dfn}

\vspace{-7pt}
\begin{thm}[Basis of Field Extensions]{thm:field-extension-basis}{5.1.5}
    \vspace{-5pt}
    Let $K(\alpha) : K$ be a simple extension.
    \vspace{-7pt}
    \begin{enumerate-zero}
        \item Suppose that $\alpha$ is algebraic over $K$. Write $m\in K[t]$ for the minimal polynomial of $\alpha$ and $n = \deg(m)$. Then
            \vspace{-2pt}
            \[1,\alpha,\dots,\alpha^{n-1}\]
            \par\vspace{-7pt}is a basis of $K(\alpha)$ over $K$. In particular, $[K(\alpha) : K] = \deg(m)$ \vspace{-5pt}
        \item Suppose that $\alpha$ is transcendental over $K$. Then $1,\alpha,\alpha^{2},\dots$ are linearly independent over $K$. In particular, $[K(\alpha) : K] = \infty$
    \end{enumerate-zero}
\end{thm}

% ╭──────────────────────────────────────────────────────────╮
% │                          PAGE 4                          │
% ╰──────────────────────────────────────────────────────────╯

%TODO: Example

\vspace{-5pt}
\begin{thm}[Tower Law]{thm:tower-law}{5.1.17}
    \vspace{-5pt}
    For field extensions $M : L : K$ and (potentially infinite) sets $I,\,J$,
    \vspace{-7pt}
    \begin{enumerate-tight}
        \item If $(\alpha_{i})_{i\in I}$ is a basis of $L$ over $K$ and $(\beta_{j})_{j\in J}$ is a basis of $M$ over $L$, then $(\alpha_{i}\beta_{j})_{(i,j)\in I \times J}$ is a basis of $M$ over $K$.\vspace{-2pt}
        \item $M : K$ is finite $\iff$ $M : L$ and $L : K$ are finite.\vspace{-2pt}
        \item $[M : K] = [M : L][L : K]$
    \end{enumerate-tight}
    \vspace{-7pt}
    A family $(\alpha_{i})_{i\in I}$ of elements of a field is \textbf{finitely supported} if the set $\{i\in I \mid \alpha_{i} \ne 0\}$ is finite.
\end{thm}

% examples

\vspace{-7pt}
\begin{crl}[Degree Results]{crl:degree-results}{5.1.A}
    \vspace{-5pt}
    \textrule{Corollary 5.1.10: Degree means Algebraic}
    \par\vspace{-12pt}
    Let $M : K$ be a field extension and $\alpha\in M$, the \textbf{degree} of $\alpha$ over $K$ is $[K(\alpha) : K]$. We write it as $\deg_{K}(\alpha)$. Then
    \vspace{-3pt}\[\deg_{K}(\alpha) < \infty \iff \alpha \text{ is algebraic over $K$.}\]\par\vspace{-5pt}
    If $\alpha$ is algebraic over $K$ then the degree of $\alpha$ over $K$ is the degree of the minimal polynomial of $\alpha$ over $K$.
    \textrule{Corollary 5.1.12: Size of Nested Extension}
    \par\vspace{-12pt}
    Let $M : L : K$ be a field extension and $\beta \in M$. Then
    \[ [L(\beta) : L] \le [K(\beta) : K] \]
% TODO: visualization of 5.1.12
    \textrule{Corollary 5.1.14: Polynomial Form of Extensions}
    \par\vspace{-12pt}
    Let $M : K$ be an extension and $\alpha_{1},\dots, \alpha_{n}\in M$, with $\alpha_{i}$ algebraic over $K$ of degree $d_{i}$. Then every element $\alpha\in K(\alpha_{1},\dots,\alpha_{n})$ can be expressed as a polynomial in $\alpha_{1},\dots,\alpha_{n}$ over $K$. More exactly,
    \[\alpha = \sum_{r_{1},\dots,r_{n}} c_{r_{1},\dots,r_{n}}a^{r_{1}}_{1} \cdots a^{r_{n}}_{n}\]
    for some $c_{r_{1},\dots,r_{n}}\in K$, where $r_{i}$ ranges over $0,\dots,d_{i}-1$.

    \textrule{Corollary 5.1.19: Dividing Extensions}
    \par\vspace{-12pt}
Let $M : L' : L : K$ be field extensions. If $M : K$ is finite, then $[L' : L]$ divides $[M : K]$

\vspace{-5pt}
\textrule{Corollary 5.1.21: Triangle Tower Inequality}
    \par\vspace{-12pt}
Let $M : K$ be a field extension and $\alpha_{1},\dots,\alpha_{n}\in M$. Then
    \[[K(\alpha_{1},\dots,\alpha_{n}) : K] \le [K(\alpha_{1}) : K] \cdots [K(\alpha_{n}) : K].\]
\end{crl}

% TODO: the diagram thing
\vspace{-7pt}
\begin{dfn}[Finitely Generated Extensions]{dfn:finitely-generated}{5.2.1}
    \vspace{-5pt}
    A field extension $M : K$ is \textbf{finitely generated} if $M = K(Y)$ for some finite subset $Y \subseteq M$.
    \textrule{Definition 5.2.2: Algebraic Extension}
    \par\vspace{-12pt}
    A field ext. $M\hspace{-0.7ex}:\hspace{-0.7ex} K$ is \textbf{algebraic} if all elements of $M$ are algebraic over $K$
\end{dfn}

\vspace{-7pt}
\begin{ppn}[Algebraic and Finiteness]{ppn:algebraic-finite-extension}{5.2.4}
    \vspace{-5pt}
    The following conditions on a field extension $M : K$ are equivalent:
    \par\vspace{-10pt}
    \begin{enumerate-tight}
        \item $M : K$ is finite\vspace{-2pt}
        \item $M : K$ is finitely generated and algebraic\vspace{-2pt}
        \item $M = K(\alpha_{1},\dots,\alpha_{n})$ for some finite set $\{\alpha_{1},\dots,\alpha_{n}\}$ of elements of $M$ algebraic over $K$.
    \end{enumerate-tight}
    \vspace{-8pt}
    \textrule{Corollary 5.2.6: Variation for Simple Extensions}
    \par\vspace{-13pt}
    Let $K(\alpha) : K$ be a simple extension. The following are equivalent:
    \vspace{-5pt}
    \begin{enumerate-tight}
        \vspace{-10pt}\begin{multicols}{2}
        \item $K(\alpha) : K$ is finite
        \item $K(\alpha) : K$ is algebraic
        \end{multicols}\vspace{-15pt}
        \item $\alpha$ is algebraic over $K$.
    \end{enumerate-tight}
    \vspace{-7pt}
    \textbf{Corollary 5.2.7}: $\overline{\mathbb{Q}}$ is a subfield of $\mathbb{C}$.
\end{ppn}

\vspace{-7pt}
\begin{dfn-s}[Ruler and Compass Constructions]{dfn:compositum}{5.3.3}
    \vspace{-5pt}
    A point $C$ in the plane is \textbf{immediately constructible} from $\Sigma$ if it is a point of intersection between lines or circles. $C$ is \textbf{constructible} from $\Sigma$ if there is a finite sequence $C_{1},\dots,C_{n}=C$ of points such that $C_{i}$ is immediately constructible from $\Sigma \cup \{C_{1},\dots,C_{i-1}\}$ for each $i$.
    \par\vspace{-8pt}
    \longrule{0.08ex}
    For a subfield $K \subseteq \mathbb{R}$, an extension $K : \mathbb{Q}$ is \textbf{iterated quadratic} if there is some finite sequence of subfields
    \vspace{-2pt}
    \[\mathbb{Q} = K_{0} \subseteq K_{1} \subseteq \cdots \subseteq K_{n} = K\]
    such that $[K_{i} : K_{i-1}] = 2$ for all $i\in \{1,\dots,n\}$
    \par\vspace{-8pt}
    \longrule{0.08ex}
    Let $L$ and $L'$ be subfields of a field $M$. The \textbf{compositum $LL'$} of $L$ and $L'$ is the subfield of $M$ generated by $L \cup L'$. That is, $LL'$ is the smallest subfield of $M$ containing both $L$ and $L'$.
\end{dfn-s}

\vspace{-7pt}
\begin{lma}[Ruler and Compass Results]{lma:ruler-compass}{5.3.B}
    \vspace{-5pt}
    \textbf{Lemma 5.3.6}: For a field extension $M : K$ and $L,\, L'$ subfields of $M$ containing $K$, if $[L : K] = 2$ then $[LL':L'] \in \{1,2\}$.

    \vspace{-3pt}
    \textbf{Lemma 5.3.8}: Let $K$ and $L$ be subfields of $\mathbb{R}$ s.t. the extensions $K : \mathbb{Q}$ and $L : \mathbb{Q}$ are iterated quadratic. Then there is some subfield $M$ of $\mathbb{R}$ s.t. the $M : \mathbb{Q}$ is iterated quadratic and $K,\, L \subseteq M$.
    \textrule{Proposition 5.3.9: Iterated Quadratics from Points}
    \par\vspace{-12pt}
    Let $(x,y)\in \mathbb{R}^{2}$. If $(x,y)$ is constructable from $\{(0,0),\, (1,0)\}$ then there is an iterated quadratic extension of $\mathbb{Q}$ containing $x$ and $y$.
    \textrule{Theorem 5.3.10: Quadratics and Constructability}
    \par\vspace{-12pt}
    Let $(x,y)\in \mathbb{R}^{2}$. If $(x,y)$ is constructible from $\{(0,0),\, (1,0)\}$ then $x,\,y$ are algebraic over $\mathbb{Q}$, and their degrees over $\mathbb{Q}$ are powers of $2$.
\end{lma}

\vspace{-17pt}
\section{Splitting Fields}
\vspace{-3pt}
\begin{dfn}[Extending Homomorphism]{dfn:extending-homomorphism}{6.1.1}
    \vspace{-5pt}
    \begin{vwcol}[widths={0.75,0.25}, rule=0pt]
    Let $\iota : K \to M$ and $\iota : K' \to M'$ be field extensions. Let $\psi : K \to K'$ be a homomorphism of fields. A homomorphism $\phi : M \to M'$ \textbf{extends} $\psi$ if the square commutes $(\phi \circ \iota = \iota' \circ \psi)$. 

    \columnbreak
% https://q.uiver.app/#q=WzAsNCxbMCwwLCJNIl0sWzEsMCwiTSciXSxbMCwxLCJLJyJdLFsxLDEsIksiXSxbMiwzLCJcXHBzaSIsMl0sWzIsMCwiXFxpb3RhIl0sWzAsMSwiXFxwaGkiXSxbMywxLCJcXGlvdGEnIiwyXV0=
$\begin{tikzcd}[ampersand replacement=\&,cramped,column sep=scriptsize]
	M \& {M'} \\
	{K'} \& K
	\arrow["\phi", from=1-1, to=1-2]
	\arrow["\iota", from=2-1, to=1-1]
	\arrow["\psi", from=2-1, to=2-2]
	\arrow["{\iota'}"', from=2-2, to=1-2]
\end{tikzcd}$

    \end{vwcol}
    \vspace{-22pt}
Usually we view $K$ as a subset of $M$, and $K'$ as a subset of $M'$, with inclusions $\iota$ and $\iota'$. In this case, for $\phi$ to extend $\psi$ means that
\[\pi(a) = \psi(a) \text{ for all } a\in K\]
\end{dfn}

\vspace{-7pt}
\begin{figure}[H]
    \begin{lma}[Extending Isomorphisms]{lma:induced-homomorphism-sum}{6.1.3}
        \vspace{-5pt}
        \textbf{Induced Homomorphism 2}: Let $M : K$ and $M' : K'$ be field extensions, let $\phi : K \to K'$ be a homomorphism, and let $\phi : M \to M'$ be a homomorphism extending $\psi$. Let $\alpha\in M$ and $f(t) \in K[t]$. Then
        \vspace{-5pt}
        \[f(\alpha) = 0 \iff (\psi_{\ast} f)(\phi(\alpha)) = 0.\]
        \vspace{-12pt}
        \textrule{Prop 6.1.6: Extending Isomorphisms}
        \par\vspace{-10pt}
        Let $\psi : K \to K'$ be an isomorphism of fields, $K(\alpha) : K$ a simple extension where $\alpha$ has minimal polynomial $m$ over $K$, and $K'(\alpha') : K'$ a simple extension where $\alpha'$ has minimal polynomial $\psi_{\ast}m$ over $K'$.
        \begin{vwcol}[widths={0.63,0.37}, rule=0pt]
            Then there is exactly one isomorphism $\phi : K(\alpha) \to K'(\alpha')$ that extends $\psi$ and satisfies $\phi(\alpha) = \alpha'$. \textit{(Dotted arrow: a map whose existence is part of the conclusion.)}
            \columnbreak
% https://q.uiver.app/#q=WzAsNCxbMCwwLCJLKFxcYWxwaGEpIl0sWzIsMCwiSycoXFxhbHBoYScpIl0sWzAsMSwiSyciXSxbMiwxLCJLJyJdLFsyLDMsIlxccHNpLCBcXGNvbmcnIiwyXSxbMiwwXSxbMCwxLCJcXHBoaSwgXFxjb25nJyIsMCx7InN0eWxlIjp7ImJvZHkiOnsibmFtZSI6ImRvdHRlZCJ9fX1dLFszLDFdXQ==
            \[\begin{tikzcd}[ampersand replacement=\&,cramped,column sep=scriptsize, row sep=scriptsize]
                {K(\alpha)} \& {K'(\alpha')} \\
                {K'} \& {K'}
                \arrow["{\phi}", "\cong"', dotted, from=1-1, to=1-2]
                \arrow[from=2-1, to=1-1]
                \arrow["\psi"', "\cong", from=2-1, to=2-2]
                \arrow[from=2-2, to=1-2]
            \end{tikzcd}\]
        \end{vwcol}
        \vspace{-10pt}
    \end{lma}
\end{figure}

\newpage

% ╭──────────────────────────────────────────────────────────╮
% │                          PAGE 4                          │
% ╰──────────────────────────────────────────────────────────╯

\begin{dfn}[Splitting Polynomial]{dfn:splitting-polynomial}{6.2.2}
    \vspace{-6pt}
    Let $f$ be a polynomial over a field $M$. Then $f$ \textbf{splits} in $M$ if
    \[f(t) = \beta(t - \alpha_{1}) \cdots (t - a_{n})\]
    \par\vspace{-5pt}
    for some $n \ne 0$ and $\beta, \alpha_{1},\dots,\alpha_{n}\in M$. Equivalently, $f$ splits in $M$ if all its irreducible factors in $M[t]$ are linear.
    \textrule{Definition 6.2.6: Splitting Field}
    \par\vspace{-14pt}
    Let $f$ be a nonzero polynomial over a field $K$. A \textbf{splitting field} of $f$ over $K$ is an extension $M$ of $K$ such that:
    \vspace{-7pt}
    \begin{enumerate-zero}
        \item $f$ splits in $M$
        \item $M = K(\alpha_{1},\dots,\alpha_{n})$, where $\alpha_{1},\dots,\alpha_{n}$ are the roots of $f$ in $M$. \textit{``If $L$ is a subfield of $M$ containing $K$, and $f$ splits in $L$, then $L = M$''}
    \end{enumerate-zero}
\end{dfn}

\vspace{-7pt}
\begin{lma}[Splitting Field Results]{lma:splitting-field-results}{6.2.A}
    \vspace{-5pt}
    \textbf{Lemma 6.2.10}: Let $f\ne 0$ be a polynomial over a field $K$. Then there exists a splitting field $M$ of $f$ over $K$ s.t. $[M : K] \le \deg(f)!$.

    \vspace{-2pt}
    \textrule{Prop 6.2.11: Splitting Fields and Isomorphisms}
    \par\vspace{-12pt}
    Let $\psi : K \to K'$ be an isomorphism of fields, $0 \ne f \in K[t]$, $M$ be a splitting field of $f$ over $K$, and $M'$ be a splitting field of $\psi_{\ast}f$ over $K'$. Then
    \vspace{-7pt}
    \begin{enumerate-tight}
        \item There exists an isomorphism $\phi : M \to M'$ extending $\psi$.\vspace{-2pt}
        \item There are at most $[M : K]$ such extensions $\phi$.
    \end{enumerate-tight}
    \vspace{-7pt}
    We often use this result when $K' = K$ and $\psi = \id_{K}$.
    \vspace{-1pt}
    \textrule{Theorem 6.2.13: Isos and Autos of a Splitting Field}
    \par\vspace{-12pt}
    Let $f$ be a nonzero polynomial over a field $K$. Then
    \vspace{-7pt}
    \begin{enumerate-zero}
        \item There exists a splitting field of $f$ over $K$ \vspace{-2pt}
        \item Any two splitting fields of $f$ are isomorphic over $K$\vspace{-2pt}
        \item When $M$ is a splitting field of $f$ over $K$,
            \vspace{-2pt}
            \[\text{num. of automorphisms of $M$ over $K$}\le [M : K] \le \deg(f)\]
    \end{enumerate-zero}
    \vspace{-5pt}
    \textrule{Lemma 6.2.14: Splitting Fields and Extensions}
    \vspace{-20pt}
    \begin{enumerate-zero}
        \item Let $M : S : K$ be field extensions, $0 \ne f \in K[t]$, and $Y \subseteq M$. Suppose that $S$ is the splitting field of $f$ over $K$. Then $S(Y)$ is the splitting field of $f$ over $K(Y)$ \vspace{-2pt}
        \item Let $f\ne 0$ be a polynomial over a field $K$, and let $L$ be a subfield of $\mathrm{SF}_{K}(f)$ containing $K$ (so that $\mathrm{SF}_{K}(f) : L : K$). Then $\mathrm{SF}_{K}(f)$ is the splitting field of $f$ over $L$.
    \end{enumerate-zero}
\end{lma}

\vspace{-7pt}
\begin{dfn}[Galois Group of an Extension]{dfn:galois-group-extension}{6.3.1}
    \vspace{-5pt}
    The \textbf{Galois Group} $\Gal(M : K)$ of a field extension $M : K$ is the group of automorphisms of $M$ over $K$, with composition as the group operation. In other words, an element of $\Gal(M : K)$ is an isomorphism $\theta : M \to M$ such that $\theta(a) = a$ for all $a\in K$.
    \textrule{Definition 6.3.5: Galois Group of a Polynomial}
    \par\vspace{-12pt}
    Let $f$ be a nonzero polynomial over a field $K$. The \textbf{Galois Group} $\Gal_{K}(f)$ of $f$ over $K$ is $\Gal(\SF_{K}(f) : K)$.
    \vspace{-2pt}
    \[\text{polynomial} \quad \longmapsto\quad \text{field extension}\quad\longmapsto\quad\text{group}\]
    \par\vspace{-10pt}
    \longrule{0.08ex}
    Via Theoerem $6.2.13$,
    \vspace{-2pt}
    \[\lvert \Gal_{K}(f) \rvert \le [\SF_{K}(f) :K] \le \deg(f)!\]
    In particular, $\Gal_{K}(f)$ is always a finite group.
\end{dfn}

% ╭──────────────────────────────────────────────────────────╮
% │                          PAGE 5                          │
% ╰──────────────────────────────────────────────────────────╯

\vspace{-7pt}
\begin{lma}[Restriction of Actions on GGs]{lma:action-restriction}{6.3.7}
    \vspace{-5pt}
    For a nonzero polynom $F$ over a field $K$, the action of $\Gal_{K}(f)$ on $\SF_{K}(f)$ \textbf{restricts} to an action on the set of roots of $f$ in $\SF_{K}(f)$.

    \par\vspace{-7pt}
    \longrule{0.08ex}
    \textbf{Terminology}: Given a group $G$ acting on a set $X$ and a subset $A \subseteq X$, the action \textbf{restricts} to $A$ if $ga \in A$,  $\forall g\in G$ and $a\in A$.

    \vspace{-3pt}
    \textrule{Lemma 6.3.8: Galois Actions are Faithful}
    \par\vspace{-12pt}
    Let $f$ be a nonzero polynomial over a field $K$. Then the action of $\Gal_{K}(f)$ on the roots of $f$ is \textbf{faithful}.
\end{lma}

\vspace{-7pt}
\begin{dfn}[Conjugacy for real this time]{dfn:conjugacy-2}{6.3.9}
    \vspace{-5pt}
    Let $M : K$ be a field extension, let $k \ge 0$, and let $(\alpha_{1},\dots,\alpha_{k})$ and $(\alpha_{1}',\dots,\alpha_{k}')$ be $k$-tuples of elements of $M$. Then $(\alpha_{1},\dots,\alpha_{k})$ and $(\alpha_{1}',\dots,\alpha_{k}')$ are \textbf{conjugate} over $K$ if for all $p\in K[t_{1},\dots,t_{k}]$,
    \vspace{-2pt}
    \[p(\alpha_{1},\dots,\alpha_{k})= 0 \iff p(\alpha_{1}',\dots,\alpha_{k}') = 0\]
    \par\vspace{-3pt}
    If $k=1$ we omit the brackets and say $\alpha$ and $\alpha'$ are conjugate.
\end{dfn}

\vspace{-7pt}
\begin{rem}[What The Galois Group Actually Means]{rem:what-galois-group-means}{6.3.B}
    \vspace{-5pt}
    An element of $\Gal_{K}(f)$ is completely determined by how it permutes the roots of $f$. So you can view elements of $\Gal_{K}(f)$ as \textit{being} permutations of the roots. However, not every permutation of the roots belongs to the Galois group. Suppose $f\in K[t]$ has distinct roots $\alpha_{1},\dots,\alpha_{k}$ in its splitting field. For each $\theta\in \Gal_{K}(f)$ there is a permutation $\sigma_{\theta}\in S_{k}$ defined by
    \[\theta(\alpha_{i}) = \alpha_{\sigma_{\theta}(i)} \quad\text{ for } i\in \{1,\dots,k\}\]
    Then $\Gal_{K}(f)$ is isomorphic to the subgroup $\{\sigma_{\theta} \mid \theta \in \Gal_{K}(f)\}$ of $S_{K}$. The isomorphism is given by $\theta \mapsto \sigma_{\theta}$.
\end{rem}

\vspace{-8pt}
\begin{ppn}[Permutation Definition of Galois]{ppn:permutation-galois-group}{6.3.10}
    \vspace{-5pt}
    Let $f$ be a nonzero polynomial over a field $K$ with distinct roots $\alpha_{1},\dots,\alpha_{k}$ in $\SF_{k}(f)$. Then
    \[\{\sigma\in S_{k} \mid (\alpha_{1},\dots,\alpha_{k})\text{ and }(\alpha_{\sigma(1)},\dots,\alpha_{\sigma(k)}) \text{ are conj. over $K$}\}\]
    is a subgroup of $S_{k}$ isomorphic to $\Gal_{K}(f)$
    \textrule{Corollary 6.3.12: Galois Groups and Extensions}
    \par\vspace{-12pt}
    Let $L : K$ be a field extension and $0 \ne f \in K[t]$. Then $\Gal_{L}(f)$ is isomorphic to a subgroup of $\Gal_{K}(f)$.
    \textrule{Corollary 6.3.14: Division of Roots in Galois}
    \par\vspace{-12pt}
    Let $f$ be a nonzero polynomial over a field $K$, with $k$ distinct roots in $\SF_{K}(f)$. Then $\lvert \Gal_{K}(f) \rvert$ divides $k!$.
\end{ppn}

% TODO: actually read page 97 lol
\vspace{-18pt}
\section{Preparation for the Fundamental Theorem}
\vspace{-5pt}
\begin{minipage}[c]{0.65\linewidth}
\begin{dfn}[Normal Extensions]{dfn:normal-extension}{7.1.1}
    \vspace{-5pt}
    An algebraic field extension $M : K$ is \textbf{normal} if for all $\alpha\in M$, the minimal polynomial of $\alpha$ splits in $M$. We also say \textbf{$M$ is normal over $K$} to mean that $M : K$ is normal.

    \textrule{Lemma 7.1.2}
    \par\vspace{-10pt}
    Let $M : K$ be an algebraic extension. Then $M : K$ is normal iff every irreducible polynomial over $K$ either has no roots in $M$ or splits in $M$. Put another way, normality means that any irreducible polynomial over $K$ with \textit{at least one} root in $M$ has \textit{all} its roots in $M$.
\end{dfn}

\vspace{-9pt}
\begin{thm-s}[Splitting and Normality]{thm:splitting-normal}{7.1.5}
    \vspace{-5pt}
    Let $M : K$ be a field extension. Then
    \vspace{-5pt}
    \begin{multline*}
        M = \SF_{K}(f) \text{ for some nonzero } f \in K[t]\\
        \iff M : K \text{ is finite and normal}
    \end{multline*}
    \vspace{-10pt}
    \textrule{Corollary 7.1.6}
    Let $M : L : K$ be field extensions. If $M : K$ is finite and normal then so is $M : L$.
    \par\vspace{-5pt}
    \longrule{0.08ex}
    \textbf{Warning}: This does \textit{not} follow that $L : K$ is normal.
    % TODO: warning 7.1.8
\end{thm-s}

\end{minipage} \begin{minipage}[c]{0.35\linewidth}
    \begin{thm-s}[Maps]{thm:thing}{7.1.5}
        \begin{sideways}
% https://q.uiver.app/#q=WzAsNixbMSwzLCJLIl0sWzAsMiwiSyhcXGRlbHRhKSJdLFswLDEsIk0gPSBLKFxcYWxwaGFfMSxcXGRvdHMsXFxhbHBoYV9uKSJdLFsyLDEsIk0gPSBLKFxcYWxwaGFfMSxcXGRvdHMsXFxhbHBoYV9uLFxcZXBzaWxvbikiXSxbMiwyLCJLKFxcZXBzaWxvbikiXSxbMiwwLCJcXFNGX00obSkiXSxbMCwxXSxbMSwyXSxbMiwzLCJcXHBoaSIsMCx7InN0eWxlIjp7ImJvZHkiOnsibmFtZSI6ImRvdHRlZCJ9fX1dLFs0LDNdLFswLDRdLFsxLDQsIlxcdGhldGEiLDIseyJzdHlsZSI6eyJib2R5Ijp7Im5hbWUiOiJkb3R0ZWQifX19XSxbMyw1XV0=
            \begin{tikzcd}[ampersand replacement=\&,cramped,column sep=small, row sep=scriptsize]
                \&\& {\SF_M(m)} \\
                {M = K(\alpha_1,\dots,\alpha_n)} \&\& {M = K(\alpha_1,\dots,\alpha_n,\epsilon)} \\
                {K(\delta)} \&\& {K(\epsilon)} \\
                \& K
                \arrow["\phi", dotted, from=2-1, to=2-3]
                \arrow[from=2-3, to=1-3]
                \arrow[from=3-1, to=2-1]
                \arrow["\theta"', dotted, from=3-1, to=3-3]
                \arrow[from=3-3, to=2-3]
                \arrow[from=4-2, to=3-1]
                \arrow[from=4-2, to=3-3]
            \end{tikzcd}
        \end{sideways}
\end{thm-s}
\end{minipage}

\vspace{-2pt}
\begin{ppn}[Conjugacy and Orbits]{ppn:conjugacy-orbits}{7.1.9}
    \vspace{-5pt}
    Let $M : K$ be a finite normal extension and $\alpha,\, \alpha'\in M$. Then
    \setlength{\jot}{0pt}
    \begin{equation*}
        \alpha \text{ and } \alpha' \text{ conjugate over }K
        \iff \alpha' = \phi(\alpha) \text{ for some } \phi \in \Gal(M : K)
    \end{equation*}

    \vspace{-5pt}
    \textrule{Corollary 7.1.11: Transitivity of Actions}
    \par\vspace{-12pt}
    Let $f$ be an irreducible polynomial over a field $K$. Then the action of $\Gal_{K}(f)$ on the roots of $f$ in $\SF_{K}(f)$ is transitive, i.e. for all $x, x'\in X$ there exists $g\in G$ such that $gx = x'$
\end{ppn}

% TODO: why is this spectactular?! also the fact at the end right before 7.1.14

\vspace{-7pt}
\begin{thm}[Quotients of Normal Extensions]{thm:quotient-field-extension}{7.1.15}
    \vspace{-5pt}
    Let $M : L : K$ be field extensions with $M : K$ finite and normal.
    \vspace{-7pt}
    \begin{enumerate-tight}
        \item $L : K$ is a normal extension $\iff$ $\phi L = L$ for all $\phi \in \Gal(M : K)$
        \item If $L : K$ is a normal extension then $\Gal(M : L)$ is a normal subgroup of $\Gal(M : K)$ and
            \vspace{-4pt}
            \[\frac{\Gal(M : K)}{\Gal(M : L)} \cong \Gal(L : K)\]
    \end{enumerate-tight}
\end{thm}

\vspace{-8pt}
\begin{dfn}[Separable Polynomial]{dfn:separable}{7.2.2}
    \vspace{-7pt}
    For a polynomial $f(t)\in K[t]$ and a root $\alpha$ of $f$ in some extension $M$ of $K$, we say that $\alpha$ is a \textbf{repeated} root if $(t - a)^{2} \mid f(t)$ in $M[t]$.
    \par\vspace{-7pt}
    \longrule{0.08ex}
    An irreducible polynomial over a field is \textbf{separable} if it has no repeated roots in its splitting field. Equivalently, an irreducible polynomial $f\in K[t]$ is separable if it splits into \textit{distinct} linear factors in $\SF_{K}(f)$:
    \vspace{-2pt}
    \[f(t) = a(t - \alpha_{1}) \cdots (t - a_{n})\]
    \vspace{-2pt}
    for some $a\in K$ and \textit{distinct} $\alpha_{1},\dots,\alpha_{n}\in \SF_{K}(f)$. Put another way, an irreducible $f$ is separable iff it has $\deg(f)$ distinct roots in its splitting field.
    \newline\textbf{Warning}: this only works for \textit{irreducible polynomials}.
\end{dfn}

\vspace{-7pt}
\begin{dfn}[Formal Derivative]{dfn:formal-derivative}{7.2.6}
    \vspace{-7pt}
    For a field $K$ and $f(t) = \sum_{i = 0}^{n} i_{i}t^{i}\in K[t]$, the \textbf{formal derivative} of $f$ is
    \vspace{-7pt}
    \[(Df)(t) = \sum_{i = 1}^{n} i a_{i}t^{i-1} \in K[t]\]
    \vspace{-5pt}
    \textrule{Lemma 7.2.7: Basic Derivative Rules}
    Let $K$ be a field. Then
    \vspace{-2pt}
    \[D(f + g) = Df + Dg, \quad D(fg) = f \cdot Dg + Df \cdot g, \quad Da=0\]
    \par\vspace{-4pt}
    for all $f,\,g\in K[t]$ and $\alpha\in K$.
\end{dfn}

\vspace{-7pt}
\begin{lma}[Separability Results]{lma:separability-results}{7.2.9}
    \vspace{-7pt}
    \textrule{Lemma 7.2.9: Repeated Roots}
    \par\vspace{-12pt}
    Let $f$ be a nonzero polynomial over a field $K$. The following are equivalent:
    \vspace{-5pt}
    \begin{enumerate-tight}
        \item $f$ has a repeated root in $\SF_{K}(f)$\vspace{-2pt}
        \item $f$ and $Df$ have a common root in $\SF_{K}(f)$\vspace{-2pt}
        \item $f$ and $Df$ have a nonconstant common factor in $K[t]$
    \end{enumerate-tight}
    \vspace{-5pt}
    \textrule{Lemma 7.2.10: Inseparability of Zero}
    \par\vspace{-12pt}
    Let $f$ be an irreducible polynomial over a field. $f$ is inseparable iff $Df = 0$
    \textrule{Corollary 7.2.11: Separability of Irreducibles}
    \par\vspace{-12pt}
    Let $K$ be a field.
    \vspace{-7pt}
    \begin{enumerate-zero}
        \item If $\Char K = 0$, every irreducible polynomial over $K$ is separable.\vspace{-2pt}
        \item If $\Char K = p > 0$, an irreducible polynomial $f\in K[t]$ is inseparable iff\vspace{-2pt}
            \[f(t) = b_{0} + b_{1}t^{p} + \cdots + b_{r}t^{rp}\]
            \vspace{-2pt}for some $b_{0},\dots,b_{r}\in K$
    \end{enumerate-zero}
    \vspace{-7pt}
    i.e. the only irreducible inseparable polynomials are ones in $t^{p}$ in $\Char p$.
\end{lma}

\vspace{-7pt}
\begin{dfn}[Separable Elements]{dfn:separable-element}{7.2.13}
    \vspace{-7pt}
    Let $M : K$ be an algebraic extension. An element of $M$ is \textbf{separable} over $K$ if its miminal polynomial over $K$ is separable. The extension $M : K$ is \textbf{separable} if every element of $M$ is separable over $K$.
    \par\vspace{-7pt}
    \longrule{0.08ex}
    \textbf{Lemma 7.2.16}: Let $M : L : K$ be field extensions, with $M : K$ algebraic. If $M : K$ is separable then so are $M : L$ and $L : K$.
    \textrule{Proposition 7.2.17: Splitting Field Isomorphisms}
    \par\vspace{-12pt}
    Let $\phi : K \to K'$ be an isomorphism of fields, let $0 \ne f \in K[t]$, let $M$ be a splitting field of $f$ over $K$, and let $M'$ be a splitting field of $\phi_{\ast}f$ over $K'$. Suppose that the extension $M' : K'$ is separable. Then there are exactly $[M : K]$ isomorphisms $\phi : M \to M'$ extending $\psi$.
    \textrule{Theorem 7.2.18: Size of Galois Extensions}
    \par\vspace{-12pt}
    $\lvert \Gal(M : K) \rvert = [M : K]$ for every finite normal separable extension $M : K$
\end{dfn}

\vspace{-9pt}
\begin{figure}[H]
\begin{lma}[Fixed Fields]{lma:fixer-subfield}{7.3.1}
    \vspace{-7pt}
    $\Aut(M)$ is the group of automorphisms of a field $M$, which acts naturally on $M$. Given $S \subseteq \Aut(M)$, $\Fix(S)$ is the set of elements of $M$ fixed by $S$.
    \par\vspace{-5pt}
    \longrule{0.08ex}
    $\Fix(S)$ is a subfield of $M$, for any $S \subseteq \Aut(M)$.
    \textrule{Thm 7.3.3: Size of Fixed Field}
    \par\vspace{-12pt}
    Let $M$ be a field and $H$ a finite subgroup of $\Aut(M)$. Then $[M : \Fix(H)] \le \lvert H \rvert$.
    This is actually an equality.
    \textrule{Fixed Field Normal Extensions}
    \par\vspace{-12pt}
    Let $M : K$ be a finite normal extension and $H$ a normal subgroup of $\Gal(M : K)$. Then $\Fix(H)$ is a normal extension of $K$.
\end{lma}
\end{figure}

\newpage
\section{The Fundamental Theorem of Galois Theory!}

\begin{rem}[Intermediate Field]{rem:intermediate-field}{8.1.A}
    \vspace{-5pt}
    Let $M : K$ be a field extension, with $K$ viewed as a subfield of $M$. An \newline\textbf{intermediate field} of $M : K$ is a subfield of $M$ containing $K$. 
    \vspace{-15pt}
    \setlength{\columnseprule}{0.5pt}
    \begin{multicols}{2}
        Write
    \[\mathscr{F} = \{\text{intermediate fields of } M : K\}\]
        
    For $L \in \mathscr{F}$, we draw diagrams like this:
% https://q.uiver.app/#q=WzAsMyxbMCwwLCJNIl0sWzAsMSwiTCJdLFswLDIsIksiXSxbMSwyLCIiLDAseyJzdHlsZSI6eyJoZWFkIjp7Im5hbWUiOiJub25lIn19fV0sWzAsMSwiIiwwLHsic3R5bGUiOnsiaGVhZCI6eyJuYW1lIjoibm9uZSJ9fX1dXQ==
\[\begin{tikzcd}[cramped, row sep=small]
	M \\
	L \\
	K
	\arrow[no head, from=1-1, to=2-1]
	\arrow[no head, from=2-1, to=3-1]
\end{tikzcd}\]
with the bigger fields \textit{higher up}.
\columnbreak

We also write
\[\mathscr{G} = \{\text{subgroups of }\Gal(M : K)\}\]
For $H \in \mathscr{G}$, we draw diagrams like this:
% https://q.uiver.app/#q=WzAsMyxbMCwwLCJJIl0sWzAsMSwiSCJdLFswLDIsIlxcR2FsKE0gOiBLKSJdLFsxLDIsIiIsMCx7InN0eWxlIjp7ImhlYWQiOnsibmFtZSI6Im5vbmUifX19XSxbMCwxLCIiLDAseyJzdHlsZSI6eyJoZWFkIjp7Im5hbWUiOiJub25lIn19fV1d
\[\begin{tikzcd}[cramped, row sep=small]
	I \\
	H \\
	{\Gal(M : K)}
	\arrow[no head, from=1-1, to=2-1]
	\arrow[no head, from=2-1, to=3-1]
\end{tikzcd}\]
with the bigger groups \textit{lower down}.
    \end{multicols}
\vspace{-10pt}
For $L \in \mathscr{F}$, the group $\Gal(M : K)$ consists of all automorphisms $\phi$ of $M$ that fix each element of $L$. Since $K \subseteq L$, any such $\phi$ certainly fixes each element of $K$. Hence, $\Gal(M : L)$ is a subgroup of $\Gal(M : K)$. this process defines a function
\begin{align*}
    \Gal(M : -) : \mathscr{F} &\mapsto \mathscr{G} \\
    L &\mapsto \Gal(M : L)
\end{align*}
In the expression $\Gal(M : -)$, the symbol $-$ should be seen as a blank space into which arguments can be inserted.

In the other direction, for $H \in \mathscr{G}$, the subfield $\Fix(H)$ of $M$ contains $K$. Indeed, $H \subseteq \Gal(M : K)$, and by definition, every element of $\Gal(M : K)$ fixes every element of $K$, so $\Fix(H) \supseteq K$. Hence, $\Fix(H)$ is an intermediate field of $M : K$. This process defines a function
\begin{align*}
    \Fix : \mathscr{G} &\mapsto \mathscr{F} \\
    H &\mapsto \Fix(H)
\end{align*}
We have now defined functions
% https://q.uiver.app/#q=WzAsMixbMCwwLCJcXG1hdGhzY3J7Rn0iXSxbMiwwLCJcXG1hdGhzY3J7R30iXSxbMCwxLCJcXEdhbChNIDogLSkiLDAseyJvZmZzZXQiOi0xfV0sWzEsMCwiXFxGaXgiLDAseyJvZmZzZXQiOi0xfV1d
\[\begin{tikzcd}[cramped]
	{\mathscr{F}} && {\mathscr{G}}
	\arrow["{\Gal(M : -)}", shift left, from=1-1, to=1-3]
	\arrow["\Fix", shift left, from=1-3, to=1-1]
\end{tikzcd}\]
\[\]
\end{rem}

% TODO: warning 8.1.1

\begin{lma}[Ordering of Intermediates]{lma:intermediate-ordering}{8.1.2}
    \vspace{-5pt}
    Let $M : K$ be a field extension, and define $\mathscr{F}$ and $\mathscr{G}$ as above.

    \begin{minipage}[c]{0.65\linewidth}
    \begin{enumerate-zero}
        \item For $L_{1},\,L_{2}\in \mathscr{F}$,
            \[L_{1} \subseteq L_{2} \implies \Gal(M : L_{1}) \supseteq \Gal(M : L_{2})\]
            For $H_{1},\,H_{2}\in \mathscr{G}$,
            \[H_{1} \subseteq H_{2} \implies \Fix(H_{1}) \supseteq \Fix(H_{2})\]
        \item For $L \in \mathscr{F}$ and $H \in \mathscr{G}$,
            \[L \subseteq \Fix(H) \iff H \supseteq \Gal(M : L)\]
        \item For all $L\in \mathscr{F}$, $L \subseteq \Fix(\Gal(M : L))$ \newline
            For all $H\in \mathscr{G}$, $H \subseteq \Gal(M : \Fix(H))$
    \end{enumerate-zero}
    \end{minipage}\vline\begin{minipage}[c]{0.35\linewidth}
% https://q.uiver.app/#q=WzAsOCxbMCwwLCJNIl0sWzAsMSwiTF8yIl0sWzAsMiwiTF8xIl0sWzAsMywiSyJdLFsxLDMsIlxcR2FsKE0gOiBLKSJdLFsxLDIsIlxcR2FsKE0gOiBMXzEpIl0sWzEsMSwiXFxHYWwoTSA6IExfMikiXSxbMSwwLCIxIl0sWzAsMSwiIiwwLHsic3R5bGUiOnsiaGVhZCI6eyJuYW1lIjoibm9uZSJ9fX1dLFsxLDIsIiIsMCx7InN0eWxlIjp7ImhlYWQiOnsibmFtZSI6Im5vbmUifX19XSxbMiwzLCIiLDAseyJzdHlsZSI6eyJoZWFkIjp7Im5hbWUiOiJub25lIn19fV0sWzQsNSwiIiwwLHsic3R5bGUiOnsiaGVhZCI6eyJuYW1lIjoibm9uZSJ9fX1dLFs1LDYsIiIsMCx7InN0eWxlIjp7ImhlYWQiOnsibmFtZSI6Im5vbmUifX19XSxbNiw3LCIiLDAseyJzdHlsZSI6eyJoZWFkIjp7Im5hbWUiOiJub25lIn19fV1d
\[\begin{tikzcd}[ampersand replacement=\&,cramped, column sep=scriptsize]
	M \& 1 \\
	{L_2} \& {\Gal(M : L_2)} \\
	{L_1} \& {\Gal(M : L_1)} \\
	K \& {\Gal(M : K)}
	\arrow[no head, from=1-1, to=2-1]
	\arrow[no head, from=2-1, to=3-1]
	\arrow[no head, from=2-2, to=1-2]
	\arrow[no head, from=3-1, to=4-1]
	\arrow[no head, from=3-2, to=2-2]
	\arrow[no head, from=4-2, to=3-2]
\end{tikzcd}\]
        
    \end{minipage}
    % Warning 8.1.3
\end{lma}

\begin{rem}[Galois Correspondence]{rem:galois-correspondence}{8.1.B}
    \vspace{-5pt}
    The functions
    \vspace{-2pt}
% https://q.uiver.app/#q=WzAsMixbMCwwLCJcXG1hdGhzY3J7Rn0iXSxbMiwwLCJcXG1hdGhzY3J7R30iXSxbMCwxLCJcXEdhbChNIDogLSkiLDAseyJvZmZzZXQiOi0xfV0sWzEsMCwiXFxGaXgiLDAseyJvZmZzZXQiOi0xfV1d
\[\begin{tikzcd}[cramped]
	{\mathscr{F}} && {\mathscr{G}}
	\arrow["{\Gal(M : -)}", shift left, from=1-1, to=1-3]
	\arrow["\Fix", shift left, from=1-3, to=1-1]
\end{tikzcd}\]
    are called the \textbf{Galois correspondence} for $M : K$. This terminology is mostly used in the case where the functions are \textbf{mutually inverse}, i.e.
    \[L = \Fix(\Gal(M : L)), \quad H = \Gal(M : \Fix(H))\]
    for all $L\in \mathscr{F}$ and $H\in \mathscr{G}$. In both cases, the LHS is a subset of the RHS. (But they are not always equal.) If $\Gal(M : -)$ and $\Fix$ \textit{are} mutually inverse then they set up a one-to-one correspondence between $\mathscr{F}$ and $\mathscr{G}$.
\end{rem}

\begin{thm-s}[The Fundamental Theorem of Galois Theory]{thm:the-theorem}{8.2.1}
    \vspace{-5pt}
    Let $M : K$ be a finite normal separable extension. Write
    \begin{align*}
        \mathscr{F} &= \{\text{intermediate fields of } M : K\}\\
        \mathscr{G} &= \{\text{subgroups of } \Gal(M : K)\}\\
    \end{align*}
    \vspace{-25pt}
    \begin{enumerate-zero}
        \item The functions $\begin{tikzcd}[cramped]
                {\mathscr{F}} && {\mathscr{G}}
                \arrow["{\Gal(M : -)}", shift left, from=1-1, to=1-3]
                \arrow["\Fix", shift left, from=1-3, to=1-1]
            \end{tikzcd}$ are mutually inverse.\vspace{-1pt}
        \item $\lvert \Gal(M : L) \rvert = [M : L]$ for all $L\in \mathscr{F}$ and $[M : \Fix(H)] = \lvert H \rvert$ for all $H\in \mathscr{G}$
        \item Let $L\in \mathscr{F}$. Then
            \vspace{-5pt}
            \begin{multline*}
                L \text{ is a normal extension of } K \iff \\
                \Gal(M : L) \text{ is a normal subgroup of } \Gal(M : K).
            \end{multline*}
            \par\vspace{-5pt}
            and in that case,
            \vspace{-3pt}
            \[\frac{\Gal(M : K)}{\Gal(M : L)} \cong \Gal(L : K)\]
    \end{enumerate-zero}
\end{thm-s}

\vspace{-7pt}
\begin{rem}[Useful Results]{rem:useful-results}{8.2.3}
    \vspace{-5pt}
    \begin{enumerate-tight}
        \item Lemmas 6.3.7 and 6.3.8 say that $\Gal_{K}(f)$ acts faithfully on the set of roots of $f$ in $\SF_{K}(f)$. i.e. an element of the Galois group can be understood as a permutation of the roots
        \item Corollary $6.3.14$ states that $\lvert \Gal_{K}(f) \rvert$ divides $k!$, where $k$ is the number of distinct foots of $f$ in its splitting field.
        \item Let $\alpha$ and $\beta$ be roots of $f$ in $\SF_{K}(f)$. Then there is an element of the Galois group mapping $\alpha$ to $\beta$ iff $\alpha$ and $\beta$ are conjugate over $K$ (have the same minimal polynomial). This follows from Prop 7.1.9.
        \item In particular, when $f$ is irreducible, the action of the Galois group on the roots is transitive (Corollary 7.1.11).
    \end{enumerate-tight}
\end{rem}

\vspace{-7pt}
\begin{crl}[Automorphisms with FTGT]{crl:automorphisms-ftgt}{8.2.7}
    \vspace{-5pt}
    Let $M : K$ be a finite normal separable extension. Then for every $\alpha\in M \backslash K$, there is some automorphism $\phi$ of $M$ over $K$ such that $\phi(\alpha) \ne \alpha$
\end{crl}

% 8.3

% Warning 9.1.1

\vspace{-18pt}
\section{Solvability by Radicals}
\vspace{-3pt}
\begin{dfn}[Radical Number]{dfn:radical-number}{9.1.2}
    \vspace{-5pt}
    Let $\Qrad$ be the smallest subfield of $\mathbb{C}$ such that for $\alpha\in \mathbb{C}$,
    \[\alpha^{n} \in \Qrad \text{ for some }n \ge 1 \implies \alpha\in \Qrad.\]
    A complex number is \textbf{radical} if it belongs to $\Qrad$

    \textrule{Definition 9.1.5: Solvability by Radicals}
    \par\vspace{-12pt}
    A nonzero polynomial over $\mathbb{Q}$ is \textbf{solvable by radicals} if all of its complex roots are radical.
\end{dfn}

\vspace{-7pt}
\begin{lma}[Abelian Groups]{lma:abelian-groups}{9.1.6}
    \vspace{-5pt}
    \textbf{Lemma 9.1.6}: For all $n \ge 1$, the group $\Gal_{\mathbb{Q}}(t^{n} - 1)$ is abelian.

    \textbf{Lemma 9.1.8}: Let $K$ be a field and $n\ge 1$. Suppose that $t^{n} - 1$ splits in $K$. Then $\Gal_{K}(t^{n} - a)$ is abelian for all $a\in K$.
\end{lma}

\vspace{-7pt}
\begin{dfn}[Solvable Extension]{dfn:solvable-extension}{9.2.1}
    \vspace{-5pt}
    Roughly, the diagram of solvable polynomials is
    \[\text{solvable polynomial} \longmapsto \text{solvable extension} \longmapsto \text{solvable group}\]
    In other words, we define ``solvable extension'' in such a way that
    \vspace{-7pt}
    \begin{enumerate-tight}
        \item If $f\in \mathbb{Q}[t]$ is a polynomial solvable by radicals then $\SF_{\mathbb{Q}}(f) : \mathbb{Q}$ is a solvable extension.\vspace{-3pt}
        \item If $M : K$ is a solvable extension then $\Gal(M : K)$ is a solvable group. Hence, if $f$ is solvable by radicals then $\Gal_{\mathbb{Q}}(f)$ is solvable.
    \end{enumerate-tight}
    \vspace{-12pt}
    \longrule{0.08ex}
    Let $M : K$ be a finite normal separable extension. Then $M : K$ is \textbf{solvable} (or $M$ is \textbf{solvable over $K$}) if there exist $r \ge 0$ and intermediate fields
    \[K = L_{0} \subseteq L_{1} \subseteq \cdots \subseteq L_{r} = M\]
    s.t. $L_{i} : L_{i - 1}$ is normal and $\Gal(L_{i} : L_{i - 1})$ is abelian for each $i\in \{1,. .,r\}$.
\end{dfn}

\begin{lma}[Solvable Results]{lma:solvable-results}{9.2.A}
    \vspace{-5pt}
    \textrule{Lemma 9.2.4: Solvable Galois and Extensions}
    \par\vspace{-12pt}
    Let $M : K$ be a finite normal separable extension. Then
    \[M : K \text{ is solvable } \iff \Gal(M : K) \text{ is solvable}\]
    \vspace{-8pt}
    \textrule{Lemma 9.2.6: Finite Normal Results}
    Let $M : K$ be a field extension and let $L$ and $L'$ be intermediate fields.
    \vspace{-5pt}
    \begin{enumerate-zero}
        \item If $L : K$ and $L' : K$ are finite and normal, then so is $LL' : K$.
        \item If $L : K$ is finite and normal, then so is $LL' : L'$.
        \item If $K : K$ is finite, normal with abelian Galois group, then so is $LL' : L'$
    \end{enumerate-zero}
    % TODO: the diagram
    \vspace{-5pt}
    \textrule{Lemma 9.2.7: Iterated Subfields}
    Let $L$ and $M$ be subfields of $\mathbb{C}$ such that the extensions $L : \mathbb{Q}$ and $M : \mathbb{Q}$ are finite, normal, and solvable. Then there is some subfield $M$ of $\mathbb{C}$ such that $N : \mathbb{Q}$ is finite, normal, and solvable and $L,\, M \subseteq N$.
    \textrule{Working with the Rationals}
    \textbf{Lemma 9.2.8}: Let $\Qsol$ be defined as
    \vspace{-5pt}
   \begin{multline*}
       \Qsol = \{\alpha\in \mathbb{C} \mid \alpha\in L \text{ for some subfield } L \subseteq \mathbb{C} \\\text{that is finite, normal, and solvable over } \mathbb{Q}\}.
   \end{multline*}
   Then $\Qsol$ is a subfield of $\mathbb{C}$.

   \textbf{Lemma 9.2.9}: Let $\alpha\in \mathbb{C}$ and $n \ge 1$. If $\alpha^{n}\in \Qsol$ then $\alpha\in \Qsol$.

   \textbf{Proposition 9.2.12}: $\Qrad \subseteq \Qsol$. That is, every radical number is contained in some subfield of $\mathbb{C}$ that is a finite, normal, solvable extension of $\mathbb{Q}$.
\end{lma}

\vspace{-5pt}
\begin{thm}[Solvability of Galois Group]{thm:poly-implies-gal-solvable}{9.2.13}
    \vspace{-5pt}
    Let $0 \ne f \in \mathbb{Q}[t]$. If the polynomial $f$ is solvable by radicals then the group $\Gal_{\mathbb{Q}}(f)$ is solvable.
\end{thm}

\vspace{-5pt}
\begin{lma}[Unsolvable Polynomials]{lma:unsolvable-polynomials}{9.3}
    \vspace{-5pt}
    \textbf{Lemma 9.3.1}: Let $f$ be an irreducible polynomial over a field $K$, with $\SF_{K}(f) : K$ separable. Then $\deg(f)$ divides $\lvert \Gal_{K}(f) \rvert.$
    
    \textbf{Lemma 9.3.2}: For $n\ge 2$, the symmetric group $S_{n}$ is generated by $(12)$ and $(12 \dots n)$.

    \textbf{Lemma 9.3.3}: Let $p$ be a prime number, and let $f\in \mathbb{Q}[t]$ be an irreducible polynomial of degree $p$ with exactly $p - 2$ real roots. Then $\Gal_{\mathbb{Q}}(f) \cong S_{p}$.
\end{lma}

\vspace{-5pt}
\begin{thm}[Unsolvability of the Quintics]{thm:quintics-are-unsolvable}{9.3.5}
    \vspace{-5pt}
    Not every polynomial over $\mathbb{Q}$ of degree $5$ is solvable by radicals.
\end{thm}

\vspace{-18pt}
\section{Finite Fields}
\vspace{-3pt}
\begin{lma}[Classification of the Finite Fields]{lma:fin-fields}{10.1}
    \vspace{-5pt}
    \textbf{Lemma 10.1.1}: Let $M$ be a finite field. Then $\Char M$ is a prime number $p$, and $\lvert M \rvert = p^{n}$ where $n = [M : \mathbb{F}_{p}] \ge 1$. In particular, the order of a finite field is a prime power.

    \textbf{Lemma 10.1.5}: Let $p$ be a prime number and $n \ge 1$. Then the splitting field of $t^{p^{n}} - t$ over $\mathbb{F}_{p}$ has order $p^{n}$.

    \textbf{Lemma 10.1.6} Let $M$ be a finite field of order $q$. Then $\alpha^{q} = \alpha$ for all $\alpha\in M$.

    \textbf{Lemma 10.1.8}: Every finite field of order $q$ is a splitting field of $t^{q} - t$ over $\mathbb{F}_{p}$
    \textrule{Theorem 10.1.9: Classification of Finite Fields}
    \vspace{-18pt}
    \begin{enumerate-tight}
        \item Every finite field has order $p^{n}$ for some prime $p$ and integer $n\ge 1$.
        \item For each prime $p$ and integer $n \ge 1$, there is exactly one field of order $p^{n}$, up to isomorphism. It has characteristic $p$ and is a splitting field for $t^{p^{n}} - t$ over $\mathbb{F}_{p}$.
    \end{enumerate-tight}
\end{lma}

\begin{lma}[Multiplicative Structure]{lma:multiplicative-structure}{10.2}
    \textbf{Proposition 10.2.1}: For an arbitrary field $K$, every finite subgroup of $K^{\times}$ is cyclic. In particular, if $K$ is finite, then $K^{\times}$ is cyclic.

    \textbf{Corollary 10.2.5}: Every extension of one finite field over another is simple.

    \textbf{Corollary 10.2.8}: For every prime number $p$ and integer $n\ge 1$, there exists an irreducible polynomial over $\mathbb{F}_{p}$ of degree $n$.
\end{lma}

\begin{lma}[Galois Groups for Finite Fields]{lma:finfield-galois}{10.3}
    \textbf{Lemma 10.3.2}: Let $M : K$ be a field extension.
    \begin{enumerate-tight}
        \item If $K$ is finite then $M : K$ is separable.
        \item If $M$ is also finite then $M : K$ is finite and normal.
    \end{enumerate-tight}

    \textbf{Proposition 10.3.3}: Let $p$ be a prime and $n \ge 1$. Then $\Gal(\mathbb{F}_{p^{n}} : \mathbb{F}_{p})$ is cyclic of order $n$, generated by the Frobenius Automorphism of $\mathbb{F}_{p^{n}}$

    \textbf{Proposition 10.3.6}: Let $p$ be a prime and $n \ge 1$. Then $\mathbb{F}_{p^{n}}$ has exactly one subfield of order $p^{m}$ for each divisor $m$ of $n$, and no others. It is
    \[\{\alpha \in \mathbb{F}_{p^{n}} : \alpha^{p^{m}} = \alpha\}\]

    \textbf{Proposition 10.3.8}: Let $M : K$ be a field extension with $M$ finite. Then $\Gal(M : K)$ is cyclic of order $[M : K]$.
\end{lma}


\end{multicols}
\end{document}
