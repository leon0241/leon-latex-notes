\documentclass[landscape, 8pt]{extarticle}

\usepackage{../../preamble}
\usepackage{symbols}


\begin{document}
\setlength{\abovedisplayskip}{3.5pt}
\setlength{\belowdisplayskip}{3.5pt}
\setlength{\abovedisplayshortskip}{3.5pt}
\setlength{\belowdisplayshortskip}{3.5pt}

\begin{multicols}{3}
\raggedcolumns


\section*{\huge General Topology Notes}
Made by Leon :) \textit{Note: Any reference numbers are to the lecture notes}

\section{Topological Spaces and Examples}

\begin{dfn}[Topological Space]{dfn:topology}{1.1}
    A \textbf{topological space} is a pair $(X, \mathcal{T})$, where $X$ is a nonempty set, and $\mathcal{T}$ is a collection of subsets of $X$ which satisfies:
    \begin{enumerate-a-tight}
        \item $\emptyset \in \mathcal{T}$ and $X \in \mathcal{T}$
        \item if $U_{\lambda}\in \mathcal{T}$ for each $\lambda\in \Lambda$ (where $\Lambda$ is some indexing set), then $\bigcup_{\lambda\in \Lambda} U_{\lambda} \in \mathcal{T}$
        \item if $U_{1},\,U_{2}\in \mathcal{T}$, then $U_{1} \cap U_{2} \in \mathcal{T}$
    \end{enumerate-a-tight}
    The collection $\mathcal{T}$ is called the \textbf{topology} of the topological space, and the members of $\mathcal{T}$ are called the \textbf{open sets} of the topology
\end{dfn}

\begin{xmp}[Euclidean Spaces]{dfn:euclidean-spaces}{1.7}
    Let $\mathbb{R}^{n}$ enote the $n$-dimensional Euclidean vector space with elements $x = (x_{1},\,x_{2},\dots,x_{n})$ and $x_{i}\in\mathbb{R}$, and let
    \[\lvert x \rvert = \sqrt{\sum_{i = 1}^{n} x_{i}^{2}} \ge 0\]
    be the length of $x$. ($\mathbb{R}^{1} = \mathbb{R}$ is the real line). A subset $U$ of $\mathbb{R}^{n}$ is \textbf{open (for the usual topology)} iff for each $a\in U$ there exists an $r > 0$ such that
    \[\lvert x - a \rvert < r \implies x \in U.\]
    The collection of open sets thus defined is called the \textbf{usual topology} on $\mathbb{R}^{n}$. Note that open balls $B(y, \rho) = \{x\in \mathbb{R}^{n} : \lvert x - y \rvert < \rho\}$ are open sets under this definition.
\end{xmp}


\begin{xmp}[Metric Spaces]{def:metric}{1.8}
    A \textbf{metric space} $(X, d)$ is a nonempty set $X$ together with a function $d : X \times X \to \mathbb{R}$ with the following properties:
    \begin{enumerate-a-tight}
        \item $d(x,y)\ge 0$ and $d(x,y)=0 \iff x = y$
        \item $d(x,y)=d(y,x)$
        \item $d(x,y)\le d(x,z)+d(z,y)$ (Triangle Inequality)
    \end{enumerate-a-tight}
    The function $d$ is called the \textbf{metric}.
    \longrule{0.08ex}
    Let $(X, d)$ be a metric space, $x$ be a point in $X$, and $r > 0$. The \textbf{open ball} with center $x$ and radius $r$ is defined by
    \[B(x,r) = \{y,,\in X: d(x,y) < r\}.\]
    A subset $U$ of $X$ is \textbf{open (in the metric topology given by d)} iff for each $a\in U$ there is an $r > 0$ such that $B(a,r) \subseteq U$. Just like euclidean spaces, open balls are open in this sense.
\end{xmp}

\begin{xmp}[Other Topologies and Metrics]{xmp:standard-tops-and-metrics}{}
    If $(X, \mathcal{T})$ is a topological space, and if $X$ admits a metric whose metric topology is precisely $\mathcal{T}$, then we say that $(X, \mathcal{T})$ is \textbf{metrisable}
    \begin{itemize-tight}
        \item Euclidean spaces with their usual topologies are metrisable.
        \item[\textbf{1.9)}] The \textbf{Discrete Topology} is the topology of all subsets of a set $X$. We can define the \textbf{discrete metric} of $X$ to be
            \[d(x, y) = \begin{cases}
                0 & \text{if } x = y \\
                1 & \text{otherwise}
            \end{cases}.\]
        \item[\textbf{1.10)}] The \textbf{Trivial} or \textbf{Indiscrete Topology} is the topology $\mathcal{T} := \{\emptyset, X\}$ for a set $X$. This is a non-metrisable topology when $X$ has more than one member.
        \item[\textbf{1.14)}] Let $X = \{a,\,b,\,c\}$, where $a,\,b,\,c$ are distinct. Then
            \[\mathcal{T} = \{\emptyset, X, \{a\},\,\{a,b\},\,\{a,c\}\}\]
            is a topology on $X$
        \item[\textbf{1.15)}] Give $\mathbb{R}$ the topolgoy whose open subsets $U \subseteq \mathbb{R}$ are precisesly the subsets with finite complement $\mathbb{R} \backslash U$, or $U = \emptyset$. Then $\mathbb{R}$ with this topology is not metrisable. This is an example of a \textbf{Zariski Topology}
    \end{itemize-tight}
\end{xmp}

\begin{ppn}[Topology Equality]{ppn:topology-equality}{1.11}
    Let $d,\,d'$ be metrics on the same set $X$, and let $\mathcal{T},\,\mathcal{T}'$ be the corresponding metric topologies. If for real numbers $A,\, B > 0$ we have
    \[d(x, y) \le Ad'(x, y),\, d'(x, y) \le Bd(x, y) \text{ for all } x,\,y\in X,\]
    then $\mathcal{T} = \mathcal{T}'$.
\end{ppn}

\begin{xmp}[Example of Topology Equality]{xmp:topology-equality}{1.12}
    \begin{itemize-tight}
        \item The \textbf{Euclidean metric} on $\mathbb{R}^{n}$ is defined as:
            \[d\bigl((x_{1},x_{2},\dots,x_{n}),\, (y_{1},y_{2},\dots,y_{n})\bigr) = \sqrt{\sum_{i = 1}^{n}(x_{i} - y_{i})^{2}}\]
        \item The \textbf{Box metric} on $\mathbb{R}^{n}$ is defined as:
            \[d(x, y) \le \sqrt{n} d'(x,y),\, d'(x,y) \le d(x, y)\]
    \end{itemize-tight}
    By \ref{ppn:ppn:topology-equality}, these have the same topology.
\end{xmp}

\begin{dfn}[Subspace Topology]{dfn:subspace-topology}{1.16}
    Let $(X, \mathcal{T})$ be a topological space, and let $A \subseteq X$ be any subset. Then teh \textbf{subspace topology} on $A$ consists of all sets of the form $U \cap A$ where $U \in \mathcal{T}$.
\end{dfn}

\begin{dfn}[Closed Set]{dfn:closed-set}{1.17}
    Let $(X, \mathcal{T})$ be a topological space. A subset $A \subseteq X$ is \textbf{closed} iff its complement $X \backslash A := \{x \in X \mid x \not\in A\}$ is open in $X$. Note that a set being \textit{closed} does not mean it isn't \textit{open}. Sets that are both \textit{closed} and \textit{open} are called \textbf{clopen}.
\end{dfn}

\begin{thm}[Properties of open and closed sets]{thm:open-set-props}{1.19}
    \vspace{-5pt}
    Let $(X,\mathcal{T})$ be a topological space.
    \vspace{-5pt}
    \begin{enumerate-zero}
        \item $\emptyset$ and $X$ are closed.
        \item The union of \textbf{finitely many} closed sets is an closed set.
        \item The intersection of \textbf{any collection} of closed sets is a closed set.
        \item The union of \textbf{any collection} of open sets is an open set.
        \item The intersection of \textbf{finitely many} open sets is an open set
\end{enumerate-zero}
\end{thm}

\begin{dfn}[Properties of Topological Spaces]{dfn:topological-props}{1.20}
    \begin{enumerate-tight}
        \item The \textbf{closure} of a set $A \subseteq X$ is
            \[\overline{A} := \bigcap_{C \subseteq X \text{ closed; }A \subseteq C} C.\]
        \item The \textbf{interior} of a set $A \subseteq X$ is
            \[\Int A = A^{\circ} := \bigcap_{C \subseteq X \text{ open; }A \subseteq C} C.\]
        \item The \textbf{boundary} (or \textbf{frontier}) of a subset $A \subseteq X$ is
            \[\partial A := \overline{A} \backslash A^{\circ}.\]
        \item A subset $A$ of $X$ is \textbf{dense} in $X$ iff $\overline{A} = X$.
    \end{enumerate-tight}
    $\overline{A}$ is closed, and contains $A$ and is the smallest set with this property. So $A$ is closed iff $\overline{A} = A$.

    $A^{\circ}$ is open, and is contained in $A$, and is the largest set with this proprety. So $A$ is open iff $A^{\circ} = A$.

    % way more compact version
    % $A$ is closed iff $\overline{A} = A$, and $A$ is open iff $A^{\circ} = A$.
\end{dfn}

\begin{ppn}[Relating Topological Properties]{ppn:topological-prop-relations}{1.22}
    The closure of the complement is the complement of the interior:
    \[\overline{X \backslash A} = X \backslash (A^{\circ}).\]

    The interior of the complement is the complement of the closure:
    \[(X \backslash A)^{\circ} = X \backslash \overline{A}.\]
\end{ppn}

\begin{dfn}[Limit Points]{dfn:limit-points}{1.23}
    Let $(X, \mathcal{T})$ be a topological space, and let $A \subseteq X$ be a subset. A \textbf{limit point} of $A$ is a point $x\in X$ s.t. for every open subset $U \subseteq X$ with $x\in U$ there exists an element $a\in A \cup U$ with $a \ne x$. Let $A'$ be the set of limit points of $A$.\newline
    Note that this has nothing to do with limits of sequences.
\end{dfn}

\newpage
\begin{lma}[Limit Points and Open Balls]{lma:limit-point-balls}{1.24}
    An element $x\in X$ in a metric space $(X, d)$ is a limit point of a subset $A \subseteq X$ iff for every $\epsilon > 0$ there exists $a\in A$ with $0<d(x,a)<\epsilon$, or iff there exists a sequence $a_{1},\,a_{2},\,a_{3},\cdots$ of elements $a_{i}\in A$, with $a_{i}  \ne x$ for all $i$, such that $d(x_{i}, a_{i}) \to 0$ as $i \to\infty$. This interpretation does not extend to general topological spaces.
\end{lma}

\begin{xmp}[Examples of limit points]{xmp:}{}
    P7 in the notes
\end{xmp}

\begin{ppn}[Union of Limit points]{ppn:limit-point-unions}{1.26}
    Let $(X, \mathcal{T})$ be a topological space, and suppose $A \subseteq X$. Then
    \[\overline{A} = A \cup A'\]
\end{ppn}

\begin{crl}[]{crl:closed-limit-points}{1.27}
    A subset $A \subseteq X$ is closed iff it contains all its limit points.
\end{crl}

\begin{thm}[Open and Closed sets in \texorpdfstring{$\mathbb{R}$}{R}]{thm:open-closed-sets-in-R}{1.30}
    Consider $\mathbb{R}$ with the usual topology.
    \begin{enumerate-zero}
        \item A nonempty set $U$ is open iff it can be written as a countable union of disjoint nonempty open intervals $I_{j}$:
            \[U = \bigcup_{j=1}^{\infty} I_{j}.\]
        \item A set $F$ is closed iff it can be written as a countable intersection
            \[F = \bigcap_{j=1}^{\infty} F_{j}\]
            where each $F_{j}$ is a finite union of closed intervals.
    \end{enumerate-zero}
\end{thm}

\begin{dfn}[Hausdorff Spaces]{dfn:hausdorff}{1.32}
    A topological space $(X, \mathcal{T})$ is \textbf{Hausdorff} if for each $x,\,y\in X$ with $x \ne y$ there exist \textbf{disjoint} open sets $U$ and $V$ such that $x\in U$ and $y\in V$.
    \longrule{0.08ex}
    Any metrisable space is Hausdorff, The trivial topology n a set with more than one element is not Hausdorff.
\end{dfn}

\begin{dfn}[Convergence of a Topological space]{dfn:convergence-topological}{1.33}
    A sequence $(x_{n})$ of members of a topological space $X$ converges to $x\in X$ if for every open set $U$ containing $x$, there exists an $N$ such that $n \ge N \implies x_{n} \in U$
\end{dfn}

\begin{ppn}[Convergence of Hausdorff Spaces]{ppn:convergence-hausdorff}{1.34}
    Suppose $(X, \mathcal{T})$ is Hausdorff. Then a sequence $(x_{n})$ can converge to at most one limit.
\end{ppn}

\begin{dfn}[Cauchy and Completeness]{dfn:cauchy-completeness}{1.36}
    Let $(X, d)$ be a metric space.
    \begin{enumerate-tight}
        \item A \textbf{Cauchy sequence} is a sequence $(x_{n})$ with each $x_{n}\in X$ with the property that for each $\epsilon > 0$, there exists an $N$ such that $m,\,n\in N \implies d(x_{m},\,x_{n})< \epsilon$
        \item $(X, d)$ is \textbf{complete} if every Cauchy sequence converges.
    \end{enumerate-tight}
\end{dfn}

\begin{dfn}[Topology Basis]{dfn:topology-basis}{1.37}
    A \textbf{basis for a topology} on a set $X$ is a collection $\mathcal{B}$ of subsets $B \subseteq X$ such that:
    \begin{enumerate-tight}
        \item $X = \bigcup_{B\in \mathcal{B}} B$
        \item The intersection of sets $B_{1},\,B_{2}\in \mathcal{B}$ in a set $B_{1} \cap B_{2} \in \mathcal{B}$
    \end{enumerate-tight}

    \longrule{0.08ex}
    The \textbf{topology $\mathcal{T}$ generated by a basis $\mathcal{B}$} has open sets the arbitrary unions of basis elements $B_{\lambda} \in \mathcal{B}$:
    \[U = \bigcup_{\lambda\in \Lambda} B_{\lambda}\]
    (Don't forget to check that this really is a topology)
\end{dfn}

\begin{xmp}[Finite Intersections of open balls]{xmp:open-balls-finite-intersections}{1.38}
    For any metric space $(X, \mathcal{T})$ the finite intersections of open balls
    \[B(x,r) = \{y \in X \mid d(x, y) < r\} \subseteq X \quad (r > 0,\, x\in X)\]
    constitute a basis for the metric topology on $X$
    \begin{multline*}
        \mathcal{B} = \{B(x_{1},r_{1}) \cap B(x_{2},r_{2}) \cap \cdots \cap B(x_{k}, r_{k}) \mid \\
        x_{1},x_{2},\dots,x_{k}\in X,\,r_{1},r_{2},\dots,r_{k}>0\}
    \end{multline*}
\end{xmp}

\section{Continuous functions and Homeomorphisms}

\begin{dfn}[Continuity]{dfn:continuity}{2.1}
    Let $(X, \mathcal{T})$, $(Y, \mathcal{U})$ be topological spaces. A function $f : X \to Y$ is \textbf{continuous} iff
    \[U \in \mathcal{U} \text{ implies } f^{-1}(U) \in \mathcal{T}.\]
    That is, \textbf{inverse} images of open sets are open. Continuous functions are often called \textbf{maps} or \textbf{mappings} of topological spaces.
\end{dfn}

\begin{ppn}[Topological and Analytic Continuity]{ppn:topological-analytic-continuity}{2.6}
    Let $(X, d)$ and $(Y, \rho)$ be metric spaces with their induced topologies $\mathcal{T}$ and $\mathcal{U}$ respectively. A function $f: X \to Y $ is continuous (topologically) iff it is continuous analytically: for every $a\in X$ and every $\epsilon > 0$ there exists $\delta > 0$ such that
    \[d(x, a) < \delta \implies \rho(f(x), f(a)) < \epsilon\]
\end{ppn}

\begin{dfn}[Homeomorphism]{dfn:homeomorphism}{2.7}
    Let $(X, \mathcal{T})$ and $(Y, \mathcal{U})$ be topological spaces. A \textbf{homeomorphism} is a bijection $f: X \to Y$ such that both $f$ and $f^{-1}$ are continuous. Two topological spaces are \textbf{homeomorphic} if there is a homeomorphism between them.
\end{dfn}

\begin{ppn}[Open Homeomorphisms]{ppn:open-homeomorphisms}{2.8}
    Let $f : (X, \mathcal{T}) \to (Y, \mathcal{U})$ be a homeomorphism. Then $U$ is open in $Y$ iff $f^{-1}(U)$ is open in $X$.
\end{ppn}

%TODO: Some stuff on topological invariants

\begin{xmp}[Examples of homeomorphisms]{xmp:homeomorphisms}{2.10}
    \begin{enumerate-tight}
        \item Let $(X, \mathcal{T})$ be an arbitrary topological space. Then the identity map
            \[\iota : X \to X \,;\quad x \mapsto x\]
            is continuous, and indeed a homeomorphism.
        \item Suppose $(X, \mathcal{T})$, $(Y, \mathcal{U})$, and $(Z, \mathcal{W})$ are topological spaces, and that $f : X \to Y$ and $g : Y \to Z$ are continuous functions. Then their composition
            \[g \circ f : X \to Z \,; \quad x \mapsto g(f(x))\]
            is continuous.
        \item For any topological spaces $X$, $Y$, and any element $y_{0} \in Y$ the constant function
            \[f_{0} : X \to Y \,; \quad x \mapsto y_{0}\]
            is continuous.
    \end{enumerate-tight}
\end{xmp}

\begin{ppn}[Continuity and Closed sets]{ppn:continuity-closed}{2.14}
    \begin{itemize-zero}
        \item Suppose $(X, \mathcal{T})$ and $(Y, \mathcal{U})$ are topological spaces and that $f : X \to Y$. Then $f$ is continuous iff for every closed subset $F \subseteq Y$ its inverse image $f^{-1}(F)$ is closed in $X$.
        \item $f$ is continuous iff the image of the closure of every subset $A \subseteq X$ is contained in the closure of the image, i.e., $\forall A \subseteq X$,
            \[f(\overline{A}) \subseteq \overline{f(A)}\]
    \end{itemize-zero}
\end{ppn}

\newpage
\begin{ppn}[The Punctured Sphere]{ppn:punctured-sphere}{2.18}
    Consider the $n$-dimensional sphere
    \[\mathbb{S}^{n} = \{x\in \mathbb{R}^{n+1} : \lvert x \rvert = 1\}\]
    with the metric topology inherited from $\mathbb{R}^{n+1}$. Let $x_{0} \in \mathbb{S}^{n}$. Then $\mathbb{S}^{n} \backslash \{x_{0}\}$ is homeomorphic to $\mathbb{R}^{n}$.
\end{ppn}

% TODO: whole proof

% TODO: entire section....

\section{Subspaces Revisited}

\begin{ppn}[Hausdorff and Subspaces]{ppn:hausdorff-subspaces}{3.4}
    Suppose $(X, \mathcal{T})$ is a Hausdorff topological space and suppose $A$ is a subspace. Then $A$ is Hausdorff.
\end{ppn}

\begin{ppn}[Continuity and Subspaces]{ppn:continuity-subspaces}{3.5}
    Suppose $(X, \mathcal{T})$ and $(Y, \mathcal{U})$ are topological spaces and suppose $A$ is a subspace of $X$. Let $f: X \to Y $ be continuous. Then $f \rvert_{A} : A \to Y$ is continuous.
\end{ppn}

\begin{crl}[Homeomorphisms and Exclusions]{crl:homeomorphisms-exclusion}{3.6}
    Suppose $(X, \mathcal{T})$ and $(Y, \mathcal{U})$ are homeomorphic via $f$. Then $X \backslash \{x_{0}\}$ is homeomorphic to $Y \backslash \{f(x_{0})\}$
\end{crl}

\begin{dfn}[Disjoint Unions]{dfn:disjoint-union}{3.65}
    Let $(X, \mathcal{T})$ and $(Y, \mathcal{U})$ be topological spaces. Their \textbf{disjoint union} $X + Y$ is the set $(X \times \{0\}) \cup (Y \times \{1\})$ with the topology consisting of all sets of the form
    \[(T \times \{0\}) \cup (U \times \{1\})\text{ such that }T \in \mathcal{T},\,U \in \mathcal{U}\]
\end{dfn}

\begin{dfn}[Product Topology]{dfn:product-topology}{3.8}
    Let $(X, \mathcal{T})$, $(Y, \mathcal{U})$ be topological spaces. The \textbf{product topology} on their product $X \times Y$ consists of all sets of the form
    \[T = \bigcup_{\alpha\in A} (U_{\alpha} \times V_{\alpha})\]
    where $\mathcal{A}$ is an arbitrary indexing set, and $U_{\alpha}\in \mathcal{U}$ and $V_{\alpha}\in \mathcal{V}$.
\end{dfn}

\begin{lma}[Openness in Product Topologies]{lma:product-topology-open}{3.9}
    Let $(X, \mathcal{T})$ $(Y, \mathcal{U})$ be topological spaces. Then $T \subseteq X \times Y$ is open in the product topology if and only if for all $t\in T$ there exists $U \in \mathcal{U}$ and $V \in \mathcal{V}$ such that $t \in U \times V$ and $U \times V \subseteq T$.
\end{lma}

\begin{lma}[Product Topology is a topology]{lma:product-topology-topology}{3.10}
    The product topology is indeed a topology. (lol)
\end{lma}

\begin{dfn}[Projection Maps]{dfn:projection-maps}{3.11.5}
    Let $(X, \mathcal{T})$ and $(Y, \mathcal{U})$ be topological spaces, and consider their product $X \times Y$ with the product topology. There are two natural maps $\Pi_{X}$ and $\Pi_{Y}$, the projections of $X \times Y$ onto $X$ and $Y$ respectively, given by
    \[\Pi_{X} : X \times Y \to X, \quad (x, y) \mapsto x\]
    and
    \[\Pi_{Y} : X \times Y \to Y, \quad (x, y) \mapsto y.\]
\end{dfn}

\begin{thm}[Continuity of Projection Maps]{thm:projection-map-continuity}{3.12}
    Let $(X, \mathcal{T})$ and $(Y, \mathcal{U})$ be topological spaces and $\mathcal{T}$ the product topology on $X \times Y$. Then the projection maps $\Pi_{X}$ and $\Pi_{Y}$ are continuous. Moreover, $\mathcal{T}$ is the smallest topology on $X \times Y$ such that the projection maps are continuous.
\end{thm}

\begin{ppn}[Continuity of compositions]{ppn:composition-continuity}{3.13}
    Let $X,\,Y,\,Z$ be topological spaces. Endow $X \times Y$ with the product topology. A function $f : Z \to X \times Y$ is continuous iff the functions $\Pi_{X} \circ f : Z \to X$ and $\Pi_{Y} \circ f : Z \to Y$ are both continuous.
\end{ppn}

\begin{dfn}[Weak Topology]{dfn:weak-topology}{3.14}
    Suppose that $X$ is a set. $(X_{\lambda}, \mathcal{T}_{\lambda})$ is a family of topological spaces, and that $f_{\lambda} : X \to X_{\lambda}$ are functions. The \textbf{weak topology generated by $\{f_{\lambda}\}$} is the smallest topology on $X$ making all the $f_{\lambda}$ continuous.

    Thus the product topology on $X \times Y$ is the weak topology generated by the two maps $\Pi_{X}$ and $\Pi_{Y}$
\end{dfn}

\lipsum[1-12]
\end{multicols}
\end{document}
