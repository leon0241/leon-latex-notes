\documentclass[landscape, 8pt]{extarticle}
\usepackage{geometry}
% \usepackage{showframe}
\usepackage[dvipsnames]{xcolor}

\colorlet{colour1}{Red}
\colorlet{colour2}{Green}
\colorlet{colour3}{Cerulean}

\geometry{
    a4paper, 
    margin=0.17in
}

\pretolerance=0
\hyphenpenalty=0

\usepackage{lmodern}

% \usepackage[fontsize=8pt]{scrextend}


\usepackage{quiver}
\usepackage{mathtools}
\usepackage{multirow}
\usepackage{symbols}
\usepackage{graphicx} % Required for inserting images
\usepackage{amsmath}
\usepackage{amsfonts}
\usepackage{amssymb}
% \usepackage{preamble}
\usepackage{enumitem}
\usepackage{multicol}
\usepackage{lipsum}
\usepackage[framemethod=TikZ]{mdframed}
% \usepackage{../thmboxes_white}
\usepackage{../../thmboxes_v3}
\usepackage{customs}
\usepackage{float}
% \usepackage{setspace}
\usepackage[nodisplayskipstretch]{setspace}

\usepackage{hyperref} % note: this is the final package

\parindent = 0pt

\renewcommand\labelitemi{\tiny$\bullet$}

\begin{document}

\setlength{\abovedisplayskip}{3.5pt}
\setlength{\belowdisplayskip}{3.5pt}
\setlength{\abovedisplayshortskip}{3.5pt}
\setlength{\belowdisplayshortskip}{3.5pt}

\begin{multicols}{3}
\raggedcolumns


\section*{\huge Modelling Concurrent Systems Notes}
Made by Leon :)


\vspace{-5pt}
\section{Process Algebras}
\setcounter{subsection}{1}

\centering
\begin{dfn}[ACP, CCS, CSP]{dfn:acp-ccs-csp}{}
    The syntax of ACP, CCS, and CSP is defined as:
    \begin{tabular}{ |r|c|c|c|}
        \hline
        \textbf{Operation} & \textbf{ACP} & \textbf{CCS} & \textbf{CSP} \\
        \hline
        Termination & $\epsilon$ & $0$ & $\mathrm{STOP}$ \\ 
        \hline
        Deadlock & $\delta$ & &\\
        \hline
        Action & a & &\\
        Sequential Composition & $P.Q$ & & \\
        Action Prefixing &  & $a.P$& $a \to P$\\
        \hline
        Alternative Choice & $P + Q$ & $P + Q$ & \\
        External Choice & & & $P \extchoice Q$ \\
        Internal Choice &  & & $P \intchoice Q$ \\
        \hline
        Parallel Composition & $P \merge Q$ & $P \mid Q$ & $P \pcomp Q$ \\
        \hline
        Restriction & $\restrict(P)$ & $P \backslash a$ & \\
        \hline
        Abstraction & $\abstract(P)$ & & $P / a$ \\
        \hline
        Relabelling & & $P[f]$ & $P[f]$ \\
        \hline

    \end{tabular}

    \vspace{10pt}
    \longrule{0.08ex}

    \textbf{Differences between ACP, CCS, and CSP}
    \begin{itemize}
        \item \textbf{Action}: CCS and CSP require Action Prefixing, while ACP can perform sequential composition on any process. This also requires CCS and CSP processes to feature the inaction $0$/$\mathrm{STOP}$, while ACP is not restricted to this.
        \item \textbf{Choice}: ACP and CCS have an operator which can perform both External and Internal Choice. CSP Differentiates internal choices from external ones, and internal actions within $\extchoice$ do not count as a choice in CSP.
        \item \textbf{Parallel Composition}:
            \begin{itemize}
                \item  ACP actions follow a communication function to decide what to synchronise, i.e. $\gamma(a, b)$
                \item CCS actions can only synchronise with its conjugate counterpart, i.e. $a$ and $\overline{a}$
                \item CSP actions can only synchronise over the same action, i.e. $a$ and $a$
            \end{itemize}
        \item \textbf{Restriction, Abstraction, Relabelling}:
            \begin{itemize}
                \item Relabelling just doesn't exist in base ACP, lol
                \item CCS combines communication and abstraction into one step - every synchronisation results in a $\tau$.
                \item CSP combines Parallel Composition and Restriction into one step, as CSP Parallel Composition doesn't feature left-over Left Merges.
            \end{itemize}
    \end{itemize}
\end{dfn}

\begin{dfn}[The GSOS Format]{dfn:gsos}{}
    General Structured Operational Semantics (GSOS) operations are compositional.

    \textbf{Rules of GSOS}
    \begin{itemize}
        \item Its source has the form $f(x_{1},\dots,x_{ar(f)})$ with $f\in \Sigma$ and $x_{i}\in V$
        \item The left hand sides of its premises are variables $x_{i}$ with $1\le i \le ar(f)$
        \item The right hand sides of its positive premises are variables that are all distinct, and that do not occur in its source
        \item Its target only contains variables that also occur in its source or premises
    \end{itemize}


\textbf{GSOS Semantics of ACP}

\framebox{$\begin{array}{ccccc}
(a. P) \prightarrow{\alpha} P &
P + Q \prightarrow{\alpha} P &
P + Q \prightarrow{\alpha} Q \\[2ex]
% \displaystyle\frac{P \prightarrow{\alpha} P'}{f(P) \prightarrow{f(\alpha)} f(P')} \\[4ex]
\displaystyle\frac{P\prightarrow{\alpha} P'}{P \merge Q \prightarrow{\alpha} P' \merge Q} &

\multicolumn{2}{c}{\displaystyle\frac{P\prightarrow{a} P'~~Q\prightarrow{b} Q'~~ \scriptstyle{a \mid b = c}}{P \merge Q \prightarrow{a} P' \merge Q'}} \\[4ex]

\displaystyle\frac{Q\prightarrow{\alpha} Q'}{P \merge Q \prightarrow{\alpha} P \merge Q'} &

\displaystyle\frac{P \prightarrow{\alpha} P'~~{\scriptstyle(\alpha\notin A)}}{\restrict(P) \prightarrow{\alpha} \restrict(P')} &

\displaystyle\frac{\langle \mathcal{S}_{X} \mid \mathcal{S} \rangle \prightarrow{a} P'}{\langle X \mid \mathcal{S} \rangle \prightarrow{a} P'}


\end{array}$}

\textbf{GSOS Semantics of CSP}
\begin{center}
\framebox{$\begin{array}{ccc}
(a\rightarrow P) \prightarrow{a} P &
P \intchoice Q \prightarrow{\tau} P &
P \intchoice Q \prightarrow{\tau} Q \\[2ex]
\displaystyle\frac{P\prightarrow{a} P'}{P\extchoice Q \prightarrow{a} P'} &
\displaystyle\frac{P\prightarrow{\tau} P'}{P\extchoice Q \prightarrow{\tau} P'\extchoice Q} &
\displaystyle\frac{Q\prightarrow{a} Q'}{P\extchoice Q \prightarrow{a} Q'} \\[4ex]
\displaystyle\frac{Q\prightarrow{\tau} Q'}{P\extchoice Q \prightarrow{\tau} P\extchoice Q'} &
\displaystyle\frac{P \prightarrow{\alpha} P'}{f(P) \prightarrow{f(\alpha)} f(P')} &
\displaystyle\frac{P\prightarrow{\alpha} P'~~{\scriptstyle(\alpha\notin A)}}{P\|_AQ \prightarrow{\alpha} P'\|_AQ} \\[4ex]

\multicolumn{2}{c}{
\displaystyle\frac{P\prightarrow{a} P'~~Q\prightarrow{a} Q'~~{\scriptstyle(a\in A)}}{P\|_AQ \prightarrow{a} P'\|_AQ'}} &
\displaystyle\frac{Q\prightarrow{\alpha} Q'~~{\scriptstyle(\alpha\notin A)}}{P\|_AQ \prightarrow{\alpha} P\|_AQ'} \\[4ex]
\multicolumn{3}{c}{\mu p.P \xrightarrow{\mathmakebox[10pt]{\tau}} P[\mu p.P/p]}
\end{array}$}
\end{center}

\end{dfn}

\begin{dfn}[CSP Expansion Theorem]{dfn:expansion-theorem}{}
    Let $P := \sum_{i\in I} \alpha_{i} P_{i}$ and $Q := \sum_{j\in J} \beta_{j}. Q_{j}$. Then
    \[P \mid Q = \sum_{i\in I} \alpha_{i}(P_{i} \mid Q) + \sum_{j\in J} \beta_{j}(P \mid Q_{j}) + \sum_{\substack{i\in I, j\in J \\ \alpha_{i} = \overline{b}_{j}}} \tau.(P_{i} \mid Q_{j})\]
    Any guarded CCS expression can be written into a bisimulation equivalent CCS expression of the form $\sum_{i\in I} \alpha_{i}.P_{i}$. This is called \textbf{head normal form}
\end{dfn}

\newpage
\section{Semantics and shit like that}
\begin{dfn}[Trace Semantice]{dfn:trace-semantics}{}
    \begin{itemize}
        \item \textbf{Completed Trace}: A start to finish trace of a process.
        \item \textbf{Partial Trace}: From the start of a process to any point, including the end. Clearly, $CT(P) \subseteq PT(Q)$
        \item \textbf{Strong vs Weak}: Weak PT and CT means that two processes are equivalent with all instances of $\tau$ omitted.
        \item \textbf{Infinite Trace Semantics}: Differs from different types of divergence. Stronger than CT and PT
    \end{itemize}
\end{dfn}

\begin{dfn}[Bisimulation Semantice]{dfn:bisimulation}{}
    \begin{itemize}
        \item True Bisimulation
        \item Weak Bisimilarity
        \item Branching Bisimilarity
        \item Rooted Branching Bisimilarity
    \end{itemize}    
\end{dfn}

\begin{dfn}[The GOAT of Process algebra]{dfn:bisim-counterexample}{}
% https://q.uiver.app/#q=WzAsMTEsWzUsMSwiXFxjaXJjIl0sWzQsMiwiXFxidWxsZXQiXSxbNiwyLCJcXGJ1bGxldCJdLFs2LDMsIlxcYnVsbGV0Il0sWzQsMywiXFxidWxsZXQiXSxbMSwxLCJcXGNpcmMiXSxbMSwyLCJcXGJ1bGxldCJdLFsyLDMsIlxcYnVsbGV0Il0sWzAsMywiXFxidWxsZXQiXSxbNSwwXSxbMSwwXSxbMCwxLCJhIl0sWzAsMiwiYSIsMl0sWzIsMywiYyIsMl0sWzEsNCwiYiJdLFs1LDYsImEiXSxbNiw3LCJjIl0sWzYsOCwiYiJdLFs5LDAsIiIsMCx7InNob3J0ZW4iOnsic291cmNlIjo1MH19XSxbMTAsNSwiIiwwLHsic2hvcnRlbiI6eyJzb3VyY2UiOjUwfX1dXQ==
\[\begin{tikzcd}[cramped]
	& {} &&&& {} \\
	& \circ &&&& \circ \\
	& \bullet &&& \bullet && \bullet \\
	\bullet && \bullet && \bullet && \bullet
	\arrow[shorten <=5pt, from=1-2, to=2-2]
	\arrow[shorten <=5pt, from=1-6, to=2-6]
	\arrow["a", from=2-2, to=3-2]
	\arrow["a", from=2-6, to=3-5]
	\arrow["a"', from=2-6, to=3-7]
	\arrow["b", from=3-2, to=4-1]
	\arrow["c", from=3-2, to=4-3]
	\arrow["b", from=3-5, to=4-5]
	\arrow["c"', from=3-7, to=4-7]
\end{tikzcd}\]
\end{dfn}

\begin{dfn}[Failure Semantics]{dfn:failure}{}
    
\end{dfn}


\begin{dfn}[Consistent Colouring]{dfn:colouring}{}
    
\end{dfn}

\begin{dfn}[Safety]{dfn:}{}
    
\end{dfn}

\lipsum[1-12]
\end{multicols}
\end{document}
