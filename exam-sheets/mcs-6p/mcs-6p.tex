\documentclass[landscape, 8pt]{extarticle}

\colorlet{colour1}{Red}
\colorlet{colour2}{Green}
\colorlet{colour3}{Cerulean}

\geometry{
    a4paper, 
    margin=0.17in
}

\pretolerance=0
\hyphenpenalty=0

\usepackage{lmodern}

% \usepackage[fontsize=8pt]{scrextend}


\usepackage{quiver}
\usepackage{mathtools}
\usepackage{multirow}
\usepackage{symbols}

\begin{document}

\setlength{\abovedisplayskip}{3.5pt}
\setlength{\belowdisplayskip}{3.5pt}
\setlength{\abovedisplayshortskip}{3.5pt}
\setlength{\belowdisplayshortskip}{3.5pt}

\begin{multicols}{3}
\raggedcolumns


\section*{\huge Modelling Concurrent Systems Notes}
Made by Leon :)


\vspace{-5pt}
\section{Process Algebras}
\setcounter{subsection}{1}

\centering
\begin{dfn}[ACP, CCS, CSP]{dfn:acp-ccs-csp}{}
The syntax of ACP, CCS, and CSP is defined as:
\begin{tabular}{ |r|c|c|c|}
	\hline
	\textbf{Operation}     & \textbf{ACP}   & \textbf{CCS}     & \textbf{CSP}     \\
	\hline
	Termination            & $\epsilon$     & $0$              & $\mathrm{STOP}$  \\
	\hline
	Deadlock               & $\delta$       &                  &                  \\
	\hline
	Action                 & a              &                  &                  \\
	Sequential Composition & $P.Q$          &                  &                  \\
	Action Prefixing       &                & $a.P$            & $a \to P$        \\
	\hline
	Alternative Choice     & $P + Q$        & $P + Q$          &                  \\
	External Choice        &                &                  & $P \extchoice Q$ \\
	Internal Choice        &                &                  & $P \intchoice Q$ \\
	\hline
	Parallel Composition   & $P \merge Q$   & $P \mid Q$       & $P \pcomp Q$     \\
	\hline
	Restriction            & $\restrict(P)$ & $P \backslash a$ &                  \\
	\hline
	Abstraction            & $\abtau(P)$    &                  & $P / a$          \\
	\hline
	Relabelling            &                & $P[f]$           & $P[f]$           \\
	\hline
\end{tabular}

\vspace{10pt}
\longrule{0.08ex}

\textbf{Differences between ACP, CCS, and CSP}
\begin{itemize-zero}
\item \textbf{Action}: CCS and CSP require Action Prefixing, while ACP can perform sequential composition on any process. This also requires CCS and CSP processes to feature the inaction $0$/$\mathrm{STOP}$, while ACP is not restricted to this.
\item \textbf{Choice}: ACP and CCS have an operator which can perform both External and Internal Choice. CSP Differentiates internal choices from external ones, and internal actions within $\extchoice$ do not count as a choice in CSP.
\item \textbf{Parallel Composition}:
\begin{itemize-tight}
\item  ACP actions follow a communication function to decide what to synchronise, i.e. $\gamma(a, b)$
\item CCS actions can only synchronise with its conjugate counterpart, i.e. $a$ and $\overline{a}$
\item CSP actions can only synchronise over the same action, i.e. $a$ and $a$
\end{itemize-tight}
\item \textbf{Restriction, Abstraction, Relabelling}:
\begin{itemize-tight}
\item Relabelling just doesn't exist in base ACP, lol
\item CCS combines communication and abstraction into one step - every synchronisation results in a $\tau$.
\item CSP combines Parallel Composition and Restriction into one step, as CSP Parallel Composition doesn't feature left-over Left Merges.
\end{itemize-tight}
\end{itemize-zero}
\end{dfn}

\begin{dfn}[The GSOS Format]{dfn:gsos}{}
General Structured Operational Semantics (GSOS) operations are compositional.

\textbf{Rules of GSOS}
\begin{itemize-zero}
\item Its source has the form $f(x_{1},\dots,x_{ar(f)})$ with $f\in \Sigma$ and $x_{i}\in V$
\item The left hand sides of its premises are variables $x_{i}$ with $1\le i \le ar(f)$
\item The right hand sides of its positive premises are variables that are all distinct, and that do not occur in its source
\item Its target only contains variables that also occur in its source or premises
\end{itemize-zero}


\textbf{GSOS Semantics of ACP}

\framebox{$\begin{array}{ccccc}
			(a. P) \prightarrow{\alpha} P                                                                                                  &
			P + Q \prightarrow{\alpha} P                                                                                                   &
			P + Q \prightarrow{\alpha} Q                                                                                                                           \\[2ex]
			% \displaystyle\frac{P \prightarrow{\alpha} P'}{f(P) \prightarrow{f(\alpha)} f(P')} \\[4ex]
			\displaystyle\frac{P\prightarrow{\alpha} P'}{P \merge Q \prightarrow{\alpha} P' \merge Q}                                      &

			\multicolumn{2}{c}{\displaystyle\frac{P\prightarrow{a} P'~~Q\prightarrow{b} Q'~~ \scriptstyle{a \mid b = c}}{P \merge Q \prightarrow{a} P' \merge Q'}} \\[4ex]

			\displaystyle\frac{Q\prightarrow{\alpha} Q'}{P \merge Q \prightarrow{\alpha} P \merge Q'}                                      &

			\displaystyle\frac{P \prightarrow{\alpha} P'~~{\scriptstyle(\alpha\notin A)}}{\restrict(P) \prightarrow{\alpha} \restrict(P')} &

			\displaystyle\frac{\langle \mathcal{S}_{X} \mid \mathcal{S} \rangle \prightarrow{a} P'}{\langle X \mid \mathcal{S} \rangle \prightarrow{a} P'}
		\end{array}$}

\textbf{GSOS Semantics of CSP}
\begin{center}
	\framebox{$\begin{array}{ccc}
				(a\rightarrow P) \prightarrow{a} P                                                                                     &
				P \intchoice Q \prightarrow{\tau} P                                                                                    &
				P \intchoice Q \prightarrow{\tau} Q                                                                                      \\[2ex]
				\displaystyle\frac{P\prightarrow{a} P'}{P\extchoice Q \prightarrow{a} P'}                                              &
				\displaystyle\frac{P\prightarrow{\tau} P'}{P\extchoice Q \prightarrow{\tau} P'\extchoice Q}                            &
				\displaystyle\frac{Q\prightarrow{a} Q'}{P\extchoice Q \prightarrow{a} Q'}                                                \\[4ex]
				\displaystyle\frac{Q\prightarrow{\tau} Q'}{P\extchoice Q \prightarrow{\tau} P\extchoice Q'}                            &
				\displaystyle\frac{P \prightarrow{\alpha} P'}{f(P) \prightarrow{f(\alpha)} f(P')}                                      &
				\displaystyle\frac{P\prightarrow{\alpha} P'~~{\scriptstyle(\alpha\notin A)}}{P\|_AQ \prightarrow{\alpha} P'\|_AQ}        \\[4ex]

				\multicolumn{2}{c}{
				\displaystyle\frac{P\prightarrow{a} P'~~Q\prightarrow{a} Q'~~{\scriptstyle(a\in A)}}{P\|_AQ \prightarrow{a} P'\|_AQ'}} &
				\displaystyle\frac{Q\prightarrow{\alpha} Q'~~{\scriptstyle(\alpha\notin A)}}{P\|_AQ \prightarrow{\alpha} P\|_AQ'}        \\[4ex]
				\multicolumn{3}{c}{\mu p.P \xrightarrow{\mathmakebox[10pt]{\tau}} P[\mu p.P/p]}
			\end{array}$}
\end{center}

\end{dfn}

\begin{dfn}[CSP Expansion Theorem]{dfn:expansion-theorem}{}
	Let $P := \sum_{i\in I} \alpha_{i} P_{i}$ and $Q := \sum_{j\in J} \beta_{j}. Q_{j}$. Then
	\[P \mid Q = \sum_{i\in I} \alpha_{i}(P_{i} \mid Q) + \sum_{j\in J} \beta_{j}(P \mid Q_{j}) + \sum_{\substack{i\in I, j\in J \\ \alpha_{i} = \overline{b}_{j}}} \tau.(P_{i} \mid Q_{j})\]
	Any guarded CCS expression can be written into a bisimulation equivalent CCS expression of the form $\sum_{i\in I} \alpha_{i}.P_{i}$. This is called \textbf{head normal form}
\end{dfn}

\begin{dfn}[CCS Axioms]{dfn:csp-axions}{}
\begin{itemize-tight}
\item Axioms of CCS
\begin{align*}
	(P + Q ) + R & = P + (Q + R) \tag{associativity}  \\
	P + Q        & = Q + Q \tag{commutativity}        \\
	P + P        & = P \tag{idempotence}              \\
	P + 0        & = P \tag{$0$ is a neutral element}
\end{align*}
\item Axiomatisation of Rooted Weak Bisimulation
\begin{align*}
	\alpha . \tau . P       & = \alpha.P \tag{T1}                       \\
	\tau . P                & = \tau.P + P \tag{T2}                     \\
	\alpha . (\tau . P + Q) & = \alpha.(\tau.P + Q) + \alpha.P \tag{T3}
\end{align*}
\item Axiomatisation of Branching Bisimularity
\begin{align*}
	\alpha . (\tau . (P + Q) + Q) & = \alpha.(P + Q) \tag{P}
\end{align*}
\item Axiomatisation of strong partial trace equivalence
\[\alpha . (P + Q) = \alpha . P + \alpha . Q\]
\item Axiomatisation of weak partial trace equivalence
\[\tau.P = P\]
\end{itemize-tight}
\end{dfn}

\begin{dfn}[Axioms of ACP]{dfn:acp-axioms}{}
	is this really necessary.. who knows
\end{dfn}

\newpage
\section{Semantics and shit like that}
\begin{dfn}[Trace Semantice]{dfn:trace-semantics}{}
\begin{itemize-zero}
\item \textbf{Completed Trace}: A start to finish trace of a process.
\item \textbf{Partial Trace}: From the start of a process to any point, including the end. Clearly, $CT(P) \subseteq PT(Q)$
\item \textbf{Strong vs Weak}: Weak PT and CT means that two processes are equivalent with all instances of $\tau$ omitted.
\item \textbf{Infinite Trace Semantics}: Differs from different types of divergence. Stronger than CT and PT
\end{itemize-zero}
\end{dfn}

\begin{dfn}[Bisimulation Semantice]{dfn:bisimulation}{}
\begin{itemize-zero}
\item \textbf{True Bisimulation} ($\leftrightarroweq$):
\begin{itemize-tight}
\item if $sRt$ and $s \prightarrow{a} s'$ then $\exists t'$ s.t. $t \prightarrow{a} t'$ and $s' R t'$
\item if $sRt$ and $t \prightarrow{a} t'$ then $\exists s'$ s.t. $s \prightarrow{a} s'$ and $s' R t'$
\item if $sRt$ then $s\models p \iff t \models p$ for all $p\in P$
\end{itemize-tight}
\item \textbf{Branching Bisimilarity} ($\rbrb$)
\begin{itemize-tight}
\item if $sRt$ and $s \prightarrow{a} s'$ then either:
\begin{itemize-tight}
\item $a = \tau$ and $s' R t$
\item $\exists t_{1},\,t'$ such that $t \Rightarrow t_{1} \prightarrow{a} t'$, $s R t_{1}$ and $s' R t'$
\end{itemize-tight}
\item if $sRt$ and $t \prightarrow{a} t'$ then either:
\begin{itemize-tight}
\item $a = \tau$ and $t' R s$
\item $\exists s_{1},\,s'$ such that $s \Rightarrow s_{1} \prightarrow{a} s'$, $s_{1} R t$ and $s' R t'$
\end{itemize-tight}
\item if $sRt$ and $s\models p$ then $\exists t_{1}$ s.t. $t \Rightarrow t_{1} \models p$, and $sRt_{1}$
\item if $sRt$ and $t\models p$ then $\exists s_{1}$ s.t. $s \Rightarrow s_{1} \models p$, and $s_{1}Rt$
\end{itemize-tight}
\item Other notions:
\begin{itemize-tight}
\item \textbf{Rooted Branching Bisimilarity}: The same as Branching Bisimilarity except the first action is Strongly bisimilar. (This makes RBB a congruence on $+$)
\item \textbf{Delay Bisimilarity}: Same as \textit{branching bisimilarity}, but with the requirements $sRt_{1}$ and $s_{1}Rt$ dropped.
\item \textbf{Weak Bisimilarity}: The same as \textit{delay bisimularity} except the action requirements are also relaxed:
\begin{itemize-tight}
\item If $sRt$ and $s \prightarrow{a} s'$ then either:
\begin{itemize-tight}
\item $a = \tau$ and $s' R t$
\item $\exists t_{1},\,t_{2},\,t'$ such that $t \Rightarrow t_{1} \prightarrow{a} t_{2} \Rightarrow t'$ and $s'Rt'$
\end{itemize-tight}
\item If $sRt$ and $t \prightarrow{a} t'$ then either:
\begin{itemize-tight}
\item $a = \tau$ and $s R t'$
\item $\exists s_{1},\,s_{2},\,s'$ such that $s \Rightarrow s_{1} \prightarrow{a} s_{2} \Rightarrow s'$ and $s'Rt'$
\end{itemize-tight}
\end{itemize-tight}
\end{itemize-tight}

\end{itemize-zero}
\end{dfn}

\begin{dfn}[Compositionality, Congruence]{dfn:congruence}{}
	An equivalence $\sim$ is a \textbf{congruence}\footnote{We can also say \textbf{the language is compositional for the equivalence}} for a language $L$ if $P \sim Q$ implies that $C[P] \sim C[Q]$ for every context $C[\ ]$, where $C[\ ]$ is an $L$-expression with a \textbf{hole} in it, and $C[P]$ is the result of plugging in $P$ for the hole. An alternative definition for a congruence $\sim$ is if every $n$-ary operator $f$ of $L$, we have
	\[P_{i} \sim Q_{i} \text{ for } i = 1,\dots, n \text{ implies } f(P_{1},\dots,P_{n}) \sim f(Q_{1} ,\dots, Q_{n})\]
	\longrule{0.08ex}
	The \textbf{Congruence closure} of a language, denoted $\sim^{c}$ of a language is a modification to a language that isn't compositional to turn it compositional. The \textit{congruence closure} of Branching Bisimilarity is Rooted Branching Bisimilarity.
\end{dfn}

\begin{thm}[Semantic Equivalence Spectrum]{thm:spectrum}{}
	% https://q.uiver.app/#q=WzAsMTQsWzQsNiwiXFxtYXRocm17Q1R9Il0sWzQsMTAsIlxcbWF0aHJte1BUfSJdLFs0LDMsIlxcbWF0aHJte1dCfSJdLFsxLDAsIlxcbWF0aHJte0JCfSJdLFszLDgsIlxcbWF0aHJte1dDVH0iXSxbMywxMiwiXFxtYXRocm17V1BUfSJdLFswLDExLCJcXG1hdGhybXtXUFR9XlxcaW5mdHkiXSxbMSw1LCJcXG1hdGhybXtDVH1eXFxpbmZ0eSJdLFsxLDksIlxcbWF0aHJte1BUfV5cXGluZnR5Il0sWzAsNywiXFxtYXRocm17V0NUfV5cXGluZnR5Il0sWzMsMiwiXFxtYXRocm17QkJ9Il0sWzQsMiwiXFxtYXRocm17V0J9XkNcXFxcKFxcbWF0aHJte1JXQn0pIl0sWzMsMSwiXFxtYXRocm17QkJ9XkNcXFxcKFxcbWF0aHJte1JCQn0pIl0sWzEsNCwiXFxtYXRocm17Q1R9XkMiXSxbNSw0XSxbNCwwLCIiLDAseyJzdHlsZSI6eyJoZWFkIjp7Im5hbWUiOiJub25lIn19fV0sWzEsMCwiIiwyLHsic3R5bGUiOnsiaGVhZCI6eyJuYW1lIjoibm9uZSJ9fX1dLFs1LDFdLFswLDIsIiIsMix7InN0eWxlIjp7ImhlYWQiOnsibmFtZSI6Im5vbmUifX19XSxbNSw2LCIiLDEseyJzdHlsZSI6eyJoZWFkIjp7Im5hbWUiOiJub25lIn19fV0sWzAsNywiIiwxLHsic3R5bGUiOnsiaGVhZCI6eyJuYW1lIjoibm9uZSJ9fX1dLFsxLDgsIiIsMSx7InN0eWxlIjp7ImhlYWQiOnsibmFtZSI6Im5vbmUifX19XSxbNCw5LCIiLDEseyJzdHlsZSI6eyJoZWFkIjp7Im5hbWUiOiJub25lIn19fV0sWzksNywiIiwxLHsic3R5bGUiOnsiaGVhZCI6eyJuYW1lIjoibm9uZSJ9fX1dLFs5LDYsIiIsMSx7InN0eWxlIjp7ImhlYWQiOnsibmFtZSI6Im5vbmUifX19XSxbNiw4LCIiLDEseyJzdHlsZSI6eyJoZWFkIjp7Im5hbWUiOiJub25lIn19fV0sWzcsOCwiIiwxLHsic3R5bGUiOnsiaGVhZCI6eyJuYW1lIjoibm9uZSJ9fX1dLFsyLDEwLCIiLDAseyJzdHlsZSI6eyJoZWFkIjp7Im5hbWUiOiJub25lIn19fV0sWzExLDIsIiIsMSx7InN0eWxlIjp7ImhlYWQiOnsibmFtZSI6Im5vbmUifX19XSxbMTEsMTIsIiIsMSx7InN0eWxlIjp7ImhlYWQiOnsibmFtZSI6Im5vbmUifX19XSxbMTAsMTIsIiIsMSx7InN0eWxlIjp7ImhlYWQiOnsibmFtZSI6Im5vbmUifX19XSxbNywxM10sWzEzLDMsIiIsMSx7InN0eWxlIjp7ImhlYWQiOnsibmFtZSI6Im5vbmUifX19XSxbMTIsM11d
	\[\begin{tikzcd}[cramped, row sep=small]
			& {\mathrm{BB}} \\
			&&& \begin{array}{c} \mathrm{BB}^C\\(\mathrm{RBB}) \end{array} \\
			&&& {\mathrm{BB}} & \begin{array}{c} \mathrm{WB}^C\\(\mathrm{RWB}) \end{array} \\
			&&&& {\mathrm{WB}} \\
			& {\mathrm{CT}^C} \\
			& {\mathrm{CT}^\infty} \\
			&&&& {\mathrm{CT}} \\
			{\mathrm{WCT}^\infty} \\
			&&& {\mathrm{WCT}} \\
			& {\mathrm{PT}^\infty} \\
			&&&& {\mathrm{PT}} \\
			{\mathrm{WPT}^\infty} \\
			&&& {\mathrm{WPT}}
			\arrow[from=2-4, to=1-2]
			\arrow[no head, from=3-4, to=2-4]
			\arrow[no head, from=3-5, to=2-4]
			\arrow[no head, from=3-5, to=4-5]
			\arrow[no head, from=4-5, to=3-4]
			\arrow[no head, from=5-2, to=1-2]
			\arrow[from=6-2, to=5-2]
			\arrow[no head, from=6-2, to=10-2]
			\arrow[no head, from=7-5, to=4-5]
			\arrow[no head, from=7-5, to=6-2]
			\arrow[no head, from=8-1, to=6-2]
			\arrow[no head, from=8-1, to=12-1]
			\arrow[no head, from=9-4, to=7-5]
			\arrow[no head, from=9-4, to=8-1]
			\arrow[no head, from=11-5, to=7-5]
			\arrow[no head, from=11-5, to=10-2]
			\arrow[no head, from=12-1, to=10-2]
			\arrow[from=13-4, to=9-4]
			\arrow[from=13-4, to=11-5]
			\arrow[no head, from=13-4, to=12-1]
		\end{tikzcd}\]
\end{thm}

\begin{dfn}[The GOAT of Process algebra]{dfn:bisim-counterexample}{}
	% https://q.uiver.app/#q=WzAsMTEsWzUsMSwiXFxjaXJjIl0sWzQsMiwiXFxidWxsZXQiXSxbNiwyLCJcXGJ1bGxldCJdLFs2LDMsIlxcYnVsbGV0Il0sWzQsMywiXFxidWxsZXQiXSxbMSwxLCJcXGNpcmMiXSxbMSwyLCJcXGJ1bGxldCJdLFsyLDMsIlxcYnVsbGV0Il0sWzAsMywiXFxidWxsZXQiXSxbNSwwXSxbMSwwXSxbMCwxLCJhIl0sWzAsMiwiYSIsMl0sWzIsMywiYyIsMl0sWzEsNCwiYiJdLFs1LDYsImEiXSxbNiw3LCJjIl0sWzYsOCwiYiJdLFs5LDAsIiIsMCx7InNob3J0ZW4iOnsic291cmNlIjo1MH19XSxbMTAsNSwiIiwwLHsic2hvcnRlbiI6eyJzb3VyY2UiOjUwfX1dXQ==
	\[\begin{tikzcd}[cramped]
			& {} &&&& {} \\
			& \circ &&&& \circ \\
			& \bullet &&& \bullet && \bullet \\
			\bullet && \bullet && \bullet && \bullet
			\arrow[shorten <=5pt, from=1-2, to=2-2]
			\arrow[shorten <=5pt, from=1-6, to=2-6]
			\arrow["a", from=2-2, to=3-2]
			\arrow["a", from=2-6, to=3-5]
			\arrow["a"', from=2-6, to=3-7]
			\arrow["b", from=3-2, to=4-1]
			\arrow["c", from=3-2, to=4-3]
			\arrow["b", from=3-5, to=4-5]
			\arrow["c"', from=3-7, to=4-7]
		\end{tikzcd}\]
\end{dfn}

\begin{dfn}[Failure Semantics]{dfn:failure}{}

\end{dfn}


\begin{dfn}[Consistent Colouring]{dfn:colouring}{}

\end{dfn}

\begin{dfn}[Safety]{dfn:}{}

\end{dfn}


\newpage

\section{Other models of Concurrency}
\begin{dfn}[Hennessy-Milner Logic]{dfn:hml}{}
The syntax of HML is given by:
\[\phi,\,\psi ::= \top \mid \bot \mid \phi \wedge \psi \mid \phi \vee \psi \mid \neg \phi \mid \langle \alpha \rangle \phi \mid [\alpha] \phi\]
Infinitary HML (HML$^{\infty}$) has an infinitary conjunction: $\bigwedge_{i\in I} \phi_{i}$
HML in set form. If a process $P$ has a property $\Phi$, we write $P \models \Phi$.
\begin{itemize-tight}
\item $P \models \top$
\item $P \not\models \bot$
\item $P \models \Phi \wedge \Psi$ iff $P \models \Phi$ and $P \models \Psi$
\item $P \models \Phi \vee \Psi$ iff $P \models \Phi$ or $P \models \Psi$
\item $P \models [K]\Phi$ iff $\forall Q \in \{P' : P \prightarrow{a} P' \text{ and } a\in K\}$. $Q \models \Phi$
\item $P \models \langle K \rangle\Phi$ iff $\exists Q \in \{P' : P \prightarrow{a} P' \text{ and } a\in K\}$. $Q \models \Phi$
\end{itemize-tight}

Deadlock can be represented as $P \models [\mathrm{Act}] \bot$, where $\mathrm{Act}$ is the set of all actions.
\end{dfn}

\begin{dfn}[Preorder]{dfn:preorder}{}
	A \textbf{preorder} is a relation that is \textit{transitive} and \textit{reflexive}, but not \textit{symmetric}. Preorders are denoted with $\sqsubseteq$.

	Preorders are used just like equivalence relations to compare specifications and implementations. We write
	\[\text{\textit{Spec}} \sqsubseteq \text{\textit{Impl}}\]
	For each preorder $\sqsubseteq$, there exists an associated equivalence relation $\equiv$ called its \textbf{kernel}, defined by
	\[P \equiv Q \iff (P \sqsubseteq Q \wedge Q \sqsubseteq P)\]
\end{dfn}

\begin{dfn}[Simulation Equivalence]{dfn:simulation}{}
	One process simulates the other when $P$ can do all the same moves as $Q$. We write $P \sqsubseteq_{S} Q$ if $Q$ can be simulated by $P$. Two processes are \textbf{simulation equivalent}, $P =_{S} Q$ if one simulates the other, and vice versa. This is two one-sided equivalences, and therefore is not the same as bisimulation, which needs both processes to be equivalent at the same time
\end{dfn}

\begin{dfn}[Kripke Structure]{dfn:kripke}{}
	Kripke Structures are defined on states rather than actions, called \textbf{atomic predicates}.

	Let $AP$ be a set of \textbf{atomic predicates}. A \textbf{Kripke structure} over $AP$ is a tuple $(S,\, \to,\, \models)$ with $S$ a set of states, $\to \subseteq S \times S$, the \textbf{transition relation}, and $\models \subseteq S \times AP$. The relation $s \models p$ says that predicate $p\in AP$ \textbf{holds in state $s\in S$}.

	\longrule{0.08ex}
	A \textbf{path} in a Kriple structure is a nonempty finite or infinite sequence $s_{0},\,s_{1} ,\dots$ of states, such as $(s_{i}, s_{i+1})\in \to$ for each adjacent pair of states $s_{i}, s_{i+1}$ in the sequence. A path is \textbf{complete} if it is either infinite or ends in deadlock (a state without outgoing transitions)


\end{dfn}

\begin{dfn}[Petri Nets]{dfn:petris}{}
	they exist
\end{dfn}

\begin{dfn}[CTL]{dfn:ctl}{}
Computational Tree Logic is defined on
\begin{multline*}
	\phi,\,\psi ::= p \mid \top \mid \bot \mid \neg\phi \mid \phi \wedge \psi \mid \phi \vee \psi \mid \neg \phi \mid \phi \to \psi  \mid \\
	\ctl{EX}\phi \mid \ctl{AX}\phi \mid \ctl{EF}\phi \mid \ctl{AF}\phi \mid \ctl{EG} \phi \mid \ctl{AG} \phi \mid \ctl{E}\psi \ctl{U} \phi \mid \ctl{A}\psi \ctl{U}\phi
\end{multline*}
$p\in AP$ is an atomic predicate. CTL is defined on states, the relation $\models$ between states $s$ in a Kripke structure and CTL formulae $\phi$ is inductively defined by
\begin{itemize-zero}
\item $s \models p$, $p\in AP$ iff $(s, p) \in \models$
\item $s \models \top$ always holds, and $s \models \bot$ never
\item $s \models \neg \phi$ iff $s \not\models \phi$
\item $s \models \phi \wedge \psi$ iff $s \models \phi$ and $s \models \psi$
\item $s \models \phi \vee \psi$ iff $s \models \phi$ or $s \models \psi$
\item $s \models \phi \to \psi$ iff $s \not\models \phi$ or $s \models \psi$
\item $s \models \ctl{EX}\phi$ iff there is a state $s'$ with $s\to s'$ and $s' \models \psi$
\item $s \models \ctl{AX}\phi$ iff for each state $s'$ with $s\to s'$ one has $s' \models \psi$
\item $s \models \ctl{EF}\phi$ iff some complete path starting in $s$ contains a state $s'$ with $s\models \phi$
\item $s \models \ctl{AF}\phi$ iff each complete path starting in $s$ contains a state $s'$ with $s\models \phi$
\item $s \models \ctl{EG}\phi$ iff all states $s'$ on some complete path starting in $s$ satisfy $s' \models \phi$
\item $s \models \ctl{AG}\phi$ iff all states $s'$ on all complete path starting in $s$ satisfy $s' \models \phi$
\item $s \models \ctl{E}\psi \ctl{U}\phi$ iff some complete path $\pi$ starting in $s$ contains a state $s'$ with $s'\models \phi$, and each state $s''$ on $\pi$ prior to $s'$ satisfies $s'' \models \phi$
\item $s \models \ctl{A}\psi \ctl{U}\phi$ iff each complete path $\pi$ starting in $s$ contains a state $s'$ with $s'\models \phi$, and each state $s''$ on $\pi$ prior to $s'$ satisfies $s'' \models \phi$
\end{itemize-zero}
\end{dfn}

\begin{dfn}[LTL]{dfn:ltl}{}
Linear-Time Temporal Logic is defined on
\begin{multline*}
	\phi,\,\psi ::= p \mid \top \mid \bot \mid \neg\phi \mid \phi \wedge \psi \mid \phi \vee \psi \mid \neg \phi \mid \phi \to \psi  \mid \\
	\ctl{X}\phi \mid \ctl{F}\phi \mid \ctl{G} \phi \mid \psi \ctl{U} \phi \mid
\end{multline*}
$p\in AP$ is an atomic predicate. The modalities $X,F,G,U$ are called \textbf{next-state}, \textbf{eventually}, \textbf{globally}, and \textbf{until}. The relation $\models$ between paths and LTL formulae, with $\pi \models \phi$ saying that the path $\pi$ satisfies the formula $\phi$, or that $\phi$ is valid on $\pi$, or \textbf{holds} on $\pi$, is inductively defined by
\begin{itemize-tight}
\item $\pi \models p$, $p\in AP$ iff $(s, p) \in \models$
\item $\pi \models \top$ always holds, and $\pi \models \bot$ never
\item $\pi \models \neg \phi$ iff $s \not\models \phi$
\item $\pi \models \phi \wedge \psi$ iff $\pi \models \phi$ and $\pi \models \psi$
\item $\pi \models \phi \vee \psi$ iff $\pi \models \phi$ or $\pi \models \psi$
\item $\pi \models \phi \to \psi$ iff $s \not\models \phi$ or $\pi \models \psi$
\item $\pi \models \ctl{X}\phi$ iff $\pi' \models \phi$, where $\pi'$ is the suffix of $\pi$ obtained by omitting the first state
\item $\pi \models \ctl{F}\phi$ iff $\pi' \models \phi$ for some suffix $\pi'$
\item $\pi \models \ctl{G}\phi$ iff $\pi' \models \phi$ for each suffix $\pi'$
\item $\pi \models \psi \ctl{U}\phi$ iff $\pi' \models \phi$ for some suffix $\pi'$ of $\pi$, and $\pi'' \models \phi$ for each path $\pi'' \ne \pi'$ with $\pi \Rightarrow \pi'' \Rightarrow \pi'$
\end{itemize-tight}
Here a path is seen as a state, namely the first staet on that path, together with a future that has been chosen already when evaluating an LTL formula on that state. Traditionally, these judgements $\pi\models \phi$ were defined only for infinite paths $\pi$. When also applying them to finite paths, we only have to make one adaptation, namely $\pi \models \mathrm{\mathbf{X}} \phi$ never holds if $\pi$ has only one state. So $\mathrm{\mathbf{X}} \phi$ says that there is a next state, and that $\phi$ holds in it.

$s \models \phi$ iff $\pi \models \phi$ for all complete paths $\pi$ starting in state $s$. Here a path is \textbf{complete} if it is either infinite or ends in deadlock.
\end{dfn}

\begin{lma}[Comparing LTL to CTL]{lma:ltl-to-ctl}{}
	LTL and CTL can be shown to be incomparable by proving that there cannot exist an LTL formula that is equivalent to the CTL formula $\mathrm{\mathbf{AG}} \mathrm{\mathbf{EF}} a$, and by showing that there cannot exist a CTL formula equivalent to the LTL formula $\mathrm{\mathbf{FG}} a$
\end{lma}

\newpage
\begin{thm}[Satisfaction of Strong Bisimilarity]{thm:hml-sb}{}
	\begin{itemize-zero}
	    \item Two processes are strongly bisimlar iff they satisfy the same infinitary HML formulas. Therefore, to show that two processes are not strongly bisimilar, it suffices to find an infinitary HML formula that is satisfied by one, but not the other.
	    \item Two processes $P$ and $Q$ satisfy the same CTL formulas if they are strongly bisimilar. For finitely branching processes, we have ``iff''. For general processes, we have ``iff'' if we use CTL with infinite conjunctions.
	    \item Two divergence-free processes satisfy the same CTL$_{-X}$ formulas if they are branching bisimulation equivalent. We have ``iff'' if we use CTL$_{-X}$ with infinite conjunctions.
	\end{itemize-zero}
\end{thm}

\begin{thm}[De Nicola-Vaandrager]{thm:}{}
	Translates Krikpe structures :D
	
\end{thm}

\begin{dfn}[Partition Refining]{dfn:partition-refining}{}
	Partition refining is an algorithm to turn a process into its minimal state, making it easier to compare bisimularity. Works with
	\[\mathtt{split}(B, a, P)\]
	where $B$ is an equivalence class, $a$ is an action, and $P$ is the process. $\mathtt{split}$ splits an equivalence class into two, ones with the action and ones without it.
	\[[\{s \prightarrow{a} \bullet\}]_{P}\]
	means ``state $s$ does action $a$ outside the equivalence class in process $P$''

	\begin{lstlisting}[escapeinside={(*}{*)}, caption={Pseudocode for \texttt{split}}]
	split(B, a, P) (*$\to$*) ((*$\{B_i\}$*) a set of blocks)
	   choose (*$s\in B$*)
	   (*$B_1 = \emptyset$*) (*$\color{red}(B_1\text{\textrm{ contains states equivalent to }} s*\,)$*)
	   (*$B_2 = \emptyset$*) (*$\color{red}(B_2 \text{\textrm{ contains states inequivalent to }} s*\,)$*)
	   for each (*$s'\in B$*) do
	   begin
	      if (*$[\{s \prightarrow{a} \bullet\}]_P = [\{s' \prightarrow{a} \bullet\}]_P$*) then
		(*$B_1 = B_1 \cup \{s'\}$*)
	      else
		(*$B_2 = B_2 \cup \{s'\}$*)
	   end
	   if (*$B_2 = \emptyset$*) then
	      return (*$\{B_1\}$*)
	   else
	      return (*$\{B_1,\, B_2\}$*)

	\end{lstlisting}
	Methodology:
	\begin{itemize-tight}
	    \item Start with one equivalence class for all states
	    \item Run \texttt{split} on the outermost states, this now splits into $R_{1}$ and $R_{2}$, where $R_{2}$ are outside states
	    \item Run \texttt{split} on states with outgoing actions to states in $R_{2}$, this now splits $R_{1}$ into $R_{1}$ and $R_{3}$, where $R_{3}$ are second-most outer states
	    \item Keep on running until root state is partitioned
		\item If needed, the minimal graph will have exactly one of each equivalence class
	\end{itemize-tight}
	
	\longrule{0.08ex}
	Running partition refinement for Branching Bisimularity
	\begin{itemize}
	    \item When checking whether a state in block $B$ has an $\alpha$ transition to block $C$, it is okay if we can move through block $B$ by doing only $\tau$-transitions, and then reach a state with an $\alpha$ transition to block $C$
	    \item We never check $\tau$-transitions from block $B$ to block $B$ ($\tau$-transitions to another block are fair game though)
	\end{itemize}
	This implies running the rule on
	\[[\{s \Rightarrow\prightarrow{a} \bullet\}]_{P}\]

\end{dfn}

\newpage

\section{Example Catalogue}

\begin{xmp}[Trace Semantics]{xmp:trace-semantics}{}
	A process that is PT equivalent but not CT equivalent
	% https://q.uiver.app/#q=WzAsOSxbMSwxLCJcXGJ1bGxldCJdLFswLDIsIlxcYnVsbGV0Il0sWzAsMywiXFxidWxsZXQiXSxbMiwyLCJcXGJ1bGxldCJdLFs0LDEsIlxcYnVsbGV0Il0sWzQsMiwiXFxidWxsZXQiXSxbNCwzLCJcXGJ1bGxldCJdLFsxLDBdLFs0LDBdLFswLDEsImEiLDJdLFsxLDIsImEiXSxbMCwzLCJhIl0sWzQsNSwiYSIsMl0sWzUsNiwiYSIsMl0sWzcsMCwiIiwyLHsic2hvcnRlbiI6eyJzb3VyY2UiOjUwfX1dLFs4LDQsIiIsMix7InNob3J0ZW4iOnsic291cmNlIjo1MH19XV0=
	\[\begin{tikzcd}[cramped, column sep=scriptsize]
			& {} &&& {} \\
			& \bullet &&& \bullet \\
			\bullet && \bullet && \bullet \\
			\bullet &&&& \bullet
			\arrow[shorten <=5pt, from=1-2, to=2-2]
			\arrow[shorten <=5pt, from=1-5, to=2-5]
			\arrow["a"', from=2-2, to=3-1]
			\arrow["a", from=2-2, to=3-3]
			\arrow["a"', from=2-5, to=3-5]
			\arrow["a", from=3-1, to=4-1]
			\arrow["a"', from=3-5, to=4-5]
		\end{tikzcd}\]

	A process that is CT equivalent but not Bisimulation equivalent

	% https://q.uiver.app/#q=WzAsMTEsWzUsMSwiXFxjaXJjIl0sWzQsMiwiXFxidWxsZXQiXSxbNiwyLCJcXGJ1bGxldCJdLFs2LDMsIlxcYnVsbGV0Il0sWzQsMywiXFxidWxsZXQiXSxbMSwxLCJcXGNpcmMiXSxbMSwyLCJcXGJ1bGxldCJdLFsyLDMsIlxcYnVsbGV0Il0sWzAsMywiXFxidWxsZXQiXSxbNSwwXSxbMSwwXSxbMCwxLCJhIl0sWzAsMiwiYSIsMl0sWzIsMywiYyIsMl0sWzEsNCwiYiJdLFs1LDYsImEiXSxbNiw3LCJjIl0sWzYsOCwiYiJdLFs5LDAsIiIsMCx7InNob3J0ZW4iOnsic291cmNlIjo1MH19XSxbMTAsNSwiIiwwLHsic2hvcnRlbiI6eyJzb3VyY2UiOjUwfX1dXQ==
	\[\begin{tikzcd}[cramped]
			& {} &&&& {} \\
			& \circ &&&& \circ \\
			& \bullet &&& \bullet && \bullet \\
			\bullet && \bullet && \bullet && \bullet
			\arrow[shorten <=5pt, from=1-2, to=2-2]
			\arrow[shorten <=5pt, from=1-6, to=2-6]
			\arrow["a", from=2-2, to=3-2]
			\arrow["a", from=2-6, to=3-5]
			\arrow["a"', from=2-6, to=3-7]
			\arrow["b", from=3-2, to=4-1]
			\arrow["c", from=3-2, to=4-3]
			\arrow["b", from=3-5, to=4-5]
			\arrow["c"', from=3-7, to=4-7]
		\end{tikzcd}\]

\end{xmp}



\begin{xmp}[Bisimulation Semantics]{xmp:bisim-semantics}{}

	Two processes that are Branching Bisimulation equivalent
	% https://q.uiver.app/#q=WzAsNyxbMSwyLCJcXGJ1bGxldCJdLFszLDIsIlxcYnVsbGV0Il0sWzEsMCwiXFxidWxsZXQiXSxbMywwLCJcXGJ1bGxldCJdLFs1LDIsIlxcYnVsbGV0Il0sWzAsMl0sWzAsMF0sWzAsMSwiYSIsMl0sWzIsMywiYSJdLFs1LDAsIiIsMix7InNob3J0ZW4iOnsic291cmNlIjo1MH19XSxbNiwyLCIiLDIseyJzaG9ydGVuIjp7InNvdXJjZSI6NTB9fV0sWzAsMiwiIiwwLHsiY29sb3VyIjpbMjQwLDYwLDYwXSwic3R5bGUiOnsiYm9keSI6eyJuYW1lIjoiZGFzaGVkIn0sImhlYWQiOnsibmFtZSI6Im5vbmUifX19XSxbMSw0LCJcXHRhdSJdLFszLDEsIiIsMCx7ImNvbG91ciI6WzI0MCw2MCw2MF0sInN0eWxlIjp7ImJvZHkiOnsibmFtZSI6ImRhc2hlZCJ9LCJoZWFkIjp7Im5hbWUiOiJub25lIn19fV0sWzMsNCwiIiwwLHsiY29sb3VyIjpbMjQwLDYwLDYwXSwic3R5bGUiOnsiYm9keSI6eyJuYW1lIjoiZGFzaGVkIn0sImhlYWQiOnsibmFtZSI6Im5vbmUifX19XV0=
	\[\begin{tikzcd}[cramped, column sep=scriptsize]
			{} & \bullet && \bullet \\
			\\
			{} & \bullet && \bullet && \bullet
			\arrow[shorten <=8pt, from=1-1, to=1-2]
			\arrow["a", from=1-2, to=1-4]
			\arrow[color={rgb,255:red,92;green,92;blue,214}, dashed, no head, from=1-4, to=3-4]
			\arrow[color={rgb,255:red,92;green,92;blue,214}, dashed, no head, from=1-4, to=3-6]
			\arrow[shorten <=8pt, from=3-1, to=3-2]
			\arrow[color={rgb,255:red,92;green,92;blue,214}, dashed, no head, from=3-2, to=1-2]
			\arrow["a"', from=3-2, to=3-4]
			\arrow["\tau", from=3-4, to=3-6]
		\end{tikzcd}\]

		Two processes that are Branching Bisimulation equivalent but not Rooted BB equivalent
% https://q.uiver.app/#q=WzAsNyxbMSwyLCJcXGJ1bGxldCJdLFszLDIsIlxcYnVsbGV0Il0sWzEsMCwiXFxidWxsZXQiXSxbMywwLCJcXGJ1bGxldCJdLFs1LDIsIlxcYnVsbGV0Il0sWzAsMl0sWzAsMF0sWzAsMSwiXFx0YXUiLDJdLFsyLDMsImEiXSxbNSwwLCIiLDIseyJzaG9ydGVuIjp7InNvdXJjZSI6NTB9fV0sWzYsMiwiIiwyLHsic2hvcnRlbiI6eyJzb3VyY2UiOjUwfX1dLFswLDIsIiIsMCx7ImNvbG91ciI6WzI0MCw2MCw2MF0sInN0eWxlIjp7ImJvZHkiOnsibmFtZSI6ImRhc2hlZCJ9LCJoZWFkIjp7Im5hbWUiOiJub25lIn19fV0sWzEsNCwiYSJdLFszLDQsIiIsMCx7ImNvbG91ciI6WzI0MCw2MCw2MF0sInN0eWxlIjp7ImJvZHkiOnsibmFtZSI6ImRhc2hlZCJ9LCJoZWFkIjp7Im5hbWUiOiJub25lIn19fV0sWzIsMSwiIiwwLHsiY29sb3VyIjpbMjQwLDYwLDYwXSwic3R5bGUiOnsiYm9keSI6eyJuYW1lIjoiZGFzaGVkIn0sImhlYWQiOnsibmFtZSI6Im5vbmUifX19XV0=
\[\begin{tikzcd}[cramped, column sep=scriptsize]
	{} & \bullet && \bullet \\
	\\
	{} & \bullet && \bullet && \bullet
	\arrow[shorten <=8pt, from=1-1, to=1-2]
	\arrow["a", from=1-2, to=1-4]
	\arrow[color={rgb,255:red,92;green,92;blue,214}, dashed, no head, from=1-2, to=3-4]
	\arrow[color={rgb,255:red,92;green,92;blue,214}, dashed, no head, from=1-4, to=3-6]
	\arrow[shorten <=8pt, from=3-1, to=3-2]
	\arrow[color={rgb,255:red,92;green,92;blue,214}, dashed, no head, from=3-2, to=1-2]
	\arrow["\tau"', from=3-2, to=3-4]
	\arrow["a", from=3-4, to=3-6]
\end{tikzcd}\]

	
	Two processes that are Weak Bisimulation equivalent but not Branching Bisimulation equivalent?
	% https://q.uiver.app/#q=WzAsOSxbMSwwLCJcXGJ1bGxldCJdLFsxLDEsIlxcYnVsbGV0Il0sWzAsMiwiXFxidWxsZXQiXSxbMiwyLCJcXGJ1bGxldCJdLFs1LDAsIlxcYnVsbGV0Il0sWzQsMSwiXFxidWxsZXQiXSxbNiwxLCJcXGJ1bGxldCJdLFs0LDIsIlxcYnVsbGV0Il0sWzYsMiwiXFxidWxsZXQiXSxbMCwxLCJcXHRhdSJdLFsxLDIsImEiLDJdLFsxLDMsImIiXSxbNCw1LCJcXHRhdSIsMl0sWzQsNiwiXFx0YXUiXSxbNSw3LCJhIiwyXSxbNiw4LCJiIl0sWzAsNCwiIiwyLHsiY29sb3VyIjpbMjQwLDYwLDYwXSwic3R5bGUiOnsiYm9keSI6eyJuYW1lIjoiZGFzaGVkIn0sImhlYWQiOnsibmFtZSI6Im5vbmUifX19XSxbMiw3LCIiLDEseyJjdXJ2ZSI6MiwiY29sb3VyIjpbMjQwLDYwLDYwXSwic3R5bGUiOnsiYm9keSI6eyJuYW1lIjoiZGFzaGVkIn0sImhlYWQiOnsibmFtZSI6Im5vbmUifX19XSxbMyw4LCIiLDEseyJjdXJ2ZSI6MiwiY29sb3VyIjpbMjQwLDYwLDYwXSwic3R5bGUiOnsiYm9keSI6eyJuYW1lIjoiZGFzaGVkIn0sImhlYWQiOnsibmFtZSI6Im5vbmUifX19XSxbMSw1LCIiLDEseyJjb2xvdXIiOlsyNDAsNjAsNjBdLCJzdHlsZSI6eyJib2R5Ijp7Im5hbWUiOiJkYXNoZWQifSwiaGVhZCI6eyJuYW1lIjoibm9uZSJ9fX1dLFs2LDEsIiIsMSx7ImN1cnZlIjotMiwiY29sb3VyIjpbMjQwLDYwLDYwXSwic3R5bGUiOnsiYm9keSI6eyJuYW1lIjoiZGFzaGVkIn0sImhlYWQiOnsibmFtZSI6Im5vbmUifX19XV0=
	\[\begin{tikzcd}[cramped, column sep=scriptsize]
			& \bullet &&&& \bullet \\
			& \bullet &&& \bullet && \bullet \\
			\bullet && \bullet && \bullet && \bullet
			\arrow[color={rgb,255:red,92;green,92;blue,214}, dashed, no head, from=1-2, to=1-6]
			\arrow["\tau", from=1-2, to=2-2]
			\arrow["\tau"', from=1-6, to=2-5]
			\arrow["\tau", from=1-6, to=2-7]
			\arrow[color={rgb,255:red,92;green,92;blue,214}, dashed, no head, from=2-2, to=2-5]
			\arrow["a"', from=2-2, to=3-1]
			\arrow["b", from=2-2, to=3-3]
			\arrow["a"', from=2-5, to=3-5]
			\arrow[color={rgb,255:red,92;green,92;blue,214}, curve={height=-12pt}, dashed, no head, from=2-7, to=2-2]
			\arrow["b", from=2-7, to=3-7]
			\arrow[color={rgb,255:red,92;green,92;blue,214}, curve={height=12pt}, dashed, no head, from=3-1, to=3-5]
			\arrow[color={rgb,255:red,92;green,92;blue,214}, curve={height=12pt}, dashed, no head, from=3-3, to=3-7]
		\end{tikzcd}\]

		Two processes that are Simulation equivalent but not Bisimulation equivalent
% https://q.uiver.app/#q=WzAsMjQsWzUsMiwiXFxidWxsZXQiXSxbNCwzLCJcXGJ1bGxldCJdLFs1LDEsIlxcYnVsbGV0Il0sWzYsMywiXFxidWxsZXQiXSxbMiwyLCJcXGJ1bGxldCJdLFszLDMsIlxcYnVsbGV0Il0sWzEsMywiXFxidWxsZXQiXSxbMSwxLCJcXGJ1bGxldCJdLFswLDIsIlxcYnVsbGV0Il0sWzAsMywiXFxidWxsZXQiXSxbMSwwXSxbNSwwXSxbMSw0XSxbMSw1LCJcXGJ1bGxldCJdLFsyLDYsIlxcYnVsbGV0Il0sWzMsNywiXFxidWxsZXQiXSxbMSw3LCJcXGJ1bGxldCJdLFswLDYsIlxcYnVsbGV0Il0sWzAsNywiXFxidWxsZXQiXSxbNSw0XSxbNSw1LCJcXGJ1bGxldCJdLFs1LDYsIlxcYnVsbGV0Il0sWzQsNywiXFxidWxsZXQiXSxbNiw3LCJcXGJ1bGxldCJdLFswLDEsImIiLDJdLFsyLDAsImEiXSxbMCwzLCJjIl0sWzQsNSwiYyJdLFs0LDYsImIiLDJdLFs3LDQsImEiXSxbNyw4LCJhIiwyXSxbOCw5LCJiIiwyXSxbMTAsNywiIiwwLHsic2hvcnRlbiI6eyJzb3VyY2UiOjUwfX1dLFsxMSwyLCIiLDAseyJzaG9ydGVuIjp7InNvdXJjZSI6NTB9fV0sWzcsMiwiIiwwLHsiY29sb3VyIjpbMjQwLDYwLDYwXSwic3R5bGUiOnsiYm9keSI6eyJuYW1lIjoiZGFzaGVkIn19fV0sWzQsMCwiIiwwLHsiY3VydmUiOjIsImNvbG91ciI6WzI0MCw2MCw2MF0sInN0eWxlIjp7ImJvZHkiOnsibmFtZSI6ImRhc2hlZCJ9fX1dLFs4LDAsIiIsMCx7ImN1cnZlIjotMiwiY29sb3VyIjpbMjQwLDYwLDYwXSwic3R5bGUiOnsiYm9keSI6eyJuYW1lIjoiZGFzaGVkIn19fV0sWzksMSwiIiwyLHsiY3VydmUiOjMsImNvbG91ciI6WzI0MCw2MCw2MF0sInN0eWxlIjp7ImJvZHkiOnsibmFtZSI6ImRhc2hlZCJ9fX1dLFs1LDMsIiIsMSx7ImN1cnZlIjozLCJjb2xvdXIiOlsyNDAsNjAsNjBdLCJzdHlsZSI6eyJib2R5Ijp7Im5hbWUiOiJkYXNoZWQifX19XSxbNiwxLCIiLDEseyJjdXJ2ZSI6LTIsImNvbG91ciI6WzI0MCw2MCw2MF0sInN0eWxlIjp7ImJvZHkiOnsibmFtZSI6ImRhc2hlZCJ9fX1dLFsxMiwxMywiIiwwLHsic2hvcnRlbiI6eyJzb3VyY2UiOjUwfX1dLFsxMywxNCwiYSJdLFsxNCwxNSwiYyJdLFsxNCwxNiwiYiIsMl0sWzEzLDE3LCJhIiwyXSxbMTcsMTgsImIiLDJdLFsxOSwyMCwiIiwwLHsic2hvcnRlbiI6eyJzb3VyY2UiOjUwfX1dLFsyMCwyMSwiYSJdLFsyMSwyMiwiYiIsMl0sWzIxLDIzLCJjIl0sWzIwLDEzLCIiLDAseyJjb2xvdXIiOlsyNDAsNjAsNjBdLCJzdHlsZSI6eyJib2R5Ijp7Im5hbWUiOiJkYXNoZWQifX19XSxbMjEsMTQsIiIsMSx7ImNvbG91ciI6WzI0MCw2MCw2MF0sInN0eWxlIjp7ImJvZHkiOnsibmFtZSI6ImRhc2hlZCJ9fX1dLFsyMiwxNiwiIiwxLHsiY3VydmUiOi0yLCJjb2xvdXIiOlsyNDAsNjAsNjBdLCJzdHlsZSI6eyJib2R5Ijp7Im5hbWUiOiJkYXNoZWQifX19XSxbMjMsMTUsIiIsMSx7ImN1cnZlIjotMiwiY29sb3VyIjpbMjQwLDYwLDYwXSwic3R5bGUiOnsiYm9keSI6eyJuYW1lIjoiZGFzaGVkIn19fV1d
\[\begin{tikzcd}[cramped, column sep=scriptsize]
	& {} &&&& {} \\
	& \bullet &&&& \bullet \\
	\bullet && \bullet &&& \bullet \\
	\bullet & \bullet && \bullet & \bullet && \bullet \\
	& {} &&&& {} \\
	& \bullet &&&& \bullet \\
	\bullet && \bullet &&& \bullet \\
	\bullet & \bullet && \bullet & \bullet && \bullet
	\arrow[shorten <=5pt, from=1-2, to=2-2]
	\arrow[shorten <=5pt, from=1-6, to=2-6]
	\arrow[color={rgb,255:red,92;green,92;blue,214}, dashed, from=2-2, to=2-6]
	\arrow["a"', from=2-2, to=3-1]
	\arrow["a", from=2-2, to=3-3]
	\arrow["a", from=2-6, to=3-6]
	\arrow[color={rgb,255:red,92;green,92;blue,214}, curve={height=-12pt}, dashed, from=3-1, to=3-6]
	\arrow["b"', from=3-1, to=4-1]
	\arrow[color={rgb,255:red,92;green,92;blue,214}, curve={height=12pt}, dashed, from=3-3, to=3-6]
	\arrow["b"', from=3-3, to=4-2]
	\arrow["c", from=3-3, to=4-4]
	\arrow["b"', from=3-6, to=4-5]
	\arrow["c", from=3-6, to=4-7]
	\arrow[color={rgb,255:red,92;green,92;blue,214}, curve={height=18pt}, dashed, from=4-1, to=4-5]
	\arrow[color={rgb,255:red,92;green,92;blue,214}, curve={height=-12pt}, dashed, from=4-2, to=4-5]
	\arrow[color={rgb,255:red,92;green,92;blue,214}, curve={height=18pt}, dashed, from=4-4, to=4-7]
	\arrow[shorten <=5pt, from=5-2, to=6-2]
	\arrow[shorten <=5pt, from=5-6, to=6-6]
	\arrow["a"', from=6-2, to=7-1]
	\arrow["a", from=6-2, to=7-3]
	\arrow[color={rgb,255:red,92;green,92;blue,214}, dashed, from=6-6, to=6-2]
	\arrow["a", from=6-6, to=7-6]
	\arrow["b"', from=7-1, to=8-1]
	\arrow["b"', from=7-3, to=8-2]
	\arrow["c", from=7-3, to=8-4]
	\arrow[color={rgb,255:red,92;green,92;blue,214}, dashed, from=7-6, to=7-3]
	\arrow["b"', from=7-6, to=8-5]
	\arrow["c", from=7-6, to=8-7]
	\arrow[color={rgb,255:red,92;green,92;blue,214}, curve={height=-12pt}, dashed, from=8-5, to=8-2]
	\arrow[color={rgb,255:red,92;green,92;blue,214}, curve={height=-12pt}, dashed, from=8-7, to=8-4]
\end{tikzcd}\]
\end{xmp}

\lipsum[1-12]

\end{multicols}
\end{document}
