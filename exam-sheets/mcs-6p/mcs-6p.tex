\documentclass[landscape, 8pt]{extarticle}

\usepackage{../../preamble}
\usepackage{symbols}

\begin{document}

\setlength{\abovedisplayskip}{3.5pt}
\setlength{\belowdisplayskip}{3.5pt}
\setlength{\abovedisplayshortskip}{3.5pt}
\setlength{\belowdisplayshortskip}{3.5pt}

\begin{multicols}{3}
\raggedcolumns


\section*{\huge Modelling Concurrent Systems Notes}
Made by Leon :)


\vspace{-5pt}
\section{Process Algebras}
\setcounter{subsection}{1}

\centering
\begin{dfn}[ACP, CCS, CSP]{dfn:acp-ccs-csp}{}
The syntax of ACP, CCS, and CSP is defined as:
\begin{tabular}{ |r|c|c|c|}
	\hline
	\textbf{Operation}     & \textbf{ACP}   & \textbf{CCS}     & \textbf{CSP}     \\
	\hline
	Termination            & $\epsilon$     & $0$              & $\mathrm{STOP}$  \\
	\hline
	Deadlock               & $\delta$       &                  &                  \\
	\hline
	Action                 & a              &                  &                  \\
	Sequential Composition & $P.Q$          &                  &                  \\
	Action Prefixing       &                & $a.P$            & $a \to P$        \\
	\hline
	Alternative Choice     & $P + Q$        & $P + Q$          &                  \\
	External Choice        &                &                  & $P \extchoice Q$ \\
	Internal Choice        &                &                  & $P \intchoice Q$ \\
	\hline
	Parallel Composition   & $P \merge Q$   & $P \mid Q$       & $P \pcomp Q$     \\
	\hline
	Restriction            & $\restrict(P)$ & $P \backslash a$ &                  \\
	\hline
	Abstraction            & $\abtau(P)$    &                  & $P / a$          \\
	\hline
	Relabelling            &                & $P[f]$           & $P[f]$           \\
	\hline
\end{tabular}

\vspace{10pt}
\longrule{0.08ex}

\textbf{Differences between ACP, CCS, and CSP}
\begin{itemize-zero}
\item \textbf{Action}: CCS and CSP require Action Prefixing, while ACP can perform sequential composition on any process. This also requires CCS and CSP processes to feature the inaction $0$/$\mathrm{STOP}$, while ACP is not restricted to this.
\item \textbf{Choice}: ACP and CCS have an operator which can perform both External and Internal Choice. CSP Differentiates internal choices from external ones, and internal actions within $\extchoice$ do not count as a choice in CSP.
\item \textbf{Parallel Composition}:
\begin{itemize-tight}
\item  ACP actions follow a communication function to decide what to synchronise, i.e. $\gamma(a, b)$
\item CCS actions can only synchronise with its conjugate counterpart, i.e. $a$ and $\overline{a}$
\item CSP actions can only synchronise over the same action, i.e. $a$ and $a$
\end{itemize-tight}
\item \textbf{Restriction, Abstraction, Relabelling}:
\begin{itemize-tight}
\item Relabelling just doesn't exist in base ACP, lol
\item CCS combines communication and abstraction into one step - every synchronisation results in a $\tau$.
\item CSP combines Parallel Composition and Restriction into one step, as CSP Parallel Composition doesn't feature left-over Left Merges.
\end{itemize-tight}
\end{itemize-zero}
\end{dfn}

\begin{dfn}[The GSOS Format]{dfn:gsos}{}
General Structured Operational Semantics (GSOS) operations are compositional.

\textbf{Rules of GSOS}
\begin{itemize-zero}
\item Its source has the form $f(x_{1},\dots,x_{ar(f)})$ with $f\in \Sigma$ and $x_{i}\in V$
\item The left hand sides of its premises are variables $x_{i}$ with $1\le i \le ar(f)$
\item The right hand sides of its positive premises are variables that are all distinct, and that do not occur in its source
\item Its target only contains variables that also occur in its source or premises
\end{itemize-zero}


\textbf{GSOS Semantics of ACP}

\framebox{$\begin{array}{ccccc}
			(a. P) \prightarrow{\alpha} P                                                                                                  &
			P + Q \prightarrow{\alpha} P                                                                                                   &
			P + Q \prightarrow{\alpha} Q                                                                                                                           \\[2ex]
			% \displaystyle\frac{P \prightarrow{\alpha} P'}{f(P) \prightarrow{f(\alpha)} f(P')} \\[4ex]
			\displaystyle\frac{P\prightarrow{\alpha} P'}{P \merge Q \prightarrow{\alpha} P' \merge Q}                                      &

			\multicolumn{2}{c}{\displaystyle\frac{P\prightarrow{a} P'~~Q\prightarrow{b} Q'~~ \scriptstyle{a \mid b = c}}{P \merge Q \prightarrow{a} P' \merge Q'}} \\[4ex]

			\displaystyle\frac{Q\prightarrow{\alpha} Q'}{P \merge Q \prightarrow{\alpha} P \merge Q'}                                      &

			\displaystyle\frac{P \prightarrow{\alpha} P'~~{\scriptstyle(\alpha\notin A)}}{\restrict(P) \prightarrow{\alpha} \restrict(P')} &

			\displaystyle\frac{\langle \mathcal{S}_{X} \mid \mathcal{S} \rangle \prightarrow{a} P'}{\langle X \mid \mathcal{S} \rangle \prightarrow{a} P'}
		\end{array}$}

\textbf{GSOS Semantics of CSP}
\begin{center}
	\framebox{$\begin{array}{ccc}
				(a\rightarrow P) \prightarrow{a} P                                                                                     &
				P \intchoice Q \prightarrow{\tau} P                                                                                    &
				P \intchoice Q \prightarrow{\tau} Q                                                                                      \\[2ex]
				\displaystyle\frac{P\prightarrow{a} P'}{P\extchoice Q \prightarrow{a} P'}                                              &
				\displaystyle\frac{P\prightarrow{\tau} P'}{P\extchoice Q \prightarrow{\tau} P'\extchoice Q}                            &
				\displaystyle\frac{Q\prightarrow{a} Q'}{P\extchoice Q \prightarrow{a} Q'}                                                \\[4ex]
				\displaystyle\frac{Q\prightarrow{\tau} Q'}{P\extchoice Q \prightarrow{\tau} P\extchoice Q'}                            &
				\displaystyle\frac{P \prightarrow{\alpha} P'}{f(P) \prightarrow{f(\alpha)} f(P')}                                      &
				\displaystyle\frac{P\prightarrow{\alpha} P'~~{\scriptstyle(\alpha\notin A)}}{P\|_AQ \prightarrow{\alpha} P'\|_AQ}        \\[4ex]

				\multicolumn{2}{c}{
				\displaystyle\frac{P\prightarrow{a} P'~~Q\prightarrow{a} Q'~~{\scriptstyle(a\in A)}}{P\|_AQ \prightarrow{a} P'\|_AQ'}} &
				\displaystyle\frac{Q\prightarrow{\alpha} Q'~~{\scriptstyle(\alpha\notin A)}}{P\|_AQ \prightarrow{\alpha} P\|_AQ'}        \\[4ex]
				\multicolumn{3}{c}{\mu p.P \xrightarrow{\mathmakebox[10pt]{\tau}} P[\mu p.P/p]}
			\end{array}$}
\end{center}

\end{dfn}

\begin{dfn}[CSP Expansion Theorem]{dfn:expansion-theorem}{}
	Let $P := \sum_{i\in I} \alpha_{i} P_{i}$ and $Q := \sum_{j\in J} \beta_{j}. Q_{j}$. Then
	\[P \mid Q = \sum_{i\in I} \alpha_{i}(P_{i} \mid Q) + \sum_{j\in J} \beta_{j}(P \mid Q_{j}) + \sum_{\substack{i\in I, j\in J \\ \alpha_{i} = \overline{b}_{j}}} \tau.(P_{i} \mid Q_{j})\]
	Any guarded CCS expression can be written into a bisimulation equivalent CCS expression of the form $\sum_{i\in I} \alpha_{i}.P_{i}$. This is called \textbf{head normal form}
\end{dfn}

\begin{dfn}[CCS Axioms]{dfn:csp-axions}{}
\begin{itemize-tight}
\item Axioms of CCS
\begin{align*}
	(P + Q ) + R & = P + (Q + R) \tag{associativity}  \\
	P + Q        & = Q + Q \tag{commutativity}        \\
	P + P        & = P \tag{idempotence}              \\
	P + 0        & = P \tag{$0$ is a neutral element}
\end{align*}
\item Axiomatisation of Rooted Weak Bisimulation
\begin{align*}
	\alpha . \tau . P       & = \alpha.P \tag{T1}                       \\
	\tau . P                & = \tau.P + P \tag{T2}                     \\
	\alpha . (\tau . P + Q) & = \alpha.(\tau.P + Q) + \alpha.P \tag{T3}
\end{align*}
\item Axiomatisation of Branching Bisimularity
\begin{align*}
	\alpha . (\tau . (P + Q) + Q) & = \alpha.(P + Q) \tag{P}
\end{align*}
\item Axiomatisation of strong partial trace equivalence
\[\alpha . (P + Q) = \alpha . P + \alpha . Q\]
\item Axiomatisation of weak partial trace equivalence
\[\tau.P = P\]
\end{itemize-tight}
\end{dfn}

% \begin{dfn}[Axioms of ACP]{dfn:acp-axioms}{}
% 	is this really necessary.. who knows
% \end{dfn}

\begin{dfn}[The Recursive Specification]{dfn:rsp}{}
	The \textbf{recursive definition principle} (RDP) says that a system satisfies its recursive definition. The \textbf{recursive specification principle} (RSP) says that two processes satisfying the same \textbf{guarded} recursive equation must be equal. It holds for SB and Strong Finite PT.

	\longrule{0.08ex}
	If $S$ is a recursive specification, i.e. a set of equations $X_{i} = S_{i}$ for each $i\in I$ where $I$ is some index set, then $S[P_{i} /X_{i}]_{i\in I}$ consists of the equations $P_{i} = S_{i}[P_{i} /X_{i}]_{i\in I}$ for $i\in I$. The family $(P_{i})_{i\in I}$ of processes $P_{i}$ constitutes a \textbf{solution} of $S$, up to a semantic equivalence $\sim$ iff the equations in $S[P_{i} /X_{i}]_{i\in I}$ become true when interpreting $=$ as $\sim$

	\longrule{0.08ex}
	RDP says that each recursive specification has a solution (up to $\sim$), and RSP says that two solutions (up to $\sim$) of the same guarded recursive specification must be $\sim$-equivalent. Thus RSP can be formulated as the following proof rule
	\[\prftree{P_{i} = S_{i}[P_{i} /X_{i}]_{i\in I} \quad Q_{i} = S_{i}[Q_{i} /X_{i}]_{i\in I} \quad \text{ for } i\in I}{P_{i} = Q_{i} \quad \text{ for } i \in I}\]
\end{dfn}

\newpage
\section{Semantics and shit like that}
\setcounter{subsection}{1}
\begin{dfn}[Trace Semantice]{dfn:trace-semantics}{}
\begin{itemize-zero}
\item \textbf{Completed Trace}: A start to finish trace of a process.
\item \textbf{Partial Trace}: From the start of a process to any point, including the end. Clearly, $CT(P) \subseteq PT(Q)$
\item \textbf{Strong vs Weak}: Weak PT and CT means that two processes are equivalent with all instances of $\tau$ omitted.
\item \textbf{Infinite Trace Semantics}: Differs from different types of divergence. Stronger than CT and PT
\end{itemize-zero}
\end{dfn}

\begin{dfn}[Bisimulation Semantice]{dfn:bisimulation}{}
\begin{itemize-zero}
\item \textbf{True Bisimulation} ($\leftrightarroweq$):
\begin{itemize-tight}
\item if $sRt$ and $s \prightarrow{a} s'$ then $\exists t'$ s.t. $t \prightarrow{a} t'$ and $s' R t'$
\item if $sRt$ and $t \prightarrow{a} t'$ then $\exists s'$ s.t. $s \prightarrow{a} s'$ and $s' R t'$
\item if $sRt$ then $s\models p \iff t \models p$ for all $p\in P$
\end{itemize-tight}
\item \textbf{Branching Bisimilarity} ($\rbrb$)
\begin{itemize-tight}
\item if $sRt$ and $s \prightarrow{a} s'$ then either:
\begin{itemize-tight}
\item $a = \tau$ and $s' R t$
\item $\exists t_{1},\,t'$ such that $t \Rightarrow t_{1} \prightarrow{a} t'$, $s R t_{1}$ and $s' R t'$
\end{itemize-tight}
\item if $sRt$ and $t \prightarrow{a} t'$ then either:
\begin{itemize-tight}
\item $a = \tau$ and $t' R s$
\item $\exists s_{1},\,s'$ such that $s \Rightarrow s_{1} \prightarrow{a} s'$, $s_{1} R t$ and $s' R t'$
\end{itemize-tight}
\item if $sRt$ and $s\models p$ then $\exists t_{1}$ s.t. $t \Rightarrow t_{1} \models p$, and $sRt_{1}$
\item if $sRt$ and $t\models p$ then $\exists s_{1}$ s.t. $s \Rightarrow s_{1} \models p$, and $s_{1}Rt$
\end{itemize-tight}
\item Other notions:
\begin{itemize-tight}
\item \textbf{Rooted Branching Bisimilarity}: The same as Branching Bisimilarity except the first action is Strongly bisimilar. (This makes RBB a congruence on $+$)
\item \textbf{Delay Bisimilarity}: Same as \textit{branching bisimilarity}, but with the requirements $sRt_{1}$ and $s_{1}Rt$ dropped.
\item \textbf{Weak Bisimilarity}: The same as \textit{delay bisimularity} except the action requirements are also relaxed:
\begin{itemize-tight}
\item If $sRt$ and $s \prightarrow{a} s'$ then either:
\begin{itemize-tight}
\item $a = \tau$ and $s' R t$
\item $\exists t_{1},\,t_{2},\,t'$ such that $t \Rightarrow t_{1} \prightarrow{a} t_{2} \Rightarrow t'$ and $s'Rt'$
\end{itemize-tight}
\item If $sRt$ and $t \prightarrow{a} t'$ then either:
\begin{itemize-tight}
\item $a = \tau$ and $s R t'$
\item $\exists s_{1},\,s_{2},\,s'$ such that $s \Rightarrow s_{1} \prightarrow{a} s_{2} \Rightarrow s'$ and $s'Rt'$
\end{itemize-tight}
\end{itemize-tight}
\item \textbf{Simulation}: One process simulates the other when $P$ can do all the same moves as $Q$. We write $P \sqsubseteq_{S} Q$ if $Q$ can be simulated by $P$. Two processes are \textbf{simulation equivalent}, $P =_{S} Q$ if one simulates the other, and vice versa. This is two one-sided equivalences, and therefore is not the same as bisimulation, which needs both processes to be equivalent at the same time
\end{itemize-tight}

\end{itemize-zero}
\end{dfn}

\begin{dfn}[Compositionality, Congruence]{dfn:congruence}{}
	An equivalence $\sim$ is a \textbf{congruence}\footnote{We can also say \textbf{the language is compositional for the equivalence}} for a language $L$ if $P \sim Q$ implies that $C[P] \sim C[Q]$ for every context $C[\ ]$, where $C[\ ]$ is an $L$-expression with a \textbf{hole} in it, and $C[P]$ is the result of plugging in $P$ for the hole. An alternative definition for a congruence $\sim$ is if every $n$-ary operator $f$ of $L$, we have
	\[P_{i} \sim Q_{i} \text{ for } i = 1,\dots, n \text{ implies } f(P_{1},\dots,P_{n}) \sim f(Q_{1} ,\dots, Q_{n})\]
	\longrule{0.08ex}
	The \textbf{Congruence closure} of a language, denoted $\sim^{c}$ of a language is a modification to a language that isn't compositional to turn it compositional. The \textit{congruence closure} of Branching Bisimilarity is Rooted Branching Bisimilarity.
\end{dfn}

\begin{thm}[Semantic Equivalence Spectrum]{thm:spectrum}{}
	% https://q.uiver.app/#q=WzAsMTQsWzQsNiwiXFxtYXRocm17Q1R9Il0sWzQsMTAsIlxcbWF0aHJte1BUfSJdLFs0LDMsIlxcbWF0aHJte1dCfSJdLFsxLDAsIlxcbWF0aHJte0JCfSJdLFszLDgsIlxcbWF0aHJte1dDVH0iXSxbMywxMiwiXFxtYXRocm17V1BUfSJdLFswLDExLCJcXG1hdGhybXtXUFR9XlxcaW5mdHkiXSxbMSw1LCJcXG1hdGhybXtDVH1eXFxpbmZ0eSJdLFsxLDksIlxcbWF0aHJte1BUfV5cXGluZnR5Il0sWzAsNywiXFxtYXRocm17V0NUfV5cXGluZnR5Il0sWzMsMiwiXFxtYXRocm17QkJ9Il0sWzQsMiwiXFxtYXRocm17V0J9XkNcXFxcKFxcbWF0aHJte1JXQn0pIl0sWzMsMSwiXFxtYXRocm17QkJ9XkNcXFxcKFxcbWF0aHJte1JCQn0pIl0sWzEsNCwiXFxtYXRocm17Q1R9XkMiXSxbNSw0XSxbNCwwLCIiLDAseyJzdHlsZSI6eyJoZWFkIjp7Im5hbWUiOiJub25lIn19fV0sWzEsMCwiIiwyLHsic3R5bGUiOnsiaGVhZCI6eyJuYW1lIjoibm9uZSJ9fX1dLFs1LDFdLFswLDIsIiIsMix7InN0eWxlIjp7ImhlYWQiOnsibmFtZSI6Im5vbmUifX19XSxbNSw2LCIiLDEseyJzdHlsZSI6eyJoZWFkIjp7Im5hbWUiOiJub25lIn19fV0sWzAsNywiIiwxLHsic3R5bGUiOnsiaGVhZCI6eyJuYW1lIjoibm9uZSJ9fX1dLFsxLDgsIiIsMSx7InN0eWxlIjp7ImhlYWQiOnsibmFtZSI6Im5vbmUifX19XSxbNCw5LCIiLDEseyJzdHlsZSI6eyJoZWFkIjp7Im5hbWUiOiJub25lIn19fV0sWzksNywiIiwxLHsic3R5bGUiOnsiaGVhZCI6eyJuYW1lIjoibm9uZSJ9fX1dLFs5LDYsIiIsMSx7InN0eWxlIjp7ImhlYWQiOnsibmFtZSI6Im5vbmUifX19XSxbNiw4LCIiLDEseyJzdHlsZSI6eyJoZWFkIjp7Im5hbWUiOiJub25lIn19fV0sWzcsOCwiIiwxLHsic3R5bGUiOnsiaGVhZCI6eyJuYW1lIjoibm9uZSJ9fX1dLFsyLDEwLCIiLDAseyJzdHlsZSI6eyJoZWFkIjp7Im5hbWUiOiJub25lIn19fV0sWzExLDIsIiIsMSx7InN0eWxlIjp7ImhlYWQiOnsibmFtZSI6Im5vbmUifX19XSxbMTEsMTIsIiIsMSx7InN0eWxlIjp7ImhlYWQiOnsibmFtZSI6Im5vbmUifX19XSxbMTAsMTIsIiIsMSx7InN0eWxlIjp7ImhlYWQiOnsibmFtZSI6Im5vbmUifX19XSxbNywxM10sWzEzLDMsIiIsMSx7InN0eWxlIjp7ImhlYWQiOnsibmFtZSI6Im5vbmUifX19XSxbMTIsM11d
	\[\begin{tikzcd}[cramped, row sep=tiny, column sep=scriptsize]
			& {\mathrm{BB}} \\
			&&& \begin{array}{c} \mathrm{BB}^C\\(\mathrm{RBB}) \end{array} \\
			&&& {\mathrm{BB}} & \begin{array}{c} \mathrm{WB}^C\\(\mathrm{RWB}) \end{array} \\
			&&&& {\mathrm{WB}} \\
			& {\mathrm{CT}^C} \\
			& {\mathrm{CT}^\infty} \\
			&&&& {\mathrm{CT}} \\
			{\mathrm{WCT}^\infty} \\
			&&& {\mathrm{WCT}} \\
			& {\mathrm{PT}^\infty} \\
			&&&& {\mathrm{PT}} \\
			{\mathrm{WPT}^\infty} \\
			&&& {\mathrm{WPT}}
			\arrow[from=2-4, to=1-2]
			\arrow[no head, from=3-4, to=2-4]
			\arrow[no head, from=3-5, to=2-4]
			\arrow[no head, from=3-5, to=4-5]
			\arrow[no head, from=4-5, to=3-4]
			\arrow[no head, from=5-2, to=1-2]
			\arrow[from=6-2, to=5-2]
			\arrow[no head, from=6-2, to=10-2]
			\arrow[no head, from=7-5, to=4-5]
			\arrow[no head, from=7-5, to=6-2]
			\arrow[no head, from=8-1, to=6-2]
			\arrow[no head, from=8-1, to=12-1]
			\arrow[no head, from=9-4, to=7-5]
			\arrow[no head, from=9-4, to=8-1]
			\arrow[no head, from=11-5, to=7-5]
			\arrow[no head, from=11-5, to=10-2]
			\arrow[no head, from=12-1, to=10-2]
			\arrow[from=13-4, to=9-4]
			\arrow[from=13-4, to=11-5]
			\arrow[no head, from=13-4, to=12-1]
		\end{tikzcd}\]
	Simulation is finer than PT, coarser than B, incomparable to CT.
\end{thm}

% \begin{dfn}[The GOAT of Process algebra]{dfn:bisim-counterexample}{}
% 	% https://q.uiver.app/#q=WzAsMTEsWzUsMSwiXFxjaXJjIl0sWzQsMiwiXFxidWxsZXQiXSxbNiwyLCJcXGJ1bGxldCJdLFs2LDMsIlxcYnVsbGV0Il0sWzQsMywiXFxidWxsZXQiXSxbMSwxLCJcXGNpcmMiXSxbMSwyLCJcXGJ1bGxldCJdLFsyLDMsIlxcYnVsbGV0Il0sWzAsMywiXFxidWxsZXQiXSxbNSwwXSxbMSwwXSxbMCwxLCJhIl0sWzAsMiwiYSIsMl0sWzIsMywiYyIsMl0sWzEsNCwiYiJdLFs1LDYsImEiXSxbNiw3LCJjIl0sWzYsOCwiYiJdLFs5LDAsIiIsMCx7InNob3J0ZW4iOnsic291cmNlIjo1MH19XSxbMTAsNSwiIiwwLHsic2hvcnRlbiI6eyJzb3VyY2UiOjUwfX1dXQ==
% 	\[\begin{tikzcd}[cramped]
% 			& {} &&&& {} \\
% 			& \circ &&&& \circ \\
% 			& \bullet &&& \bullet && \bullet \\
% 			\bullet && \bullet && \bullet && \bullet
% 			\arrow[shorten <=5pt, from=1-2, to=2-2]
% 			\arrow[shorten <=5pt, from=1-6, to=2-6]
% 			\arrow["a", from=2-2, to=3-2]
% 			\arrow["a", from=2-6, to=3-5]
% 			\arrow["a"', from=2-6, to=3-7]
% 			\arrow["b", from=3-2, to=4-1]
% 			\arrow["c", from=3-2, to=4-3]
% 			\arrow["b", from=3-5, to=4-5]
% 			\arrow["c"', from=3-7, to=4-7]
% 		\end{tikzcd}\]
% \end{dfn}
% %
% \begin{dfn}[Failure Semantics]{dfn:failure}{}
%
% \end{dfn}
%
% %
% % \begin{dfn}[Consistent Colouring]{dfn:colouring}{}
% %
% % \end{dfn}
%
\begin{dfn}[Safety]{dfn:}{}
	A process $p$ satisfies the \textbf{canonical safety property}, $P \models \mathrm{safety}(b)$ if no trace of $P$ contains $B$

	\longrule{0.08ex}
	A \textbf{safety property} of processes in an LTS is given by a set $B \subseteq A^{*}$. A process $p$ satisfies this safety property, $P \models \mathrm{safety}(B)$ when $\mathrm{WPT}(P) \cap B = \emptyset.$

	\longrule{0.08ex}
	If $P =_{\mathrm{WPT}} Q$ and $P \models \mathrm{safety}(B)$ for some $B \subseteq A^{*}$ then $Q \models \mathrm{safety}(B)$.
\end{dfn}

\begin{dfn}[Deadlock Types]{dfn:deadlock}{}
	\vspace{-10pt}
% https://q.uiver.app/#q=WzAsMTgsWzAsMCwiXFxidWxsZXQiXSxbMCwxLCJcXGJ1bGxldCJdLFswLDIsIlxcYnVsbGV0Il0sWzAsMywibm9ybWFsIFxcXFxwcm9jZXNzIl0sWzEsMCwiXFxidWxsZXQiXSxbMSwxLCJcXGJ1bGxldCJdLFsxLDIsIlxcYnVsbGV0Il0sWzEsMywiRGVhZGxvY2siXSxbMiwxLCJcXGJ1bGxldCJdLFszLDAsIlxcYnVsbGV0Il0sWzMsMSwiXFxidWxsZXQiXSxbMywyLCJcXGJ1bGxldCJdLFs0LDEsIlxcYnVsbGV0Il0sWzUsMCwiXFxidWxsZXQiXSxbMywzLCJMaXZlbG9jayBcXFxcIERpdmVyZ2VudCJdLFs1LDEsIlxcYnVsbGV0Il0sWzUsMiwiXFxidWxsZXQiXSxbNSwzLCJFc2NhcGFibGUgXFxcXCBEaXZlcmdlbmNlIl0sWzAsMSwiYSJdLFsxLDIsImciXSxbNCw1LCJhIl0sWzUsNiwiZyJdLFs1LDgsIlxcdGF1Il0sWzksMTAsImEiXSxbMTAsMTEsImciXSxbMTAsMTIsIlxcdGF1Il0sWzEyLDEyLCJcXHRhdSJdLFsxMywxNSwiYSJdLFsxNSwxNiwiZyJdLFsxNSwxNSwiXFx0YXUiLDAseyJhbmdsZSI6OTB9XV0=
\[\begin{tikzcd}[cramped,column sep=tiny,row sep=small]
	\bullet & \bullet && \bullet && \bullet \\
	\bullet & \bullet & \bullet & \bullet & \bullet & \bullet \\
	\bullet & \bullet && \bullet && \bullet \\
	\begin{array}{c} normal \\process \end{array} & Deadlock && \begin{array}{c} Livelock \\ Divergent \end{array} && \begin{array}{c} Escapable \\ Divergence \end{array}
	\arrow["a", from=1-1, to=2-1]
	\arrow["a", from=1-2, to=2-2]
	\arrow["a", from=1-4, to=2-4]
	\arrow["a", from=1-6, to=2-6]
	\arrow["g", from=2-1, to=3-1]
	\arrow["\tau", from=2-2, to=2-3]
	\arrow["g", from=2-2, to=3-2]
	\arrow["\tau", from=2-4, to=2-5]
	\arrow["g", from=2-4, to=3-4]
	\arrow["\tau", from=2-5, to=2-5, loop, in=55, out=125, distance=10mm]
	\arrow["\tau", from=2-6, to=2-6, loop, in=325, out=35, distance=10mm]
	\arrow["g", from=2-6, to=3-6]
\end{tikzcd}\]
NP: $a.g.0$, ED: $a.\triangle g.0$, LL: $a.(g.0+\tau.\triangle 0)$, DL: $\alpha .(g.0 + \tau.0)$
\end{dfn}

\begin{dfn}[Distinguishing Interleaving]{dfn:distinguishing-interleaving}{}
	\vspace{-5pt}
	When using Petri nets we take into account \textbf{interleaving semantics}, which doesn't distinguish $a \merge b = ab+ba$. This is rejected in causal semantics because in $a \merge b$, $a$ and $b$ are causally independent, whereas in $ab+ba$, either $a$ depends on $b$ or vice versa.
	
	\longrule{0.08ex}
	\textbf{Interval Semantics} lie between step semantics and causal semantics. It takes causality into account only to the extent that it manifests itself by durational actions overlapping in time.
\end{dfn}

\begin{dfn}[Step Bisimulation]{dfn:step-bisim}{}
	A \textbf{step bisimulation} is a binary relation $B$ on the markings of two petri nets, such that
	\begin{itemize-tight}
	    \item the initial markings of the nets are related
	    \item If $M_{1} B M_{2}$ and $M_{1} \prightarrow{L} M_{1}'$ then there is a marking $M_{2}'$ such that $M_{2} \prightarrow{L} M_{2}'$ and $M_{1}' B M_{2}'$
	    \item If $M_{1} B M_{2}$ and $M_{2} \prightarrow{L} M_{2}'$ then there is a marking $M_{1}'$ such that $M_{1} \prightarrow{L} M_{1}'$ and $M_{1}' B M_{2}'$
	\end{itemize-tight}
	Two nets are \textbf{step bisimilar} if there exists a step bisimulation between them.

% https://q.uiver.app/#q=WzAsOCxbMSwwLCJhIFxcbWVyZ2UgYiJdLFswLDEsIjAgXFxtZXJnZSBiIl0sWzIsMSwiYSBcXG1lcmdlIDAiXSxbMSwyLCIwIG1lcmdlIDAiXSxbNCwwLCJcXGJ1bGxldCJdLFszLDEsIlxcYnVsbGV0Il0sWzUsMSwiXFxidWxsZXQiXSxbNCwyLCJcXGJ1bGxldCJdLFswLDEsImEiLDJdLFswLDIsImIiXSxbMiwzLCJhIl0sWzEsMywiYiIsMl0sWzQsNSwiYSIsMl0sWzQsNiwiYiJdLFs1LDcsImIiLDJdLFs2LDcsImEiXSxbMCwzLCIiLDEseyJzdHlsZSI6eyJib2R5Ijp7Im5hbWUiOiJzcXVpZ2dseSJ9fX1dXQ==
\[\begin{tikzcd}[cramped, column sep=small, row sep=small]
	& {a \merge b} &&& \bullet \\
	{0 \merge b} && {a \merge 0} & \bullet && \bullet \\
	& {0 \merge 0} &&& \bullet
	\arrow["a"', from=1-2, to=2-1]
	\arrow["b", from=1-2, to=2-3]
	\arrow[squiggly, from=1-2, to=3-2]
	\arrow["a"', from=1-5, to=2-4]
	\arrow["b", from=1-5, to=2-6]
	\arrow["b"', from=2-1, to=3-2]
	\arrow["a", from=2-3, to=3-2]
	\arrow["b"', from=2-4, to=3-5]
	\arrow["a", from=2-6, to=3-5]
\end{tikzcd}\]
% In this definition, a \textbf{marking} is a global state of the Petri net; it is given by a function that tells for each place how many tokens it contains. We write $M \prightarrow{L} M'$ if a net can go from state/marking $M$ to state/marking $M'$ be simultaneously firing the transitions in a multiset $U$ of transitions, such that $L$ is the multiset of labels of the transitions in $U$. If a place is the preplace of $k$ transitions in $U$, it needs to contain $k$ tokens for this multiset to fire.
\end{dfn}

\begin{dfn}[Pomset]{dfn:pomset}{}
	Causal trace semantics can be represented in terms of \textbf{pomsets}, partially ordered multisets. The process $a(b \merge (c + de))$ has two completed pomsets, $b \leftarrow a \to c$ and $b \leftarrow a \to d \to e$.
\end{dfn}

\newpage

\section{Other models of Concurrency}
\setcounter{subsection}{1}
\begin{dfn}[Hennessy-Milner Logic]{dfn:hml}{}
The syntax of HML is given by:
\[\phi,\,\psi ::= \top \mid \bot \mid \phi \wedge \psi \mid \phi \vee \psi \mid \neg \phi \mid \langle \alpha \rangle \phi \mid [\alpha] \phi\]
Infinitary HML (HML$^{\infty}$) has an infinitary conjunction: $\bigwedge_{i\in I} \phi_{i}$
HML in set form. If a process $P$ has a property $\Phi$, we write $P \models \Phi$.
\begin{itemize-tight}
\item $P \models \top$
\item $P \not\models \bot$
\item $P \models \Phi \wedge \Psi$ iff $P \models \Phi$ and $P \models \Psi$
\item $P \models \Phi \vee \Psi$ iff $P \models \Phi$ or $P \models \Psi$
\item $P \models [K]\Phi$ iff $\forall Q \in \{P' : P \prightarrow{a} P' \text{ and } a\in K\}$. $Q \models \Phi$
\item $P \models \langle K \rangle\Phi$ iff $\exists Q \in \{P' : P \prightarrow{a} P' \text{ and } a\in K\}$. $Q \models \Phi$
\end{itemize-tight}

Deadlock can be represented as $P \models [\mathrm{Act}] \bot$, where $\mathrm{Act}$ is the set of all actions.
\end{dfn}

\begin{dfn}[Preorder]{dfn:preorder}{}
	A \textbf{preorder} is a relation that is \textit{transitive} and \textit{reflexive}, but not \textit{symmetric}. Preorders are denoted with $\sqsubseteq$.

	Preorders are used just like equivalence relations to compare specifications and implementations. We write
	\[\text{\textit{Spec}} \sqsubseteq \text{\textit{Impl}}\]
	For each preorder $\sqsubseteq$, there exists an associated equivalence relation $\equiv$ called its \textbf{kernel}, defined by
	\[P \equiv Q \iff (P \sqsubseteq Q \wedge Q \sqsubseteq P)\]
\end{dfn}


\begin{dfn}[Petri Nets]{dfn:petris}{}
	Petri Nets are defined with actions as squares. Initial states have a number of markings that can transfer to other states.

	Parallel Composition adds a new state, and replaces both actions with the new one.


\end{dfn}

\begin{dfn}[Kripke Structure]{dfn:kripke}{}
	Kripke Structures are defined on states rather than actions, called \textbf{atomic predicates}.

	Let $AP$ be a set of \textbf{atomic predicates}. A \textbf{Kripke structure} over $AP$ is a tuple $(S,\, \to,\, \models)$ with $S$ a set of states, $\to \subseteq S \times S$, the \textbf{transition relation}, and $\models \subseteq S \times AP$. The relation $s \models p$ says that predicate $p\in AP$ \textbf{holds in state $s\in S$}.

	\longrule{0.08ex}
	A \textbf{path} in a Kriple structure is a nonempty finite or infinite sequence $s_{0},\,s_{1} ,\dots$ of states, such as $(s_{i}, s_{i+1})\in \to$ for each adjacent pair of states $s_{i}, s_{i+1}$ in the sequence. A path is \textbf{complete} if it is either infinite or ends in deadlock (a state without outgoing transitions)
\end{dfn}


\begin{dfn}[CTL]{dfn:ctl}{}
Computational Tree Logic is defined on
\begin{multline*}
	\phi,\,\psi ::= p \mid \top \mid \bot \mid \neg\phi \mid \phi \wedge \psi \mid \phi \vee \psi \mid \neg \phi \mid \phi \to \psi  \mid \\
	\ctl{EX}\phi \mid \ctl{AX}\phi \mid \ctl{EF}\phi \mid \ctl{AF}\phi \mid \ctl{EG} \phi \mid \ctl{AG} \phi \mid \ctl{E}\psi \ctl{U} \phi \mid \ctl{A}\psi \ctl{U}\phi
\end{multline*}
$p\in AP$ is an atomic predicate. CTL is defined on states, the relation $\models$ between states $s$ in a Kripke structure and CTL formulae $\phi$ is inductively defined by
\begin{itemize-zero}
\item $s \models p$, $p\in AP$ iff $(s, p) \in \models$
\item $s \models \top$ always holds, and $s \models \bot$ never
\item $s \models \neg \phi$ iff $s \not\models \phi$
\item $s \models \phi \wedge \psi$ iff $s \models \phi$ and $s \models \psi$
\item $s \models \phi \vee \psi$ iff $s \models \phi$ or $s \models \psi$
\item $s \models \phi \to \psi$ iff $s \not\models \phi$ or $s \models \psi$. aka $\phi$ implies $\psi$
\item $s \models \ctl{EX}\phi$ iff there is a state $s'$ with $s\to s'$ and $s' \models \phi$
\item $s \models \ctl{AX}\phi$ iff for each state $s'$ with $s\to s'$ one has $s' \models \phi$
\item $s \models \ctl{EF}\phi$ iff some complete path starting in $s$ contains a $s'$ with $s\models \phi$
\item $s \models \ctl{AF}\phi$ iff each complete path starting in $s$ contains an $s'$ with $s;\models \phi$
\item $s \models \ctl{EG}\phi$ iff all states $s'$ on some complete path starting in $s$ satisfy $s' \models \phi$
\item $s \models \ctl{AG}\phi$ iff all states $s'$ on all complete path starting in $s$ satisfy $s' \models \phi$
\item $s \models \ctl{E}\psi \ctl{U}\phi$ iff some complete path $\pi$ starting in $s$ contains an $s'$ with $s'\models \phi$, and each state $s''$ on $\pi$ prior to $s'$ satisfies $s'' \models \phi$
\item $s \models \ctl{A}\psi \ctl{U}\phi$ iff each complete path $\pi$ starting in $s$ contains an $s'$ with $s'\models \phi$, and each state $s''$ on $\pi$ prior to $s'$ satisfies $s'' \models \phi$
\end{itemize-zero}
\end{dfn}

\begin{xmp}[]{xmp:ctl}{2}
	Examples of CTL formulae that hold in the initial state:
	\vspace{-10pt}
	\begin{multicols}{2}
		\begin{enumerate-zero}
			\item $\mathrm{AF}(p \wedge q)$
			\item $\mathrm{AF}\,\mathrm{AG}(p \vee q)$
			\item $\mathrm{AF}(\mathrm{EG} p \wedge \mathrm{EG} q)$
			\item $\mathrm{AG}(q \to \mathrm{A}(q \mathrm{U} p))$
			\item $\mathrm{A}(\mathrm{E}(\neg q \mathrm{U} p) \mathrm{U} q)$
			\item $\mathrm{AG}((p \wedge q) \to \mathrm{AX}\,\mathrm{E}(q \mathrm{U} p))$
		\end{enumerate-zero}


		\[\begin{tikzcd}[cramped, column sep=scriptsize, row sep=scriptsize]
	& {} \\
	& \circ \\
			p && q \\
			p & {q,p} & q \\
			  & {p,q}
			  \arrow[shorten <=5pt, from=1-2, to=2-2]
			  \arrow[from=2-2, to=3-1]
			  \arrow[from=2-2, to=3-3]
			  \arrow[from=3-1, to=4-1]
			  \arrow[from=3-3, to=4-3]
			  \arrow[from=4-1, to=5-2]
			  \arrow[from=4-2, to=3-1]
			  \arrow[from=4-2, to=3-3]
			  \arrow[from=4-3, to=5-2]
			  \arrow[from=5-2, to=4-2]
		\end{tikzcd}\]
		\vspace{5pt}

	\end{multicols}
\end{xmp}

\begin{lma}[Comparing LTL to CTL]{lma:ltl-to-ctl}{}
	LTL and CTL can be shown to be incomparable by proving that there cannot exist an LTL formula that is equivalent to the CTL formula $\mathrm{\mathbf{AG}} \mathrm{\mathbf{EF}} a$, and by showing that there cannot exist a CTL formula equivalent to the LTL formula $\mathrm{\mathbf{FG}} a$
\end{lma}


\begin{dfn}[LTL]{dfn:ltl}{}
	\vspace{-5pt}
Linear-Time Temporal Logic is defined on
\begin{multline*}
	\phi,\,\psi ::= p \mid \top \mid \bot \mid \neg\phi \mid \phi \wedge \psi \mid \phi \vee \psi \mid \neg \phi \mid \phi \to \psi  \mid \\
	\ctl{X}\phi \mid \ctl{F}\phi \mid \ctl{G} \phi \mid \psi \ctl{U} \phi \mid
\end{multline*}
$p\in AP$ is an atomic predicate. The modalities $X,F,G,U$ are called \textbf{next-state}, \textbf{eventually}, \textbf{globally}, and \textbf{until}. The relation $\models$ between paths and LTL formulae, with $\pi \models \phi$ saying that the path $\pi$ satisfies the formula $\phi$, or that $\phi$ is valid on $\pi$, or \textbf{holds} on $\pi$, is inductively defined by
\begin{itemize-tight}
\item $\pi \models p$, $p\in AP$ iff $(s, p) \in \models$
\item $\pi \models \top$ always holds, and $\pi \models \bot$ never
\item $\pi \models \neg \phi$ iff $s \not\models \phi$
\item $\pi \models \phi \wedge \psi$ iff $\pi \models \phi$ and $\pi \models \psi$
\item $\pi \models \phi \vee \psi$ iff $\pi \models \phi$ or $\pi \models \psi$
\item $\pi \models \phi \to \psi$ iff $s \not\models \phi$ or $\pi \models \psi$
\item $\pi \models \ctl{X}\phi$ iff $\pi' \models \phi$, where $\pi'$ is the suffix of $\pi$ obtained by omitting the first state
\item $\pi \models \ctl{F}\phi$ iff $\pi' \models \phi$ for some suffix $\pi'$
\item $\pi \models \ctl{G}\phi$ iff $\pi' \models \phi$ for each suffix $\pi'$
\item $\pi \models \psi \ctl{U}\phi$ iff $\pi' \models \phi$ for some suffix $\pi'$ of $\pi$, and $\pi'' \models \phi$ for each path $\pi'' \ne \pi'$ with $\pi \Rightarrow \pi'' \Rightarrow \pi'$
\end{itemize-tight}
Here a path is seen as a state, namely the first state on that path, together with a future that has been chosen already when evaluating an LTL formula on that state. When applying to finite paths, we have to make one adaptation, namely $\pi \models \mathrm{\mathbf{X}} \phi$ never holds if $\pi$ has only one state. So $\mathrm{\mathbf{X}} \phi$ says that there is a next state, and that $\phi$ holds in it.

\longrule{0.08ex}
$s \models \phi$ iff $\pi \models \phi$ for all complete paths $\pi$ starting in state $s$. Here a path is \textbf{complete} if it is either infinite or ends in deadlock.
\end{dfn}



\begin{dfn}[X variants]{dfn:x-variant}{}
	\vspace{-5pt}
	The variants $\mathrm{CTL}_{-X}$ and $\mathrm{LTL}_{-X}$ are the same thing but without the $X$ operators.
\end{dfn}

\begin{thm}[Satisfaction of Strong Bisimilarity]{thm:hml-sb}{}
	\vspace{-5pt}
	\begin{itemize-zero}
	    \item Two processes are strongly bisimlar iff they satisfy the same infinitary HML formulas. Therefore, to show that two processes are not strongly bisimilar, it suffices to find an infinitary HML formula that is satisfied by one, but not the other.
	    \item Two processes $P$ and $Q$ satisfy the same CTL formulas if they are strongly bisimilar. For finitely branching processes, we have ``iff''. For general processes, we have ``iff'' if we use CTL with infinite conjunctions.
	    \item Two divergence-free processes satisfy the same CTL$_{-X}$ formulas if they are branching bisimulation equivalent. We have ``iff'' if we use CTL$_{-X}$ with infinite conjunctions.
	\end{itemize-zero}
\end{thm}

\newpage

\section{Other stuff and Algorithms}
\setcounter{subsection}{1}

\begin{dfn}[Completeness Criteria]{dfn:completeness-criteria}{}
	\vspace{-5pt}
	A completeness criterion $D$ is \textbf{stronger} than a criterion $C$ if it rules out more paths as incomplete, i.e. if the set of complete paths according to $D$ is a subset of the set of complete paths according to $C$. If $D$ is stronger than $C$, then $P \models^{c} \phi$ implies $P \models^{D} \phi$ but not necessarily vice versa. For to check $P \models^{D} \phi$ we have to check that $\phi$ holds for all complete paths, and under $D$ there are fewer complete paths than under $C$, so there is less to check.
\end{dfn}

\begin{thm}[The Completeness Hierarchy]{thm:completeness-hierarchy}{}
	\vspace{-5pt}
	The strongest completeness criteron says that no path is complete. We have $P \models^{\infty} \phi$ for all processes $P$ and properties $\phi$.

	The weakest completeness criterion says that all paths are complete, which we call $\emptyset$.

	\vspace{-5pt}
	\longrule{0.08ex}
	We define a \textbf{task} as a set of transitions in a Kripke structure. A task $T$ is \textbf{enabled} in a state $s\in S$ if $s$ has an outgoing transition that belongs to the task, \textbf{perpetually enabled} on a path $\pi$ if it is enabled in every state of $\pi$, \textbf{occurs} in $\pi$ if $\pi$ contains a transition that belongs to $T$, and \textbf{relentlessly enabled} on $\pi$ if each suffix of $\pi$ contains a state in which it is enabled. This is the case if the task is enabled in infinitely many states of $\pi$, in a state that occurs infinitely often in $\pi$, or in the last state of a finite $\pi$.

	\vspace{-5pt}
	\longrule{0.08ex}
	A \textbf{finite prefix} of a path $\pi$ is the part of $\pi$ from its beginning until a given state. A \textbf{suffix} of a path $\pi$ is the path that remains after you remove a finite prefix of it. Note that if $\pi$ is infinite, then all its suffixes are infinite as well.

	\begin{enumerate-tight}
	    \item \textbf{Progress}: A path is complete if it is either infinite, or ends in a state in deadlock. Used for CTL/LTL formulae, and for CT semantics.
		\item \textbf{Justness}: A path is complete 
		\item \textbf{Weak Fairness}: A path is \textbf{weakly fair} if, for each suffix $\pi'$ of $\pi$, each task that is perpetually enabled on $\pi'$, occurs in $\pi'$. Equivalently, if each task that from some state onwards is perpetually enabled on $\pi'$, occurs infinitely often in $\pi$.

			Weak fairness can be expressed by the LTL formula
			\[\G(\G(\text{enabled}(T)) \implies \F(\text{occurs}(T)))\]
			for each task $T$.
		\item \textbf{Strong Fairness}: A path $\pi$ is \textbf{strongly fair} if for every suffix $\pi'$ of $\pi$, each task that is relentlessly enabled on $\pi'$ occurs on $\pi'$. Equivalently, a path $\pi$ is \textbf{strongly fair} if each task that is relentless enabled on $\pi$ occurs infinitely often in $\pi$.

			Strong fairness can be expressed in LTL as
			\[\G(\G\F(\text{enabled}(T)) \implies \F(\text{occurs}(T)))\]
		\item \textbf{Full Fairness}: Rooted BB..? Doesn't count as a completeness criterion though.
	\end{enumerate-tight}
\end{thm}


% \begin{thm}[De Nicola-Vaandrager]{thm:}{}
% 	Translates Krikpe structures :D
%
% \end{thm}

\begin{dfn}[Partition Refining]{dfn:partition-refining}{}
	Partition refining is an algorithm to turn a process into its minimal state, making it easier to compare bisimularity. Works with
	\[\mathtt{split}(B, a, P)\]
	where $B$ is an equivalence class, $a$ is an action, and $P$ is the process. $\mathtt{split}$ splits an equivalence class into two, ones with the action and ones without it.
	\[[\{s \prightarrow{a} \bullet\}]_{P}\]
	means ``state $s$ does action $a$ outside the equivalence class in process $P$''

	\begin{lstlisting}[escapeinside={(*}{*)}, caption={Pseudocode for \texttt{split}}]
	split(B, a, P) (*$\to$*) ((*$\{B_i\}$*) a set of blocks)
	   choose (*$s\in B$*)
	   (*$B_1 = \emptyset$*) (*$\color{red}(B_1\text{\textrm{ contains states equivalent to }} s*\,)$*)
	   (*$B_2 = \emptyset$*) (*$\color{red}(B_2 \text{\textrm{ contains states inequivalent to }} s*\,)$*)
	   for each (*$s'\in B$*) do
	   begin
	      if (*$[\{s \prightarrow{a} \bullet\}]_P = [\{s' \prightarrow{a} \bullet\}]_P$*) then
		(*$B_1 = B_1 \cup \{s'\}$*)
	      else
		(*$B_2 = B_2 \cup \{s'\}$*)
	   end
	   if (*$B_2 = \emptyset$*) then
	      return (*$\{B_1\}$*)
	   else
	      return (*$\{B_1,\, B_2\}$*)

	\end{lstlisting}
	Methodology:
	\begin{itemize-tight}
	    \item Start with one equivalence class for all states
	    \item Run \texttt{split} on the outermost states, this now splits into $R_{1}$ and $R_{2}$, where $R_{2}$ are outside states
	    \item Run \texttt{split} on states with outgoing actions to states in $R_{2}$, this now splits $R_{1}$ into $R_{1}$ and $R_{3}$, where $R_{3}$ are second-most outer states
	    \item Keep on running until root state is partitioned
		\item If needed, the minimal graph will have exactly one of each equivalence class
	\end{itemize-tight}
	
	\longrule{0.08ex}
	Running partition refinement for Branching Bisimularity
	\begin{itemize}
	    \item When checking whether a state in block $B$ has an $\alpha$ transition to block $C$, it is okay if we can move through block $B$ by doing only $\tau$-transitions, and then reach a state with an $\alpha$ transition to block $C$
	    \item We never check $\tau$-transitions from block $B$ to block $B$ ($\tau$-transitions to another block are fair game though)
	\end{itemize}
	This implies running the rule on
	\[[\{s \Rightarrow\prightarrow{a} \bullet\}]_{P}\]

\end{dfn}

\begin{dfn}[Alternating Bit Protocol]{dfn:alternating-bit-protocol}{}
% https://q.uiver.app/#q=WzAsMTIsWzAsMiwiMSJdLFswLDQsIjIiXSxbMCw2LCIzIl0sWzIsNCwiNCJdLFs0LDQsIjUiXSxbNiw0LCI3Il0sWzQsNiwiNiJdLFs2LDIsIjgiXSxbNiwwLCI5Il0sWzQsMiwiMTAiXSxbMiwyLCIxMSJdLFsyLDAsIjEyIl0sWzAsMSwicl9BKGQpIiwyXSxbMSwyLCJjX0J6KFxcYm90KSIsMix7ImN1cnZlIjoyfV0sWzIsMSwiY19EKFxcYm90KSBcXFxcY19EKDEpIiwyLHsiY3VydmUiOjJ9XSxbMSwzLCJjX0IoZCwgMCkiXSxbMyw0LCJzX0MoZCkiXSxbNCw1LCJjX0QoMCkiXSxbNCw2LCJjX0QoXFxib3QpIiwyLHsiY3VydmUiOjJ9XSxbNiw0LCJjX0QoXFxib3QpIFxcXFxjX0QoZCwwKSIsMix7ImN1cnZlIjoyfV0sWzUsNywicl9BKGQpIiwyXSxbOCw3LCJjX0QoXFxib3QpIFxcXFxjX0QoMCkiLDAseyJjdXJ2ZSI6LTJ9XSxbNyw4LCJjX0IoXFxib3QpIiwwLHsiY3VydmUiOi0yfV0sWzcsOSwiY19CKGQsMSkiXSxbOSwxMCwic19DKGQpIl0sWzExLDEwLCJjX0QoXFxib3QpIFxcXFxjX0IoZCwxKSIsMCx7ImN1cnZlIjotMn1dLFsxMCwxMSwiY19EKFxcYm90KSIsMCx7ImN1cnZlIjotMn1dLFsxMCwwLCJjX0QoMSkiXV0=
\[\begin{tikzcd}[cramped]
	&& 12 &&&& 9 \\
	\\
	1 && 11 && 10 && 8 \\
	\\
	2 && 4 && 5 && 7 \\
	\\
	3 &&&& 6
	\arrow["\substack{{c} c_D(\bot) \\c_B(d,1) }", curve={height=-12pt}, from=1-3, to=3-3]
	\arrow["\substack{{c} c_D(\bot) \\c_D(0) }", curve={height=-12pt}, from=1-7, to=3-7]
	\arrow["{r_A(d)}"', from=3-1, to=5-1]
	\arrow["{c_D(\bot)}", curve={height=-12pt}, from=3-3, to=1-3]
	\arrow["{c_D(1)}", from=3-3, to=3-1]
	\arrow["{s_C(d)}", from=3-5, to=3-3]
	\arrow["{c_B(\bot)}", curve={height=-12pt}, from=3-7, to=1-7]
	\arrow["{c_B(d,1)}", from=3-7, to=3-5]
	\arrow["{c_B(d, 0)}", from=5-1, to=5-3]
	\arrow["{c_Bz(\bot)}"', curve={height=12pt}, from=5-1, to=7-1]
	\arrow["{s_C(d)}", from=5-3, to=5-5]
	\arrow["{c_D(0)}", from=5-5, to=5-7]
	\arrow["{c_D(\bot)}"', curve={height=12pt}, from=5-5, to=7-5]
	\arrow["{r_A(d)}"', from=5-7, to=3-7]
	\arrow["\substack{{c} c_D(\bot) \\c_D(1) }"', curve={height=12pt}, from=7-1, to=5-1]
	\arrow["\substack{{c} c_D(\bot) \\c_D(d,0) }"', curve={height=12pt}, from=7-5, to=5-5]
\end{tikzcd}\]
\textbf{Specification for the Sender}
\begin{align*}
	S_{b} &= \sum_{d\in \Delta} r_{A}(d) \cdot T_{db} \\
	T_{db} &= (s_{B}(d, b) + s_{B}(\bot)) \cdot U_{db}\\
	U_{db} &= r_{D}(b) \cdot S_{1 - b} + (r_{D}(1-b) + r_{D}(\bot)) \cdot T_{db}
\end{align*}
\footnotesize
In state $S_{b}$, the Sender reads a datum $d$ from channel $A$. Then it proceeds to state $T_{db}$, in which it sends datum $d$ into channel $B$, with the bit $b$ attached to it. However, the pair $(d, b)$ may be distorted by the channel, so that it becomes the error message $\bot$. Next, the system proceeds to state $U_{db}$, in which it expects to receive the acknowledgement $b$ through channel $D$, ensuring that the pair $(d, b)$ has reached the Receiver unscathed. If the correct acknowledgement $b$ is received, then the system proceeds to state $S_{1-b}$, in which it is going to send out a datum with the bit $1 - b$ attached to it. If the acknowledgement is either the wrong bit, $1 - b$ or the error message $\bot$, then the system proceeds to state $T_{db}$, to send the pair $(d, b)$ into channel $B$ once more.

\normalsize
\textbf{Specification for the Reciever}
\begin{align*}
	\begin{split}
		R_{b} &= \sum_{d'\in \Delta} \{r_{B}(d',b) \cdot s_{C}(d')\cdot Q_{b} + r_{B}(d', 1-b) \cdot Q_{1-b}\}\\
			  &\qquad+ r_{B}(\bot) \cdot Q_{1-b}
	\end{split} \\
	Q_{b} &= (s_{D}(b) + s_{D}(\bot) \cdot R_{1-b})
\end{align*}

\footnotesize
\begin{enumerate-zero}
    \item If in Rb the Receiver reads a pair $(d', b)$ from channel $B$, then this constitutes new information, so the datum $d'$ is sent into channel $C$. Then the Receiver proceeds to state $Q_{b}$, in which it sends acknowledgement $b$ to the Sender via channel $D$. However, this acknowledgement may be distorted by the channel, so that it becomes the error message $\bot$. Next, the Receiver proceeds to state $R_{1-b}$, in which it is expecting to receive a datum with the bit $1 - b$ attached to it.
	\item If in $R_{b}$ the Receiver reads a pair $(d', 1 - b)$ or an error message $\bot$ from channel $B$, then this does not constitute new information. So then the Receiver proceeds to state $Q_{1-b}$ straight away, to send acknowledgement $1 - b$ to the Sender via channel $D$. However, this acknowledgement may be distorted by the channel, so that it becomes the error message $\bot$. Next, the Receiver proceeds to state $R_{b}$ again.
\end{enumerate-zero}


\end{dfn}


\newpage

\section{Example Catalogue}

\begin{xmp}[LTL]{xmp:ltl}{1}
	\vspace{-5pt}
    Examples of LTL that hold in the initial state:
    
	\vspace{-15pt}
    \begin{multicols}{2}
    \begin{enumerate-zero}
        \item $\G(p \vee q)$ \textcolor{red}{false!}
        \item $\F\G(p \vee q)$
        \item $\F(\G p \vee \G q)$ \textcolor{red}{false!}
        \item $\G(q \to q \U p)$
        \item $\G((p \wedge \neg q) \to \X q)$ \textcolor{red}{false!}
        \item $\G((p \wedge q) \to \X (q \U p))$
    \end{enumerate-zero}
    
    \columnbreak

\[\begin{tikzcd}[cramped, column sep=scriptsize, row sep=scriptsize]
	& {} \\
	& \circ \\
	p_{1} && q_{1} \\
	p_{2} & {q,p} & q_{2} \\
	& {p,q}
	\arrow[shorten <=5pt, from=1-2, to=2-2]
	\arrow[from=2-2, to=3-1]
	\arrow[from=2-2, to=3-3]
	\arrow[from=3-1, to=4-1]
	\arrow[from=3-3, to=4-3]
	\arrow[from=4-1, to=5-2]
	\arrow[from=4-2, to=3-1]
	\arrow[from=4-2, to=3-3]
	\arrow[from=4-3, to=5-2]
	\arrow[from=5-2, to=4-2]
\end{tikzcd}\]
\vspace{5pt}
    \end{multicols}
\end{xmp}

\begin{xmp}[Trace Semantics]{xmp:trace-semantics}{}
	A process that is PT equivalent but not CT equivalent
	% https://q.uiver.app/#q=WzAsOSxbMSwxLCJcXGJ1bGxldCJdLFswLDIsIlxcYnVsbGV0Il0sWzAsMywiXFxidWxsZXQiXSxbMiwyLCJcXGJ1bGxldCJdLFs0LDEsIlxcYnVsbGV0Il0sWzQsMiwiXFxidWxsZXQiXSxbNCwzLCJcXGJ1bGxldCJdLFsxLDBdLFs0LDBdLFswLDEsImEiLDJdLFsxLDIsImEiXSxbMCwzLCJhIl0sWzQsNSwiYSIsMl0sWzUsNiwiYSIsMl0sWzcsMCwiIiwyLHsic2hvcnRlbiI6eyJzb3VyY2UiOjUwfX1dLFs4LDQsIiIsMix7InNob3J0ZW4iOnsic291cmNlIjo1MH19XV0=
	\[\begin{tikzcd}[cramped, column sep=scriptsize]
			& {} &&& {} \\
			& \bullet &&& \bullet \\
			\bullet && \bullet && \bullet \\
			\bullet &&&& \bullet
			\arrow[shorten <=5pt, from=1-2, to=2-2]
			\arrow[shorten <=5pt, from=1-5, to=2-5]
			\arrow["a"', from=2-2, to=3-1]
			\arrow["a", from=2-2, to=3-3]
			\arrow["a"', from=2-5, to=3-5]
			\arrow["a", from=3-1, to=4-1]
			\arrow["a"', from=3-5, to=4-5]
		\end{tikzcd}\]

	A process that is CT equivalent but not Bisimulation equivalent

	% https://q.uiver.app/#q=WzAsMTEsWzUsMSwiXFxjaXJjIl0sWzQsMiwiXFxidWxsZXQiXSxbNiwyLCJcXGJ1bGxldCJdLFs2LDMsIlxcYnVsbGV0Il0sWzQsMywiXFxidWxsZXQiXSxbMSwxLCJcXGNpcmMiXSxbMSwyLCJcXGJ1bGxldCJdLFsyLDMsIlxcYnVsbGV0Il0sWzAsMywiXFxidWxsZXQiXSxbNSwwXSxbMSwwXSxbMCwxLCJhIl0sWzAsMiwiYSIsMl0sWzIsMywiYyIsMl0sWzEsNCwiYiJdLFs1LDYsImEiXSxbNiw3LCJjIl0sWzYsOCwiYiJdLFs5LDAsIiIsMCx7InNob3J0ZW4iOnsic291cmNlIjo1MH19XSxbMTAsNSwiIiwwLHsic2hvcnRlbiI6eyJzb3VyY2UiOjUwfX1dXQ==
	\[\begin{tikzcd}[cramped]
			& {} &&&& {} \\
			& \circ &&&& \circ \\
			& \bullet &&& \bullet && \bullet \\
			\bullet && \bullet && \bullet && \bullet
			\arrow[shorten <=5pt, from=1-2, to=2-2]
			\arrow[shorten <=5pt, from=1-6, to=2-6]
			\arrow["a", from=2-2, to=3-2]
			\arrow["a", from=2-6, to=3-5]
			\arrow["a"', from=2-6, to=3-7]
			\arrow["b", from=3-2, to=4-1]
			\arrow["c", from=3-2, to=4-3]
			\arrow["b", from=3-5, to=4-5]
			\arrow["c"', from=3-7, to=4-7]
		\end{tikzcd}\]

\end{xmp}



\begin{xmp}[Bisimulation Semantics]{xmp:bisim-semantics}{}

	Two processes that are Branching Bisimulation equivalent
	% https://q.uiver.app/#q=WzAsNyxbMSwyLCJcXGJ1bGxldCJdLFszLDIsIlxcYnVsbGV0Il0sWzEsMCwiXFxidWxsZXQiXSxbMywwLCJcXGJ1bGxldCJdLFs1LDIsIlxcYnVsbGV0Il0sWzAsMl0sWzAsMF0sWzAsMSwiYSIsMl0sWzIsMywiYSJdLFs1LDAsIiIsMix7InNob3J0ZW4iOnsic291cmNlIjo1MH19XSxbNiwyLCIiLDIseyJzaG9ydGVuIjp7InNvdXJjZSI6NTB9fV0sWzAsMiwiIiwwLHsiY29sb3VyIjpbMjQwLDYwLDYwXSwic3R5bGUiOnsiYm9keSI6eyJuYW1lIjoiZGFzaGVkIn0sImhlYWQiOnsibmFtZSI6Im5vbmUifX19XSxbMSw0LCJcXHRhdSJdLFszLDEsIiIsMCx7ImNvbG91ciI6WzI0MCw2MCw2MF0sInN0eWxlIjp7ImJvZHkiOnsibmFtZSI6ImRhc2hlZCJ9LCJoZWFkIjp7Im5hbWUiOiJub25lIn19fV0sWzMsNCwiIiwwLHsiY29sb3VyIjpbMjQwLDYwLDYwXSwic3R5bGUiOnsiYm9keSI6eyJuYW1lIjoiZGFzaGVkIn0sImhlYWQiOnsibmFtZSI6Im5vbmUifX19XV0=
	\[\begin{tikzcd}[cramped, column sep=scriptsize]
			{} & \bullet && \bullet \\
			\\
			{} & \bullet && \bullet && \bullet
			\arrow[shorten <=8pt, from=1-1, to=1-2]
			\arrow["a", from=1-2, to=1-4]
			\arrow[color={rgb,255:red,92;green,92;blue,214}, dashed, no head, from=1-4, to=3-4]
			\arrow[color={rgb,255:red,92;green,92;blue,214}, dashed, no head, from=1-4, to=3-6]
			\arrow[shorten <=8pt, from=3-1, to=3-2]
			\arrow[color={rgb,255:red,92;green,92;blue,214}, dashed, no head, from=3-2, to=1-2]
			\arrow["a"', from=3-2, to=3-4]
			\arrow["\tau", from=3-4, to=3-6]
		\end{tikzcd}\]

		Two processes that are Branching Bisimulation equivalent but not Rooted BB equivalent
% https://q.uiver.app/#q=WzAsNyxbMSwyLCJcXGJ1bGxldCJdLFszLDIsIlxcYnVsbGV0Il0sWzEsMCwiXFxidWxsZXQiXSxbMywwLCJcXGJ1bGxldCJdLFs1LDIsIlxcYnVsbGV0Il0sWzAsMl0sWzAsMF0sWzAsMSwiXFx0YXUiLDJdLFsyLDMsImEiXSxbNSwwLCIiLDIseyJzaG9ydGVuIjp7InNvdXJjZSI6NTB9fV0sWzYsMiwiIiwyLHsic2hvcnRlbiI6eyJzb3VyY2UiOjUwfX1dLFswLDIsIiIsMCx7ImNvbG91ciI6WzI0MCw2MCw2MF0sInN0eWxlIjp7ImJvZHkiOnsibmFtZSI6ImRhc2hlZCJ9LCJoZWFkIjp7Im5hbWUiOiJub25lIn19fV0sWzEsNCwiYSJdLFszLDQsIiIsMCx7ImNvbG91ciI6WzI0MCw2MCw2MF0sInN0eWxlIjp7ImJvZHkiOnsibmFtZSI6ImRhc2hlZCJ9LCJoZWFkIjp7Im5hbWUiOiJub25lIn19fV0sWzIsMSwiIiwwLHsiY29sb3VyIjpbMjQwLDYwLDYwXSwic3R5bGUiOnsiYm9keSI6eyJuYW1lIjoiZGFzaGVkIn0sImhlYWQiOnsibmFtZSI6Im5vbmUifX19XV0=
\[\begin{tikzcd}[cramped, column sep=scriptsize]
	{} & \bullet && \bullet \\
	\\
	{} & \bullet && \bullet && \bullet
	\arrow[shorten <=8pt, from=1-1, to=1-2]
	\arrow["a", from=1-2, to=1-4]
	\arrow[color={rgb,255:red,92;green,92;blue,214}, dashed, no head, from=1-2, to=3-4]
	\arrow[color={rgb,255:red,92;green,92;blue,214}, dashed, no head, from=1-4, to=3-6]
	\arrow[shorten <=8pt, from=3-1, to=3-2]
	\arrow[color={rgb,255:red,92;green,92;blue,214}, dashed, no head, from=3-2, to=1-2]
	\arrow["\tau"', from=3-2, to=3-4]
	\arrow["a", from=3-4, to=3-6]
\end{tikzcd}\]

	
	Two processes that are Weak Bisimulation equivalent but not Branching Bisimulation equivalent?
	% https://q.uiver.app/#q=WzAsOSxbMSwwLCJcXGJ1bGxldCJdLFsxLDEsIlxcYnVsbGV0Il0sWzAsMiwiXFxidWxsZXQiXSxbMiwyLCJcXGJ1bGxldCJdLFs1LDAsIlxcYnVsbGV0Il0sWzQsMSwiXFxidWxsZXQiXSxbNiwxLCJcXGJ1bGxldCJdLFs0LDIsIlxcYnVsbGV0Il0sWzYsMiwiXFxidWxsZXQiXSxbMCwxLCJcXHRhdSJdLFsxLDIsImEiLDJdLFsxLDMsImIiXSxbNCw1LCJcXHRhdSIsMl0sWzQsNiwiXFx0YXUiXSxbNSw3LCJhIiwyXSxbNiw4LCJiIl0sWzAsNCwiIiwyLHsiY29sb3VyIjpbMjQwLDYwLDYwXSwic3R5bGUiOnsiYm9keSI6eyJuYW1lIjoiZGFzaGVkIn0sImhlYWQiOnsibmFtZSI6Im5vbmUifX19XSxbMiw3LCIiLDEseyJjdXJ2ZSI6MiwiY29sb3VyIjpbMjQwLDYwLDYwXSwic3R5bGUiOnsiYm9keSI6eyJuYW1lIjoiZGFzaGVkIn0sImhlYWQiOnsibmFtZSI6Im5vbmUifX19XSxbMyw4LCIiLDEseyJjdXJ2ZSI6MiwiY29sb3VyIjpbMjQwLDYwLDYwXSwic3R5bGUiOnsiYm9keSI6eyJuYW1lIjoiZGFzaGVkIn0sImhlYWQiOnsibmFtZSI6Im5vbmUifX19XSxbMSw1LCIiLDEseyJjb2xvdXIiOlsyNDAsNjAsNjBdLCJzdHlsZSI6eyJib2R5Ijp7Im5hbWUiOiJkYXNoZWQifSwiaGVhZCI6eyJuYW1lIjoibm9uZSJ9fX1dLFs2LDEsIiIsMSx7ImN1cnZlIjotMiwiY29sb3VyIjpbMjQwLDYwLDYwXSwic3R5bGUiOnsiYm9keSI6eyJuYW1lIjoiZGFzaGVkIn0sImhlYWQiOnsibmFtZSI6Im5vbmUifX19XV0=
	\[\begin{tikzcd}[cramped, column sep=scriptsize]
			& \bullet &&&& \bullet \\
			& \bullet &&& \bullet && \bullet \\
			\bullet && \bullet && \bullet && \bullet
			\arrow[color={rgb,255:red,92;green,92;blue,214}, dashed, no head, from=1-2, to=1-6]
			\arrow["\tau", from=1-2, to=2-2]
			\arrow["\tau"', from=1-6, to=2-5]
			\arrow["\tau", from=1-6, to=2-7]
			\arrow[color={rgb,255:red,92;green,92;blue,214}, dashed, no head, from=2-2, to=2-5]
			\arrow["a"', from=2-2, to=3-1]
			\arrow["b", from=2-2, to=3-3]
			\arrow["a"', from=2-5, to=3-5]
			\arrow[color={rgb,255:red,92;green,92;blue,214}, curve={height=-12pt}, dashed, no head, from=2-7, to=2-2]
			\arrow["b", from=2-7, to=3-7]
			\arrow[color={rgb,255:red,92;green,92;blue,214}, curve={height=12pt}, dashed, no head, from=3-1, to=3-5]
			\arrow[color={rgb,255:red,92;green,92;blue,214}, curve={height=12pt}, dashed, no head, from=3-3, to=3-7]
		\end{tikzcd}\]

		Two processes that are Simulation equivalent but not Bisimulation equivalent
% https://q.uiver.app/#q=WzAsMjQsWzUsMiwiXFxidWxsZXQiXSxbNCwzLCJcXGJ1bGxldCJdLFs1LDEsIlxcYnVsbGV0Il0sWzYsMywiXFxidWxsZXQiXSxbMiwyLCJcXGJ1bGxldCJdLFszLDMsIlxcYnVsbGV0Il0sWzEsMywiXFxidWxsZXQiXSxbMSwxLCJcXGJ1bGxldCJdLFswLDIsIlxcYnVsbGV0Il0sWzAsMywiXFxidWxsZXQiXSxbMSwwXSxbNSwwXSxbMSw0XSxbMSw1LCJcXGJ1bGxldCJdLFsyLDYsIlxcYnVsbGV0Il0sWzMsNywiXFxidWxsZXQiXSxbMSw3LCJcXGJ1bGxldCJdLFswLDYsIlxcYnVsbGV0Il0sWzAsNywiXFxidWxsZXQiXSxbNSw0XSxbNSw1LCJcXGJ1bGxldCJdLFs1LDYsIlxcYnVsbGV0Il0sWzQsNywiXFxidWxsZXQiXSxbNiw3LCJcXGJ1bGxldCJdLFswLDEsImIiLDJdLFsyLDAsImEiXSxbMCwzLCJjIl0sWzQsNSwiYyJdLFs0LDYsImIiLDJdLFs3LDQsImEiXSxbNyw4LCJhIiwyXSxbOCw5LCJiIiwyXSxbMTAsNywiIiwwLHsic2hvcnRlbiI6eyJzb3VyY2UiOjUwfX1dLFsxMSwyLCIiLDAseyJzaG9ydGVuIjp7InNvdXJjZSI6NTB9fV0sWzcsMiwiIiwwLHsiY29sb3VyIjpbMjQwLDYwLDYwXSwic3R5bGUiOnsiYm9keSI6eyJuYW1lIjoiZGFzaGVkIn19fV0sWzQsMCwiIiwwLHsiY3VydmUiOjIsImNvbG91ciI6WzI0MCw2MCw2MF0sInN0eWxlIjp7ImJvZHkiOnsibmFtZSI6ImRhc2hlZCJ9fX1dLFs4LDAsIiIsMCx7ImN1cnZlIjotMiwiY29sb3VyIjpbMjQwLDYwLDYwXSwic3R5bGUiOnsiYm9keSI6eyJuYW1lIjoiZGFzaGVkIn19fV0sWzksMSwiIiwyLHsiY3VydmUiOjMsImNvbG91ciI6WzI0MCw2MCw2MF0sInN0eWxlIjp7ImJvZHkiOnsibmFtZSI6ImRhc2hlZCJ9fX1dLFs1LDMsIiIsMSx7ImN1cnZlIjozLCJjb2xvdXIiOlsyNDAsNjAsNjBdLCJzdHlsZSI6eyJib2R5Ijp7Im5hbWUiOiJkYXNoZWQifX19XSxbNiwxLCIiLDEseyJjdXJ2ZSI6LTIsImNvbG91ciI6WzI0MCw2MCw2MF0sInN0eWxlIjp7ImJvZHkiOnsibmFtZSI6ImRhc2hlZCJ9fX1dLFsxMiwxMywiIiwwLHsic2hvcnRlbiI6eyJzb3VyY2UiOjUwfX1dLFsxMywxNCwiYSJdLFsxNCwxNSwiYyJdLFsxNCwxNiwiYiIsMl0sWzEzLDE3LCJhIiwyXSxbMTcsMTgsImIiLDJdLFsxOSwyMCwiIiwwLHsic2hvcnRlbiI6eyJzb3VyY2UiOjUwfX1dLFsyMCwyMSwiYSJdLFsyMSwyMiwiYiIsMl0sWzIxLDIzLCJjIl0sWzIwLDEzLCIiLDAseyJjb2xvdXIiOlsyNDAsNjAsNjBdLCJzdHlsZSI6eyJib2R5Ijp7Im5hbWUiOiJkYXNoZWQifX19XSxbMjEsMTQsIiIsMSx7ImNvbG91ciI6WzI0MCw2MCw2MF0sInN0eWxlIjp7ImJvZHkiOnsibmFtZSI6ImRhc2hlZCJ9fX1dLFsyMiwxNiwiIiwxLHsiY3VydmUiOi0yLCJjb2xvdXIiOlsyNDAsNjAsNjBdLCJzdHlsZSI6eyJib2R5Ijp7Im5hbWUiOiJkYXNoZWQifX19XSxbMjMsMTUsIiIsMSx7ImN1cnZlIjotMiwiY29sb3VyIjpbMjQwLDYwLDYwXSwic3R5bGUiOnsiYm9keSI6eyJuYW1lIjoiZGFzaGVkIn19fV1d
\[\begin{tikzcd}[cramped, column sep=scriptsize]
	& {} &&&& {} \\
	& \bullet &&&& \bullet \\
	\bullet && \bullet &&& \bullet \\
	\bullet & \bullet && \bullet & \bullet && \bullet \\
	& {} &&&& {} \\
	& \bullet &&&& \bullet \\
	\bullet && \bullet &&& \bullet \\
	\bullet & \bullet && \bullet & \bullet && \bullet
	\arrow[shorten <=5pt, from=1-2, to=2-2]
	\arrow[shorten <=5pt, from=1-6, to=2-6]
	\arrow[color={rgb,255:red,92;green,92;blue,214}, dashed, from=2-2, to=2-6]
	\arrow["a"', from=2-2, to=3-1]
	\arrow["a", from=2-2, to=3-3]
	\arrow["a", from=2-6, to=3-6]
	\arrow[color={rgb,255:red,92;green,92;blue,214}, curve={height=-12pt}, dashed, from=3-1, to=3-6]
	\arrow["b"', from=3-1, to=4-1]
	\arrow[color={rgb,255:red,92;green,92;blue,214}, curve={height=12pt}, dashed, from=3-3, to=3-6]
	\arrow["b"', from=3-3, to=4-2]
	\arrow["c", from=3-3, to=4-4]
	\arrow["b"', from=3-6, to=4-5]
	\arrow["c", from=3-6, to=4-7]
	\arrow[color={rgb,255:red,92;green,92;blue,214}, curve={height=18pt}, dashed, from=4-1, to=4-5]
	\arrow[color={rgb,255:red,92;green,92;blue,214}, curve={height=-12pt}, dashed, from=4-2, to=4-5]
	\arrow[color={rgb,255:red,92;green,92;blue,214}, curve={height=18pt}, dashed, from=4-4, to=4-7]
	\arrow[shorten <=5pt, from=5-2, to=6-2]
	\arrow[shorten <=5pt, from=5-6, to=6-6]
	\arrow["a"', from=6-2, to=7-1]
	\arrow["a", from=6-2, to=7-3]
	\arrow[color={rgb,255:red,92;green,92;blue,214}, dashed, from=6-6, to=6-2]
	\arrow["a", from=6-6, to=7-6]
	\arrow["b"', from=7-1, to=8-1]
	\arrow["b"', from=7-3, to=8-2]
	\arrow["c", from=7-3, to=8-4]
	\arrow[color={rgb,255:red,92;green,92;blue,214}, dashed, from=7-6, to=7-3]
	\arrow["b"', from=7-6, to=8-5]
	\arrow["c", from=7-6, to=8-7]
	\arrow[color={rgb,255:red,92;green,92;blue,214}, curve={height=-12pt}, dashed, from=8-5, to=8-2]
	\arrow[color={rgb,255:red,92;green,92;blue,214}, curve={height=-12pt}, dashed, from=8-7, to=8-4]
\end{tikzcd}\]
\end{xmp}

\begin{xmp}[ACP Represented in Petri Nets]{xmp:petri-nets}{}
	\textbf{Sequential Composition}
% https://q.uiver.app/#q=WzAsMjUsWzAsMCwiKFxcYnVsbGV0KSJdLFswLDEsIlthXSJdLFsxLDAsIihcXGJ1bGxldCkiXSxbMSwxLCJbYl0iXSxbMiwxLCJbY10iXSxbMiwwLCIoXFxxdWFkKSJdLFs0LDAsIihcXGJ1bGxldCkiXSxbNCwxLCJbZF0iXSxbNSwwLCIoXFxidWxsZXQpIl0sWzUsMSwiW2VdIl0sWzMsMCwiXFxjZG90Il0sWzEsMiwiKFxcYnVsbGV0KSJdLFsxLDMsIlthXSJdLFsyLDIsIihcXGJ1bGxldCkiXSxbMiwzLCJbYl0iXSxbMywzLCJbY10iXSxbMywyLCIoXFw6XFw6KSJdLFsxLDQsIihcXDpcXDopIl0sWzIsNCwiKFxcOlxcOikiXSxbMyw0LCIoXFw6XFw6KSJdLFsyLDYsIltkXSJdLFszLDYsIltlXSJdLFswLDQsIihcXDpcXDopIl0sWzQsMywiKFxcYnVsbGV0KSJdLFs0LDQsIihcXGJ1bGxldCkiXSxbMCwxXSxbMiwzXSxbMiw0XSxbNCw1XSxbNiw3XSxbOCw5XSxbMTEsMTJdLFsxMywxNF0sWzEzLDE1XSxbMTUsMTZdLFsxMiwxN10sWzE0LDE4XSxbMTUsMTldLFsxNSwxOF0sWzE0LDE5XSxbMTgsMjBdLFsxOSwyMV0sWzEyLDIyXSxbMjIsMjAsIiIsMSx7ImN1cnZlIjoxfV0sWzE3LDIxLCIiLDEseyJjdXJ2ZSI6MX1dLFsyMywxNV0sWzI0LDE1XSxbMjQsMjAsIiIsMSx7ImN1cnZlIjotMX1dLFsyMywyMSwiIiwxLHsiY3VydmUiOi01fV1d
\[\begin{tikzcd}[cramped, column sep=tiny, row sep=small]
	{(\bullet)} & {(\bullet)} & {(\quad)} & \cdot & {(\bullet)} & {(\bullet)} \\
	{[a]} & {[b]} & {[c]} && {[d]} & {[e]} \\
	& {(\bullet)} & {(\bullet)} & {(\:\:)} \\
	& {[a]} & {[b]} & {[c]} & {(\bullet)} \\
	{(\:\:)} & {(\:\:)} & {(\:\:)} & {(\:\:)} & {(\bullet)} \\
	\\
	&& {[d]} & {[e]}
	\arrow[from=1-1, to=2-1]
	\arrow[from=1-2, to=2-2]
	\arrow[from=1-2, to=2-3]
	\arrow[from=1-5, to=2-5]
	\arrow[from=1-6, to=2-6]
	\arrow[from=2-3, to=1-3]
	\arrow[from=3-2, to=4-2]
	\arrow[from=3-3, to=4-3]
	\arrow[from=3-3, to=4-4]
	\arrow[from=4-2, to=5-1]
	\arrow[from=4-2, to=5-2]
	\arrow[from=4-3, to=5-3]
	\arrow[from=4-3, to=5-4]
	\arrow[from=4-4, to=3-4]
	\arrow[from=4-4, to=5-3]
	\arrow[from=4-4, to=5-4]
	\arrow[from=4-5, to=4-4]
	\arrow[curve={height=-30pt}, from=4-5, to=7-4]
	\arrow[curve={height=6pt}, from=5-1, to=7-3]
	\arrow[curve={height=6pt}, from=5-2, to=7-4]
	\arrow[from=5-3, to=7-3]
	\arrow[from=5-4, to=7-4]
	\arrow[from=5-5, to=4-4]
	\arrow[curve={height=-6pt}, from=5-5, to=7-3]
\end{tikzcd}\]

\textbf{Alternative Composition}
% https://q.uiver.app/#q=WzAsMzAsWzMsMCwiKFxcYnVsbGV0KSJdLFsyLDAsIlthXSJdLFs1LDAsIihcXGJ1bGxldCkiXSxbNSwxLCJbYl0iXSxbNiwxLCJbYl0iXSxbNiwwLCIoXFxidWxsZXQpIl0sWzgsNCwiKFxcYnVsbGV0KSJdLFs3LDQsIlthXSJdLFs5LDQsIihcXGJ1bGxldCkiXSxbOSw1LCJbY10iXSxbOCw1LCJbYl0iXSxbNyw1LCIoXFw6XFw6KSJdLFs5LDYsIihcXDpcXDopIl0sWzYsMiwiKFxcOlxcOikiXSxbNiwzLCJbYV0iXSxbNCwwLCIrIl0sWzYsNSwiW2FdIl0sWzksNywiW2FdIl0sWzUsNCwiPSJdLFsxLDQsIihcXGJ1bGxldCkiXSxbMSw1LCJbYV0iXSxbMSw2LCIoXFw6XFw6KSJdLFswLDYsIlthXSJdLFszLDQsIihcXGJ1bGxldCkiXSxbMyw1LCJbYl0iXSxbNCw0LCIoXFxidWxsZXQpIl0sWzQsNSwiW2JdIl0sWzQsNiwiKFxcOlxcOikiXSxbNCw3LCJbYV0iXSxbMiw0LCIrIl0sWzAsMSwiIiwwLHsiY3VydmUiOi0yfV0sWzIsM10sWzYsNywiIiwyLHsiY3VydmUiOjJ9XSxbOCw5XSxbNywxMV0sWzksMTJdLFs1LDRdLFs0LDEzXSxbMTMsMTRdLFsxMSwxNiwiIiwwLHsiY3VydmUiOi0yfV0sWzEsMCwiIiwwLHsiY3VydmUiOi0yfV0sWzE5LDIwXSxbMjEsMjIsIiIsMCx7ImN1cnZlIjotMn1dLFsyMiwyMSwiIiwwLHsiY3VydmUiOi0yfV0sWzIzLDI0XSxbMjYsMjddLFsyNywyOF0sWzYsMTBdLFsxMiwxN10sWzE2LDExLCIiLDAseyJjdXJ2ZSI6LTJ9XSxbOCw3LCIiLDIseyJjdXJ2ZSI6NH1dLFsyMCwyMV0sWzI1LDI2XV0=
\[\begin{tikzcd}[cramped,column sep=tiny,row sep=small]
	&& {[a]} & {(\bullet)} & {+} & {(\bullet)} & {(\bullet)} \\
	&&&&& {[b]} & {[b]} \\
	&&&&&& {(\:\:)} \\
	&&&&&& {[a]} \\
	& {(\bullet)} & {+} & {(\bullet)} & {(\bullet)} & {=} && {[a]} & {(\bullet)} & {(\bullet)} \\
	& {[a]} && {[b]} & {[b]} && {[a]} & {(\:\:)} & {[b]} & {[c]} \\
	{[a]} & {(\:\:)} &&& {(\:\:)} &&&&& {(\:\:)} \\
	&&&& {[a]} &&&&& {[a]}
	\arrow[curve={height=-12pt}, from=1-3, to=1-4]
	\arrow[curve={height=-12pt}, from=1-4, to=1-3]
	\arrow[from=1-6, to=2-6]
	\arrow[from=1-7, to=2-7]
	\arrow[from=2-7, to=3-7]
	\arrow[from=3-7, to=4-7]
	\arrow[from=5-2, to=6-2]
	\arrow[from=5-4, to=6-4]
	\arrow[from=5-5, to=6-5]
	\arrow[from=5-8, to=6-8]
	\arrow[curve={height=12pt}, from=5-9, to=5-8]
	\arrow[from=5-9, to=6-9]
	\arrow[curve={height=24pt}, from=5-10, to=5-8]
	\arrow[from=5-10, to=6-10]
	\arrow[from=6-2, to=7-2]
	\arrow[from=6-5, to=7-5]
	\arrow[curve={height=-12pt}, from=6-7, to=6-8]
	\arrow[curve={height=-12pt}, from=6-8, to=6-7]
	\arrow[from=6-10, to=7-10]
	\arrow[curve={height=-12pt}, from=7-1, to=7-2]
	\arrow[curve={height=-12pt}, from=7-2, to=7-1]
	\arrow[from=7-5, to=8-5]
	\arrow[from=7-10, to=8-10]
\end{tikzcd}\]

\textbf{Parallel Composition}
% https://q.uiver.app/#q=WzAsMzUsWzEsMCwiKFxcYnVsbGV0KSJdLFsxLDEsIlthXSJdLFszLDAsIihcXGJ1bGxldCkiXSxbMywxLCJbYl0iXSxbMywzLCJbYl0iXSxbMywyLCIoXFw6XFw6KSJdLFswLDQsIihcXGJ1bGxldCkiXSxbMCw1LCJbYV0iXSxbMiw0LCIoXFxidWxsZXQpIl0sWzEsNSwiW2NdIl0sWzIsNSwiW2JdIl0sWzAsNiwiKFxcOlxcOikiXSxbMiw2LCIoXFw6XFw6KSJdLFsxLDgsIltjXSJdLFsyLDcsIltiXSJdLFsxLDIsIihcXDpcXDopIl0sWzEsMywiW2JdIl0sWzQsMywiKFxcOlxcOikiXSxbNSwzLCJbYV0iXSxbMiwwLCJcXG1pZFxcbWlkIl0sWzAsNywiW2JdIl0sWzEsNywiW2NdIl0sWzIsOCwiKFxcOlxcOikiXSxbMiw5LCJbYV0iXSxbNCw0LCIoXFxidWxsZXQpIl0sWzUsNSwiW2NdIl0sWzYsNCwiKFxcYnVsbGV0KSJdLFs1LDcsIltjXSJdLFs0LDYsIihcXDpcXDopIl0sWzYsNiwiKFxcOlxcOikiXSxbNiw4LCIoXFw6XFw6KSJdLFs1LDgsIltjXSJdLFszLDQsIj0iXSxbMCwwLCJcXHBhcnRpYWxfSCgiXSxbNCwwLCIpIl0sWzAsMV0sWzIsM10sWzYsN10sWzgsOV0sWzgsMTBdLFs3LDExXSxbMTAsMTJdLFs5LDEyXSxbMTIsMTRdLFsxLDE1XSxbMTUsMTZdLFszLDVdLFs1LDRdLFs0LDE3XSxbMTcsMThdLFs2LDldLFs5LDExXSxbMTEsMjBdLFsxMiwyMV0sWzExLDEzXSxbMjEsMTFdLFsxNCwyMl0sWzIyLDEzXSxbMjIsMjNdLFsyMSwyMl0sWzYsMjFdLFsyNCwyNV0sWzI2LDI1XSxbMjQsMjddLFsyNywyOF0sWzI1LDI4XSxbMjUsMjldLFsyOSwyN10sWzI3LDMwXSxbMzAsMzFdLFsyOCwzMV1d
\[\begin{tikzcd}[cramped,column sep=tiny,row sep=small]
	{\partial_H(} & {(\bullet)} & {\mid\mid} & {(\bullet)} & {)} \\
	& {[a]} && {[b]} \\
	& {(\:\:)} && {(\:\:)} \\
	& {[b]} && {[b]} & {(\:\:)} & {[a]} \\
	{(\bullet)} && {(\bullet)} & {=} & {(\bullet)} && {(\bullet)} \\
	{[a]} & {[c]} & {[b]} &&& {[c]} \\
	{(\:\:)} && {(\:\:)} && {(\:\:)} && {(\:\:)} \\
	{[b]} & {[c]} & {[b]} &&& {[c]} \\
	& {[c]} & {(\:\:)} &&& {[c]} & {(\:\:)} \\
	&& {[a]}
	\arrow[from=1-2, to=2-2]
	\arrow[from=1-4, to=2-4]
	\arrow[from=2-2, to=3-2]
	\arrow[from=2-4, to=3-4]
	\arrow[from=3-2, to=4-2]
	\arrow[from=3-4, to=4-4]
	\arrow[from=4-4, to=4-5]
	\arrow[from=4-5, to=4-6]
	\arrow[from=5-1, to=6-1]
	\arrow[from=5-1, to=6-2]
	\arrow[from=5-1, to=8-2]
	\arrow[from=5-3, to=6-2]
	\arrow[from=5-3, to=6-3]
	\arrow[from=5-5, to=6-6]
	\arrow[from=5-5, to=8-6]
	\arrow[from=5-7, to=6-6]
	\arrow[from=6-1, to=7-1]
	\arrow[from=6-2, to=7-1]
	\arrow[from=6-2, to=7-3]
	\arrow[from=6-3, to=7-3]
	\arrow[from=6-6, to=7-5]
	\arrow[from=6-6, to=7-7]
	\arrow[from=7-1, to=8-1]
	\arrow[from=7-1, to=9-2]
	\arrow[from=7-3, to=8-2]
	\arrow[from=7-3, to=8-3]
	\arrow[from=7-5, to=9-6]
	\arrow[from=7-7, to=8-6]
	\arrow[from=8-2, to=7-1]
	\arrow[from=8-2, to=9-3]
	\arrow[from=8-3, to=9-3]
	\arrow[from=8-6, to=7-5]
	\arrow[from=8-6, to=9-7]
	\arrow[from=9-3, to=9-2]
	\arrow[from=9-3, to=10-3]
	\arrow[from=9-7, to=9-6]
\end{tikzcd}\]

\end{xmp}



\newpage


\lipsum[1-12]

\end{multicols}
\end{document}
