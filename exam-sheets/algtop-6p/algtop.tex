\documentclass[landscape, 8pt]{extarticle}
\usepackage{geometry}
% \usepackage{showframe}
\usepackage[dvipsnames]{xcolor}

\colorlet{colour1}{Red}
\colorlet{colour2}{Green}
\colorlet{colour3}{Cerulean}

\geometry{
    a4paper, 
    margin=0.17in
}

\pretolerance=0
\hyphenpenalty=0

\usepackage{lmodern}

\usepackage[fontsize=7pt]{scrextend}

\usepackage{graphicx} % Required for inserting images
\usepackage{amsmath}
\usepackage{amsfonts}
\usepackage{amssymb}
% \usepackage{preamble}
\usepackage{enumitem}
\usepackage{multicol}
\usepackage{lipsum}
\usepackage[framemethod=TikZ]{mdframed}
% \usepackage{../thmboxes_white}
\usepackage{../../thmboxes_v3}
\usepackage{float}
% \usepackage{setspace}
\usepackage[nodisplayskipstretch]{setspace}





% \setlength{\parskip}{0pt}

% Custom Definitions of operators
% \DeclareMathOperator{\im}{im}
% \DeclareMathOperator{\Fix}{Fix}
% \DeclareMathOperator{\Orb}{Orb}
% \DeclareMathOperator{\Stab}{Stab}
% \DeclareMathOperator{\send}{send}
% \DeclareMathOperator{\dom}{dom}
% \DeclareMathOperator{\Maps}{Maps}
% \DeclareMathOperator{\sgn}{sgn}
% \DeclareMathOperator{\Mat}{Mat}
% \DeclareMathOperator{\scale}{sc}
% \DeclareMathOperator{\Hom}{Hom}
% \DeclareMathOperator{\id}{id}
% \DeclareMathOperator{\rk}{rk}
% \DeclareMathOperator{\Tr}{tr}
% \DeclareMathOperator{\diag}{diag}
% \DeclareMathOperator{\can}{can}

\usepackage{hyperref} % note: this is the final package

\parindent = 0pt

\renewcommand\labelitemi{\tiny$\bullet$}

\begin{document}

\setlength{\abovedisplayskip}{3.5pt}
\setlength{\belowdisplayskip}{3.5pt}
\setlength{\abovedisplayshortskip}{3.5pt}
\setlength{\belowdisplayshortskip}{3.5pt}

\begin{multicols}{3}
\raggedcolumns


\section*{\huge Galois Theory Notes}
Made by Leon :) \textit{Note: Any reference numbers are to the lecture notes}

\section{Topologies and Connectedness}\setcounter{subsection}{1}
\begin{rcl}[Topology]{rcl:topology}{}
    An (open) topology on $X$ is a collection of subsets $\tau \subset P(X)$ such that
    \begin{enumerate}[leftmargin=*]
        \item $\emptyset, X \in \tau$,
        \item Closed under finite intersections: If $\{U_{1},\dots,U_{n}\} \subset \tau$ then $\bigcap_{i=1,\dots,n} U_{i} \in \tau$
        \item Closed under arbitrary unions: If $\{U_{i}\}_{i\in I} \subset \tau$ is a family of opens, then $\bigcup_{i\in I} U_{i}\in \tau$
    \end{enumerate}
    The subsets $U\in \tau$ are called \textbf{open} and their complements in $X$ define \textbf{closed} subsets

    A subset $A \subset X $ is called \textbf{clopen} if it is both closed and open
\end{rcl}

\begin{dfn}[Connected spaces]{dfn:connectedness}{}
    A topological space $X$ is \textbf{connected} if $X = A \amalg B$ with $A, B \subset X$ open implies that $A = \emptyset$ or $A = X$
\end{dfn}

\begin{ppn}[Connectedness relations]{ppn:connectedness-clopen}{}
    \begin{itemize}
        \item A topological space $X$ is connected iff the only copens are $\emptyset$ and $X$
        \item Let $f : X \to Y$ be a continuous map of topological spaces and let $X$ be connected. Then $f(X)$ is connected.
    \end{itemize}
\end{ppn}


\lipsum[1-12]
\end{multicols}
\end{document}
